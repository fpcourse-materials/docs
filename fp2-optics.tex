%! suppress = MissingLabel

В функциональном программировании принято персистентно работать с данными.
В этой главе мы рассмотрим, когда и почему это важно.
А так же~--- функциональную оптику~--- набор абстракций для удобной работы с данными.

\subsection{Персистентные структуры данных}

Структуры данных разделяют на \vocab{изменяемые (mutable)} и \vocab{неизменяемые (immutable)}.
Это классификация лишь по внутренней реализации, используются изменяемые ячейки памяти или нет.

По использованию выделают \vocab{эфемерные} и \vocab{персистентные} структуры данных. % todo verify
Изменения эфемерной структуры данных можно наблюдать по той же ссылке:
\begin{minted}{Swift}
    let xs = MutableList<Int>.empty()
    xs.add(42)
    print(xs)
\end{minted}
В то время как изменение персистентной порождает каждый раз новые ссылки, а по старым остаются доступны предыдущие версии структуры:
\begin{minted}{haskell}
    let xs = []
    let xs' = 42 : xs
    print xs
\end{minted}

Может показаться, что эфемерные являются изменяемыми, а персистентные --- неизменяемыми.
На самом деле это более-менее ортогональные классификации.
Так, с помощью копирования можно к изменяемой структуре данных предоставить персистентный интерфейс, а неизменяемой --- эфемерный (с помощью монады \mintinline{haskell}|State|).

В речи, когда говорят о персистентных структурах данных, часто имеют в виду структуры данных с персистентным интерфейсом, специально оптимизированные для него (не требуют полного копирования на каждую операцию).
Такие структуры данных бывают на удивление оптимальными и разнообразными~\cite{okasaki1999purely}.

Например, можно реализовать эффективный персистентный массив с логарифмической сложностью всех операций.
В Haskell такой структурой данных является \mintinline{haskell}|Seq|\footnote{\url{https://hackage.haskell.org/package/containers-0.7/docs/Data-Sequence.html}}\footnote{\url{http://www.staff.city.ac.uk/~ross/papers/FingerTree.html}}.
Если в вершинах хранить небольшие массивы, которые современные аритектуры процессоров могут эффективно копировать, можно существенно уменьшить высоту дерева и алгоритмическую сложность операций (e.g. \mintinline{scala}|scala.immutable.Vector|).

\begin{task}
    Реализуйте не Haskell персистентное декартово дерево по неявному ключу.
    Какие особенности Haskell усложняют использование этой структуры?
\end{task}

Когда использовать эфемерные, а когда персистентные структуры данных?
Если сравнивать, то можно обнаружить, что эфемерные можно реализовывать эффективнее во многих случаях: кеши процессоров лучше работают с локальными данными, меньше аллокаций и индирекций.
В то же время персистентные структуры обречены быть деревьями, чтобы реаллоцировать не структуру целиком, а только путь до корня.

Персистентные структуры же позволяют писать более модульный и безопасный с точки зрения многопоточности код, который может не учитывать возможность изменения структуры по ссылке.
В то время как работа с эфемерной структурой не является чистым кодом, а включает в себя порождение наблюдаемых побочных эффектов.
Также объединение персистентных результатов разных вызовов может быть дешевле, как, например, конкатенация персистентных массивов дешевле конкатенации эфемерных (логарифмическая сложность против линейной).

Таким образом, в рамках ограниченного, легко обозреваемого скоупа лучше использовать эфемерные структуры ввиду их эффективности (например, чтобы изначально заполнить коллекцию элементами).
Однако, через границы абстракции лучше пропускать только персистентные структуры (или же положиться на систему эффектов, см.~\ref{sec:effect-systems}).
То есть каждая структура данных должна поддерживать две фазы своей жизни.
Например, так сделано в Scala, где у многих персистентных коллекций есть \mintinline{scala}|Builder| версия.

\subsection{Идея оптики}

\vocab{Функциональная оптика} позволяет строить функциональные ссылки, фокусирующиеся на определённых свойствах структур данных.
В дальнейшем, с помощью элиминирующих функций, можно по функциональной ссылке и объекту прочитать свойство или персистентно обновить его.
Например, мы можем сфокусироваться на первой компоненте пары с помощью ссылки \texttt{\_1} и прочитать первую компоненту:
\begin{minted}{haskell}
    ghci> view _1 (42, 43)
    42
\end{minted}
Или можем сфокусироваться на свойстве наличия в множестве определённого элемента \texttt{member}, выставить свойство в \mintinline{haskell}{True} и получить новое множество с нужным элементом:
\begin{minted}{haskell}
    ghci> set (member 42) (Set.fromList [1, 2]) True
    Set.fromList [1, 2, 42]
\end{minted}

Функциональные ссылки композируются (тут, с помощью \texttt{\%}).
Так, если множество будет в первой компоненте пары, мы всё ещё можем добавить в него элемент:
\begin{minted}{haskell}
    ghci> set (_1 % member 42) (Set.fromList [1, 2], 3) True
    (Set.fromList [1, 2, 42], 3)
\end{minted}

Важно заметить, что функциональные ссылки, в отличие от обычных указателей, с одной стороны более абстрактны и не обязательно просто ссылаются на область памяти, а с другой~--- отделены от конкретного объекта, чем скорее напоминают указатели на методы в C++.

Таким образом, оптика помогает приблизить удобство программирования с персистентными структурами данных к удобству программирования с мутабельными.
Это важно, поскольку хорошие практики программирования должны быть как минимум так же доступны, как и менее оптимальные.

Функциональные ссылки могут предоставлять различные интерфейсы использования, которые называют \vocab{оптическими девайсами}.
Девайсы образуют иерархию (рис.~\ref{fig:optics-hierarchy}): при композиции двух девайсов, результирующий является ближайшим общим предком с функциональностью пересечения функциональностей исходных.

\begin{figure}
    \centering
    \includegraphics[width=\textwidth]{figs/optics-hierarchy}
    \caption{Иерархия оптических девайсов библиотеки \texttt{optics}.}
    \label{fig:optics-hierarchy}
\end{figure}

Обе ссылки выше являются оптическим девайсом \vocab{линза}, который исторически появился первым.
Изначально линзы были предложены как решение проблемы view-update в базах данных~\cite{bohannon2006relational, foster2008quotient}.
А именно --- как нужно изменить реальную базу при изменении view.
Или как восстанавливать целое после извлечения и изменения части.
Линзы тут являются средством двустороннего программирования --- позволяют одновременно описывать как view, так и способ обновления.
Параллельно\footnote{\url{https://github.com/ekmett/lens/wiki/History-of-Lenses}}, линзы упоминаются в серии блог-постов, описывающих попытку удобнее работать с изменяемым состоянием при написании игр\footnote{\href{https://web.archive.org/web/20140402193032/https://lukepalmer.wordpress.com/2007/07/26/making-haskell-nicer-for-game-programming/}{(post) Making Haskell nicer for game programming}}\footnote{\href{https://web.archive.org/web/20120303223802/https://lukepalmer.wordpress.com/2007/08/05/haskell-state-accessors-second-attempt-composability/}{(post) Haskell State Accessors (second attempt: Composability)}} (линзы в играх продолжают радовать\footnote{\url{http://www.timphilipwilliams.com/posts/2019-07-25-minecraft.html}}).

Существует множество библиотек оптики как для Haskell\footnote{\url{https://hackage.haskell.org/package/optics-0.1/docs/Optics.html}}\footnote{\url{http://lens.github.io/}}\footnote{\url{https://github.com/marcosh/existential-optics/tree/main}}, так и для других языков: Kotlin\footnote{\url{https://arrow-kt.io/}}, Scala\footnote{\url{https://github.com/optics-dev/Monocle}}, Swift\footnote{\url{https://github.com/swiftlang/swift-evolution/blob/main/proposals/0161-key-paths.md}} (языковая поддержка).

\subsection{Использование оптики}

Рассмотрим для примера использование нескольких оптических девайсов библиотеки \texttt{optics}, имеющей прекрасную документацию\footnote{\url{https://hackage.haskell.org/package/optics-0.4.2.1/docs/Optics.html}}.

% todo

Примеры:
\begin{minted}{haskell}
    data Pet = Cat String | Dog Int deriving Show
    newtype UserName = UserName { _getUserName :: String } deriving Show
    data User = User { _userName :: UserName, _userCats :: [Pet] } deriving Show

    exampleUser :: User
    exampleUser = User (UserName "Bob") [Cat "Kitty", Dog 42, Dog 4]

    exampleName :: User -> User
    exampleName = over (userName % getUserName) (++ "!")

    exampleIx :: User -> User
    exampleIx = set (userName % getUserName % ix 2) 'k'

    exampleFold :: User -> Int
    exampleFold = getSum $ foldMapOf (userName % getUserName % folded) (const 1)

    examplePrint :: User -> IO User
    examplePrint = traverseOf (userCats % traversed % _Dog) (\x -> x <$ print x)
\end{minted}

TODO % todo

\subsection{Data optics}

\subsubsection{Линзы}

Простейшую линзу образуют пара функций --- просмотр и установка свойства:
\begin{minted}{haskell}
    data SimpleLens' s a = SimpleLens'
      { view' :: s -> a
      , set' :: s -> a -> s
      }
\end{minted}

На эти функции накладываются естественные законы:
\begin{minted}{haskell}
    view l (set l s x) ?$\equiv$? x
    set l s (view l s) ?$\equiv$? s
    set l (view l s x) y ?$\equiv$? set l s y
\end{minted}

Например, можно легко изменить возраст пользователя:
\begin{minted}{haskell}
    newtype Age = Age { _getAge :: Int }
    data User = User { _userName :: String, _userAge :: Age }

    userAge :: SimpleLens' User Age
    userAge = SimpleLens' { view' = _userAge, set' = \s x -> s { _userAge = x } }

    getAge :: SimpleLens' Age Int
    getAge = SimpleLens' { view' = _getAge, set' = \s x -> s { _getAge = x } }

    ghci> set (userAge % getAge) user 1
\end{minted}

Можно обобщить линзы до полиморфных линз, которые позволяют пересоздавать структуру с новыми типовыми параметрами:
\begin{minted}{haskell}
    data SimpleLens s t a b = SimpleLens
      { view :: s -> a
      , set :: s -> b -> t
      }

    _1 :: SimpleLens (a, c) (b, c) a b
    _1 = SimpleLens { view = \(x, _) -> x, set = \(x, c) y -> (y, c) }
\end{minted}

Можно заметить, что линза в таком представлении является комонадой коалгеброй коstate\footnote{\href{https://youtu.be/9_iYlp8smc8?si=NLka0vnnhYDdTfgm}{(youtube) Category Theory II 9.1: Lenses.}}.
Действительно, пару функций можно заменить на функцию, возвращающую пару:
\begin{minted}{haskell}
    data Store a s = Store (a, a -> s)
    data DataLens s a = DataLens (s -> Store a s)
\end{minted}

Если определить инстанс комонады для \mintinline{haskell}|Store a s| и выписать законы совместимости с коалгеброй, мы получим в точности законы линз.

TODO % todo

\subsubsection{Призмы}

Призмы позволяют получить свойство, которое может отсутствовать, и устанавливать его при наличии.
Так, конкретное поле типа-суммы может не удаться извлечь, при передаче не ожидаемого конструктора.

Простые призмы можно определить аналогично линзам, с учётом возможного отсутствия свойства:

\begin{minted}{haskell}
    data SimplePrism' s a = SimplePrism'
      { preview' :: s -> Maybe a
      , review' :: a -> s
      }
\end{minted}

Полиморфная призма должна предоставить структуру с обновлённым типом в случае неуспеха просмотра:
\begin{minted}{haskell}
    data SimplePrism s t a b = SimplePrism
      { preview :: s -> Either t a
      , review :: b -> t
      }
\end{minted}

Можно определить призму для работы с содержимым конструктора \mintinline{haskell}|Left|:
\begin{minted}{haskell}
    _Left :: SimplePrism (Either a c) (Either b c) a b
    _Left = SimplePrism
      { preview = \case Left a -> Right a; Right c -> Left (Right c)
      , review = Left
      }
\end{minted}

Композиция призм определяется естественным образом.

% todo сериализация это призма

\subsubsection{Композиция линз и призм}

Заведём класс типов, с помощью которого научимся композировать различные комбинации линз и призм.
Результирующий девайс однозначно определяется операндами композиции.

\begin{minted}{haskell}
    class Composable o1 o2 o3 | o1 o2 -> o3 where
      compose :: o1 s t a b -> o2 a b c d -> o3 s t c d
\end{minted}

Теперь можем естественным образом определить композицию для различных девайсов.
Можно заметить, что этот подход требует квадратичное количество инстансов от количества оптических девайсов.

\begin{minted}{haskell}
    instance Composable SimpleLens SimpleLens SimpleLens where
      compose :: SimpleLens s t a b -> SimpleLens a b c d -> SimpleLens s t c d
      compose l2 l1 = SimpleLens
        { view = view l1 . view l2
        , set = \s x -> set l2 s (set l1 (view l2 s) x)
        }
\end{minted}

Композиция линз и призм образует другой девайс --- аффинные траверсы, которые являются чем-то средним.
Наличие призмы в композиции не позволяет гарантированно просматривать свойство, а линзы --- создавать новое значение без изначального.
\begin{minted}{haskell}
    data AffineTraversal s t a b = AffineTraversal
      { apreview :: s -> Either t a
      , aset :: s -> b -> t
      }

    instance Composable SimplePrism SimpleLens SimplePrism where
      compose :: SimplePrism s t a b -> SimpleLens a b c d -> SimplePrism s t c d
      compose p l = SimplePrism
        { preview = fmap (view l) . preview p
        , pset = \s d -> case preview p s of
            Left t -> t
            Right a -> pset p s (set l a d)
        }
\end{minted}

\subsection{Путь к profunctor optics}

TODO % todo

\subsection{Генерация оптики}

TODO % todo


















%Например, линза может фокусироваться на знак числа, позволяя его просматривать и устанавливать.
%Это свойство имеет тип \mintinline{haskell}|Bool|:
%\begin{minted}{haskell}
%    sign :: Lens' Int Bool
%\end{minted}
%


%
%\subsubsection{Линзы --- costate coalgebra comonad}
%
% todo картинки
%

%

%

%




%\subsection{Другие представления оптики}
%
%Ранее мы рассмотрели тривиальную оптику, составленную непосредственно из кортежа элиминирующих функций (пары разбора-пересборки значений).
%Однако, у такого подхода есть существенные недостатки.
%Во-первых, нужно определять много операций композиции для различных видов оптики.
%То есть композиция девайсов --- что-то гораздо более сложное, чем композиция функций.
%Во-вторых, это представление не очень эффективно, так как нужно инлайнить вызовы функций из структур данных. % todo девиртуализировать в терминах llvm
%
%Было приложено много усилий и теорката\footnote{\url{https://bartoszmilewski.com/category/lens/}}, чтобы получить другие представления оптики.
%Некоторые из них являются просто одной функцией, кодирующей все нужные преобразования, а композиция определена как композиция таких функций.
%
%\subsubsection{Semantic editor combinators}
%
%TODO semantic editor combinators\footnote{\url{http://conal.net/blog/posts/semantic-editor-combinators}} % todo
%
%\subsubsection{Линзы ван Лаарховена}
%
%TODO van Laarhoven lens\footnote{\url{https://www.twanvl.nl/blog/haskell/cps-functional-references}}\footnote{\href{http://r6.ca/blog/20120623T104901Z.html}{(post) Polymorphic Update with van Laarhoven Lenses}} % todo
%
%TODO lens\footnote{\url{http://lens.github.io/}} % todo
%
%% todo uniplate
%
%\subsubsection{Profunctor optics}
%
%TODO % todo немножко profunctor optics
%
%% todo ссылки на Бартоша
%
%\subsubsection{Existential (coend) optics}
%
%TODO\footnote{\url{https://github.com/marcosh/existential-optics/tree/main}}\footnote{\url{https://www.tweag.io/blog/2022-05-05-existential-optics/}}\footnote{\url{https://www.twanvl.nl/blog/haskell/isomorphism-lenses}}\footnote{\url{https://www.brunogavranovic.com/posts/2022-01-05-lenses-to-the-left-of-me.html}} % todo
%
%
%\subsection{Генерация оптики}
%
%Чтобы использовать оптику для работы со своими структурами данных, нужно так или иначе получить для них реализацию девайсов.
%
%Некоторую специфичную оптику можно написать руками.
%Например, позволяющую работать с нетривиальным свойством значения (знак числа, высота дерева\ldots).
%Библиотеки предоставляют специальные билдеры, упрощающие конструирование\footnote{\url{https://hackage.haskell.org/package/optics-core-0.1/docs/Optics-Lens.html\#v:lens}}.
%
%Некоторая оптика получается автоматически из инстансов стандартных классов типов.
%Например, свёртки и траверсы\footnote{\url{https://hackage.haskell.org/package/optics-core-0.1/docs/Optics-Traversal.html\#g:6}}.
%Дальше уже встаёт вопрос о генерации инстансов классов типов (см.~\ref{sec:datatype-generic}).
%
%Поскольку линзы являются просто обобщением полей, а призмы --- конструкторов, их можно генерировать автоматически с помощью макросов по структуре данных\footnote{\url{https://hackage.haskell.org/package/optics-th-0.1/docs/Optics-TH.html}}.
%Для этого используются конвенции, поля называются с подчёркиванием, а макрос сгенерирует линзы без подчёркивания.
%С призмами наоборот.
%Так, генерация для примера выше (\ref{subsec:optics}) будет выглядеть следующим образом:
%\begin{minted}{haskell}
%    makeLenses ''User
%    makeLenses ''UserName
%    makePrisms ''Pet
%\end{minted}
%
%TODO библиотека  % todo
%
%TODO data-generic optics\footnote{\url{https://hackage.haskell.org/package/optics-core-0.4.1.1/docs/Optics-Label.html}}\footnote{\url{https://ghc.gitlab.haskell.org/ghc/doc/users_guide/exts/overloaded_record_update.html}}\footnote{\url{https://ghc.gitlab.haskell.org/ghc/doc/users_guide/exts/overloaded_labels.html}} % todo
%
%TODO % todo


% todo библиотеки с оптикой для других языков

% todo transducers

% todo слайды Беляева

% todo zippers

% todo сравнение с datatype generic programming

% todo оптика как альтернатива стримам?

% todo first-class patterns should also be able to re-build the values that they match, Pattern Synonyms paper, кастомные паттерны это та же оптика только с фиксированной структурой, которая помогает с exhaustiveness

% todo data processing

% todo https://blog.poisson.chat/
% todo lens and serialization


% todo общее между оптикой и data-дженериками - нужно универсально работать с разными структурами, хоть они и все так или иначе суммы, например, мы это знаем. То есть вид структуры знаем, а конкретные конструкторы и прочее - нет.

%Рассмотрим самую вершину башни интерпретаторов.
%Там интерпретируется программа, которая сама по себе уже не является интерпретатором, а представляет собой API или UI запрос от внешнего мира.
%Эта программа интерпретируется в некоторое преобразование данных $d_{in}\rightsquigarrow d_{out}$, которые в свою очередь уже не являются программой (поскольку им не даётся семантика).
%Как выглядят эти преобразования, как их композиционно описывать?
%
%\[
%    d_{out} =
%    U^{App}\left(
%    \left<
%    U_{App}^{API/UI},
%    \left<
%    p_{API/UI},
%    d_{in}
%    \right>
%    \right>
%    \right)
%\]
