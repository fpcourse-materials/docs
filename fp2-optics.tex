%! suppress = MissingLabel

Рассмотрим самую вершину башни интерпретаторов.
Там интерпретируется программа, которая сама по себе уже не является интерпретатором, а представляет собой API или UI запрос от внешнего мира.
Эта программа интерпретируется в некоторое преобразование данных $d_{in}\rightsquigarrow d_{out}$, которые в свою очередь уже не являются программой (поскольку им не даётся семантика).
Как выглядят эти преобразования, как их композиционно описывать?

\[
    d_{out} =
    U^{Has}\left(
    \left<
    U_{Has}^{API/UI},
    \left<
    p_{API/UI},
    d_{in}
    \right>
    \right>
    \right)
\]

Решением является \vocab{функциональная оптика}.
Она позволяет фокусироваться на некоторые свойства объекта и персистентно восстанавливать целый объект с новым значением свойств.
Разные средства фокусировки классифицируют как различные \vocab{оптические девайсы} по интерфейсу использования.
Так, \vocab{линзы} позволяют сфокусироваться на поле типа-произведения, \vocab{призмы} --- типа-суммы, \vocab{траверсы} --- коллекции\ldots

\subsection{Персистентные структуры данных}

Структуры данных разделяют на \vocab{изменяемые (mutable)} и \vocab{неизменяемые (immutable)}.
Это классификация лишь по внутренней реализации, используются изменяемые ячейки памяти или нет.

По использованию выделают \vocab{эфемерные} и \vocab{персистентные} структуры данных.
После работы с эфемерной структурой, по той же ссылке может быть доступна другая структура.
\begin{minted}{kotlin}
    let xs = emptyMutableList<Int>()
    xs.add(42)
    print(xs)
\end{minted}

Работа с персистентными структурами порождает каждый раз новые ссылки, в то время как по старым доступна изначальная структура.
\begin{minted}{haskell}
    let xs = []
    let xs' = 42 : xs
    print xs'
\end{minted}

Может показаться, что эфемерные являются изменяемыми, а персистентные --- неизменяемыми.
На самом деле это более-менее ортогональные классификации.
Так, с помощью копирования можно к изменяемой структуре данных предоставить персистентный интерфейс, а неизменяемой --- эфемерный (с помощью монады \mintinline{haskell}|State|).

В речи, когда говорят о персистентных структурах данных, часто имеют в виду структуры данных с персистентным интерфейсом, специально оптимизированные для него (не требуют полного копирования на каждую операцию).
Способы построения таких структур и работы с ними на удивление довольно оптимальны и разнообразны~\cite{okasaki1999purely}.

Например, можно реализовать эффективный персистентный массив с логарифмической сложностью всех операций.
В Haskell такой структурой данных является \mintinline{haskell}|Seq|\footnote{\url{https://hackage.haskell.org/package/containers-0.7/docs/Data-Sequence.html}}\footnote{\url{http://www.staff.city.ac.uk/~ross/papers/FingerTree.html}}.
Если в вершинах хранить небольшие массивы, которые современные аритектуры процессоров могут эффективно копировать, можно существенно уменьшить высоту дерева и алгоритмическую сложность операций (e.g. \mintinline{scala}|scala.immutable.Vector|).

\begin{task}
    Реализуйте не Haskell персистентное декартово дерево по неявному ключу.
    Какие особенности Haskell усложняют использование этой структуры?
\end{task}

Когда использовать эфемерные, а когда персистентные структуры данных?
Если сравнивать, то можно обнаружить, что эфемерные можно реализовывать эффективнее во многих случаях, так как память изменяемая и кеши процессоров лучше работают с локальными данными (в то время как персистентные структуры обречены быть деревьями, чтобы реаллоцировать не структуру целиком, а только путь до корня).

Персистентные структуры же позволяют писать более модульный и безопасный с точки зрения многопоточности код, который может не учитывать возможность изменения структуры по ссылке.
В то время как работа с эфемерной структурой не является чистым кодом, а включает в себя порождение побочных эффектов.
Также объединение персистентными результатов разных вызовов может быть дешевле, как, например, конкатенация персистентных массивов дешевле конкатенации эфемерных (логарифмическая сложность против линейной).

Таким образом, в рамках ограниченного, легко обозреваемого скоупа лучше использовать эфемерные структуры ввиду их эффективности (например, чтобы изначально заполнить коллекцию элементами).
Однако, через границы абстракции лучше пропускать только персистентные структуры (или же положиться на систему эффектов, см.~\ref{sec:effect-systems}).
То есть каждая структура данных должна поддерживать две фазы своей жизни.
Например, так сделано в Scala, где у многих персистентных коллекций есть \mintinline{scala}|Builder| версия.

\subsection{Простейшая оптика}

Линзы являются простейшим примером оптического девайса.
Они позволяют гарантированно обозревать одно свойство и устанавливать его.

Изначально линзы были предложены как решение проблемы view-update в базах данных~\cite{bohannon2006relational, foster2008quotient}.
А именно --- как нужно изменить реальную базу при изменении view.
Или как восстанавливать целое после извлечения и изменения части.
Линзы тут являются средством двустороннего программирования --- позволяют одновременно описывать как view, так и способ обновления.

Параллельно\footnote{\url{https://github.com/ekmett/lens/wiki/History-of-Lenses}}, линзы упоминаются в серии блог-постов, описывающих попытку удобнее работать с изменяемым состоянием при написании игр\footnote{\href{https://web.archive.org/web/20140402193032/https://lukepalmer.wordpress.com/2007/07/26/making-haskell-nicer-for-game-programming/}{(post) Making Haskell nicer for game programming}}\footnote{\href{https://web.archive.org/web/20120303223802/https://lukepalmer.wordpress.com/2007/08/05/haskell-state-accessors-second-attempt-composability/}{(post) Haskell State Accessors (second attempt: Composability)}} (линзы в играх продолжают радовать\footnote{\url{http://www.timphilipwilliams.com/posts/2019-07-25-minecraft.html}}).

\subsubsection{Линзы --- costate coalgebra comonad}

% todo картинки

Простейшую линзу образуют пара функций --- просмотр и установка свойства:
\begin{minted}{haskell}
    data SimpleLens' s a = SimpleLens'
      { view' :: s -> a
      , set' :: s -> a -> s
      }
\end{minted}

На эти функции накладываются естественные законы:
\begin{minted}{haskell}
    view l (set l s x) ?$\equiv$? x
    set l s (view l s) ?$\equiv$? s
    set l (view l s x) y ?$\equiv$? set l s y
\end{minted}

Линзы можно композировать и обращаться к вложенным полям произведений.
Так, они могут быть морфизмами в категории типов.
При таком определении, правда, композиция работает в обратном, от желаемого, порядке.
\begin{minted}{haskell}
    instance Category SimpleLens' where
      id :: SimpleLens' s s
      id = SimpleLens' { view' = id, set' = flip const }

      (.) :: SimpleLens' a b -> SimpleLens' s a -> SimpleLens' s b
      l1 . l2 = SimpleLens'
        { view' = view' l1 . view' l2
        , set' = \s x -> set' l2 s (set' l1 (view' l2 s) x)
        }
\end{minted}

Например, можно легко изменить возраст пользователя:
\begin{minted}{haskell}
    newtype Age = Age { _getAge :: Int }
    data User = User { _userName :: String, _userAge :: Age }

    userAge :: SimpleLens' User Age
    userAge = SimpleLens' { view' = _userAge, set' = \s x -> s { _userAge = x } }

    getAge :: SimpleLens' Age Int
    getAge = SimpleLens' { view' = _getAge, set' = \s x -> s { _getAge = x } }

    ghci> set (getAge . userAge) user 1
\end{minted}

Можно обобщить линзы до полиморфных линз, которые позволяют пересоздавать структуру с новыми типовыми параметрами:
\begin{minted}{haskell}
    data SimpleLens s t a b = SimpleLens
      { view :: s -> a
      , set :: s -> b -> t
      }

    _1 :: SimpleLens (a, c) (b, c) a b
    _1 = SimpleLens { view = \(x, _) -> x, set = \(x, c) y -> (y, c) }
\end{minted}

Можно заметить, что линза в таком представлении является комонадой коалгеброй коstate.
Действительно, пару функций можно заменить на функцию, возвращающую пару.
Эта пара~--- коstate, инвертировав стрелки в \mintinline{haskell}|State s a| получим \mintinline{haskell}|Either s a -> s| или \mintinline{haskell}|Store a s|.
\begin{minted}{haskell}
    data Store a s = Store (a, a -> s)
    data DataLens s a = DataLens (s -> Store a s)
\end{minted}

Если определить инстанс комонады для \mintinline{haskell}|Store a s| и выписать законы совместимости с коалгеброй, мы получим в точности законы линз.

\subsubsection{Призмы}

Призмы позволяют получить свойство, которое может отсутствовать, и устанавливать его при наличии.
Так, конкретное поле типа-суммы может не удаться извлечь, при передаче не ожидаемого конструктора.

Простые призмы можно определить аналогично линзам, с учётом возможного отсутствия свойства:

\begin{minted}{haskell}
    data SimplePrism' s a = SimplePrism'
      { preview' :: s -> Maybe a
      , pset' :: s -> a -> s
      }
\end{minted}

Полиморфная призма должна предоставить структуру с обновлённым типом в случае неуспеха просмотра:
\begin{minted}{haskell}
    data SimplePrism s t a b = SimplePrism
      { preview :: s -> Either t a
      , pset :: s -> b -> t
      }
\end{minted}

Можно определить призму для работы с содержимым конструктора \mintinline{haskell}|Left|:
\begin{minted}{haskell}
    _Left :: SimplePrism (Either a c) (Either b c) a b
    _Left = SimplePrism
      { preview = \case Left a -> Right a; Right c -> Left (Right c)
      , pset = \case Left _ -> Left; Right c -> const (Right c)
      }
\end{minted}

Композиция призм определяется естественным образом.

\subsubsection{Композиция линз и призм}

Заведём класс типов, с помощью которого научимся композировать различные комбинации линз и призм.
Результирующий девайс однозначно определяется операндами композиции.

\begin{minted}{haskell}
    class Composable o1 o2 o3 | o1 o2 -> o3 where
      compose :: o1 s t a b -> o2 a b c d -> o3 s t c d
\end{minted}

Теперь можем естественным образом определить композицию для различных девайсов.
Можно заметить, что этот подход требует квадратичное количество инстансов от количества оптических девайсов.

\begin{minted}{haskell}
    instance Composable SimplePrism SimpleLens SimplePrism where
      compose :: SimplePrism s t a b -> SimpleLens a b c d -> SimplePrism s t c d
      compose p l = SimplePrism
        { preview = fmap (view l) . preview p
        , pset = \s d -> case preview p s of
            Left t -> t
            Right a -> pset p s (set l a d)
        }

    instance Composable SimpleLens SimpleLens SimpleLens where
      compose :: SimpleLens s t a b -> SimpleLens a b c d -> SimpleLens s t c d
      compose l2 l1 = SimpleLens
        { view = view l1 . view l2
        , set = \s x -> set l2 s (set l1 (view l2 s) x)
        }
\end{minted}

В действительности мы определили не совсем призмы\footnote{\url{https://hackage.haskell.org/package/optics-core-0.1/docs/Optics-Prism.html}}, а скорее аффинные траверсы\footnote{\url{https://hackage.haskell.org/package/optics-core-0.1/docs/Optics-AffineTraversal.html}}, которые и получаются в результате композиции линз и призм.
Призмы же, в отличие от аффинных траверсов, имеют дополнительную операцию \mintinline{haskell}|review :: b -> t|, которая позволяет реконструировать тип-суммы по содержимому компоненты.

\subsection{Разнообразие оптических девайсов, \texttt{optics}}



TODO библиотека optics\footnote{\url{https://hackage.haskell.org/package/optics-0.4.2.1/docs/Optics.html}} % todo

TODO data-generic optics\footnote{\url{https://hackage.haskell.org/package/optics-core-0.4.1.1/docs/Optics-Label.html}}\footnote{\url{https://ghc.gitlab.haskell.org/ghc/doc/users_guide/exts/overloaded_record_update.html}}\footnote{\url{https://ghc.gitlab.haskell.org/ghc/doc/users_guide/exts/overloaded_labels.html}} % todo

\subsection{Другие реализации оптики}

\subsubsection{Semantic editor combinators}

TODO semantic editor combinators\footnote{\url{http://conal.net/blog/posts/semantic-editor-combinators}} % todo

\subsubsection{Линзы ван Лаарховена}

TODO van Laarhoven lens\footnote{\url{https://www.twanvl.nl/blog/haskell/cps-functional-references}}\footnote{\href{http://r6.ca/blog/20120623T104901Z.html}{(post) Polymorphic Update with van Laarhoven Lenses}} % todo

TODO lens\footnote{\url{http://lens.github.io/}} % todo

\subsubsection{Profunctor optics}

% todo немножко profunctor optics


% todo transducers

% todo слайды Беляева

% todo zippers

% todo сравнение с datatype generic programming

% todo оптика как альтернатива стримам?

% todo first-class patterns should also be able to re-build the values that they match, Pattern Synonyms paper

% todo data processing
