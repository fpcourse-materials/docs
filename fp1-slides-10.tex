%! suppress = EscapeHashOutsideCommand
%! suppress = Quote
%! suppress = MissingImport
%! suppress = MissingLabel
%! suppress = LineBreak

% CLI args https://tex.stackexchange.com/a/1501
\newif\ifhandout
\input{flags}

%! suppress = MissingLabel
%! suppress = DocumentclassNotInRoot
%! suppress = DiscouragedUseOfDef

% * Make friends tikz & colors
%   https://en.wikibooks.org/wiki/LaTeX/Colors
% * To enable vertical top alignment globally
%   https://tex.stackexchange.com/questions/9889/positioning-content-at-the-top-of-a-beamer-slide-by-default
% * Set handout from CLI
%   https://tex.stackexchange.com/a/1501
\ifhandout
\documentclass[usenames, dvipsnames, handout]{beamer} % https://tex.stackexchange.com/questions/224091/beamer-how-to-disable-pause-temporarily
\else
\documentclass[usenames, dvipsnames]{beamer}
\fi
% ------------------------------------------------

% Graphics
\usepackage{color}
\usepackage{tabularx}
\usepackage{tikz}
% https://tikz.dev/tikz-graphs
\usetikzlibrary{positioning, shapes.geometric, arrows, automata, graphs}
\tikzset{
    expr/.style={ellipse, draw=gray!60, fill=gray!5, very thick, minimum size=7mm, yshift=0.7cm},
    hexpr/.style={ellipse, draw=gray!60, fill=blue!15, very thick, minimum size=7mm, yshift=0.7cm},
    stmt/.style={rectangle, draw=gray!60, fill=gray!5, very thick, minimum size=5mm, yshift=0.7cm},
    decl/.style={rectangle, draw=blue!60, fill=gray!5, very thick, minimum size=5mm, yshift=0.7cm},
    hdecl/.style={rectangle, draw=blue!60, fill=blue!15, very thick, minimum size=5mm, yshift=0.7cm},
    subtree/.style={shape border rotate=90, isosceles triangle, draw=gray!60, fill=gray!5, very thick, minimum size=5mm, yshift=0.0cm},
}
\usepackage{blkarray}
\usepackage{graphicx}
\usepackage{forest} % https://tex.stackexchange.com/questions/198405/how-to-change-the-color-of-subtrees-in-tikz-qtree
% ------------------------------------------------

% Math
\usepackage{amsmath, amsfonts}
\usepackage{amssymb}
\usepackage{proof}
\usepackage{mathrsfs}
% Crossed-out symbols
% https://tex.stackexchange.com/questions/75525/how-to-write-crossed-out-math-in-latex
\usepackage[makeroom]{cancel}
\usepackage{mathtools}
% ------------------------------------------------

% Additional font sizes
% https://www.overleaf.com/learn/latex/Questions/How_do_I_adjust_the_font_size%3F
\usepackage{moresize}
% Additional colors
% https://www.overleaf.com/learn/latex/Using_colours_in_LaTeX
\usepackage{xcolor}
% Textual math symbols
\usepackage{textcomp}
% ------------------------------------------------

% Language
\usepackage[utf8] {inputenc}
\usepackage[T2A] {fontenc}
\usepackage[english, russian] {babel}
\usepackage{indentfirst, verbatim}
\usetikzlibrary{cd, babel}
% ------------------------------------------------

% Fonts: https://sites.math.washington.edu/~reu/docs/latex_symbols.pdf
\usepackage{stmaryrd}
\usepackage{cmbright}
\usepackage{wasysym}
\usepackage[weather]{ifsym} % https://tex.stackexchange.com/questions/100424/how-to-use-the-ifsym-package
% https://tex.stackexchange.com/questions/615300/pdflatex-builtin-glyph-names-is-empty
\pdfmapline{=dictsym DictSym <dictsym.pfb}
\pdfmapline{=pigpen <pigpen.pfa}
\usepackage{dictsym}
% ------------------------------------------------

% Code
% * Needs -shell-escape build flag
%   https://tex.stackexchange.com/questions/99475/how-to-invoke-latex-with-the-shell-escape-flag-in-texstudio-former-texmakerx
% * Set build directory
%   https://tex.stackexchange.com/questions/339931/latex-minted-package-using-custom-output-directory-build
\usepackage{minted}
\setminted{xleftmargin=\parindent, autogobble, escapeinside=\#\#}
% ------------------------------------------------

% Template
\usetheme{CambridgeUS}
\usecolortheme{dolphin}
% https://tex.stackexchange.com/questions/231439/beamer-how-to-make-font-larger-for-page-numbers
\setbeamerfont{headline}{size=\scriptsize}
\setbeamerfont{footline}{size=\scriptsize}
% Remove heddline
% https://tex.stackexchange.com/questions/33146/how-could-i-remove-a-header-in-a-beamer-presentation
%\setbeamertemplate{headline}{}
% Slide sizes
% https://tex.stackexchange.com/questions/56768/how-to-set-a-small-default-font-size-with-beamer
%\geometry{paperwidth=140mm,paperheight=105mm} % 4:3
\geometry{paperwidth=168mm,paperheight=105mm} % 16:10
% Remove navigation bar
% https://stackoverflow.com/questions/3210205/how-to-get-rid-of-navigation-bars-in-beamer
\beamertemplatenavigationsymbolsempty
% ------------------------------------------------

% Bullets
% https://9to5science.com/change-bullet-style-formatting-in-beamer
% https://tex.stackexchange.com/questions/185742/i-need-to-change-color-of-beamer-itemize-and-subitem-separately
\setbeamertemplate{itemize item}{\scriptsize\raise1.25pt\hbox{\donotcoloroutermaths$\blacktriangleright$}}
\setbeamertemplate{itemize subitem}{\scriptsize\raise1.5pt\hbox{\donotcoloroutermaths$\blacktriangleright$}}
\setbeamertemplate{itemize subsubitem}{\tiny\raise1.5pt\hbox{\donotcoloroutermaths$\blacktriangleright$}}
\setbeamertemplate{enumerate item}{\insertenumlabel.}
\setbeamertemplate{enumerate subitem}{\insertenumlabel.\insertsubenumlabel}
\setbeamertemplate{enumerate subsubitem}{\insertenumlabel.\insertsubenumlabel.\insertsubsubenumlabel}
% ------------------------------------------------

% Table of contents format
% https://tex.stackexchange.com/questions/642927/format-table-of-contents-in-beamer
\setbeamertemplate{section in toc}{%
        {\color{blue}\inserttocsectionnumber.}
    \inserttocsection\par%
}
\setbeamertemplate{subsection in toc}{%
        {\color{blue}\hspace{1em}\scriptsize\raise1.25pt\hbox{\donotcoloroutermaths$\blacktriangleright$}}
    \inserttocsubsection\par%
}
\setbeamertemplate{subsubsection in toc}{%
        {\color{blue}\hspace{2em}\tiny\raise1.25pt\hbox{\donotcoloroutermaths$\blacktriangleright$}}
    \inserttocsubsubsection\par%
}
% ------------------------------------------------

% Misc
\usepackage{multicol}
\usepackage{hyperref}
\usepackage{soul} % https://tex.stackexchange.com/questions/23711/strikethrough-text
% ------------------------------------------------

% Fix \pause for amsmath package envs (black black magic)
% https://tex.stackexchange.com/questions/16186/equation-numbering-problems-in-amsmath-environments-with-pause/75550#75550
% https://tex.stackexchange.com/questions/6348/problem-with-beamers-pause-in-alignments
%! suppress = Makeatletter
\makeatletter
\let\save@measuring@true\measuring@true
\def\measuring@true{%
    \save@measuring@true
    \def\beamer@sortzero##1{\beamer@ifnextcharospec{\beamer@sortzeroread{##1}}{}}%
    \def\beamer@sortzeroread##1<##2>{}%
    \def\beamer@finalnospec{}%
}
%! suppress = Makeatletter
\makeatother
% ------------------------------------------------

% Sections
\newcommand{\sectionplan}[1]{\section{#1}%
    \begin{frame}[noframenumbering]{Содержание}
        \tableofcontents[currentsection]
    \end{frame}
}
\newcommand{\subsectionplan}[1]{\subsection{#1}%
    \begin{frame}[noframenumbering]{Содержание}
        \tableofcontents[currentsubsection]
    \end{frame}
}
% ------------------------------------------------

% Footnotes
\renewcommand{\thefootnote}{\arabic{footnote}}
\renewcommand{\thempfootnote}{\arabic{mpfootnote}}
% https://tex.stackexchange.com/questions/28465/multiple-footnotes-at-one-point
\usepackage{fnpct}
% ------------------------------------------------

% Links
% Colors also links on slide foot.
%\hypersetup{
%    colorlinks=true,
%    citecolor=blue,
%    linkcolor=blue,
%    urlcolor=blue
%}
% ------------------------------------------------

% Appendix
% Slide numbers
% https://tex.stackexchange.com/questions/70448/dont-count-backup-slides
\usepackage{appendixnumberbeamer}
\newcommand{\backupbegin}{
    \newcounter{framenumbervorappendix}
    \setcounter{framenumbervorappendix}{\value{framenumber}}
}
\newcommand{\backupend}{
    \addtocounter{framenumbervorappendix}{-\value{framenumber}}
    \addtocounter{framenumber}{\value{framenumbervorappendix}}
}
% ------------------------------------------------

% Custom commands
% * Decor
\newcommand{\newtopic}[0]{$+$} % item: new topic on "in previous series"
\newcommand{\then}{$\Rightarrow$} % item: consequences
\newcommand{\pop}[0]{\SunCloud} %item:  general eduation
\newcommand{\popslide}[0]{(\pop)}
\newcommand{\advanced}[0]{$\varhexstar$} % item: advanced science
\newcommand{\advancedslide}[0]{(\advanced)}
\newcommand{\practical}[0]{\dstechnical} % item: practical programming notions
\newcommand{\practicalslide}[0]{(\practical)}
\newcommand{\todo}[0]{todo} % item: question
\newcommand{\answer}[0]{\Lightning} % item: answer to the previous question
\newcommand{\eg}[0]{e.g.} % item: example
\newcommand{\defi}[0]{$\Delta$} % item: definition on smth
\newcommand{\textdefi}[1]{\textbf{#1}}
\newcommand{\positive}{$+$} % item: pros
\newcommand{\negative}{{\color{red} $-$}} % item: cons
\newcommand%! suppress = EscapeHashOutsideCommand
\NB[1][0.3]{N\kern-#1em{B}} % default kern amount: -0.3em
\renewcommand{\emph}[1]{{\color{blue} \textit{#1}}}
\newcommand{\vocab}[1]{\textbf{#1}} % item: important new word
% * Lambda calculi
\newcommand{\comb}[1]{\mathbf{#1}} % defined combinator
\newcommand{\term}[1]{\mathbf{#1}} % predefined lambda-term reference
\newcommand{\termdef}{\coloneqq} % lamda term binding
\newcommand{\step}{\rightsquigarrow} % reduction step
\newcommand{\sstep}{\twoheadrightarrow} % multiple steps reduction
\newcommand{\ap}{~} % lambda-term application
\newcommand{\subst}[3]{\left[#2 \mapsto #3 \right] #1} % substitution
\newcommand{\eqbeta}{=_\beta} % beta equality
\newcommand{\eqeta}{=_\eta} % eta-equality
\newcommand{\eqt}{=} % tree-equality of terms
\newcommand{\tlist}[1]{\term{[}#1\term{]}} % list-term
% * Legacy
%\newcommand{\err}[0]{\textcolor{red}{ошибка}} % compilation error

% ------------------------------------------------

% Speaker notes
% https://tex.stackexchange.com/questions/114219/add-notes-to-latex-beamer
% https://tex.stackexchange.com/questions/35444/split-beamer-notes-across-multiple-notes-pages/35496#35496
%\setbeameroption{show notes on second screen=right} % enable speaker notes
%--------------------------------------

\author[]{Андрей Стоян, Илья Колегов, Дмитрий Халанский}
\institute[MSE ITMO]{MSE ITMO}


\title[10. Монады]{Практика 10. Монады}
\date{осень 2024}

\begin{document}

    \setcounter{framenumber}{-1}
    \maketitle

    \begin{frame}[fragile]{В предыдущих сериях}
        \begin{itemize}
            \item Классы типов \mintinline{haskell}|Functor| и \mintinline{haskell}|Applicative|
            \item[\newtopic] Класс типов \mintinline{haskell}|Monad|
            \item[\newtopic] Стандартные представители: \mintinline{haskell}|Identity|, \mintinline{haskell}|Maybe| и \mintinline{haskell}|[]|
        \end{itemize}
    \end{frame}

    \begin{frame}[noframenumbering]{Содержание}
        \tableofcontents
    \end{frame}

    \sectionplan{Работа с монадами}

    \begin{frame}[fragile]{Монады}
        \begin{block}{Moggi's principle}
            \begin{center}
                <<Computations of type $\alpha$ correspond to values of type $f\ap\alpha$>>
            \end{center}
        \end{block}
        \vspace{0.5em}
        \begin{minted}{haskell}
            class Applicative m => Monad m where
              (>>=) :: m a -> (a -> m b) -> m b
        \end{minted}
        \begin{itemize}
            \item Средство последовательной композиции вычислений
            \item[\todo] Реализуйте \mintinline{haskell}|liftA2| через \mintinline{haskell}|(>>=)|
            \item[\todo] Реализуйте \mintinline{haskell}|(>>=)| через \mintinline{haskell}|join :: m (m a) -> m a|
            \item[\answer] \pause \mintinline{haskell}|liftA2 op mx my = mx >>= \x -> my >>= \y -> pure (x + y)|
            \item[\answer] \pause Bind часто называют \texttt{flatMap}
            \begin{minted}{haskell}
                m >>= k = join $ k <$> m
            \end{minted}
        \end{itemize}
    \end{frame}

    \begin{frame}[fragile]{Монада \mintinline{haskell}|Either|}
        \begin{minted}{haskell}
            class Applicative m => Monad m where
              (>>=) :: m a -> (a -> m b) -> m b
        \end{minted}
        \begin{itemize}
            \item[\todo] Сделайте \mintinline{haskell}|Either| монадой
            \item[\todo] Реализуйте \mintinline{haskell}|join| для \mintinline{haskell}|Either|
            \item[\todo] Распечатайте имена всех котов человека, если База Котов отвечает
            \begin{minted}{haskell}
                data Person { personName :: String, personCats :: [CatId] }
                getPerson :: Db -> PersonId -> Either DbError Person
                getCat :: Db -> CatId -> Either DbError Person
            \end{minted}
            \item[\answer] \pause
            \begin{minted}{haskell}
                instance Monad (Either e) where
                  (>>=) :: Either e a -> (a -> Either e b) -> Either e b
                   m >>= k = case m of Left e -> Left e; Right x -> k x
            \end{minted}
            \item[\answer] \pause
            \begin{minted}{haskell}
                catsOf :: Db -> PersonId -> Either DbError [Cat]
                catsOf db pid = getPerson db pid >>= \person ->
                                traverse (getCat db) (personCats person)
            \end{minted}
        \end{itemize}
    \end{frame}

    \begin{frame}[fragile]{Монада списка}
        \begin{itemize}
            \item[\todo] Реализуйте монаду списка
            \item[\todo] Реализуйте \mintinline{haskell}|join| для монады списка
            \item[\todo] Реализуйте \mintinline{haskell}|filter|, используя монаду списка
            \item[\todo] Функция \mintinline{haskell}|next :: Board -> [Board]| выдаёт всевозможные состояния шахматной доски после следующего хода, может ли игра закончиться за \texttt{n} ходов (\mintinline{haskell}|finished :: Board -> Bool|)?
            \item[\answer] \pause
            \begin{minted}{haskell}
                (>>=) = flip concatMap
            \end{minted}
            \item[\answer] \pause
            \begin{minted}{haskell}
                filter p xs = xs >>= \x -> if p x then [x] else []
            \end{minted}
            \item[\answer] \pause
            \begin{minted}{haskell}
                canFinish :: Int -> Board -> Bool
                canFinish nSteps board = any finished $ allOutcomes nSteps board where
                  allOutcomes nSteps board
                    | nSteps <= 0 = board
                    | otherwise = next board >>= allOutcomes (nSteps - 1)
            \end{minted}
        \end{itemize}
    \end{frame}
    
    \begin{frame}[fragile]{``Консоль'' и \mintinline{haskell}|do|-нотация}
        \begin{itemize}
            \item[\todo] Два числа с ``консоли'', поделите одно на другое нацело и распечатайте результат, если остаток не нулевой, распечатайте его тоже
            \item[\answer] \pause
        \end{itemize}
        \vspace{-1.7em}
        \begin{columns}[onlytextwidth]
            \begin{column}[t]{0.485\textwidth}
                \pause
                \begin{block}{Через явное связывание}
                    \begin{minted}{haskell}
                        main :: MyIO ()
                        main =
                          readInt >>= \n ->
                          readInt >>= \k ->
                          let (d, m) = n `divMod` k in
                          writeInt d
                          if m == 0 then pure () else
                            writeInt m
                    \end{minted}
                \end{block}
            \end{column}\hfill%
            \begin{column}[t]{0.485\textwidth}
                \pause
                \begin{block}{Синтаксический сахар}
                    \begin{minted}{haskell}
                        main :: MyIO ()
                        main = do
                          n <- readInt
                          k <- readInt
                          let (d, m) = n `divMod` k
                          writeInt d
                          if m == 0 then pure () else
                            writeInt m
                    \end{minted}
                \end{block}
            \end{column}
        \end{columns}
    \end{frame}

\end{document}
