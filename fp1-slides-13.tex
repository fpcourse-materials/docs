%! suppress = EscapeHashOutsideCommand
%! suppress = Quote
%! suppress = MissingImport
%! suppress = MissingLabel
%! suppress = LineBreak

% CLI args https://tex.stackexchange.com/a/1501
\newif\ifhandout
\input{flags}

%! suppress = MissingLabel
%! suppress = DocumentclassNotInRoot
%! suppress = DiscouragedUseOfDef

% * Make friends tikz & colors
%   https://en.wikibooks.org/wiki/LaTeX/Colors
% * To enable vertical top alignment globally
%   https://tex.stackexchange.com/questions/9889/positioning-content-at-the-top-of-a-beamer-slide-by-default
% * Set handout from CLI
%   https://tex.stackexchange.com/a/1501
\ifhandout
\documentclass[usenames, dvipsnames, handout]{beamer} % https://tex.stackexchange.com/questions/224091/beamer-how-to-disable-pause-temporarily
\else
\documentclass[usenames, dvipsnames]{beamer}
\fi
% ------------------------------------------------

% Graphics
\usepackage{color}
\usepackage{tabularx}
\usepackage{tikz}
% https://tikz.dev/tikz-graphs
\usetikzlibrary{positioning, shapes.geometric, arrows, automata, graphs}
\tikzset{
    expr/.style={ellipse, draw=gray!60, fill=gray!5, very thick, minimum size=7mm, yshift=0.7cm},
    hexpr/.style={ellipse, draw=gray!60, fill=blue!15, very thick, minimum size=7mm, yshift=0.7cm},
    stmt/.style={rectangle, draw=gray!60, fill=gray!5, very thick, minimum size=5mm, yshift=0.7cm},
    decl/.style={rectangle, draw=blue!60, fill=gray!5, very thick, minimum size=5mm, yshift=0.7cm},
    hdecl/.style={rectangle, draw=blue!60, fill=blue!15, very thick, minimum size=5mm, yshift=0.7cm},
    subtree/.style={shape border rotate=90, isosceles triangle, draw=gray!60, fill=gray!5, very thick, minimum size=5mm, yshift=0.0cm},
}
\usepackage{blkarray}
\usepackage{graphicx}
\usepackage{forest} % https://tex.stackexchange.com/questions/198405/how-to-change-the-color-of-subtrees-in-tikz-qtree
% ------------------------------------------------

% Math
\usepackage{amsmath, amsfonts}
\usepackage{amssymb}
\usepackage{proof}
\usepackage{mathrsfs}
% Crossed-out symbols
% https://tex.stackexchange.com/questions/75525/how-to-write-crossed-out-math-in-latex
\usepackage[makeroom]{cancel}
\usepackage{mathtools}
% ------------------------------------------------

% Additional font sizes
% https://www.overleaf.com/learn/latex/Questions/How_do_I_adjust_the_font_size%3F
\usepackage{moresize}
% Additional colors
% https://www.overleaf.com/learn/latex/Using_colours_in_LaTeX
\usepackage{xcolor}
% Textual math symbols
\usepackage{textcomp}
% ------------------------------------------------

% Language
\usepackage[utf8] {inputenc}
\usepackage[T2A] {fontenc}
\usepackage[english, russian] {babel}
\usepackage{indentfirst, verbatim}
\usetikzlibrary{cd, babel}
% ------------------------------------------------

% Fonts: https://sites.math.washington.edu/~reu/docs/latex_symbols.pdf
\usepackage{stmaryrd}
\usepackage{cmbright}
\usepackage{wasysym}
\usepackage[weather]{ifsym} % https://tex.stackexchange.com/questions/100424/how-to-use-the-ifsym-package
% https://tex.stackexchange.com/questions/615300/pdflatex-builtin-glyph-names-is-empty
\pdfmapline{=dictsym DictSym <dictsym.pfb}
\pdfmapline{=pigpen <pigpen.pfa}
\usepackage{dictsym}
% ------------------------------------------------

% Code
% * Needs -shell-escape build flag
%   https://tex.stackexchange.com/questions/99475/how-to-invoke-latex-with-the-shell-escape-flag-in-texstudio-former-texmakerx
% * Set build directory
%   https://tex.stackexchange.com/questions/339931/latex-minted-package-using-custom-output-directory-build
\usepackage{minted}
\setminted{xleftmargin=\parindent, autogobble, escapeinside=\#\#}
% ------------------------------------------------

% Template
\usetheme{CambridgeUS}
\usecolortheme{dolphin}
% https://tex.stackexchange.com/questions/231439/beamer-how-to-make-font-larger-for-page-numbers
\setbeamerfont{headline}{size=\scriptsize}
\setbeamerfont{footline}{size=\scriptsize}
% Remove heddline
% https://tex.stackexchange.com/questions/33146/how-could-i-remove-a-header-in-a-beamer-presentation
%\setbeamertemplate{headline}{}
% Slide sizes
% https://tex.stackexchange.com/questions/56768/how-to-set-a-small-default-font-size-with-beamer
%\geometry{paperwidth=140mm,paperheight=105mm} % 4:3
\geometry{paperwidth=168mm,paperheight=105mm} % 16:10
% Remove navigation bar
% https://stackoverflow.com/questions/3210205/how-to-get-rid-of-navigation-bars-in-beamer
\beamertemplatenavigationsymbolsempty
% ------------------------------------------------

% Bullets
% https://9to5science.com/change-bullet-style-formatting-in-beamer
% https://tex.stackexchange.com/questions/185742/i-need-to-change-color-of-beamer-itemize-and-subitem-separately
\setbeamertemplate{itemize item}{\scriptsize\raise1.25pt\hbox{\donotcoloroutermaths$\blacktriangleright$}}
\setbeamertemplate{itemize subitem}{\scriptsize\raise1.5pt\hbox{\donotcoloroutermaths$\blacktriangleright$}}
\setbeamertemplate{itemize subsubitem}{\tiny\raise1.5pt\hbox{\donotcoloroutermaths$\blacktriangleright$}}
\setbeamertemplate{enumerate item}{\insertenumlabel.}
\setbeamertemplate{enumerate subitem}{\insertenumlabel.\insertsubenumlabel}
\setbeamertemplate{enumerate subsubitem}{\insertenumlabel.\insertsubenumlabel.\insertsubsubenumlabel}
% ------------------------------------------------

% Table of contents format
% https://tex.stackexchange.com/questions/642927/format-table-of-contents-in-beamer
\setbeamertemplate{section in toc}{%
        {\color{blue}\inserttocsectionnumber.}
    \inserttocsection\par%
}
\setbeamertemplate{subsection in toc}{%
        {\color{blue}\hspace{1em}\scriptsize\raise1.25pt\hbox{\donotcoloroutermaths$\blacktriangleright$}}
    \inserttocsubsection\par%
}
\setbeamertemplate{subsubsection in toc}{%
        {\color{blue}\hspace{2em}\tiny\raise1.25pt\hbox{\donotcoloroutermaths$\blacktriangleright$}}
    \inserttocsubsubsection\par%
}
% ------------------------------------------------

% Misc
\usepackage{multicol}
\usepackage{hyperref}
\usepackage{soul} % https://tex.stackexchange.com/questions/23711/strikethrough-text
% ------------------------------------------------

% Fix \pause for amsmath package envs (black black magic)
% https://tex.stackexchange.com/questions/16186/equation-numbering-problems-in-amsmath-environments-with-pause/75550#75550
% https://tex.stackexchange.com/questions/6348/problem-with-beamers-pause-in-alignments
%! suppress = Makeatletter
\makeatletter
\let\save@measuring@true\measuring@true
\def\measuring@true{%
    \save@measuring@true
    \def\beamer@sortzero##1{\beamer@ifnextcharospec{\beamer@sortzeroread{##1}}{}}%
    \def\beamer@sortzeroread##1<##2>{}%
    \def\beamer@finalnospec{}%
}
%! suppress = Makeatletter
\makeatother
% ------------------------------------------------

% Sections
\newcommand{\sectionplan}[1]{\section{#1}%
    \begin{frame}[noframenumbering]{Содержание}
        \tableofcontents[currentsection]
    \end{frame}
}
\newcommand{\subsectionplan}[1]{\subsection{#1}%
    \begin{frame}[noframenumbering]{Содержание}
        \tableofcontents[currentsubsection]
    \end{frame}
}
% ------------------------------------------------

% Footnotes
\renewcommand{\thefootnote}{\arabic{footnote}}
\renewcommand{\thempfootnote}{\arabic{mpfootnote}}
% https://tex.stackexchange.com/questions/28465/multiple-footnotes-at-one-point
\usepackage{fnpct}
% ------------------------------------------------

% Links
% Colors also links on slide foot.
%\hypersetup{
%    colorlinks=true,
%    citecolor=blue,
%    linkcolor=blue,
%    urlcolor=blue
%}
% ------------------------------------------------

% Appendix
% Slide numbers
% https://tex.stackexchange.com/questions/70448/dont-count-backup-slides
\usepackage{appendixnumberbeamer}
\newcommand{\backupbegin}{
    \newcounter{framenumbervorappendix}
    \setcounter{framenumbervorappendix}{\value{framenumber}}
}
\newcommand{\backupend}{
    \addtocounter{framenumbervorappendix}{-\value{framenumber}}
    \addtocounter{framenumber}{\value{framenumbervorappendix}}
}
% ------------------------------------------------

% Custom commands
% * Decor
\newcommand{\newtopic}[0]{$+$} % item: new topic on "in previous series"
\newcommand{\then}{$\Rightarrow$} % item: consequences
\newcommand{\pop}[0]{\SunCloud} %item:  general eduation
\newcommand{\popslide}[0]{(\pop)}
\newcommand{\advanced}[0]{$\varhexstar$} % item: advanced science
\newcommand{\advancedslide}[0]{(\advanced)}
\newcommand{\practical}[0]{\dstechnical} % item: practical programming notions
\newcommand{\practicalslide}[0]{(\practical)}
\newcommand{\todo}[0]{todo} % item: question
\newcommand{\answer}[0]{\Lightning} % item: answer to the previous question
\newcommand{\eg}[0]{e.g.} % item: example
\newcommand{\defi}[0]{$\Delta$} % item: definition on smth
\newcommand{\textdefi}[1]{\textbf{#1}}
\newcommand{\positive}{$+$} % item: pros
\newcommand{\negative}{{\color{red} $-$}} % item: cons
\newcommand%! suppress = EscapeHashOutsideCommand
\NB[1][0.3]{N\kern-#1em{B}} % default kern amount: -0.3em
\renewcommand{\emph}[1]{{\color{blue} \textit{#1}}}
\newcommand{\vocab}[1]{\textbf{#1}} % item: important new word
% * Lambda calculi
\newcommand{\comb}[1]{\mathbf{#1}} % defined combinator
\newcommand{\term}[1]{\mathbf{#1}} % predefined lambda-term reference
\newcommand{\termdef}{\coloneqq} % lamda term binding
\newcommand{\step}{\rightsquigarrow} % reduction step
\newcommand{\sstep}{\twoheadrightarrow} % multiple steps reduction
\newcommand{\ap}{~} % lambda-term application
\newcommand{\subst}[3]{\left[#2 \mapsto #3 \right] #1} % substitution
\newcommand{\eqbeta}{=_\beta} % beta equality
\newcommand{\eqeta}{=_\eta} % eta-equality
\newcommand{\eqt}{=} % tree-equality of terms
\newcommand{\tlist}[1]{\term{[}#1\term{]}} % list-term
% * Legacy
%\newcommand{\err}[0]{\textcolor{red}{ошибка}} % compilation error

% ------------------------------------------------

% Speaker notes
% https://tex.stackexchange.com/questions/114219/add-notes-to-latex-beamer
% https://tex.stackexchange.com/questions/35444/split-beamer-notes-across-multiple-notes-pages/35496#35496
%\setbeameroption{show notes on second screen=right} % enable speaker notes
%--------------------------------------

\author[]{Андрей Стоян, Илья Колегов, Дмитрий Халанский}
\institute[MSE ITMO]{MSE ITMO}


\title[13. Вывод типов]{Практика 13. Вывод типов}
\date{осень 2024}

\begin{document}

    \setcounter{framenumber}{-1}
    \maketitle

    \begin{frame}[noframenumbering]{Содержание}
        \tableofcontents
    \end{frame}

    \sectionplan{Вспоминаем типы}

    \begin{frame}{Вспомнить всё}
%        \vspace{-0.5em}
        \pause
        \begin{block}{Синтаксис типов в $\lambda_{\rightarrow}$}
            \pause
            \begin{description}
                \item[Типовые переменные] \pause \vspace{-1em} $\alpha, \beta, \ldots \in \mathbb{T}$
                \pause
                \hspace{2em}
                \begin{tikzpicture}
                    \node [expr] (a) {$\alpha$};
                \end{tikzpicture}
                \item[Стрелочные типы] \pause \vspace{-2em} $\sigma, \tau \in \mathbb{T} \Rightarrow (\sigma\rightarrow\tau)\in\mathbb{T}$\footnote{Стрелочка --- правоассоциативный оператор: $\tau \to (\sigma \to \zeta) \eqt \tau \to \sigma \to \zeta$.}
                \hspace{2em}
                \pause
                \begin{tikzpicture}
                    \node [expr] (arr) {$\to$};
                    \node [subtree] (t) [below left= of arr] {$\sigma$};
                    \node [subtree] (s) [below right= of arr] {$\tau$};
                    \draw[->] (t.north) -- (arr);
                    \draw[->] (s.north) -- (arr);
                \end{tikzpicture}
            \end{description}
        \end{block}
        \pause
        \begin{block}{Утверждение типизации}
            \pause
            \begin{itemize}
                \item \pause Синтаксис: $\Gamma$ --- контекст, $M \in \Lambda, \sigma \in \mathbb{T}$, то \pause $\Gamma \vdash M : \sigma$ --- утверждение типизации
                \item \pause Синтаксическая категория, увязывающая вместе термы и типы с контекстом
                \item \pause Утверждает, что в контексте $\Gamma$ терм $M$ имеет тип $\sigma$
            \end{itemize}
        \end{block}
        \pause
        \begin{block}{Правила вывода утверждений типизации в $\lambda_{\rightarrow}$}
            \pause
            \vspace{-1em}
            \begin{center}
                \[
                    \begin{array}{l c r}
                        \pause \infer[ctx]{\Gamma \vdash x: \sigma}{(x: \sigma) \in \Gamma}
                        &
                        \pause \infer[elim\to]{\Gamma \vdash M\;N : \tau}{\Gamma \vdash M : \sigma \to \tau & \Gamma \vdash N : \sigma}
                        &
                        \pause \infer[intro\to]{\Gamma \vdash \lambda x^{\color{red} \sigma}\ldotp M : \sigma \to \tau}{\{x : \sigma\} \cup \Gamma \vdash M : \tau}
                    \end{array}
                \]
            \end{center}
        \end{block}
    \end{frame}

    \sectionplan{Пример вывода типа комбинатора $\comb{B}$}

    %! suppress = TrimWhitespace
    \newcommand{\btypename}{Вывод типа для комбинатора $\term{B}$ }

    \begin{frame}[t]{\btypename (0)}
        \begin{block}{Правила вывода утверждений типизации в $\lambda_{\rightarrow}$}
            \vspace{-1em}
            \begin{center}
                \[
                    \begin{array}{l c r}
                        \infer[ctx]{\Gamma \vdash x: \sigma}{(x: \sigma) \in \Gamma}
                        &
                        \infer[elim\to]{\Gamma \vdash M\;N : \tau}{\Gamma \vdash M : \sigma \to \tau & \Gamma \vdash N : \sigma}
                        &
                        \infer[intro\to]{\Gamma \vdash \lambda x^{\color{red} \sigma}\ldotp M : \sigma \to \tau}{\{x : \sigma\} \cup \Gamma \vdash M : \tau}
                    \end{array}
                \]
            \end{center}
        \end{block}
        \[{\color{red}\emptyset \vdash \lambda f~g~x.~f~(g~x) :~ ?_0}\]
        \vspace{-1em}
        \begin{itemize}
            \item Требуется вывести тип терма
            \item Внешней информации нет, типизируем в пустом контексте
            \item Пока для неизвестного типа используем мета-переменную $?_0$
        \end{itemize}
    \end{frame}

    \begin{frame}[t, noframenumbering]{\btypename (1)}
        \begin{block}{Правила вывода утверждений типизации в $\lambda_{\rightarrow}$}
            \vspace{-1em}
            \begin{center}
                \[
                    \begin{array}{l c r}
                        \infer[ctx]{\Gamma \vdash x: \sigma}{(x: \sigma) \in \Gamma}
                        &
                        \infer[elim\to]{\Gamma \vdash M\;N : \tau}{\Gamma \vdash M : \sigma \to \tau & \Gamma \vdash N : \sigma}
                        &
                        \infer[intro\to]{\Gamma \vdash \lambda x^{\color{red} \sigma}\ldotp M : \sigma \to \tau}{\{x : \sigma\} \cup \Gamma \vdash M : \tau}
                    \end{array}
                \]
            \end{center}
        \end{block}
        \[
            \infer[intro\to]{\emptyset \vdash \lambda f~g~x.~f~(g~x) :~ ?_1 \to ?_2}
            {\color{red}\{f :~ ?_1\} \vdash \lambda g~x.~f~(g~x) :~ ?_2}
        \]
        \vspace{-1em}
        \begin{itemize}
            \item Видим абстракцию на верхнем уровне: правило $intro\to$
            \item Правило требует стрелочной структуры типа, запоминаем $?_0 =~ ?_1\to ?_2$
            \item Обогащаем контекст переменной типа аргумента функции
        \end{itemize}
    \end{frame}

    \begin{frame}[t, noframenumbering]{\btypename (2)}
        \begin{block}{Правила вывода утверждений типизации в $\lambda_{\rightarrow}$}
            \vspace{-1em}
            \begin{center}
                \[
                    \begin{array}{l c r}
                        \infer[ctx]{\Gamma \vdash x: \sigma}{(x: \sigma) \in \Gamma}
                        &
                        \infer[elim\to]{\Gamma \vdash M\;N : \tau}{\Gamma \vdash M : \sigma \to \tau & \Gamma \vdash N : \sigma}
                        &
                        \infer[intro\to]{\Gamma \vdash \lambda x^{\color{red} \sigma}\ldotp M : \sigma \to \tau}{\{x : \sigma\} \cup \Gamma \vdash M : \tau}
                    \end{array}
                \]
            \end{center}
        \end{block}
        \[
            \infer[intro\to]{\emptyset \vdash \lambda f~g~x.~f~(g~x) : ?_1 \to \color{red} ?_3 \to ?_4}{
                \color{red}\infer[intro\to]{\color{black}\{f :~ ?_1\} \vdash \lambda g~x.~f~(g~x) :~ \color{red}?_3 \to ?_4}{
                        {\{f :~ ?_1, g :~ ?_3\} \vdash \lambda x.~f~(g~x) :~ ?_4 }
                }}
        \]
        \vspace{-1em}
        \begin{itemize}
            \item Снова абстракция на верхнем уровне: $intro\to$
            \item Тип $?_2$ должен иметь стрелочную структуру, запоминаем $?_2 =~ ?_3 \to ?_4$
            \item Обогащаем контекст
        \end{itemize}
    \end{frame}

    \begin{frame}[t, noframenumbering]{\btypename (3)}
        \begin{block}{Правила вывода утверждений типизации в $\lambda_{\rightarrow}$}
            \vspace{-1em}
            \begin{center}
                \[
                    \begin{array}{l c r}
                        \infer[ctx]{\Gamma \vdash x: \sigma}{(x: \sigma) \in \Gamma}
                        &
                        \infer[elim\to]{\Gamma \vdash M\;N : \tau}{\Gamma \vdash M : \sigma \to \tau & \Gamma \vdash N : \sigma}
                        &
                        \infer[intro\to]{\Gamma \vdash \lambda x^{\color{red} \sigma}\ldotp M : \sigma \to \tau}{\{x : \sigma\} \cup \Gamma \vdash M : \tau}
                    \end{array}
                \]
            \end{center}
        \end{block}
        \[
            \infer[intro\to]{\emptyset \vdash \lambda f~g~x.~f~(g~x) :~ ?_1 \to ?_3 \to \color{red}?_5 \to ?_6}{
                \infer[intro\to]{\{f:~ ?_1\} \vdash \lambda g~x.~f~(g~x) :~ ?_3 \to \color{red}?_5 \to ?_6}{
                    \color{red}\infer[intro\to]{\color{black}\{f:~ ?_1, g:~ ?_3\} \vdash \lambda x.~f~(g~x) :~ \color{red}?_5 \to ?_6 }{
                            {\{f:~ ?_1, g:~ ?_3, x:~ ?_5\} \vdash f~(g~x) :~ ?_6 }
                    }}}
        \]
        \vspace{-1em}
        \begin{itemize}
            \item Снова абстракция на верхнем уровне: $intro\to$
            \item Запоминаем: $?_4 =~ ?_5 \to ?_6$
            \item Обогащаем контекст
            \item Обозначим для краткости $\Gamma = \{f:~ ?_1, g:~ ?_3, x:~ ?_5\}$
        \end{itemize}
    \end{frame}

    \begin{frame}[t, noframenumbering]{\btypename (4)}
        \begin{block}{Правила вывода утверждений типизации в $\lambda_{\rightarrow}$}
            \vspace{-1em}
            \begin{center}
                \[
                    \begin{array}{l c r}
                        \infer[ctx]{\Gamma \vdash x: \sigma}{(x: \sigma) \in \Gamma}
                        &
                        \infer[elim\to]{\Gamma \vdash M\;N : \tau}{\Gamma \vdash M : \sigma \to \tau & \Gamma \vdash N : \sigma}
                        &
                        \infer[intro\to]{\Gamma \vdash \lambda x^{\color{red} \sigma}\ldotp M : \sigma \to \tau}{\{x : \sigma\} \cup \Gamma \vdash M : \tau}
                    \end{array}
                \]
            \end{center}
        \end{block}
        %! suppress = EscapeAmpersand
        \[
            \infer[intro\to]{\emptyset \vdash \lambda f~g~x.~f~(g~x) : ?_1 \to ?_3 \to ?_5 \to ?_6}{
                \infer[intro\to]{\{f: ?_1\} \vdash \lambda g~x.~f~(g~x) : ?_3 \to ?_5 \to ?_6}{
                    \infer[intro\to]{\{f: ?_1, g: ?_3\} \vdash \lambda x.~f~(g~x) : ?_5 \to ?_6 }{
                        \color{red}\infer[elim\to]{\color{black}\{f: ?_1, g: ?_3, x: ?_5\} \vdash f~(g~x) :~ ?_6}{
                            \Gamma \vdash f : ?_7 \to ?_6 &
                            \Gamma \vdash g~x : ?_7
                        }}}}
        \]
        \vspace{-1em}
        \begin{itemize}
            \item Аппликация на верхнем уровне: правило $elim\to$
            \item Тип аргумента неизвестен, заводим мета-переменную $?_7$
            \item Типы аргумента и параметра должны совпадать
            \item Тип результата функции равен типу аппликации
        \end{itemize}
    \end{frame}

    \begin{frame}[t, noframenumbering]{\btypename (5)}
        \begin{block}{Правила вывода утверждений типизации в $\lambda_{\rightarrow}$}
            \vspace{-1em}
            \begin{center}
                \[
                    \begin{array}{l c r}
                        \infer[ctx]{\Gamma \vdash x: \sigma}{(x: \sigma) \in \Gamma}
                        &
                        \infer[elim\to]{\Gamma \vdash M\;N : \tau}{\Gamma \vdash M : \sigma \to \tau & \Gamma \vdash N : \sigma}
                        &
                        \infer[intro\to]{\Gamma \vdash \lambda x^{\color{red} \sigma}\ldotp M : \sigma \to \tau}{\{x : \sigma\} \cup \Gamma \vdash M : \tau}
                    \end{array}
                \]
            \end{center}
        \end{block}
        %! suppress = EscapeAmpersand
        \[
            \infer[intro\to]{\emptyset \vdash \lambda f~g~x.~f~(g~x) : {\color{red}(?_7 \to ?_6)} \to ?_3 \to ?_5 \to ?_6}{
                \infer[intro\to]{\{f: {\color{red}?_7 \to ?_6}\} \vdash \lambda g~x.~f~(g~x) : ?_3 \to ?_5 \to ?_6}{
                    \infer[intro\to]{\{f: {\color{red}?_7 \to ?_6}, g: ?_3\} \vdash \lambda x.~f~(g~x) : ?_5 \to ?_6 }{
                        \infer[elim\to]{\{f: {\color{red}?_7 \to ?_6}, g: ?_3, x: ?_5\} \vdash f~(g~x) }{
                            \color{red}
                            \infer[1]{\color{black}\Gamma \vdash f : {\color{red}?_7 \to ?_6}}{} &
                            \Gamma \vdash g~x : ?_7
                        }}}}
        \]
        \vspace{-1em}
        \begin{itemize}
            \item В левой ветке правило типизации $ctx$
            \item Получаем, что $?_1 =~ ?_7 \to ?_6$
        \end{itemize}
    \end{frame}

    \begin{frame}[t, noframenumbering]{\btypename (6)}
        \begin{block}{Правила вывода утверждений типизации в $\lambda_{\rightarrow}$}
            \vspace{-1em}
            \begin{center}
                \[
                    \begin{array}{l c r}
                        \infer[ctx]{\Gamma \vdash x: \sigma}{(x: \sigma) \in \Gamma}
                        &
                        \infer[elim\to]{\Gamma \vdash M\;N : \tau}{\Gamma \vdash M : \sigma \to \tau & \Gamma \vdash N : \sigma}
                        &
                        \infer[intro\to]{\Gamma \vdash \lambda x^{\color{red} \sigma}\ldotp M : \sigma \to \tau}{\{x : \sigma\} \cup \Gamma \vdash M : \tau}
                    \end{array}
                \]
            \end{center}
        \end{block}
        %! suppress = EscapeAmpersand
        \[
            \infer[intro\to]{\emptyset \vdash \lambda f~g~x.~f~(g~x) : (?_7 \to ?_6) \to ?_3 \to ?_5 \to ?_6}{
                \infer[intro\to]{\{f: ?_7 \to ?_6\} \vdash \lambda g~x.~f~(g~x) : ?_3 \to ?_5 \to ?_6}{
                    \infer[intro\to]{\{f: ?_7 \to ?_6, g: ?_3\} \vdash \lambda x.~f~(g~x) : ?_5 \to ?_6 }{
                        \infer[elim\to]{\{f: ?_7 \to ?_6, g: ?_3, x: ?_5\} \vdash f~(g~x) }{
                            \infer[ctx]{\Gamma \vdash f : ?_7 \to ?_6}{} &
                            \color{red}\infer[elim\to]{\color{black}\Gamma \vdash g~x : ?_7}{
                                \Gamma \vdash g : ?_8 \to ?_7 &
                                \Gamma \vdash x : ?_8
                            }}}}}
        \]
        \vspace{-1em}
        \begin{itemize}
            \item В правой ветке снова видим аппликацию: $elim\to$
            \item Заводим мета-переменную $?_8$
        \end{itemize}
    \end{frame}

    \begin{frame}[t, noframenumbering]{\btypename (7)}
        \begin{block}{Правила вывода утверждений типизации в $\lambda_{\rightarrow}$}
            \vspace{-1em}
            \begin{center}
                \[
                    \begin{array}{l c r}
                        \infer[ctx]{\Gamma \vdash x: \sigma}{(x: \sigma) \in \Gamma}
                        &
                        \infer[elim\to]{\Gamma \vdash M\;N : \tau}{\Gamma \vdash M : \sigma \to \tau & \Gamma \vdash N : \sigma}
                        &
                        \infer[intro\to]{\Gamma \vdash \lambda x^{\color{red} \sigma}\ldotp M : \sigma \to \tau}{\{x : \sigma\} \cup \Gamma \vdash M : \tau}
                    \end{array}
                \]
            \end{center}
        \end{block}
        %! suppress = EscapeAmpersand
        \[
            \infer[intro\to]{\emptyset \vdash \lambda f~g~x.~f~(g~x) : (?_7 \to ?_6) \to ({\color{red} ?_8 \to ?_7}) \to ?_5 \to ?_6}{
                \infer[intro\to]{\{f: ?_7 \to ?_6\} \vdash \lambda g~x.~f~(g~x) : ({\color{red} ?_8 \to ?_7}) \to ?_5 \to ?_6}{
                    \infer[intro\to]{\{f: ?_7 \to ?_6, g: {\color{red} ?_8 \to ?_7}\} \vdash \lambda x.~f~(g~x) : ?_5 \to ?_6 }{
                        \infer[elim\to]{\Gamma \vdash f~(g~x) }{
                            \infer[ctx]{\Gamma \vdash f : ?_7 \to ?_6}{} &
                            \infer[elim\to]{\Gamma \vdash g \ap x : ?_7}{
                                \color{red}\infer[ctx]{\color{black}\Gamma \vdash g : {\color{red} ?_8 \to ?_7}}{} &
                                \Gamma \vdash x : ?_8
                            }}}}}
        \]
        \vspace{-1em}
        \begin{itemize}
            \item В левой ветке первое правило: $ctx$
            \item Получаем $?_3 = ?_8 \to ?_7$
        \end{itemize}
    \end{frame}

    \begin{frame}[t, noframenumbering]{\btypename (8)}
        \begin{block}{Правила вывода утверждений типизации в $\lambda_{\rightarrow}$}
            \vspace{-1em}
            \begin{center}
                \[
                    \begin{array}{l c r}
                        \infer[ctx]{\Gamma \vdash x: \sigma}{(x: \sigma) \in \Gamma}
                        &
                        \infer[elim\to]{\Gamma \vdash M\;N : \tau}{\Gamma \vdash M : \sigma \to \tau & \Gamma \vdash N : \sigma}
                        &
                        \infer[intro\to]{\Gamma \vdash \lambda x^{\color{red} \sigma}\ldotp M : \sigma \to \tau}{\{x : \sigma\} \cup \Gamma \vdash M : \tau}
                    \end{array}
                \]
            \end{center}
        \end{block}
        %! suppress = EscapeAmpersand
        \[
            \infer[intro\to]{\emptyset \vdash \lambda f~g~x.~f~(g~x) : (?_7 \to ?_6) \to ({\color{red} ?_5} \to ?_7) \to ?_5 \to ?_6}{
                \infer[intro\to]{\{f: ?_7 \to ?_6\} \vdash \lambda g~x.~f~(g~x) : ({\color{red} ?_5} \to ?_7) \to ?_5 \to ?_6}{
                    \infer[intro\to]{\{f: ?_7 \to ?_6, g: {\color{red}?_5} \to ?_7\} \vdash \lambda x.~f~(g~x) : ?_5 \to ?_6 }{
                        \infer[elim\to]{\{f: ?_7 \to ?_6, g: {\color{red}?_5} \to ?_7, x: ?_5\} \vdash f~(g~x) }{
                            \infer[ctx]{\Gamma \vdash f : ?_7 \to ?_6}{} &
                            \infer[elim\to]{\Gamma \vdash g \ap x : ?_7}{
                                \infer[ctx]{\Gamma \vdash g : {\color{red}?_5} \to ?_7}{} &
                                \color{red} \infer[ctx]{\color{black} \Gamma \vdash x : {\color{red} ?_5}}{}
                            }}}}}
        \]
        \vspace{-1em}
        \begin{itemize}
            \item В правой ветке снова срабатывает первое правило: $ctx$
            \item Получаем $?_8 = ?_5$
            \item[\answer] Заменяем мета-переменные на типовые $(\beta \to \gamma) \to (\alpha \to \beta) \to \alpha \to \gamma$
        \end{itemize}
    \end{frame}

    \sectionplan{Материалы}

    \begin{frame}[fragile]{Материалы}
        См. 3ю пару.
    \end{frame}

\end{document}
