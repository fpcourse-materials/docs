%! suppress = MissingLabel

Ранее мы признали работу со сложностью главной задачей программиста, а построение встроенных языков~--- основным инструментом её решения (\ref{subsec:interpreters-rules}).
В данной главе мы рассмотрим понятие эффекта.
Оно тесно связано со встроенными языками и даст нам лучшее понимание, когда их конструировать, что это даёт, и с чем нужно быть осторожным.

Для реализации встроенных языков мы предпочли shallow embedding в форме tagless final (\ref{subsec:tagless-final}), который максимально переиспользует возможности мета-языка и позволяет давать различные интерпретации одной программе.
Далее мы исследовали процесс вычисления и извлекли понятие продолжения (\ref{sec:continuations}).
Оказалось, что tagless final языки, которые мы строили вокруг монад, можно выразить через продолжения удобнее и проще (\ref{subsubsec:god-cont},\ \ref{subsubsec:monadic-reflection}).
В этой главе мы поймём, как это поможет решить expression problem (\ref{subsec:expression-problem}) до конца.

\subsection{Понятие эффекта} \label{subsec:about-effects}

Начнём разговор от обратного, со свойства чистоты.
\vocab{Чистая функция} обладает следующими свойствами:
\begin{itemize}
    \item Её результат всегда одинаков при одинаковом наборе аргументов (никак более нетривиально не зависит ни от чего более);
    \item Её единственный наблюдаемый результат --- её возвращаемое значение.
\end{itemize}

% todo obsidian What is a purely functional language?

В целом стиль программирования с использованием чистых функций приветствуется, так как он обладает множеством хороших свойств.
Так, про них можно удобно рассуждать с помощью equational reasoning; всё, что нужно для понимания кода, явно написано в этом коде; классические системы типов хорошо работают, предоставляя полноту абстракции, качественную документацию и частичную спецификацию\ldots
Также известно, что всё можно записать с помощью чистых вычислений, даже работу с IO~\cite{jones2001tackling}.

Однако, используя только чистые функции, всё приходится делать вручную.
В случае с IO (с состоянием аналогично) --- передавать результирующий мир в аргументы раз за разом:
\begin{minted}{haskell}
    getList :: Int -> World -> (World, [Int])
    getList n w | n == 0 = (w, [])
                | otherwise =
      let (w', x) = getInt w in
      let (w'', xs) = getList (n - 1) w' in
      (w'', x : xs)
\end{minted}

Таким образом, код из чистых функций заполнен несущественными церемониями, за которыми не видно бизнес-логики и сути.
Чтобы сосредоточиться на важных деталях, нужно делегировать весь этот bookkeeping стороннему коду, замести неинтересные детали под ковёр.
Тогда код выше можно будет переписать, например, следующим образом:
\begin{minted}{haskell}
    getList :: Int -> IO [Int]
    getList n | n == 0 = pure []
              | otherwise = do
      x <- getInt
      xs <- getList (n - 1)
      return (x : xs)
\end{minted}

Абстрактную сущность, которой мы будем делегировать несущественные для данного фрагмента бизнес-логики детали, мы будем называть \vocab{контекстом исполнения (execution context)}\footnote{\url{https://okmij.org/ftp/Computation/having-effect.html} \label{lab:having-effect}}.
А \vocab{эффектом}~--- взаимодействие функции с контекстом исполнения, которое происходит с помощью вызова \vocab{effect-операций} (например, \texttt{getInt} из примера выше).
Так, на контекст исполнения можно смотреть как на сервер, которому функция-клиент шлёт запросы и получает ответы (см.\ рис.\ \ref{fig:effect-server}).
В этой модели такая функция может нарушать оба свойства чистых функций.

\begin{figure}
    \centering
    \includegraphics[width=0.5\textwidth]{figs/effects}
    \caption{Эффекты как клиент-серверное взаимодействие.}
    \label{fig:effect-server}
\end{figure}

На практике этим контекстом является интерпретатор (встроенного) языка, а effect-операциями --- его конструкции.
Если язык является встроенным, то говорят о \vocab{пользовательских (user-defined) эффектах}.
Так мы снова возвращаемся к задаче построения модульных интерпретаторов (\ref{subsec:expression-problem}).
К тому же, реализация вычислительного контекста также может делегировать реализацию некоторой функциональности другому контексту исполнения, и так далее.
Получаем уже знакомую нам башню интерпретаторов (\ref{subsubsec:interpreters-tower}).

Рассмотрим некоторые примеры вычислительных контекстов и операций:
\begin{itemize}
    \item Контекст --- подсистема управления памятью, \texttt{modify} --- effect-операция: контекст для нас поддерживает состояние ячеек памяти;
    \item Контекст~--- хендлер исключения, \texttt{throw MyException}~--- effect-операция: контекст за нас определяет, как ошибка будет обрабатываться (обратите внимание, что тут управление не возвращается терму);
    \item Контекст~--- настройки инъекции зависимостей, запрос функциональности~--- effect-операция: контекст за нас определяет реализацию функциональности, которой нам пользоваться\footnote{Существует термин \vocab{contextual polymorphism}~--- код в разных контекстах может иметь различное поведение.}.
    Как мы увидим далее, этот и подобные простые эффекты можно реализовать просто и эффективно (см.\ \ref{subsubsec:evidence-passing}).
\end{itemize}

\begin{task}
    Приведите ещё примеры вычислительных контекстов и операций.
\end{task}

Когда мы говорим про standalone язык, на котором мы программируем (например, Haskell), любое наше действие в программе исполняется им.
То есть, например, сложение --- effect-операция?
В таком случае разумно выделить подмножество конструкций языка, достаточно ``интересных'', чтобы считать, что они порождают эффект.

Какие конструкции языка считать ``интересными''?
Заметим, что с одной стороны это хорошо, что эффекты скрывают от нас некоторую сложность, позволяя сосредоточиться на других вещах.
С другой стороны, это же и плохо, ведь мы эту сложность перестаём наблюдать, а она пронизывает наш код, поддерживает неявные зависимости между его частями.
Таким образом, эффекты требуют дополнительной аккуратности со стороны программиста\footnote{\href{https://youtu.be/_nG09Z_tdUU?si=lo9It6299rsB1vAr}{(youtube) Kris Jenkins --- Side-Effects Are The Complexity Iceberg.}}.
Соответственно, именно такие непростые конструкции и стоит считать ``интересными''.
Как минимум точно стоит считать ``интересными'' конструкции, использование которых выводит функцию из категории чистых.
Также, это могут быть операции, делающие сложные нелокальные модификации потока управления (поддерживаемого интерпретатором в виде продолжения).

В конечном итоге выбор ``интересных'' конструкций зависит от задачи и перспективы разработчика\footref{lab:having-effect}.
Так, конструкции, влияющие на произвольные наблюдаемые свойства кода, как, например, терминируемость или вычислительная сложность, могут мотивировать считать рекурсивные вызовы или долгие операции эффектами.

Далее мы научимся отслеживать и контролировать использование эффектов на уровне типов с помощью систем эффектов (см.\ далее\ \ref{sec:effect-systems}).

\subsection{Хендлеры эффектов} \label{subsec:effect-handlers}

Хендлеры эффектов --- это универсальный метод построения модульных интерпретаторов встроенных языков, напрямую реализующий клиент-серверную метафору.
Как обычно бывает, хендлеры были изобретены множество раз.
В этой главе мы посмотрим на основные реализации, которые лучше всего помогут нам понять концепцию.

Основная идея хендлеров эффектов довольно проста.
Вводится языковая конструкция \texttt{handle}, позволяющая задать вычислительный контекст для определённого скоупа, предоставляющий реализации effect-операций.
Также вводится конструкция \texttt{perform}, позволяющая вызвать effect-операцию (отправить запрос контексту).
Каждая операция имеет набор параметров, а также ``обратный адрес'', продолжение места вызова, в который она вернёт результат.
Например, на экспериментальном языке Koka\footnote{\url{https://koka-lang.github.io/koka/doc/index.html}} контекст, предоставляющий некоторую константу, может быть реализован следующим образом (\texttt{resume}~--- имя продолжения места вызова, \texttt{perform} вставляется неявно):
\begin{minted}{cpp}
    with handler
      ctl ask() resume(21)
    ask() + ask()
\end{minted}

Если ближайший контекст нужный запрос обработать не может, запрос делегируется внешнему контексту, и так пока подходящий контекст не будет найден.
На этой неделе основывается модульность интерпретаторов, заданных хендлерами.

% todo историческая справка

% todo больше впечатляющих примеров

\subsubsection{Хендлеры через ограниченные продолжения}

Как мы уже видели ранее, различные эффекты можно реализовывать с помощью доступа к текущему продолжению (\ref{subsubsec:god-cont}, \ \ref{subsubsec:monadic-reflection}).
Хендлеры эффектов дополняют эту идею тем, что используют ограниченные продолжения, чтобы передавать управление различным интерпретаторам (хендлерам).

Известно, что классические операторы манипуляции ограниченными продолжениями, monadic reflection и хендлеры эффектов выразимы друг через друга~\cite{forster2017expressive}.

\subsubsection{Эффективная реализация хендлеров} \label{subsubsec:evidence-passing}

В общем виде скорость работы \texttt{perform} определяется скоростью захвата и восстановления ограниченных продолжений.
Однако, существует класс операций, которые можно реализовать гораздо эффективнее.

Если мы посмотрим на реализацию операции \texttt{ask}, то мы увидим, что она последним действием вызывает продолжение, возвращая управление вызвавшему коду.
Такие операции называют \vocab{tail-resumptive}, они очень сильно напоминают обычные функции, за исключением того, что их реализации определяются контекстом (хендлером).
Таким образом, tail-resumptive операции можно реализовать как неявную передачу словаря функций от хендлера к \texttt{perform}, и тем самым избежать дорогих манипуляций продолжениями~\cite{xie2020effect}\footnote{Хендлеры tail-resumptive операций напоминают co-pattern-matching (см.\ \ref{subsubsec:data-codata}).}.

\subsubsection{Встроенные хендлеры как явная клиент-серверная коммуникация} \label{subsubsec:extensible-effects}

Чтобы лучшим образом понять семантику хендлеров, реализуем язык с хендлерами как встроенный в Haskell.
Начнём с варианта, предложенного Олегом Киселёвым, максимально прямолинейно кодирующего идею клиент-серверной коммуникации терма и контекста~\cite{kiselyov2013extensible}.

Начнём с эффекта \texttt{ask}, запрашивающего числа у контекста.
Зададим тип данных сообщений к контексту, это либо конечный результат вычисления, либо запрос \mintinline{haskell}{Ask}, содержащий ``обратный адрес''~--- текущее продолжение:
\begin{minted}{haskell}
    data Message res = Val res | Ask (Int -> Message res)
\end{minted}

Продолжения будем собирать в специализированный \mintinline{haskell}{Monad Cont} с подходящим response type (см.~\ref{subsubsec:monad-cont}, \ \ref{subsubsec:god-cont}):
\begin{minted}{haskell}
    newtype Eff res = Eff
      { runEff :: forall res' . (res -> Message res') -> Message res' }
\end{minted}

Тогда эффект \texttt{ask} реализуется просто как ``отправка'' запроса \mintinline{haskell}{Ask} с текущим продолжением:\footnote{Ранее мы аналогично реализовывали генераторы, см.\ \ref{subsubsec:generators-coroutines}.}
\begin{minted}{haskell}
    ask :: Eff Int
    ask = Eff \k -> Ask k
\end{minted}

Хендлер мы реализуем как ``сервер'', который в цикле обрабатывает запросы, пока вычисление не пришлёт конечный результат:
\begin{minted}{haskell}
    run :: Eff res -> Message res
    run comp = runEff comp Val

    runReader :: Eff res -> Int -> res
    runReader comp env = loop (run comp)
      where
        loop = \case
          Val res -> res
          Ask k -> loop (k env)
\end{minted}

Наконец, мы можем писать effectful код:
\begin{minted}{haskell}
    exampleReader :: Int -> Int
    exampleReader = runReader do
      x <- ask
      y <- ask
      pure (x + y)
\end{minted}

\subsubsection{Расширяемые сообщения и пересылка}

Абстрагируем тип сообщений по ``форме'' запросов, которые в них могут участвовать (см.\ \ref{subsubsec:functor-fixpoint}):
\begin{minted}{haskell}
    data Message effs res = Val res | Request (effs (Message effs res))
\end{minted}

Предыдущий тип сообщений получается передачей следующего функтора формы:
\begin{minted}{haskell}
    newtype Reader env msg = Ask (env -> msg)
\end{minted}

\begin{task}
    Убедитесь, что \mintinline{haskell}{Message (Reader Int) res} эквивалентно предыдущему типу сообщений.
\end{task}

Копроизведение функторов формы является функтором формы (см.\ \ref{subsubsec:functor-coprod}):
\begin{minted}{haskell}
    data (eff |> effs) a = L (eff a) | R (effs a)
\end{minted}

Теперь операция \texttt{ask} допускает существование других типов запросов:
\begin{minted}{haskell}
    ask :: Eff (Reader env |> effs) env
    ask = Eff \k -> Msg $ L $ Ask k
\end{minted}

Новый хендлер обрабатывает только часть запросов, остальные пересылает хендлеру выше (скомпозировав правильным образом продолжения):
\begin{minted}{haskell}
runReader
  :: forall effs env res . Functor es
  => Eff (Reader env |> effs) res
  -> env -> Eff effs res
runReader comp env = loop (run comp)
  where
    loop :: Request (Reader env |> effs) res -> Eff effs res
    loop = \case
      Val res -> pure res
      Msg (L (Ask k)) -> loop (k env)
      Msg (R unknownReq) -> do
        response <- send \k -> fmap k unknownReq
        loop response
\end{minted}

\subsubsection{Free monads}

% todo free folk in general

% todo Stackless Scala with free monads
% todo freer monads
% todo iteratee O. Kiselyov. Iteratees. In Proc. of the 11th International Symposium on Functional and Logic Programming, pages 166–181, 2012.

% todo deep vs shallow handlers

% todo

\subsubsection{Приложения хендлеров}

% todo  custom schedulers

% todo

\subsection{Трансформеры монад}

% TODO\cite{liang1995monad}

% todo remind of tagless final

% todo lift - monad morphism

% todo ContT, lift = (>>=)

% todo fused effects and CPS

% todo

\subsection{Алгебраичность и эффекты высших порядков}

% todo abstracting definitional interpreters & github semantics

% todo Monads and algebras

% todo https://www.eff-lang.org/handlers-tutorial.pdf

% todo


% todo compare open type families & extensible interpreters

% todo Polymorphic Symmetric Multiple Dispatch with Variance
% todo \textit{multimethods}
%    Languages with \textit{multimethods}, like Common Lisp’s CLOS, Dylan, and Julia do support adding both new types and operations easily.
%    What they typically sacrifice is either static type checking, or separate compilation.

% todo ZIO, TS Effect

% todo call-by-push value and how it is related to effects

% todo порядок интерпретаторов и interleaving of effects

% todo abstracting definitional interpreters, github semantics

% todo сравните хендлеры с генераторами ранее

% todo break и continue в домашку
