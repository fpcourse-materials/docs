
% todo merriage of effects and monads

% todo https://github.com/yallop/effects-bibliography

TODO % todo


%Чёткого определения контекста исполнения не существует, однако можно выделить свойства, по которым такие контексты можно распознавать в коде.
%Так, контекст исполнения должен обладать хотя бы одним из следующих двух свойств:
%\begin{enumerate} % todo
%    \item Взаимодействие с контекстом наблюдаемо.
%    Например, в качестве контекста мы можем рассматривать подсистему управления памятью, а в качестве effectful операции --- операцию изменения значения ячейки, так как изменение можно пронаблюдать последующей операцией чтения.
%    \item Действие контекста ограничено определённым скоупом.
%    Например, обработчик исключения является контекстом исполнения для кода внутри \mintinline{kotlin}{try-catch} блока: \mintinline{kotlin}{throw} передаст управление соответствующему обработчику.
%    Также, за пределами такого скоупа операции могут иметь другой смысл.
%    Например, система распространения зависимостей (dependency injection) в различных скоупах может выдавать различную функциональность.
%\end{enumerate}

