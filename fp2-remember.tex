В этом разделе мы вспомним основные концепции функционального программирования и языка Haskell.

\subsection{Термы и редукция} \label{subsec:terms-reduction}

В ФП программы представляют собой выражения.
Выполнение программ --- редукция таких выражений до более ``простых''.
Выражения можно представлять как в виде линейной записи символов, так и в виде дерева, для понимания которого не требуется знания вспомогательных правил ассоциативности и проч.

Простейший функциональный язык --- $\lambda$-исчисление.
Выражения в нём называются $\lambda$-термами, которые состоят из вершин трёх видов:
\begin{description}
    \item[Переменные] $x \in \Lambda$, если $x \in V$ \hspace{2em}
    \begin{tikzpicture}
        \node [expr] (x) {x};
    \end{tikzpicture}
    \item[Абстракция] $(\lambda x\ldotp M) \in \Lambda$, если $x \in V, M \in \Lambda$ \hspace{2em}
    \begin{tikzpicture}
        \node [expr] (lam) {$\lambda$};
        \node [decl] (x) [below right = of lam] {$x$};
        \node [subtree] (m) [below = of lam] {$M$};
        \draw[->] (m.north) -- (lam);
        \draw[->] (x) -- (lam);
    \end{tikzpicture}
    \item[Аппликация] $(M~N) \in \Lambda$, если $M \in \Lambda, N \in \Lambda$ \hspace{2em}
    \begin{tikzpicture}
        \node [expr] (dog) {$@$};
        \node [subtree] (m) [below left= of lam] {$M$};
        \node [subtree] (n) [below right= of lam] {$N$};
        \draw[->] (m.north) -- (lam);
        \draw[->] (n.north) -- (lam);
    \end{tikzpicture}
\end{description}

В произвольном выражении можно заменить некоторый его фрагмент на формальный параметр, который должен быть задекларирован выше по дереву с помощью специальной вершины $\lambda$.
Вместо формального параметра можно в дальнейшем подставлять различные конкретные параметры с помощью вершины-аппликации $@$, то есть переиспользовать это выражение для различных целей (например, рис.\ \ref{fig:lam}).
Редукция как раз определяется как следующее правилом переписывания: ищется применение $\lambda$-функции к аргументу и в её тело осуществляется подстановка аргумента во все свободные вхождения переменной, связанной лямбдой (рис.\ \ref{fig:reduction}).

\begin{figure}
    \centering
    \begin{tikzpicture}
        \node [expr] (div) {$\div$};
        \node [expr] (x) [below left=of div] {$x$};
        \node [expr] (plus) [below right = of div] {$+$};
        \node [expr] (two) [below right = of plus] {$2$};
        \node [hexpr] (mult) [below left = of plus] {$\times$};
        \node [hexpr] (ten) [below left = of mult] {$10$};
        \node [hexpr] (four) [below right = of mult] {$4$};
        \draw[->] (plus) -- (div);
        \draw[->] (x) -- (div);
        \draw[->] (mult) -- (plus);
        \draw[->] (ten) -- (mult);
        \draw[->] (four) -- (mult);
        \draw[->] (two) -- (plus);
    \end{tikzpicture}
    \begin{tikzpicture}
        \node [hdecl] (f) {$f$};
        \node [hexpr] (lam) [below=of f] {$\lambda$};
        \node [hdecl] (y) [below right =of lam] {$y$};
        \node [expr] (div) [below=of lam] {$\div$};
        \node [expr] (x) [below left=of div] {$x$};
        \node [expr] (plus) [below right = of div] {$+$};
        \node [hexpr] (param) [below left=of plus] {$y$};
        \node [expr] (two) [below right = of plus] {$2$};
        \draw[->] (lam) -- (f);
        \draw[->] (div) -- (lam);
        \draw[->] (plus) -- (div);
        \draw[->] (x) -- (div);
        \draw[->] (two) -- (plus);
        \draw[->] (param) -- (plus);
        \draw[->] (y) -- (lam);
    \end{tikzpicture}
    \begin{tikzpicture}
        \node [hexpr] (app) {$@$};
        \node [hexpr] (f) [below left =of app] {$f$};
        \node [expr] (mult) [below right= of app] {$\times$};
        \node [expr] (ten) [below left= of mult] {$10$};
        \node [expr] (four) [below right= of mult] {$4$};
        \draw[->] (mult) -- (app);
        \draw[->] (f) -- (app);
        \draw[->] (ten) -- (mult);
        \draw[->] (four) -- (mult);
    \end{tikzpicture}
    \caption{Выражение с помощью $\lambda$ вершины преобразуется в функцию одного аргумента.}
    \label{fig:lam}
\end{figure}

\begin{figure}
    \centering
    \begin{tikzpicture}
        \node [expr] (top) {$\cdots$};
        \node [hexpr] (ap) [below = of top] {$@$};
        \node [hexpr] (lam) [below left = of ap] {$\lambda$};
        \node [decl] (x) [below right = of lam] {$x$};
        \node [subtree] (l) [below = of lam] {$M$};
        \node [subtree] (r) [below right = of ap] {$N$};
        \draw[->] (ap) -- (top);
        \draw[->] (lam) -- (ap);
        \draw[->] (l) -- (lam);
        \draw[->] (r.north) -- (ap);
        \draw[->] (x) -- (lam);
    \end{tikzpicture}
    \begin{tikzpicture}
        \node [expr] (top) {$\cdots$};
        \node [subtree] (bot) [below = of top] {$\subst{M}{x}{N}$};
        \draw[->] (bot) -- (top);
    \end{tikzpicture}
    \caption{Редукция переписывает дерево путём подстановки конкретного аргумента вместо формального параметра.}
    \label{fig:reduction}
\end{figure}

\subsection{Типы}

Программное обеспечение --- это сложно.
Поэтому постоянно и неизбежно в программах возникают ошибки.
Их можно искать, в том числе, статически, то есть без запуска программы.
Одним из видов статического анализа является анализ типов.

Идея анализа типов состоит в том, что мы каждой вершине дерева программы пытаемся присвоить некоторую синтаксическую метку по определённым правилам.
Если каждой вершине метку присвоить можно, то мы считаем, что программа проходит проверку типов, и она ``хорошая''.
Например, на рисунке\ \ref{fig:to-be-typed} представлено выражение, а на рисунке\ \ref{fig:typed} каждой вершине приписаны метки в согласие с некоторой системой типов.

\begin{figure}
    \centering
    \begin{tikzpicture}
        \node [expr] (ap) {$@$};
        \node [expr] (three) [below right = of ap]{$3$};
        \node [hexpr] (lam) [below left =of ap] {$\lambda$};
        \node [hdecl] (y) [below right =of lam] {$y$};
        \node [expr] (div) [below=of lam] {$\div$};
        \node [expr] (x) [below left=of div] {$x$};
        \node [expr] (plus) [below right = of div] {$+$};
        \node [hexpr] (param) [below left=of plus] {$y$};
        \node [expr] (two) [below right = of plus] {$2$};
        \draw[->] (three) -- (ap);
        \draw[->] (lam) -- (ap);
        \draw[->] (div) -- (lam);
        \draw[->] (plus) -- (div);
        \draw[->] (x) -- (div);
        \draw[->] (two) -- (plus);
        \draw[->] (param) -- (plus);
        \draw[->] (y) -- (lam);
    \end{tikzpicture}
    \caption{Дерево соответствующее выражению $(\lambda y\ldotp x \div (y + 2)) \ap 3$.}
    \label{fig:to-be-typed}
\end{figure}

\begin{figure}
    \centering
    \begin{tikzpicture}
        \node [expr] (ap) {$@ : int$};
        \node [expr] (three) [below right = of ap]{$3 : int$};
        \node [hexpr] (lam) [below left=of ap] {$\lambda : int \to int$};
        \node [decl] (y) [below right = of lam] {$y {\color{red}~: int}$};
        \node [expr] (div) [below=of lam] {$\div : int$};
        \node [expr] (x) [below left=of div] {$x : int$};
        \node [expr] (plus) [below right = of div] {$+ : int$};
        \node [hexpr] (param) [below left=of plus] {$y : int$};
        \node [expr] (two) [below right = of plus] {$2 : int$};
        \draw[->] (lam) -- (ap);
        \draw[->] (three) -- (ap);
        \draw[->] (div) -- (lam);
        \draw[->] (plus) -- (div);
        \draw[->] (x) -- (div);
        \draw[->] (two) -- (plus);
        \draw[->] (param) -- (plus);
        \draw[->] (y) -- (lam);
    \end{tikzpicture}
    \caption{Дерево выражения $(\lambda y\ldotp x \div (y + 2)) \ap 3$ после приписывания типовых меток.}
    \label{fig:typed}
\end{figure}

\vocab{Система типов} определяет синтаксис типовых меток и правила, по которым их можно приписывать.
Синтаксис обычно описывается в классических нотациях а ля BNF, а правила в виде типовых дробей.
Например, так выглядят дроби для просто-типизированного $\lambda$-исчисления:
\[
    \begin{array}{l c r}
        \infer[ctx]{\Gamma \vdash x: \sigma}{(x: \sigma) \in \Gamma}
        &
        \infer[elim\to]{\Gamma \vdash M\;N : \tau}{\Gamma \vdash M : \sigma \to \tau & \Gamma \vdash N : \sigma}
        &
        \infer[intro\to]{\Gamma \vdash \lambda x^{\color{red} \sigma}\ldotp M : \sigma \to \tau}{\{x : \sigma\} \cup \Gamma \vdash M : \tau}
    \end{array}
\]

Типовые метки имеют чисто-синтаксическую природу, однако их можно проинтерпретировать.
Самая популярная интерпретация~--- воспринимать типовую метку как множество.
Так, метке $int \to int$ можно поставить в соответствие множество функций между множествами ограниченных целых чисел.

\subsection{Функции в Haskell}

В своей основе Haskell представляет собой расширенное типизированное $\lambda$-исчисление, дополненное примитивными типами, возможностью декларировать новые имена, структурами данных и классами типов.

Примеры $\lambda$-абстракций в REPL окружении GHCi:

\begin{minted}{haskell}
    ghci> (\x -> x + 1) 4
    5
\end{minted}

Можно узнать тип функции в интерпретаторе (в реальности числа полиморфные, но об этом далее):
\begin{minted}{haskell}
    ghci> :t \x -> x + 1
    \x -> x + 1 :: Int -> Int
\end{minted}

Функциям можно давать имена.
Именам можно приписывать типы, это рекомендуется делать явно для деклараций на верхнем уровне файлов исходного кода.
\begin{minted}{haskell}
    f :: Int -> Int
    f x = x + 1
\end{minted}

Если имя типа начинается с маленькой буквы, то это не конкретный заранее заданный тип, а типовая переменная, способная принимать различные значения в зависимости от места вызова.
Такая возможность называется \vocab{параметрическим полиморфизмом}.
Так, функция, которая просто возвращает свой аргумент, никак не ограничивает тип аргумента.
Но в то же время тип результата должен совпадать с типом аргумента.
\begin{minted}{haskell}
    id :: a -> a
    id x = x

    ghci> :t id 5
    id 5 :: Int
\end{minted}

Функции могут принимать другие функции в качестве аргументов (такие функции называются \vocab{функциями высших порядков (higher-order functions)}.
Имя функции может состоять из специальных символов, тогда она считается оператором и может применяться к своим операндам в инфиксном стиле:
\begin{minted}{haskell}
    ($) :: (a -> b) -> a -> b
    f $ x = f x
\end{minted}

Пример рекурсивной функции, использующей охранные выражения для отличения базового случая рекурсии:
\begin{minted}{haskell}
    factorial :: Int -> Int
    factorial n
      | n < 1 = 1
      | otherwise = n * factorial (n - 1)
\end{minted}

\begin{task}
    Что выведет запрос \mintinline{haskell}|ghci> :t uncurry (flip const)|?
\end{task}

\begin{task}
    Что выведет запрос \mintinline{haskell}|ghci> :t first . first| при
    \begin{minted}{haskell}
        first :: (a -> a') -> (a, b) -> (a', b)
    \end{minted}
\end{task}

\begin{task}
    Реализуйте факториал с помощью техники аккумулирующего параметра.
\end{task}

\subsection{Данные в Haskell}

В Haskell есть встроенная возможность объявлять новые типы данных на основании других типов, а так же создавать их экземпляры.

Зададим тип данных, описывающий животных:
\begin{minted}{haskell}
    data Animal
      = Cat String Int
      | Dog String
\end{minted}

Мы задали тип данных \texttt{Animal} и два способа создать значения этого типа: для кошек и собак.
\texttt{Cat} и \texttt{Dog}~--- это \vocab{конструкторы данных}.
Они представляют собой функции, реализация которых находится на стороне языка.
Они выделяют память под экземпляры данного типа и позиционно размещают компоненты.
Кошек мы описываем именем и оставшимся количеством жизней, а собак~--- только именем.
\begin{minted}{haskell}
    Cat :: String -> Int -> Animal
    Dog :: String -> Animal
\end{minted}

Чтобы воспользоваться информацией, сохранённой в структуре данных, требуется деконструировать её с помощью паттерн-матчинга.
Мы сопоставляем значение типа с образцом.
Если образец похож на то, как было сконструировано значение, то он выбирается среди других образцов и переменные, задекларированные в нём, начинают ссылаться на соответствующее позиционно содержимое структуры данных:
\begin{minted}{haskell}
    show :: Animal -> String
    show animal = case animal of
      Cat name nLifes -> "This is cat " ++ name ++ show nLifes
      Dog name -> "This is dog " ++ name
\end{minted}

В Haskell есть специальный синтаксис для объявления полей с именованными метками.
\begin{minted}{haskell}
    data Penguin = Penguin { getName :: String, getAge :: Int }
    penguin = Penguin { getName = "Andrey", getAge = 500 }
\end{minted}
Haskell генерирует функции-аксессоры для доступа к полям объекта:
\begin{minted}{haskell}
    ghci> :t getName :: Penguin -> String
\end{minted}

Часто функции в программировании частичные --- при некоторых значениях аргументов они могут вернуть результат, а при некоторых~--- нет.
Давайте моделировать это с помощью специального типа данных.
Если есть вещественный результат, будем возвращать его.
Если нет, будем возвращать специально выделенное константное значение этого типа.
\begin{minted}{haskell}
    data MaybeD = NothingD | JustD Double
    sqrt :: Double -> MaybeD
    sqrt x = if x < 0 then NothingD else JustD (calcSqrt x)
\end{minted}

Можно заметить, что так нам придётся объявлять по типу \texttt{MaybeT} для каждого типа \texttt{T}.
Поэтому Haskell позволяет абстрагироваться в типе, аналогично тому как можно абстрагироваться по значениям в терме.
\begin{minted}{haskell}
    data Maybe a = Nothing | Just a
    sqrt :: Double -> Maybe Double
    sqrt x = if x < 0 then Nothing else Just (calcSqrt x)
\end{minted}

Заметьте, что сейчас \texttt{Maybe} --- это не совсем тип, так как теперь нужно передать типовой параметр, чтобы получить конкретный тип.
\texttt{Maybe} называют \vocab{типовым конструктором}.

Вместе с абстракцией на уровне типов появилась и аппликация типа к типу.
А что если дать меньше параметров типовому конструктору, чем ожидается?
А что если больше?
Контроль за корректностью типовых аппликаций обеспечивает \vocab{система кайндов}\footnote{Иногда в русскоязычной литературе кайнды называют родами типов, но мы не будем так говорить.}.
Это простейшие ``типы для типов'', то есть синтаксические метки, контролирующие корректность записанных программистом типов.
Так, обычные типы имеют метку (кайнд) \texttt{*}.
Типовые конструкторы имеют стрелочные кайнды.
Например, \texttt{Maybe :: * -> *}.
Аппликация типового конструктора к типу подходящего кайнда убирает одну стрелку:
\begin{minted}{haskell}
    ghci> :k Int
    Int :: *
    ghci> :k Maybe
    Maybe :: * -> *
    ghci> :k Maybe Int
    Maybe Int :: *
\end{minted}

Кроме совершенно новых типов данных, в Haskell можно объявлять типовые синонимы.
Это имена, которые можно использовать вместо других типов, если, например, запись оригинального типа слишком длинная для повсеместного написания.
\begin{minted}{haskell}
    type T a = VeryLongType Int (a -> AnotherLongType a)
\end{minted}

Если тип данных содержит только один конструктор и только одно поле, то отсутствует необходимость в аллокации новой памяти, содержащей тег конструктора и набор ссылок на поля.
В таком случае, в качестве значения такого типа можно всегда просто использовать значение оборачиваемого типа, оставляя новый тип присутствовать исключительно во время компиляции, снижая нагрузку во время исполнения.
Для объявления таких типов-обёрток нужно воспользоваться ключевым словом \texttt{newtype} вместо \texttt{data}:
\begin{minted}{haskell}
    newtype CourseId = CourseId Int64
    newtype ModuleId = ModuleId Int64
\end{minted}

\begin{task}
    Определите кайнд конструктора типа
    \begin{minted}{haskell}
        data Free f a = Pure a | Free (f (Free f a))
    \end{minted}
\end{task}

\subsection{Классы типов в Haskell}

Параметрический полиморфизм позволяет использовать один и тот же код для различных типов входных данных.
Классы типов же позволяют одному идентификатору ссылаться на разные реализации для разных типов данных (что аналогично механизму перегрузки (overloading) в других языках).
Классы типов, как говорят, являются механизмом \vocab{специального (ad-hoc) полиморфизма}.
Так, мы можем задекларировать символ \texttt{==}, выбор реализации которого зависит от выбора типа аргументов \texttt{a}:
\begin{minted}{haskell}
    class Eq a where
      (==) :: a -> a -> Bool
\end{minted}

Для каждого типа можно объявить свою собственную реализацию \texttt{Eq}:
\begin{minted}{haskell}
    instance Eq CourseId where
      CourseId x == CourseId y = x == y

    instance Eq a => Eq [a] where
      [] == [] = True
      x:xs == y:ys = x == y && xs == ys
\end{minted}

Теперь в зависимости от конкретного типа \texttt{a} в месте вызова, будет выбрана подходящая реализация для этого типа:
\begin{minted}{haskell}
    ghci> CourceId 1 == CourceId 2
    False
    ghci> [CourceId 1, CourceId 2] == [CourceId 1, CourceId 2]
    True
\end{minted}

Рассмотренные ранее параметрически-полиморфные функции ничего не могли делать со своими аргументами, кроме как возвращать их в качестве результата или передавать в другие полиморфные функции.
Чтобы уметь делать что-то ещё, нужна какая-то дополнительная информация про тип, потому что иначе нет никакой гарантии, что над объектом данного типа можно делать все необходимые операции.
Так, функция \texttt{suc n = n + 1} не будет работать для строчек, потому что для них, очевидно, не определена операция сложения.
Поэтому некорректно будет приписать полиморфный тип \texttt{suc :: a -> a}.

Классы типов, в отличие от перегрузки, в том числе являются механизмом ограничения полиморфности функций.
Мы можем явно задать, что функция требует не произвольный тип на вход, а произвольный тип, для которого определены обязательно нужные нам операции.
Так, для типа \texttt{suc} достаточно ограничить тип условием наличия плюса для него (операция обозначаемая символом \texttt{+} объявлена в классе типов \mintinline{haskell}|Num|):
\begin{minted}{haskell}
    suc :: Num a => a -> a
\end{minted}

\begin{task}
    Реализуйте инстанс полугруппы для функций.
\end{task}

\begin{task}
    Реализуйте проверку равенства функций.
\end{task}

\subsection{Монады в Haskell}

Класс типов \texttt{Functor} объявляется для конструкторов типов и позволяет заменить в некотором контейнере все элементы одного типа на все элементы другого, оставляя структуру контейнера неизменной.

\begin{minted}{haskell}
    class Functor (f :: * -> *) where
      fmap :: (a -> b) -> f a -> f b

    instance Functor [] where
      fmap :: (a -> b) -> [a] -> [b]
      fmap _ [] = []
      fmap f (x:xs) = f x : fmap f xs
\end{minted}

В Haskell любая функция просто вычисляет результат некоторого типа.
Однако в программирования часто требуются функции, которые не только вычисляют результат, но и делают что-то ещё.
Например, изменяют какое-то состояние или пишут в консоль.
Иными словами, производят побочные эффекты.
В любом случае в Haskell мы можем только вернуть из функции только результат, поэтому такие побочные эффекты мы кодируем в качестве дополнительной структуры, оборачивающей чистый результат.
Т.е. если функция без побочных эффектов возвращала какой-то тип \texttt{a}, то после добавления побочных эффектов в её реализацию, она будет возвращать некоторый тип \point{вычислений} \texttt{f a}.

\begin{itemize}
    \item Если функция кидает ошибку, то \texttt{f = Maybe}.
    \item Если функция читает глобальное состояние типа \texttt{e}, то \texttt{f = e -> \_}.
    \item Если функция читает глобальное состояние \texttt{s} и обновляет его, то \texttt{f = s -> (s, \_)}.
\end{itemize}

Стандартная библиотека Haskell предоставляет несколько классов типов для работы со значениями вида \texttt{f a}.
Они позволяют абстрагироваться от структуры \texttt{f} и работать со значениями \texttt{a} внутри, как будто нет никакой дополнительной структуры.

Первый такой класс типов позволяет писать выражения над вычислениями \texttt{f a}.
\begin{minted}{haskell}
    class Functor f => Applicative (f :: * -> *) where
      pure :: a -> f a
      liftA2 :: (a -> b -> c) -> f a -> f b -> f c

    instance Applicative Maybe where
      pure :: a -> Maybe a
      pure = Just

      liftA2 :: (a -> b -> c) -> Maybe a -> Maybe b -> Maybe c
      liftA2 _ Nothing _ = Nothing
      liftA2 _ _ Nothing = Nothing
      liftA2 f (Just x) (Just y) = Just (f x y)
\end{minted}

Второй класс типов позволяет делать последовательную композицию вычислений в императивном стиле:
\begin{minted}{haskell}
    class Applicative m => Monad (m :: * -> *) where
      (>>=) :: m a -> (a -> m b) -> m b

    newtype State s a = State { runState :: s -> (s, a) }

    instance Monad (State s) where
      (>>=) :: State s a -> (a -> State s b) -> State s b
      m >>= k = State \s ->
        let (s', x) = runState m s in
        runState (k x) s'
\end{minted}

Теперь если мы определим базовые операции работы с состоянием, мы сможем писать код в императивном стиле с побочными эффектами.
\begin{minted}{haskell}
    get :: State s s
    get = State \s -> (s, s)

    put :: s -> State s ()
    put newS = State \oldS -> (newS, ())

    example :: State Int Int
    example =
      get >>= \x ->
      put 42 >>= \() ->
      get >>= \y ->
      pure (x + y)

    ghci> runState example 1
    43
\end{minted}

Для таких монадических цепочек существует специальный синтаксический сахар:
\begin{minted}{haskell}
    example :: State Int Int
    example = do
      x <- get
      put 42
      y <- get
      pure (x + y)
\end{minted}

\begin{task}
    Реализуйте \mintinline{haskell}|liftA3| через \mintinline{haskell}|liftA2|.
\end{task}

\begin{task}
    Реализуйте \mintinline{haskell}|>>=| через \mintinline{haskell}|join| и наоборот.
\end{task}

\begin{task}
    Два числа с консоли, поделите одно на другое нацело и распечатайте результат, если остаток не нулевой, распечатайте его тоже.
\end{task}
