%! suppress = EscapeHashOutsideCommand
%! suppress = Quote
%! suppress = MissingImport
%! suppress = MissingLabel
%! suppress = LineBreak

% CLI args https://tex.stackexchange.com/a/1501
\newif\ifhandout
\input{flags}

%! suppress = MissingLabel
%! suppress = DocumentclassNotInRoot
%! suppress = DiscouragedUseOfDef

% * Make friends tikz & colors
%   https://en.wikibooks.org/wiki/LaTeX/Colors
% * To enable vertical top alignment globally
%   https://tex.stackexchange.com/questions/9889/positioning-content-at-the-top-of-a-beamer-slide-by-default
% * Set handout from CLI
%   https://tex.stackexchange.com/a/1501
\ifhandout
\documentclass[usenames, dvipsnames, handout]{beamer} % https://tex.stackexchange.com/questions/224091/beamer-how-to-disable-pause-temporarily
\else
\documentclass[usenames, dvipsnames]{beamer}
\fi
% ------------------------------------------------

% Graphics
\usepackage{color}
\usepackage{tabularx}
\usepackage{tikz}
% https://tikz.dev/tikz-graphs
\usetikzlibrary{positioning, shapes.geometric, arrows, automata, graphs}
\tikzset{
    expr/.style={ellipse, draw=gray!60, fill=gray!5, very thick, minimum size=7mm, yshift=0.7cm},
    hexpr/.style={ellipse, draw=gray!60, fill=blue!15, very thick, minimum size=7mm, yshift=0.7cm},
    stmt/.style={rectangle, draw=gray!60, fill=gray!5, very thick, minimum size=5mm, yshift=0.7cm},
    decl/.style={rectangle, draw=blue!60, fill=gray!5, very thick, minimum size=5mm, yshift=0.7cm},
    hdecl/.style={rectangle, draw=blue!60, fill=blue!15, very thick, minimum size=5mm, yshift=0.7cm},
    subtree/.style={shape border rotate=90, isosceles triangle, draw=gray!60, fill=gray!5, very thick, minimum size=5mm, yshift=0.0cm},
}
\usepackage{blkarray}
\usepackage{graphicx}
\usepackage{forest} % https://tex.stackexchange.com/questions/198405/how-to-change-the-color-of-subtrees-in-tikz-qtree
% ------------------------------------------------

% Math
\usepackage{amsmath, amsfonts}
\usepackage{amssymb}
\usepackage{proof}
\usepackage{mathrsfs}
% Crossed-out symbols
% https://tex.stackexchange.com/questions/75525/how-to-write-crossed-out-math-in-latex
\usepackage[makeroom]{cancel}
\usepackage{mathtools}
% ------------------------------------------------

% Additional font sizes
% https://www.overleaf.com/learn/latex/Questions/How_do_I_adjust_the_font_size%3F
\usepackage{moresize}
% Additional colors
% https://www.overleaf.com/learn/latex/Using_colours_in_LaTeX
\usepackage{xcolor}
% Textual math symbols
\usepackage{textcomp}
% ------------------------------------------------

% Language
\usepackage[utf8] {inputenc}
\usepackage[T2A] {fontenc}
\usepackage[english, russian] {babel}
\usepackage{indentfirst, verbatim}
\usetikzlibrary{cd, babel}
% ------------------------------------------------

% Fonts: https://sites.math.washington.edu/~reu/docs/latex_symbols.pdf
\usepackage{stmaryrd}
\usepackage{cmbright}
\usepackage{wasysym}
\usepackage[weather]{ifsym} % https://tex.stackexchange.com/questions/100424/how-to-use-the-ifsym-package
% https://tex.stackexchange.com/questions/615300/pdflatex-builtin-glyph-names-is-empty
\pdfmapline{=dictsym DictSym <dictsym.pfb}
\pdfmapline{=pigpen <pigpen.pfa}
\usepackage{dictsym}
% ------------------------------------------------

% Code
% * Needs -shell-escape build flag
%   https://tex.stackexchange.com/questions/99475/how-to-invoke-latex-with-the-shell-escape-flag-in-texstudio-former-texmakerx
% * Set build directory
%   https://tex.stackexchange.com/questions/339931/latex-minted-package-using-custom-output-directory-build
\usepackage{minted}
\setminted{xleftmargin=\parindent, autogobble, escapeinside=\#\#}
% ------------------------------------------------

% Template
\usetheme{CambridgeUS}
\usecolortheme{dolphin}
% https://tex.stackexchange.com/questions/231439/beamer-how-to-make-font-larger-for-page-numbers
\setbeamerfont{headline}{size=\scriptsize}
\setbeamerfont{footline}{size=\scriptsize}
% Remove heddline
% https://tex.stackexchange.com/questions/33146/how-could-i-remove-a-header-in-a-beamer-presentation
%\setbeamertemplate{headline}{}
% Slide sizes
% https://tex.stackexchange.com/questions/56768/how-to-set-a-small-default-font-size-with-beamer
%\geometry{paperwidth=140mm,paperheight=105mm} % 4:3
\geometry{paperwidth=168mm,paperheight=105mm} % 16:10
% Remove navigation bar
% https://stackoverflow.com/questions/3210205/how-to-get-rid-of-navigation-bars-in-beamer
\beamertemplatenavigationsymbolsempty
% ------------------------------------------------

% Bullets
% https://9to5science.com/change-bullet-style-formatting-in-beamer
% https://tex.stackexchange.com/questions/185742/i-need-to-change-color-of-beamer-itemize-and-subitem-separately
\setbeamertemplate{itemize item}{\scriptsize\raise1.25pt\hbox{\donotcoloroutermaths$\blacktriangleright$}}
\setbeamertemplate{itemize subitem}{\scriptsize\raise1.5pt\hbox{\donotcoloroutermaths$\blacktriangleright$}}
\setbeamertemplate{itemize subsubitem}{\tiny\raise1.5pt\hbox{\donotcoloroutermaths$\blacktriangleright$}}
\setbeamertemplate{enumerate item}{\insertenumlabel.}
\setbeamertemplate{enumerate subitem}{\insertenumlabel.\insertsubenumlabel}
\setbeamertemplate{enumerate subsubitem}{\insertenumlabel.\insertsubenumlabel.\insertsubsubenumlabel}
% ------------------------------------------------

% Table of contents format
% https://tex.stackexchange.com/questions/642927/format-table-of-contents-in-beamer
\setbeamertemplate{section in toc}{%
        {\color{blue}\inserttocsectionnumber.}
    \inserttocsection\par%
}
\setbeamertemplate{subsection in toc}{%
        {\color{blue}\hspace{1em}\scriptsize\raise1.25pt\hbox{\donotcoloroutermaths$\blacktriangleright$}}
    \inserttocsubsection\par%
}
\setbeamertemplate{subsubsection in toc}{%
        {\color{blue}\hspace{2em}\tiny\raise1.25pt\hbox{\donotcoloroutermaths$\blacktriangleright$}}
    \inserttocsubsubsection\par%
}
% ------------------------------------------------

% Misc
\usepackage{multicol}
\usepackage{hyperref}
\usepackage{soul} % https://tex.stackexchange.com/questions/23711/strikethrough-text
% ------------------------------------------------

% Fix \pause for amsmath package envs (black black magic)
% https://tex.stackexchange.com/questions/16186/equation-numbering-problems-in-amsmath-environments-with-pause/75550#75550
% https://tex.stackexchange.com/questions/6348/problem-with-beamers-pause-in-alignments
%! suppress = Makeatletter
\makeatletter
\let\save@measuring@true\measuring@true
\def\measuring@true{%
    \save@measuring@true
    \def\beamer@sortzero##1{\beamer@ifnextcharospec{\beamer@sortzeroread{##1}}{}}%
    \def\beamer@sortzeroread##1<##2>{}%
    \def\beamer@finalnospec{}%
}
%! suppress = Makeatletter
\makeatother
% ------------------------------------------------

% Sections
\newcommand{\sectionplan}[1]{\section{#1}%
    \begin{frame}[noframenumbering]{Содержание}
        \tableofcontents[currentsection]
    \end{frame}
}
\newcommand{\subsectionplan}[1]{\subsection{#1}%
    \begin{frame}[noframenumbering]{Содержание}
        \tableofcontents[currentsubsection]
    \end{frame}
}
% ------------------------------------------------

% Footnotes
\renewcommand{\thefootnote}{\arabic{footnote}}
\renewcommand{\thempfootnote}{\arabic{mpfootnote}}
% https://tex.stackexchange.com/questions/28465/multiple-footnotes-at-one-point
\usepackage{fnpct}
% ------------------------------------------------

% Links
% Colors also links on slide foot.
%\hypersetup{
%    colorlinks=true,
%    citecolor=blue,
%    linkcolor=blue,
%    urlcolor=blue
%}
% ------------------------------------------------

% Appendix
% Slide numbers
% https://tex.stackexchange.com/questions/70448/dont-count-backup-slides
\usepackage{appendixnumberbeamer}
\newcommand{\backupbegin}{
    \newcounter{framenumbervorappendix}
    \setcounter{framenumbervorappendix}{\value{framenumber}}
}
\newcommand{\backupend}{
    \addtocounter{framenumbervorappendix}{-\value{framenumber}}
    \addtocounter{framenumber}{\value{framenumbervorappendix}}
}
% ------------------------------------------------

% Custom commands
% * Decor
\newcommand{\newtopic}[0]{$+$} % item: new topic on "in previous series"
\newcommand{\then}{$\Rightarrow$} % item: consequences
\newcommand{\pop}[0]{\SunCloud} %item:  general eduation
\newcommand{\popslide}[0]{(\pop)}
\newcommand{\advanced}[0]{$\varhexstar$} % item: advanced science
\newcommand{\advancedslide}[0]{(\advanced)}
\newcommand{\practical}[0]{\dstechnical} % item: practical programming notions
\newcommand{\practicalslide}[0]{(\practical)}
\newcommand{\todo}[0]{todo} % item: question
\newcommand{\answer}[0]{\Lightning} % item: answer to the previous question
\newcommand{\eg}[0]{e.g.} % item: example
\newcommand{\defi}[0]{$\Delta$} % item: definition on smth
\newcommand{\textdefi}[1]{\textbf{#1}}
\newcommand{\positive}{$+$} % item: pros
\newcommand{\negative}{{\color{red} $-$}} % item: cons
\newcommand%! suppress = EscapeHashOutsideCommand
\NB[1][0.3]{N\kern-#1em{B}} % default kern amount: -0.3em
\renewcommand{\emph}[1]{{\color{blue} \textit{#1}}}
\newcommand{\vocab}[1]{\textbf{#1}} % item: important new word
% * Lambda calculi
\newcommand{\comb}[1]{\mathbf{#1}} % defined combinator
\newcommand{\term}[1]{\mathbf{#1}} % predefined lambda-term reference
\newcommand{\termdef}{\coloneqq} % lamda term binding
\newcommand{\step}{\rightsquigarrow} % reduction step
\newcommand{\sstep}{\twoheadrightarrow} % multiple steps reduction
\newcommand{\ap}{~} % lambda-term application
\newcommand{\subst}[3]{\left[#2 \mapsto #3 \right] #1} % substitution
\newcommand{\eqbeta}{=_\beta} % beta equality
\newcommand{\eqeta}{=_\eta} % eta-equality
\newcommand{\eqt}{=} % tree-equality of terms
\newcommand{\tlist}[1]{\term{[}#1\term{]}} % list-term
% * Legacy
%\newcommand{\err}[0]{\textcolor{red}{ошибка}} % compilation error

% ------------------------------------------------

% Speaker notes
% https://tex.stackexchange.com/questions/114219/add-notes-to-latex-beamer
% https://tex.stackexchange.com/questions/35444/split-beamer-notes-across-multiple-notes-pages/35496#35496
%\setbeameroption{show notes on second screen=right} % enable speaker notes
%--------------------------------------

\author[]{Андрей Стоян, Илья Колегов, Дмитрий Халанский}
\institute[MSE ITMO]{MSE ITMO}


\title[6. Классы типов]{Практика 6. Классы типов}
\date{осень 2025}

\begin{document}

    \setcounter{framenumber}{-1}
    \mymaketitle

    \begin{frame}[fragile]{В предыдущих сериях}
        \begin{itemize}
            \item Полиморфные функции в Haskell
            \item Структуры данных в Haskell, полиморфизм структур данных
%            \item[\NB] Стиль программирования с явной передачей иммутабельного состояния
%            \begin{equation*}
%                \framebox{$\sigma_1$} \to \sigma_2 \to \cdots \to \sigma_{k - 1} \to \framebox{$\sigma_k$}
%            \end{equation*}
            \item[\NB] Использование структур данных и newtype для контроля за выполнением инвариантов предметной области\footnote{\color{blue} \url{https://lexi-lambda.github.io/blog/2019/11/05/parse-don-t-validate/}}
            \begin{equation*}
                \framebox{\color{red}$\sigma_1$} {~\color{red} \rightarrow~} \sigma_2 {~\color{blue} \to~} \cdots {~\color{blue} \to~} \sigma_{k - 1} {~\color{blue} \to~} \framebox{$\sigma_k$}
            \end{equation*}
            \item[\newtopic] Классы типов
            \item[\newtopic] Некоторые стандартные классы типов языка Haskell
        \end{itemize}
    \end{frame}

    \begin{frame}[fragile]{Разминка}
        \begin{itemize}
            \item[\todo] Как определён \mintinline{haskell}|Either|?
            \item[\todo] Какой будет результат выполнения следующего выражения?
            \begin{minted}{haskell}
                case B 1 2 of
                  A x -> x + 3
                  B x y -> x + y
                  C -> 42
            \end{minted}
            \item[\todo] Приведите пример терма с типом \mintinline{haskell}|Maybe|
            \item[\todo] Какой тип у \mintinline{haskell}|Left 42|?
        \end{itemize}
    \end{frame}

    \begin{frame}[noframenumbering]{Содержание}
        \tableofcontents
    \end{frame}

    \sectionplan{Простые классы типов}

    \begin{frame}[fragile]{Классы типов как механизм перегрузки}
        \begin{itemize}
            \item \vocab{Перегрузка (overloading)} --- из функций с одинаковыми именами и разными типами выбирается подходящая
            \begin{minted}{cpp}
                string show(int x) { ... }
                string show(data d) { ... }
            \end{minted}
            \item \pause Классы типов требуют задекларировать перегружаемую функцию и варьируемый типовой параметр явно
            \begin{minted}{haskell}
                class Show #\framebox{a}# where
                  show :: #\framebox{a}# -> String
            \end{minted}
            \item Для каждого типа пишется по инстансу
        \end{itemize}
        \vspace{-1.2em}
        \begin{columns}[onlytextwidth]
            \begin{column}[t]{0.05\textwidth}
            \end{column}\hfill%
            \begin{column}[t]{0.485\textwidth}
                \begin{minted}{haskell}
                    instance Show #\framebox{Int}# where
                      show :: #\framebox{Int}# -> String
                      show x = ...
                \end{minted}
            \end{column}\hfill%
            \begin{column}[t]{0.485\textwidth}
                \begin{minted}{haskell}
                    instance Show #\framebox{Data}# where
                      show :: #\framebox{Data}# -> String
                      show d = ...
                \end{minted}
            \end{column}
        \end{columns}
        \vspace{0.5em}
        \begin{itemize}
            \item \pause В отличие от классической перегрузки, классы типов дружат с параметрическим полиморфизмом
            \begin{minted}{haskell}
                showWithPrefix :: Show a => a -> String -> String
            \end{minted}
        \end{itemize}
    \end{frame}

    \begin{frame}[fragile]{Задачки}
        \begin{itemize}
            \item[\todo] Реализуйте \mintinline{haskell}|Show| для следующего типа
            \begin{minted}{haskell}
                data LogLevel = Error | Warning | Debug | Info
            \end{minted}
            \item[\todo] Реализуйте \mintinline{haskell}|Show| для пары
            \item[\todo] Придумайте класс для парсинга данных \mintinline{haskell}|MyRead|
            \item[\answer] \pause
            \begin{minted}{haskell}
                instance (Show a, Show b) => Show (a, b) where
                  show (x, y) = "(" ++ show x ++ "," ++ show y ++ ")"
            \end{minted}
            \item[\answer] \pause
            \begin{minted}{haskell}
                class MyRead a where
                  readMaybe :: String -> Maybe a
                  -- варьируемый типовой параметр может использоваться где угодно
            \end{minted}
        \end{itemize}
    \end{frame}

    \begin{frame}[fragile]{Ещё задачки}
        \begin{itemize}
            \item[\todo] Реализуйте функцию \mintinline{haskell}|allValues|, перебирающую все значения типа
            \item[\todo] Реализуйте проверку равенства функций
            \item[\answer] \pause
            \begin{minted}{haskell}
                allValues :: (Enum a, Bound a) => [a]
                allValues = [minBound .. maxBound]
            \end{minted}
            \item[\answer] \pause
            \begin{minted}{haskell}
                instance (Enum a, Bound a, Eq b) => Eq (a -> b) where
                  (==) :: (a -> b) -> (a -> b) -> Bool
                  f == g = map f allValues == map g allValues
            \end{minted}
        \end{itemize}
    \end{frame}

    \sectionplan{Классы типов высших кайндов}

    \begin{frame}[fragile]{Классы высших кайндов и функторы}
        \begin{itemize}
            \item Рассмотренные ранее классы типов ожидают типы кайнда \mintinline{haskell}|*|\footnote{В современном Haskell вместо звёздочек принято использовать \mintinline{haskell}|Data.Kind.Type|.} % \footnote{С помощью расширения \href{https://ghc.gitlab.haskell.org/ghc/doc/users_guide/exts/kind_signatures.html}{\color{blue}KindSignatures} кайнды можно указывать явно.}
            \begin{minted}{haskell}
                class Ord (a :: *) where
                  less :: a -> a -> Bool
            \end{minted}
            \item Для абстрагирования по коллекциям нужны типы высших кайндов
            \begin{minted}{haskell}
                class Functor (f :: * -> *) where
                  fmap :: (a -> b) -> f a -> f b
            \end{minted}
            \item[\todo] Реализуйте \mintinline{haskell}|Functor| для \mintinline{haskell}|Maybe| и списков %\footnote{Явные в инстансах можно писать с помощью расширения \href{https://downloads.haskell.org/ghc/latest/docs/users_guide/exts/instances.html\#extension-InstanceSigs}{\color{blue}InstanceSigs}.}
            \item[\answer] \pause
            \begin{minted}{haskell}
                instance Functor Maybe where
                  fmap :: (a -> b) -> Maybe a -> Maybe b
                  fmap f mb = case mb of Nothing -> Nothing; Just x -> Just (f x)
                instance Functor [] where
                  fmap :: (a -> b) -> [] a -> [] b -- или [a], [b]
                  fmap = map
            \end{minted}
        \end{itemize}
    \end{frame}

    \begin{frame}[fragile]{Частичное применение типовых конструкторов}
        \begin{itemize}
            \item[\todo] \mintinline{haskell}|ghci> :k (,)|
            \item[\todo] \mintinline{haskell}|ghci> :k (->)|
            \item[\todo] \mintinline{haskell}|ghci> :k (,) Int|
            \item[\todo] Реализуйте функтор для пары
            \item[\answer] \pause \mintinline{haskell}|(,) :: * -> * -> *|
            \item[\answer] \pause \mintinline{haskell}|(->) :: * -> * -> *|
            \item[\answer] \pause \mintinline{haskell}|(,) Int :: * -> *|
            \item[\answer] \pause
            \begin{minted}{haskell}
                instance Functor ((,) e) where  -- (,) :: * -> * -> *
                  fmap :: (a -> b) -> ((,) e) a -> ((,) e) b
                  --   :: (a -> b) -> (e, a)    -> (e, b)
                  fmap f (w, x) = (w, f x)
            \end{minted}
        \end{itemize}
    \end{frame}

%    \sectionplan{Природа и возможности классов типов}

%    \begin{frame}[fragile]{Ad-hoc полиморфизм через типы-суммы vs классы типов}
%        \begin{itemize}
%            \item Паттерн-матчинг позволяет исполнять различный код на элементах типов-сумм
%            \item Классы типов
%            \begin{description}
%                \item[+альтернативу] Просто: завести для неё новый инстанс\footnote{Просто --- нужно внести модификацию в код локально, в одном месте, без перекомпиляции всего.}
%                \item[+функцию] Сложно: добавить реализацию в каждый инстанс\footnote{Сложно --- нужно внести модификации в код во множестве различных мест.}
%            \end{description}
%            \item Типы-суммы
%            \begin{description}
%                \item[+альтернативу] Сложно: дополнить паттерн-матчинг во всех функциях
%                \item[+функцию] Просто: завести функцию, разобрать все кейсы
%            \end{description}
%            \item {\color{blue}\url{https://en.wikipedia.org/wiki/Expression\_problem}}
%        \end{itemize}
%    \end{frame}

%    \begin{frame}[fragile]{Природа классов типов}
%        \begin{itemize}
%            \item \pause Классы типов --- не типы, нельзя \mintinline{haskell}|many :: [Functor]|\footnote{Гетерогенные списки можно делать с помощью экзистенциальных типов. Но это потом.}
%            \item \pause Тип принадлежит классу типов, если для него написан инстанс этого класса типов, удовлетворяющий необходимым законам этого класса
%            \item \pause Иногда полезно смотреть на классы типов как на предикаты на типах
%            \item[$\Rightarrow$] Мультипараметрические классы типов --- отношения
%        \end{itemize}
%    \end{frame}

    \sectionplan{Реализация классов типов в GHC \practicalslide}

    \begin{frame}[fragile]{Полиморфная сортировка: базовые классы типов}
        \begin{itemize}
            \item Хотим одной функцией сортировать списки элементов различных типов
            \begin{minted}{haskell}
                sort :: [a] -> [a]
            \end{minted}
            \item Нужно сравнивать элементы, по-разному для каждого типа
            \item[\then] Будем принимать функцию ``меньше'' для нужного типа
            \begin{minted}{haskell}
                sort :: (a -> a -> Bool) -> [a] -> [a]
            \end{minted}
            \item Потенциально функций может быть много, заведём рекорд
            \begin{minted}{haskell}
                data MyOrd a = MyOrd { less :: a -> a -> Bool }
                sort :: MyOrd a -> [a] -> [a]
            \end{minted}
            \item Заводим экземпляр для конкретного типа
            \begin{minted}{haskell}
                intMyOrd :: MyOrd Int
                intMyOrd = MyOrd { less = (<) }
            \end{minted}
            \item Вызов теперь выглядит так
            \begin{minted}{haskell}
                GHCi> sort intMyOrd [3, 2, 1]
                [1, 2, 3]
            \end{minted}
        \end{itemize}
    \end{frame}

    \begin{frame}[fragile]{Свойство упорядоченности}
        \vspace{-1.3em}
        \begin{columns}[onlytextwidth]
            \begin{column}[t]{0.485\textwidth}
                \begin{block}{Через \mintinline{haskell}|data|}
                    \begin{enumerate}
                        \item Определение словаря функций
                        \begin{minted}{haskell}
                        data MyOrd a = MyOrd
                            { less :: a -> a -> Bool }
                        \end{minted}
                        \item Экземпляр словаря для конкретного типа
                        \begin{itemize}
                            \item Именованное значение
                        \end{itemize}
                        \begin{minted}{haskell}
                        intMyOrd :: MyOrd Int
                        intMyOrd = MyOrd { less = (<) }
                        \end{minted}
                        \item Явный параметр функции
                        \begin{minted}{haskell}
                        sort :: MyOrd a -> [a] -> [a]
                        \end{minted}
                        \item Передаётся пользователем
                        \begin{minted}{haskell}
                        test = sort #\framebox{intMyOrd}# [3, 2, 1]
                        \end{minted}
                    \end{enumerate}
                \end{block}
            \end{column}\hfill%
            \begin{column}[t]{0.485\textwidth}
                \begin{block}{Через \mintinline{haskell}|class|}
                    \begin{enumerate}
                        \item Определение класса типов
                        \begin{minted}{haskell}
                         class MyOrd a where
                           less :: a -> a -> Bool
                        \end{minted}
                        \item Объявление типа представителем класса типов
                        \begin{itemize}
                            \item Не имеет имени
                        \end{itemize}
                        \begin{minted}{haskell}
                        instance MyOrd Int where
                          less = (<)
                        \end{minted}
                        \item Неявный параметр функции
                        \begin{minted}{haskell}
                        sort :: MyOrd a => [a] -> [a]
                        \end{minted}
                        \item Передаётся компилятором\footnote{Нужен импорт модуля, даже пустой, с определением нужного инстанса: \mintinline{haskell}|import Module ()|.}
                        \begin{minted}{haskell}
                        test = sort [3, 2, 1]
                        \end{minted}
                    \end{enumerate}
                \end{block}
            \end{column}
        \end{columns}
    \end{frame}

    \begin{frame}[fragile]{Сортировка полиморфных списков: расширение инстансов}
        \begin{itemize}
            \item Хотим объявить список упорядоченным, если упорядочены его элементы
            \item Должный завести функцию, которая даст реализацию для списка по реализации для его элементов:
            \begin{minted}{haskell}
                listMyOrd :: MyOrd a -> MyOrd [a]
                listMyOrd #\framebox{aOrd}# = MyOrd
                  { less = fix $ \rec xs ys -> case (xs, ys) of
                      -- ...
                      (x:xs, y:ys) | #\framebox{less aOrd}# x y = True
                      -- ...
                  }
            \end{minted}
            \item Теперь можем сортировать списки списков!
            \begin{minted}{haskell}
                test = sort (listMyOrd intMyOrd) [[3, 2], [], [1]]
            \end{minted}
        \end{itemize}
        \vspace{-1em}
        \begin{columns}[onlytextwidth]
            \begin{column}[t]{0.485\textwidth}
                \begin{minted}{haskell}
                    listMyOrd :: MyOrd a -> MyOrd [a]
                    listMyOrd aOrd = MyOrd { less = -- ...
                \end{minted}
            \end{column}\hfill%
            \begin{column}[t]{0.485\textwidth}
                \begin{minted}{haskell}
                    instance MyOrd a => MyOrd [a] where
                      less = -- ...
                \end{minted}
            \end{column}
        \end{columns}
        \vspace{0.5em}
        \begin{itemize}
            \item Всего один неявный параметр и механизм его вывода даёт столько могущества
            \item Однако вывод инстансов --- очень не простое дело на стороне компилятора
        \end{itemize}
    \end{frame}

    \begin{frame}[fragile]{Расширение классов типов}
        \begin{itemize}
            \item Иногда одно свойство влечёт другое как более слабое
            \item В Haskell инстанс расширенного класса типов  просто требует наличия инстанса базового
            \item Пример: частичный порядок (работает не для каждой пары элементов)
            \begin{minted}{haskell}
                class MyPartialOrd a where
                  mbLess :: a -> a -> Maybe Bool
            \end{minted}
            \item Частичный порядок обязателен для полного\footnote{Например, числа с плавающей точкой по стандарту IEEE 754 упорядочены только частично (NaN).}\footnote{В Haskell так не сделано, но сделано в Rust. Менее удобно, но более безопасно.}
            \begin{minted}{haskell}
                class MyPartialOrd a => MyOrd a where
                  less :: a -> a -> Bool
            \end{minted}
            \item При объявлении типа представителем базового класса типов пользоваться объявлением для расширенного
            \begin{minted}{haskell}
                instance MyPartialOrd Int where
                  mbLess = Just . #\framebox{less}#
            \end{minted}
        \end{itemize}
    \end{frame}

    \sectionplan{Материалы \popslide}

    \appendix

    \begin{frame}[fragile]{Похожие концепции в других языках \popslide}
        \begin{itemize}
            \item А ля классы типов, диспатч по нулевому аргументу функций
            \begin{itemize}
                \item[\eg] Rust traits
                \item[\eg] Swift protocols
                \item[\eg] Scala traits + implicits
            \end{itemize}
            \item Контроль за инстанциацией шаблонов
            \begin{itemize}
                \item[\eg] С++20 концепты
            \end{itemize}
            \item Неявные ресиверы
            \begin{itemize}
                \item[\eg] Kotlin context receivers
            \end{itemize}
            \item Шаблон visitor для динамического диспатча по нескольким аргументам одновременно
            \item Мультиметоды
            \begin{itemize}
                \item[\eg] Julia
            \end{itemize}
        \end{itemize}
    \end{frame}


    \begin{frame}{Полезные ссылки}
        \begin{itemize}
            \item \href{https://wiki.haskell.org/Typeclassopedia}{\color{blue} Typeclassopedia: шпаргалка по зоопарку классов типов}
        \end{itemize}
    \end{frame}

    \begin{frame}{Серьёзные материалы}
        \begin{itemize}
            \item \href{https://youtu.be/MbFqJ2NHS8M?si=LamIrSnrjbwCYrt2}{\color{blue}  Tagless-Final Style}
            \item \href{https://youtu.be/sWEtnq0ReZA?si=CxcESs7hoj73Kc7e}{\color{blue}  Олег Нижников — Современное ФП с Tagless Final}
            \item \href{https://youtu.be/p98W4bUtbO8?si=fU7dtXL3DnXEs75h}{\color{blue} The Death of Tagless Final by John A. De Goes}
            \item \href{https://reasonablypolymorphic.com/blog/review-codata/index.html}{\color{blue} Review: Codata in Action}
            \item \href{https://okmij.org/ftp/tagless-final/index.html}{\color{blue} Tagless Final style by Oleg}
            \item \href{https://okmij.org/ftp/tagless-final/course/lecture.pdf}{\color{blue} Lecture notes on Tagless Final}
            \item Wadler, Philip, and Stephen Blott. "How to make ad-hoc polymorphism less ad hoc." \textit{Proceedings of the 16th ACM SIGPLAN-SIGACT symposium on Principles of programming languages}. 1989.
        \end{itemize}
    \end{frame}

\end{document}
