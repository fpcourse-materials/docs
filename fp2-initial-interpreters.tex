%! suppress = MissingLabel

\subsubsection{Встроенные доменно-специфичные языки (eDSL)} \label{subsubsec:edsl}

Обсуждение терминологии и сравнение подходов к построению DSL можно найти в~\cite{gibbons2013functional}.
Краткое описание терминов --- в конспекте курса Language Engineering~\cite{languageEngineering}.

Под \vocab{доменно-специфичными языками (domain specific languages, DSL)}\footnote{\url{https://en.wikipedia.org/wiki/Domain-specific_language}} часто понимают специализированные языки для конкретных предметных областей, например, запросов к БД или форматирования документов.
Как правило, такие языки не являются полными по Тьюрингу.

В этом курсе, однако, мы будем считать доменно-специфичным языком любую доменно-специфичную специализацию языка общего назначения\footnote{\url{https://en.wikipedia.org/wiki/Language-oriented_programming}, на русскоязычную страницу тоже следует заглянуть.}.
Это следует из того соображения, что код должен читаться как грамотным проза с уместным словоупотреблением, предоставляющая читателю только необходимое количество подробностей, скрывая несущественное за умолчаниями и терминологией.
В извечной борьбе со сложностью, мы стремимся к такому коду, строя башню из DSL\@.

\vocab{Самостоятельные доменно-специфичные языки (standalone domain specific languages)} --- языки, имеющие свой собственный конкретный синтаксис, а так же инструменты программирования (IDE, исполняющая среда\ldots).
Примеры: SQL, AWK, Antlr\ldots

\vocab{Встроенные доменно-специфичные языки (embedded domain specific languages, eDSL)} --- языки, пользующиеся поддержкой инфраструктуры других языков.
Обычно реализуются как библиотеки для программ на уже существующем языке общего назначения.
Не имеют полностью собственного синтаксиса.
Примеры: ORM, функции обработки строк, библиотека парсер-комбинаторов\ldots

\vocab{Deep eDSL} --- термы на таком языке строят дерево абстрактного синтаксиса для дальнейшей интерпретации:
\begin{minted}{haskell}
    value :: Env -> Int
    value = eval $ Const 1 `Plus` Var "x"
\end{minted}

Можно заметить, что промежуточное дерево, которое получается, нас, как правило, не интересует.
Нам важно только получить элемент домена, которым мы уже умеем пользоваться непосредственно.
\vocab{Shallow eDSL} минуют стадию построения дерева и сразу строят значение в семантическом домене:
\begin{minted}{haskell}
    cnst :: Int -> (Env -> Int)
    cnst x _ = x

    var :: String -> (Env -> Int)
    var name env = env ! name

    plus :: (Env -> Int) -> (Env -> Int) -> (Env -> Int)
    plus l r env = l env + r env

    value :: Env -> Int
    value = cnst 1 `plus` var "x"
\end{minted}

Интерпретаторы часто называют \vocab{наблюдателями (observers)}, которые анализируют термы и дают им некоторый смысл~\cite{gibbons2013functional}.
Можно заметить, что для deep eDSL можно написать сколь угодно много различных наблюдателей.
Однако в случае shallow embedding наблюдатели всегда \texttt{id}.
Мы будем обсуждать возможные решения этой проблемы в разделе~\ref{sec:wonder-interpreters}.

Введём ещё одно важное понятие.
\vocab{Meta-circular интерпретатор}\footnote{\url{https://en.wikipedia.org/wiki/Meta-circular_evaluator}}~--- это интерпретатор, определяющий конструкции определяемого языка через конструкции мета-языка~\cite{reynolds1972definitional}.
Например:
\begin{minted}{haskell}
    interpret term = case term of
      App f t -> (interpret f) (interpret t)
      If c t e -> if interpret c then interpret t else interpret e
      ...
\end{minted}

Свойства мета-языка в таком случае во многом определяют свойства объектного~\cite{reynolds1972definitional,reynolds1998definitional}.
Мы будем в этом курсе стремиться как можно более переиспользовать возможности мета-языка.

\begin{task}
    Предположите, какие свойства наследует определяемый язык.
\end{task}

\subsubsection{Пример: библиотека Accelerate}

Интересным примером встроенного языка, находящегося где-то между deep и shallow является библиотека Accelerate\footnote{\url{https://hackage.haskell.org/package/accelerate}}~\cite[глава 6]{marlow2011parallel}.
Она позволяет на Haskell описать вычисления, которые будут исполняться на GPU\footnote{Другой подход: \href{https://youtu.be/6c0DB2kwF_Q?si=-nB7AkCsDWB_Q-hy}{Java code reflection}, чтобы в рантайме извлекать модель кода. Однако, такой подход не предоставляет статически гарантий программисту и требует глубогоко внедрения в мета-язык.}.

Чтобы исполнить что-то на GPU нужно породить и скомпилировать код на Cuda.
Таким образом, Accelerate должен быть deep embedding, чтобы иметь дерево вычисления, чтобы его транслировать в Cuda наиболее эффективным образом.

В то же время описывать численные вычисления как дерево крайне неудобно.
Неплохо было бы иметь привычные операторы и функции высших порядков для работы с массивами на GPU\@.
Поэтому Accelerate предоставляет на самом деле shallow интерфейс для построения деревьев.

Так, для деревьев выражений определена реализация численных классов типов, например, \mintinline{haskell}|Num|, где операции просто достраивают дерево.
Функции высших порядков реализованы примерно с помощью техники, которую мы будем обсуждать в~\ref{subsec:first-class-functions}.

% todo investigate actual HOF implementation

\subsection{Реализация интерпретаторов}

Итак, мы определили, что интерпретаторы --- это основа основ, потому что главная задача программиста --- борьба со сложностью, а главный инструмент этой борьбы --- использование подходящих доменно-специфичных языков, позволяющих думать только о важном, абстрагируя несущественные детали.

В течение этого курса мы будем учиться писать интерпретаторы.
В этом разделе --- классические, которые принимают деревья на вход и интерпретируют их в семантический домен.
Такие интерпретаторы задают deep eDSL\@.
Далее мы научимся миновать этап построения дерева определяемого языка, напрямую расширяя мета-язык новыми конструкциями (раздел~\ref{sec:wonder-interpreters}).
И наконец, постараемся приблизиться к священного Граалю этой науки~--- расширяемым интерпретаторам (раздел~\ref{sec:effect-handlers}).

``Классические'' интерпретаторы часто называют \vocab{инициальными} интерпретаторами.
Слово ``инициальный'' тут относится к тому, что мы имеем дело с инициальным объектом категории интерпретаций.
Всё что нам нужно понимать, --- что инициальные интерпретаторы работают с деревом программы, заданным классически с помощью \mintinline{haskell}|data| (да, деревья можно задавать и по-другому).

Есть отличная книга~\cite{nystrom2021crafting}, рассказывающая о построении классических интерпретаторов и простых виртуальных машин.
В то же время она покрывает сознание полноценного языка во всех его аспектах, от синтаксического анализа до управления памятью.

\subsubsection{Untyped tagless interpreters}

Для начала рассмотрим некоторый тривиальный нетипизироватный язык.
Под нетипизированностью понимаем отсутствие проверки типов как до исполнения программы, так и во время.
Абстрактный синтаксис этого языка зададим следующим образом:
\begin{minted}{haskell}
    data Term = Const Int | IsZero Term | If Term Term Term
\end{minted}

Значения, возникающие во время исполнения программ на этом языке будем представлять значениями типов \mintinline{haskell}|Bool| и \mintinline{haskell}|Int| языка Haskell\footnote{На самом деле эти значения сами по себе в Haskell несут типовую информацию в runtime, но пока опустим эту деталь для простоты.}.
Соответственно, семантическим доменом программы на этом языке является либо \mintinline{haskell}|Bool|, либо \mintinline{haskell}|Int|, в зависимости от самой программы.
\begin{minted}{haskell}
    interpretUnsafe :: forall res . Term -> res
    interpretUnsafe term = case term of
      Const val -> unsafeCoerce val
      IsZero cond -> unsafeCoerce $ interpretUnsafe @Int cond == 0
      If c t e -> if interpretUnsafe c then interpretUnsafe t else interpretUnsafe e
\end{minted}

Здесь \texttt{unsafeCoerce} используется, чтобы обмануть статическую систему типов Haskell и просто исполнять программы на нашем нетипизированном языке.
Неверное написание программы на этом языке или выбор неправильного домена интерпретации приводят к падению программы.

\subsubsection{Typed tagged interpreters}

Чтобы добиться некоторой безопасности исполнения, будем приписывать значениям некоторые теги, которые будут доступны во время исполнения.
Заведём следующий алгебраический тип:
\begin{minted}{haskell}
    data RtValue = RtBool Bool | RtInt Int
\end{minted}

Теперь семантическим доменов у нас будет тип \mintinline{haskell}|RtValue|, а интерпретатор сможет проверять типы во время исполнения:
\begin{minted}{haskell}
    interpretRt :: Term -> RtValue
    interpretRt (IsZero term) = case eval term of
      RtBool value -> error "Type error"
      RtInt value -> RtBool (value == 0)
    -- ...
\end{minted}

Ситуация с безопасностью программы определённо стала лучше, однако проверка типов во время исполнения~--- это уже поздно: требует дополнительных расходов производительности и удорожает тестирование.

Этот подход часто называют динамической типизацей, когда мы атрибутируем значения некоторой типовой информацией для использования во время исполнения.

\subsubsection{Equality coercions, GADTs} \label{subsubsec:gadts}

Как мы знаем, конструкторы данных в Haskell --- это обычные функции с той лишь разницей, что их реализация генерируется компилятором (аллокация памяти, размещение полей\ldots).
У функций есть сигнатура.
Например, \mintinline{haskell}|IsZero :: Term -> Term|.

В Haskell есть синтаксис определения \mintinline{haskell}|data| через задание типов конструкторов.
Он совершенно аналогичен рассмотренному ранее, только гораздо более удобен для сложно организованных структур данных.
Рассмотренный ранее тип термов \mintinline{haskell}|Term| будет выглядеть следующим образом:
\begin{minted}{haskell}
    data Term where
      Const :: Int -> Term
      IsZero :: Term -> Term
      If :: Term -> Term -> Term -> Term
\end{minted}

Современный Haskell является синтаксически богатым языком, который, однако, несмотря не многообразие конструкций, транслируется в маленький типизированный внутренний язык.
Это язык $System~F_C$, основанный на $System~F$.
Он описан в работе~\cite{sulzmann2007system}.

$System~F_C$ расширяет $System~F$ наличием сложных несинтаксических эквивалентностей типов.
Оказывается, этого достаточно, чтобы поддержать такие возможности Haskell как обобщённые алгебраические типы, ассоциированные семейства типов, функциональные зависимости и т.д.

Синтаксический сахар для типовых эквивалентностей выглядит следующим образом\footnote{Чтобы воспользоваться эквивалентностями и GADT, нужно подключить расширение \href{https://downloads.haskell.org/~ghc/9.0.1/docs/html/users_guide/exts/gadt.html}{GADTs}.}\footnote{\url{https://ghc.gitlab.haskell.org/ghc/doc/users_guide/exts/equality_constraints.html}}:
\begin{minted}{haskell}
    f :: forall a b . a ?$\sim$? b => a -> b
    f = id
\end{minted}
На самом деле это функция от четырёх параметров: двух типовых параметров, коерции, аргумента.
Коерция --- это тип, автоматически выводимый компилятором, который является свидетельством того, что два типа, записанный в кайнде этого типа, эквивалентны.
\begin{minted}{haskell}
    f :: forall (a :: *) (b :: *) . forall (co :: a ?$\sim$? b) . a -> b
\end{minted}

Воспользуемся типовой эквивалентностью следующим образом.
Добавим типу термов фантомный типовой параметр, который будет маркировать тип результата терма.
\begin{minted}{haskell}
    data Term (ty :: Type) where
      Const :: forall ty . ty ?$\sim$? Int => Int -> Term ty
      IsZero :: forall ty . ty ?$\sim$? Bool => Term Int -> Term ty
      If :: forall ty . Term Bool -> Term ty -> Term ty -> Term ty
\end{minted}
Или можно воспользоваться синтаксическим сахаром, который и называется \vocab{обобщёнными алгебраическими типами данных (generalized algebraic data types)}, чтобы скрыть явные эквивалентности:
\begin{minted}{haskell}
    data Term (ty :: Type) where
      Const :: Int -> Term Int
      IsZero :: Term Int -> Term Bool
      If :: Term Bool -> Term ty -> Term ty -> Term ty
\end{minted}

Теперь мы можем написать безопасный типизированный интерпретатор.
Обратите внимание, что при сопоставлении с образцами конструкторов, в скоуп попадают и коерции, которые использовались при применении соответствующих конструкторов.
\begin{minted}{haskell}
    interpret :: Term ty -> ty
    interpret = \case
      Const x  -> x              -- ty ?$\sim$? Int
      IsZero t -> interpret == 0 -- ty ?$\sim$? Bool
      If c t e -> if interpret c then interpret t else interpret e
\end{minted}

Заметим, что в структуре данных типа \mintinline{haskell}|Term ty| не хранятся значения типа \texttt{ty}, а этот типовой параметр существует только для обеспечения большей типовой безопасности.
Такие типовые параметры называют \vocab{фантомными}\footnote{\url{https://wiki.haskell.org/Phantom_type}}.

\subsubsection{Typed tagless initial interpreters}

Опишем систему типов нашего маленького языка.
%! suppress = EscapeAmpersand
\begin{equation*}{}
    \infer[Const]{Const~n : int}{n : Int}
    \quad
    \infer[IsZero]{IsZero~n : bool}{n : int}
    \quad
    \infer[If]{If~c~t~e : \tau}{c : bool & t : \tau & e : \tau}
\end{equation*}
Можем в качестве типовых тегов переиспользовать типы Haskell: \[int \rightsquigarrow Int, bool \rightsquigarrow Bool\]

Заметим, что с помощью обобщённого алгебраического типа данных \mintinline{haskell}|Term ty| выше, мы как раз закодировали эти правила вывода.
Иначе говоря, мы получили типизированный язык программирования, переиспользовав систему типов Haskell.

Таким образом, мы бесплатно получили статически типизированный язык, не нуждающийся в типовых тегах времени исполнения (отсюда слово tagless).

% todo https://jesper.cx/posts/1001-syntax-representations.html

\subsection{Функции первого класса} \label{subsec:first-class-functions}

В этом параграфе мы рассмотрим техники и понятия, относящиеся к реализации функций первого класса.
Эти понятия оказываются крайне полезны и продуктивны, как мы увидим далее.

Те же рассуждения справедливы и для описания \mintinline{haskell}|let|-связываний, поскольку их можно представлять следующим образом:
\[
    \term{let} \ap x \termdef N \ap \term{in} \ap M \equiv (\lambda x\ldotp M) \ap N
\]

\subsubsection{Связывание имён}

\vocab{Динамическое связывание (dynamic scoping)} --- значение свободных переменных функции зависит от области видимости в месте вызова.
То есть разрешение имени происходит в момент обращения к переменной.
Например, следующий код напечатает \texttt{42}:
\begin{minted}{kotlin}
    val f = {
        val x = 4
        fun inner() = x + 1
        inner
    }
    val x = 41
    println(f())
\end{minted}

Этот подход проще в реализации и использовался в ранних версиях Lisp'ов, например.
Однако в таком случае функции не являются надежным барьером абстракции, по-хорошему все свободные переменные должны являться частью сигнатуры (вернёмся в этому в~\ref{sec:effect-systems}).

\vocab{Лексическое/статическое связывание (lexical/static scoping)} --- переменные связываются со значениями в момент объявления функции, в момент вызова результат зависит только от параметров (по модулю изменяемого состояния\footnote{Например, в Kotlin в лямбды можно захватывать изменяемые переменные. Изменения снаружи наблюдаемы внутри лямбды, и наоборот. Иногда это может быть очень удобно, однако нередко приводит к очень неочевидному поведению.}).
Слово ``лексический'' часто употребляется в языках, когда мы что-то можем понять из исходного кода без запуска программы.
Так, код из примера выше напечатает 5.

Далее в этом разделе мы будем говорить о различных способах реализации функций первого класса со статическим связыванием переменных.

\subsubsection{Подстановки}

Как можно заметить, в классическом лямбда-исчислении подстановки от бета-редукции (вспоминали в разделе~\ref{subsec:terms-reduction}) обеспечивают статическое связывание.
Действительно, аргумент немедленно подставляется во все вхождения переменной, соответственно она не остаётся свободной, а просто исчезает.
\[
    (\lambda x\ldotp (\lambda x\ldotp \lambda y\ldotp x + y) \ap 4) \ap 41 \rightsquigarrow (\lambda x\ldotp (\lambda y\ldotp 4 + y)) \ap 41
\]

Такой подход не является самым эффективным, потому что на каждую аппликацию требуется переписывать код функции (!).
В то же время его довольно просто реализовать для некоторых представлений лямбда-термов.
Рассмотрим пример такого представления --- \vocab{locally nameless}~\cite{chargueraud2012locally}.

\begin{minted}{haskell}
    data Term var
      = Var var
      | App (Term var) (Term var)
      | Lam (Term (Maybe var))
\end{minted}

В этом представлении можно выбирать любой тип для именования свободных переменных:
\begin{minted}{haskell}
    example :: Term String
    example = Var "x" `App` Var "y" -- x y
\end{minted}
Добавление каждой связанной переменной добавляет типу переменных нового обитателя \mintinline{haskell}|Nothing| для обращения к ближайшей связанной переменной:
\begin{minted}{haskell}
    -- ?$\lambda x\ldotp x \ap y$?
    example1 = Lam $ Nothing `App` Just "y"
    -- ?$\lambda x \ap y\ldotp x \ap y \ap z$?
    example2 = Lam $ Lam $ Just Nothing `App` Nothing `App` Just (Just "z")
\end{minted}

Удивительно, но монадический bind является реализацией подстановки для таких термов.

\begin{minted}{haskell}
    instance Monad Term where
      (>>=) :: Term var -> (var -> Term var') -> Term var'
      Var var >>= subst = subst var
      App l r >>= subst = App (l >>= subst) (r >>= subst)
      Lam t >>= subst = Lam $ t >>= \case
        Nothing  -> Var Nothing
        Just var -> Just <$> subst var
\end{minted}

\begin{task}
    Подумайте, зачем нужен \texttt{fmap Just} в последней строчке.
\end{task}

Соответственно, call-by-name интерпретатор такого лямбда-исчисления будет выглядеть следующим образом:

\begin{minted}{haskell}
    eval :: Term var -> Term var
    eval = \case
      Var var -> Var var
      App f arg -> case eval f of
        Lam body -> eval $ body >>= maybe arg Var
        t -> App t (eval arg)
      Lam t -> Lam (eval t)
\end{minted}

Можно заметить, что эта реализация не самая эффективная, потому что мы делаем каждое применение функции лишним проходом по терму.

\subsubsection{Окружение}

Можно делать подстановку значений переменных лениво, распространяя окружение, которое ставит в соответствие свободным переменным термы.

\begin{minted}{haskell}
    data Term1 = Var1 String | App1 Term1 Term1 | Lam1 String Term1
    type Env = Map String Term1

    eval1 :: Term1 -> (Env -> Term1)
    eval1 = \case
      Var1 name -> \env -> Map.findWithDefault (Var1 name) name env
      App1 f arg -> \env -> case eval1 f env of
        Lam1 name body -> eval1 body (Map.insert name arg env)
        t -> App1 t (eval1 arg env)
      Lam1 name body -> \env ->
        let env' = Map.delete name env in
        Lam1 name (eval1 body env')
\end{minted}

%! suppress = UnresolvedReference
\begin{task}
    Объясните, зачем окружение модифицируется на строчке~10?
\end{task}

Если ветка \mintinline{haskell}|Lam1| не будет рекурсивно обходить подтерм и подставлять значения переменных, информация о значениях свободных переменных в нём потеряется и мы получим динамическое связывание вместо статического.
Чтобы восстановить статическое связывание, ветка \mintinline{haskell}|Lam1| интерпретатора должна конструировать замыкание, включающее текущее окружение (см. далее~\ref{subsubsec:closures}).

\subsubsection{Верифицированный контекст}

Рассмотрим кодирование, описанное, например, в~\cite{kiselyov2012typed}.

Для начала научимся с помощью системы типов Haskell проверять валидность обращения к окружению.
Представим окружение как список типов, закодированный с помощью вложенных пар:
\begin{minted}{haskell}
    (4, (4.0, "hello")) :: (Int, (Double, String))
\end{minted}

Обращение к окружению будем кодировать числом в унарной записи.
Тип числа (типизированной ссылки внутрь контекста) пусть задаёт множество окружений, из которых на такой позиции можно извлечь нужный тип.
\begin{minted}{haskell}
    data Ref env ty where
      Here :: Ref (ty, env) ty
      There :: Ref env ty -> Ref (ty', env) ty
\end{minted}
Например, тип числа 1 утверждает, что с его помощью можно извлечь значение типа \texttt{ty} из контекста, в котором значение соответствующего типа находится на первой позиции:
\begin{minted}{haskell}
    There Here :: Ref (ty', (ty, env)) ty
\end{minted}

Теперь мы можем закодировать типизированное безопасное обращение к контексту:
\begin{minted}{haskell}
    envLookup :: env -> Ref env ty -> ty
    envLookup env ref = case (ref, env) of
      (Here, (x, _)) -> x
      (There ref', (_, env')) -> envLookup env' ref'
\end{minted}

\begin{task}
    Можно ли разобрать пару сразу на строчке 2?
    Поясните.
\end{task}

\subsubsection{Meta-circular интерпретация}

Крайне не хотелось бы для eDSL самостоятельно реализовывать связывания и функции первого класса.
Построим meta-circular tagless interpreter, который будет переиспользовать функции первого класса мета-языка для реализации их в определяемом языке.

Термы теперь будут не только аннотированы результирующими типами, но и типами необходимых для интерпретации окружений.
Абстрагированному терму доступно большее окружение.

\begin{minted}{haskell}
    data Term2 env ty where
      Var2 :: Ref env ty -> Term2 env ty
      App2 :: Term2 env (arg -> res) -> Term2 env arg -> Term2 env res
      Lam2 :: Term2 (arg, env) res -> Term2 env (arg -> res)
\end{minted}

Теперь абстракцию можем проинтерпретировать в функцию Haskell, а аппликацию --- в аппликацию:
\begin{minted}{haskell}
    eval2 :: Term2 env ty -> env -> ty
    eval2 term env = case term of
      Var2 ref -> env `envLookup` ref
      App2 f arg -> (eval2 f env) (eval2 arg env)
      Lam2 t -> \arg -> eval2 t (arg, env)
\end{minted}

\begin{task}
    Как так получилось, что в последней строчке нужно принять ещё один аргумент?
\end{task}

\begin{task}
    Это call-by-value интерпретатор или call-by-name?
    От чего это зависит?
\end{task}

\begin{task}
    Подумайте, какое решение должно быть более производительно, это или предыдущее?
\end{task}

\subsubsection{Синтаксис высшего порядка} \label{subsubsec:h-syntax}

Ещё чем мы ещё занимаемся вручную --- определяем связыватели (да ещё и в унарной записи).
Хотим переиспользовать языковые идентификаторы.
Для это мы будем прямо в дереве синтаксиса хранить функции мета-языка --- использовать \vocab{синтаксис высшего порядка (higher order abstract syntax)}\footnote{\url{https://en.wikipedia.org/wiki/Higher-order_abstract_syntax}}\footnote{\href{https://cstheory.stackexchange.com/questions/20071/what-is-higher-order-in-higher-order-abstract-syntax}{What is higher-order in higher-order abstract syntax?}}~\cite{pfenning1988higher}.

Теперь мы ссылаемся не на переменные, а на значения, нода абстракции содержит честную функцию:
\begin{minted}{haskell}
    data Term3 ty where
      Val3 :: ty -> Term3 ty
      Plus :: Term3 Int -> Term3 Int -> Term3 Int
      App3 :: Term3 (arg -> res) -> Term3 arg -> Term3 res
      Lam3 :: (Term3 arg -> Term3 res) -> Term3 (arg -> res)

    example3 :: Term3 Int
    example3 = (Lam3 \x -> x `Plus` Val3 41) `App3` Val3 1
\end{minted}

Интерпретация очень простая и абсолютно meta-circular:
\begin{minted}{haskell}
    eval3 :: Term3 ty -> ty
    eval3 term = case term of
      Val3 x -> x
      Plus l r -> eval3 l + eval3 r
      App3 f arg -> (eval3 f) (eval3 arg)
      Lam3 f -> \arg -> eval3 (f (Val3 arg))
\end{minted}

\begin{task}
    Можно ли было объявить \mintinline{haskell}|Lam3| следующим образом?
    \begin{minted}{haskell}
        Lam3 :: (arg -> Term3 res) -> Term3 (arg -> res)
    \end{minted}
\end{task}

\subsubsection{Замыкания} \label{subsubsec:closures}

% todo https://www.youtube.com/watch?v=TcxT0EcA--4&list=PLxMpIvWUjaJv16ZvPEZrwYjBp-6aUIoDL&index=10

Мы можем свести работу с вложенными функциями к работе с глобальными с помощью \vocab{lambda lifting}: вместо каждой вложенной функции заводим по глобальной.
Если вложенная функция имела свободные переменные, добавляем глобальной функции аргумент контекста содержащего значения для этих переменных.
\begin{minted}{kotlin}
    fun f(x) {
        fun g(y) = x + y
        return g
    }
    // lambda lifting породит декларацию
    fun g_glob(ctx: GCtx, y) = ctx.x + y
\end{minted}

Теперь мы можем функциональные объекты представлять как пару из контекста и указателя на глобальную функцию --- то есть как \vocab{замыкание (closure)}\footnote{\url{https://en.wikipedia.org/wiki/Closure_(computer_programming)}}\footnote{Термин closure был предложен Piter Landin, вместе с кучей других вещей.}.
Плавный переход от интерпретаторов с окружением к замыканиям можно посмотреть в гарвардских слайдах~\cite{closures-slides}.
Простую и понятную реализацию для интерпретаторов --- в книжке~\cite[глава 11]{nystrom2021crafting}.

\subsubsection{Дефункционализация} \label{subsubsec:defunctionalization}

\vocab{Дефункционализация (defunctionalization)} --- техника избавление от функций высших порядков в программе\footnote{\url{https://en.wikipedia.org/wiki/Defunctionalization}}~\cite{defunctionalization-slides}.
Впервые предложена в работе~\cite{reynolds1972definitional, reynolds1998definitional}.

Идея заключается в том, чтобы заменить каждую лямбда-функцию вызовом конструктора некоторого алгебраического типа данных.
А каждый call-cite функции заменить на вызов специальной first-order функции \texttt{apply}, интерпретирующей данный алгебраический тип.
Рассмотрим пример.

\begin{minted}{haskell}
    example = (\x -> isRed x) $ (\b -> mkColor 10 50 b) 120
    -- перепишется на
    data Fun = IsRed | MkColor r g
    apply IsRed x = isRed x
    apply (MkColor r g) b -> mkColor r g b
    example = apply IsRed $ apply (MkColor 10 50) 120
\end{minted}

\subsubsection{Сериализация}

В этом разделе мы говорили о возможных реализациях функций первого класса, то есть функций, которые можно использовать так же гибко, как и данные.
Возникает закономерный вопрос: можем ли мы сериализовать функцию первого класса и послать исполняться на другую машину?

Функция состоит из кода и захваченных свободных переменных в случае статического связывания.
Соответственно, если код представлен в сериализуемом виде (например, позиционно-независимый байт-код), то его в принципе можно переслать по сети и исполнить на другом инстансе виртуальной машины.
Так, например, делает Erlang.
Однако, такой подход невероятно неэффективный, так как байт-код нужно интерпретировать.
Таким образом, Erlang жертвует скоростью исполнения ради горизонтальной масштабируемости.

Существуют забавные работы, которые вместо сериализации функции и отправки её на сервер, отправляют некоторый хендл, по которому сервер может вызвать эту функцию на клиенте и получить результат.
Такая разновидность RPC с поддержкой функций высших порядков\footnote{\url{https://github.com/winter-yuki/LambdaRPC.kt}}.

Если мы гарантируем, что на различных узлах кластера исполняется один и тот же кодЮ, как обычно и бывает на практике, можно добиться более эффективной реализации.
Например, используя дефункционализацию~(см.~\ref{subsubsec:defunctionalization}), мы можем сериализовать только объекты алгебраического типа, кодирующие функции.
Поскольку на другом узле кластера исполняется такой же код, мы там можем десериализовать объект и исполнить его с помощью \texttt{apply}.
Однако этот подход не очень поддерживает модульность (сложно один алгебраический тип разбить на много, вернёмся к этой задаче в разделе~\ref{sec:effect-handlers}), а так же \texttt{apply} каждый раз производит декодирование перед исполнением кода (чем, в прочем, можно пренебречь, учитывая работу с сетью).

Подход, реализованный в Haskell\footnote{\url{https://blog.ocharles.org.uk/blog/guest-posts/2014-12-23-static-pointers.html}}\footnote{\url{https://hackage.haskell.org/package/distributed-closure}} позволяет наделить каждую функцию без свободных переменных некоторым статически известным адресом, одинаковым для всех инстансов приложения.
Далее можно сконструировать сериализуемое замыкание путём последовательности частичных применений:
\begin{minted}{haskell}
    data Closure a where
      StaticPtr :: StaticPtr b -> Closure b
      Encoded :: ByteString -> Closure ByteString
      Ap :: Closure (b -> c) -> Closure b -> Closure c

    main = send "some-node" $
      closure (static factorial) `closureAp` closurePure 10
\end{minted}

Подробнее можно прочитать в основополагающей статье про облачный Haskell~\cite{epstein2011towards}.
С практической точки зрения --- в книжке~\cite[глава 16]{marlow2011parallel}.

\begin{task}
    Нужно ли явно добавлять в замыкание свободную переменную \texttt{(*)} (оператор умножения) в реализации факториала?
\end{task}
