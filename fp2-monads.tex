%! suppress = MissingLabel

Язык может давать возможность описывать чистые вычисления, предоставляя новый предметно-ориентированный словарный запас.
Однако реальные программы не получится описывать только чистыми вычислениями.
Поэтому нужно иметь возможность создавать языки для описания effectful вычислений.
Об эффектах мы начинаем говорить в этой главе.

\subsection{Понятие эффекта}

Начнём разговор про эффекты апофатично, с того, чем они не являются.
\vocab{Чистое вычисления} --- единственный его наблюдаемый результат --- его значение.
\vocab{Чистая функция} --- результат зависит только от значения аргументов и её аппликация --- чистое вычисление.

Программирование чистыми функциями считается хорошей практикой, так как этот стиль обладает большим количеством хороших свойств.
Так, композиция чистых функций --- чистая функция; всё, что нужно для понимания кода, явно написано в этом коде; системы типов хорошо работают, предоставляя чистоту абстракции, документацию и частичную спецификацию.

Однако, всё чистым кодом записать не получится, несмотря на все эти свойства и то, что даже ввод-вывод так можно описать~\cite{jones2001tackling}.
% todo
\begin{minted}{haskell}
    getList :: Int -> World -> (World, [Int])
    getList n w | n == 0 = (w, [])
                | otherwise =
      let (w', x) = getInt w in
      fmap (x :) (getList (n - 1) w')
\end{minted}

Нужно делегировать весь этот bookkeeping сторонней сущности, чтобы она занималась этим за нас, то есть замести неинтересное под ковёр. % todo

\begin{minted}{haskell}
        getList :: Int -> IO [Int]
        getList n | n == 0 = pure []
                  | otherwise = do
          x <- getInt
          fmap (x :) (getList (n - 1))
\end{minted}

Такой сущностью является интерпретатор языка.
А сам эффект является взаимодействием с контекстом исполнения.
А поскольку мы делаем shallow-embedded встроенные языки для переиспользования максимума мета-языка, нужны высокоуровневые конструкции для конструирования сложных денотаций.

% todo

TODO недетерминизм и matter of perspective\footnote{\url{https://okmij.org/ftp/Computation/having-effect.html}} % todo

% todo semantic domains have monad structure

\subsection{Монады для моделирования эффектов}

\vocab{Монада} --- это тройка $(m, \eta : a \to m a, \mu : m (m a) \to m a)$\footnote{\url{https://ncatlab.org/nlab/show/monad+(in+computer+science)}}.
% todo diagrams

Монады для структурирования семантики~\cite{moggi1988computational}, монады как обобщение list comprehension~\cite{wadler1990comprehending}, monads to structure functional programs~\cite{wadler1992essence} % todo

<<Computations of type $\alpha$ correspond to values of type $f\ap\alpha$>>, Moggi's principle. % todo

% todo lalalang

% todo оригинальная статья

% todo replicateM

\subsection{Трансформеры монад и модульные интерпретаторы}

% todo remind of tagless final

% todo lift - monad morphism

% todo ContT, lift = (>>=)

TODO\cite{liang1995monad} % todo

\subsection{Представления монад}

% todo Monads and composable continuations by Phill Wadler

% todo Representing Monads Filinsky
% todo paper delimited continuations in WebAssembly

% todo связь с континуэйшенами

% todo mother of all monads

% todo monads in python with nice syntax

% todo monadic reflection & direct style (lib from scala)
% todo https://github.com/lampepfl/monadic-reflection/blob/main/TUTORIAL.md

% todo reflection without remorse

% todo C# expression trees
% todo F# что-то про монады и forM https://learn.microsoft.com/en-us/dotnet/fsharp/language-reference/computation-expressions

% todo https://blog.poisson.chat/posts/2019-10-27-continuation-submonads.htm

% todo https://www.unison-lang.org/docs/fundamentals/abilities/for-monadically-inclined/

% todo https://www.schoolofhaskell.com/school/to-infinity-and-beyond/pick-of-the-week/the-mother-of-all-monads

% todo http://www.valuedlessons.com/2008/01/monads-in-python-with-nice-syntax.html

\subsection{Свободные монады}

% todo Stackless Scala with free monads

% todo free monads, freer monads

% todo iteratee O. Kiselyov. Iteratees. In Proc. of the 11th International Symposium on Functional and Logic Programming, pages 166–181, 2012.

% todo Monads and algebras

% todo codensity, reflection without remorse
