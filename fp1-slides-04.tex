%! suppress = Quote
%! suppress = MissingImport
%! suppress = MissingLabel
%! suppress = LineBreak

% CLI args https://tex.stackexchange.com/a/1501
\newif\ifhandout
\input{flags}

%! suppress = MissingLabel
%! suppress = DocumentclassNotInRoot
%! suppress = DiscouragedUseOfDef

% * Make friends tikz & colors
%   https://en.wikibooks.org/wiki/LaTeX/Colors
% * To enable vertical top alignment globally
%   https://tex.stackexchange.com/questions/9889/positioning-content-at-the-top-of-a-beamer-slide-by-default
% * Set handout from CLI
%   https://tex.stackexchange.com/a/1501
\ifhandout
\documentclass[usenames, dvipsnames, handout]{beamer} % https://tex.stackexchange.com/questions/224091/beamer-how-to-disable-pause-temporarily
\else
\documentclass[usenames, dvipsnames]{beamer}
\fi
% ------------------------------------------------

% Graphics
\usepackage{color}
\usepackage{tabularx}
\usepackage{tikz}
% https://tikz.dev/tikz-graphs
\usetikzlibrary{positioning, shapes.geometric, arrows, automata, graphs}
\tikzset{
    expr/.style={ellipse, draw=gray!60, fill=gray!5, very thick, minimum size=7mm, yshift=0.7cm},
    hexpr/.style={ellipse, draw=gray!60, fill=blue!15, very thick, minimum size=7mm, yshift=0.7cm},
    stmt/.style={rectangle, draw=gray!60, fill=gray!5, very thick, minimum size=5mm, yshift=0.7cm},
    decl/.style={rectangle, draw=blue!60, fill=gray!5, very thick, minimum size=5mm, yshift=0.7cm},
    hdecl/.style={rectangle, draw=blue!60, fill=blue!15, very thick, minimum size=5mm, yshift=0.7cm},
    subtree/.style={shape border rotate=90, isosceles triangle, draw=gray!60, fill=gray!5, very thick, minimum size=5mm, yshift=0.0cm},
}
\usepackage{blkarray}
\usepackage{graphicx}
\usepackage{forest} % https://tex.stackexchange.com/questions/198405/how-to-change-the-color-of-subtrees-in-tikz-qtree
% ------------------------------------------------

% Math
\usepackage{amsmath, amsfonts}
\usepackage{amssymb}
\usepackage{proof}
\usepackage{mathrsfs}
% Crossed-out symbols
% https://tex.stackexchange.com/questions/75525/how-to-write-crossed-out-math-in-latex
\usepackage[makeroom]{cancel}
\usepackage{mathtools}
% ------------------------------------------------

% Additional font sizes
% https://www.overleaf.com/learn/latex/Questions/How_do_I_adjust_the_font_size%3F
\usepackage{moresize}
% Additional colors
% https://www.overleaf.com/learn/latex/Using_colours_in_LaTeX
\usepackage{xcolor}
% Textual math symbols
\usepackage{textcomp}
% ------------------------------------------------

% Language
\usepackage[utf8] {inputenc}
\usepackage[T2A] {fontenc}
\usepackage[english, russian] {babel}
\usepackage{indentfirst, verbatim}
\usetikzlibrary{cd, babel}
% ------------------------------------------------

% Fonts: https://sites.math.washington.edu/~reu/docs/latex_symbols.pdf
\usepackage{stmaryrd}
\usepackage{cmbright}
\usepackage{wasysym}
\usepackage[weather]{ifsym} % https://tex.stackexchange.com/questions/100424/how-to-use-the-ifsym-package
% https://tex.stackexchange.com/questions/615300/pdflatex-builtin-glyph-names-is-empty
\pdfmapline{=dictsym DictSym <dictsym.pfb}
\pdfmapline{=pigpen <pigpen.pfa}
\usepackage{dictsym}
% ------------------------------------------------

% Code
% * Needs -shell-escape build flag
%   https://tex.stackexchange.com/questions/99475/how-to-invoke-latex-with-the-shell-escape-flag-in-texstudio-former-texmakerx
% * Set build directory
%   https://tex.stackexchange.com/questions/339931/latex-minted-package-using-custom-output-directory-build
\usepackage{minted}
\setminted{xleftmargin=\parindent, autogobble, escapeinside=\#\#}
% ------------------------------------------------

% Template
\usetheme{CambridgeUS}
\usecolortheme{dolphin}
% https://tex.stackexchange.com/questions/231439/beamer-how-to-make-font-larger-for-page-numbers
\setbeamerfont{headline}{size=\scriptsize}
\setbeamerfont{footline}{size=\scriptsize}
% Remove heddline
% https://tex.stackexchange.com/questions/33146/how-could-i-remove-a-header-in-a-beamer-presentation
%\setbeamertemplate{headline}{}
% Slide sizes
% https://tex.stackexchange.com/questions/56768/how-to-set-a-small-default-font-size-with-beamer
%\geometry{paperwidth=140mm,paperheight=105mm} % 4:3
\geometry{paperwidth=168mm,paperheight=105mm} % 16:10
% Remove navigation bar
% https://stackoverflow.com/questions/3210205/how-to-get-rid-of-navigation-bars-in-beamer
\beamertemplatenavigationsymbolsempty
% ------------------------------------------------

% Bullets
% https://9to5science.com/change-bullet-style-formatting-in-beamer
% https://tex.stackexchange.com/questions/185742/i-need-to-change-color-of-beamer-itemize-and-subitem-separately
\setbeamertemplate{itemize item}{\scriptsize\raise1.25pt\hbox{\donotcoloroutermaths$\blacktriangleright$}}
\setbeamertemplate{itemize subitem}{\scriptsize\raise1.5pt\hbox{\donotcoloroutermaths$\blacktriangleright$}}
\setbeamertemplate{itemize subsubitem}{\tiny\raise1.5pt\hbox{\donotcoloroutermaths$\blacktriangleright$}}
\setbeamertemplate{enumerate item}{\insertenumlabel.}
\setbeamertemplate{enumerate subitem}{\insertenumlabel.\insertsubenumlabel}
\setbeamertemplate{enumerate subsubitem}{\insertenumlabel.\insertsubenumlabel.\insertsubsubenumlabel}
% ------------------------------------------------

% Table of contents format
% https://tex.stackexchange.com/questions/642927/format-table-of-contents-in-beamer
\setbeamertemplate{section in toc}{%
        {\color{blue}\inserttocsectionnumber.}
    \inserttocsection\par%
}
\setbeamertemplate{subsection in toc}{%
        {\color{blue}\hspace{1em}\scriptsize\raise1.25pt\hbox{\donotcoloroutermaths$\blacktriangleright$}}
    \inserttocsubsection\par%
}
\setbeamertemplate{subsubsection in toc}{%
        {\color{blue}\hspace{2em}\tiny\raise1.25pt\hbox{\donotcoloroutermaths$\blacktriangleright$}}
    \inserttocsubsubsection\par%
}
% ------------------------------------------------

% Misc
\usepackage{multicol}
\usepackage{hyperref}
\usepackage{soul} % https://tex.stackexchange.com/questions/23711/strikethrough-text
% ------------------------------------------------

% Fix \pause for amsmath package envs (black black magic)
% https://tex.stackexchange.com/questions/16186/equation-numbering-problems-in-amsmath-environments-with-pause/75550#75550
% https://tex.stackexchange.com/questions/6348/problem-with-beamers-pause-in-alignments
%! suppress = Makeatletter
\makeatletter
\let\save@measuring@true\measuring@true
\def\measuring@true{%
    \save@measuring@true
    \def\beamer@sortzero##1{\beamer@ifnextcharospec{\beamer@sortzeroread{##1}}{}}%
    \def\beamer@sortzeroread##1<##2>{}%
    \def\beamer@finalnospec{}%
}
%! suppress = Makeatletter
\makeatother
% ------------------------------------------------

% Sections
\newcommand{\sectionplan}[1]{\section{#1}%
    \begin{frame}[noframenumbering]{Содержание}
        \tableofcontents[currentsection]
    \end{frame}
}
\newcommand{\subsectionplan}[1]{\subsection{#1}%
    \begin{frame}[noframenumbering]{Содержание}
        \tableofcontents[currentsubsection]
    \end{frame}
}
% ------------------------------------------------

% Footnotes
\renewcommand{\thefootnote}{\arabic{footnote}}
\renewcommand{\thempfootnote}{\arabic{mpfootnote}}
% https://tex.stackexchange.com/questions/28465/multiple-footnotes-at-one-point
\usepackage{fnpct}
% ------------------------------------------------

% Links
% Colors also links on slide foot.
%\hypersetup{
%    colorlinks=true,
%    citecolor=blue,
%    linkcolor=blue,
%    urlcolor=blue
%}
% ------------------------------------------------

% Appendix
% Slide numbers
% https://tex.stackexchange.com/questions/70448/dont-count-backup-slides
\usepackage{appendixnumberbeamer}
\newcommand{\backupbegin}{
    \newcounter{framenumbervorappendix}
    \setcounter{framenumbervorappendix}{\value{framenumber}}
}
\newcommand{\backupend}{
    \addtocounter{framenumbervorappendix}{-\value{framenumber}}
    \addtocounter{framenumber}{\value{framenumbervorappendix}}
}
% ------------------------------------------------

% Custom commands
% * Decor
\newcommand{\newtopic}[0]{$+$} % item: new topic on "in previous series"
\newcommand{\then}{$\Rightarrow$} % item: consequences
\newcommand{\pop}[0]{\SunCloud} %item:  general eduation
\newcommand{\popslide}[0]{(\pop)}
\newcommand{\advanced}[0]{$\varhexstar$} % item: advanced science
\newcommand{\advancedslide}[0]{(\advanced)}
\newcommand{\practical}[0]{\dstechnical} % item: practical programming notions
\newcommand{\practicalslide}[0]{(\practical)}
\newcommand{\todo}[0]{todo} % item: question
\newcommand{\answer}[0]{\Lightning} % item: answer to the previous question
\newcommand{\eg}[0]{e.g.} % item: example
\newcommand{\defi}[0]{$\Delta$} % item: definition on smth
\newcommand{\textdefi}[1]{\textbf{#1}}
\newcommand{\positive}{$+$} % item: pros
\newcommand{\negative}{{\color{red} $-$}} % item: cons
\newcommand%! suppress = EscapeHashOutsideCommand
\NB[1][0.3]{N\kern-#1em{B}} % default kern amount: -0.3em
\renewcommand{\emph}[1]{{\color{blue} \textit{#1}}}
\newcommand{\vocab}[1]{\textbf{#1}} % item: important new word
% * Lambda calculi
\newcommand{\comb}[1]{\mathbf{#1}} % defined combinator
\newcommand{\term}[1]{\mathbf{#1}} % predefined lambda-term reference
\newcommand{\termdef}{\coloneqq} % lamda term binding
\newcommand{\step}{\rightsquigarrow} % reduction step
\newcommand{\sstep}{\twoheadrightarrow} % multiple steps reduction
\newcommand{\ap}{~} % lambda-term application
\newcommand{\subst}[3]{\left[#2 \mapsto #3 \right] #1} % substitution
\newcommand{\eqbeta}{=_\beta} % beta equality
\newcommand{\eqeta}{=_\eta} % eta-equality
\newcommand{\eqt}{=} % tree-equality of terms
\newcommand{\tlist}[1]{\term{[}#1\term{]}} % list-term
% * Legacy
%\newcommand{\err}[0]{\textcolor{red}{ошибка}} % compilation error

% ------------------------------------------------

% Speaker notes
% https://tex.stackexchange.com/questions/114219/add-notes-to-latex-beamer
% https://tex.stackexchange.com/questions/35444/split-beamer-notes-across-multiple-notes-pages/35496#35496
%\setbeameroption{show notes on second screen=right} % enable speaker notes
%--------------------------------------

\author[]{Андрей Стоян, Илья Колегов, Дмитрий Халанский}
\institute[MSE ITMO]{MSE ITMO}


\title[Введение в Haskell]{Практика 4. Введение в Haskell}
\date{осень 2024}

\begin{document}

    \setcounter{framenumber}{-1}
    \maketitle

    \begin{frame}{В предыдущих сериях}
        \begin{itemize}
            \item Синтаксис и семантика лямбда-исчисления
            \item Структуры данных в чистом лямбда-исчислении
            \item Просто-типизированное лямбда-исчисление
            \item Стиль программирования с явной передачей обновлённого состояния вместо изменения старого
            \item[\newtopic] Базовый синтаксис Haskell
            \item[\newtopic] Базовые типы
            \item[\newtopic] Группировка объявлений: \mintinline{haskell}|let| и \mintinline{haskell}|where|
            \item[\newtopic] Операторы и сечения
        \end{itemize}
    \end{frame}

    \begin{frame}[noframenumbering]{Содержание}
        \tableofcontents
    \end{frame}

    \sectionplan{Введение}

    \begin{frame}{Как изучать Haskell}
        \begin{block}{Почему Haskell}
            \begin{itemize}
                \item Писать не функционально максимально неудобно
                \item Наружу вывернуты концепции и абстракции в ущерб простоте и эргономике
            \end{itemize}
        \end{block}
        \pause
        \begin{block}{\color{red} Как неправильно}
            \begin{itemize}
                \item Принимать на веру лекции и практики
                \item Пытаться по примерам и типам нащупать, как этим пользоваться
                \item Привычка от изучения языков с магическими правилами\footnote{\href{https://docs.python.org/3/reference/datamodel.html\#special-method-names}{\color{blue} Python special method names.}}
            \end{itemize}
        \end{block}
        \pause
        \begin{block}{\color{green} Как правильно}
            \begin{itemize}
                \item Открыть определение и почитать
                \item Поредуцировать руками простые примеры
                \item Лучший способ осознать что-то из ФП~--- реализовать то же самое самостоятельно
            \end{itemize}
        \end{block}
    \end{frame}

    \sectionplan{Рекурсия и аккумуляторы}

    \begin{frame}[fragile]{Задачки на рекурсию}
        \begin{itemize}
            \item[\todo] Реализуйте факториал
            \item[\todo] Реализуйте примитивную рекурсию
            \item[\answer] \pause
            \item[] % workaround because of columns
            \begin{columns}[onlytextwidth]
                \begin{column}{0.485\textwidth}
                    \begin{block}{Математика}
                        \[
                            n! =
                            \begin{cases}
                                1, n \le 1 \\
                                n * (n - 1)!, otherwise
                            \end{cases}
                        \]
                    \end{block}
                \end{column}\hfill%
                \begin{column}{0.485\textwidth}
                    \begin{block}{Haskell}
                        \begin{minted}{haskell}
                        fac n
                          | n <= 1 = 1
                          | otherwise = n * fac (n - 1)
                        \end{minted}
                    \end{block}
                \end{column}
            \end{columns}
            \item[\answer] \pause
            \begin{minted}{haskell}
                rec :: (Int -> a -> a) -> a -> Int -> a
                rec f ini n = if n <= 0 then ini else f (n - 1) (rec f ini (n - 1))
            \end{minted}
        \end{itemize}
    \end{frame}

    \begin{frame}[fragile]{Техника аккумулирующего параметра\footnote{\href{https://youtu.be/8gnhaE2nmQ0?si=A8lDF9pJqa_07x9q}{\color{blue} \advanced Continuation-passing style, defunctionalization, and associativity.}}}
        \begin{itemize}
            \item[\todo] Какая сложность по памяти у рекурсивного факториала?
            \item[\answer] \pause Линейная:
            \begin{minted}{haskell}
                fac 4 -> 4 * fac 3 -> 4 * (3 * fac 2)
                -> 4 * (3 * (2 * fac 1)) -> 4 * (3 * (2 * 1)) -> 24
            \end{minted}
            \item Будем накапливать промежуточные результаты в параметре-аккумуляторе
            \begin{minted}{haskell}
                fac' n = go n (n - 1)
                  where
                    go acc n'
                      | n' <= 0 = acc
                      | otherwise = go (acc * n') (n' - 1)
            \end{minted}
            \item[\todo] Проредуцируйте пример \mintinline{haskell}|fac' 4|
            \item[\answer] \pause
            \begin{minted}{haskell}
                fac' 4 -> go 4 3 -> go (4 * 3) 2 -> go ((4 * 3) * 2) 1
                -> go (((4 * 3) * 2) * 1) 0 -> ((4 * 3) * 2) * 1 -> 24
            \end{minted}
            \item Выражения можно сокращать сразу! Константа по памяти
        \end{itemize}
    \end{frame}

    \begin{frame}[fragile]{Злополучный Фибоначчи}
        \begin{itemize}
            \item Реализация по определению имеет экспоненциальную сложность
            \item[\todo] Реализуйте эффективное вычисление n-го числа Фибоначчи
            \item[\answer] \pause
            \begin{minted}{haskell}
                fib :: Int -> Int
                fib n = go 1 1 n
                  where
                    go prev curr n
                      | n <= 1 = prev
                      | otherwise = go curr (prev + curr) (n - 1)
            \end{minted}
        \end{itemize}
    \end{frame}

    \sectionplan{Полиморфные функции}

    \begin{frame}[fragile]{\secname}
        \begin{itemize}
            \item Типы с маленькой буквы --- типовые переменные
            \item Могут принимать разные значения в зависимости от контекста
            \item[\todo] \mintinline{haskell}|ghci> :t id|
            \item[\todo] \mintinline{haskell}|ghci> :t id "something here"|
            \item[\todo] \mintinline{haskell}|ghci> :t const|
            \item[\todo] \mintinline{haskell}|ghci> :t const id|
            \item[\todo] \mintinline{haskell}|ghci> :t flip const|
            \item[\todo] \mintinline{haskell}|ghci> :t :t uncurry (flip const)|
            \item[\answer] \pause \mintinline{haskell}|id :: a -> a|
            \item[\answer] \pause \mintinline{haskell}|id "something here" :: String|
            \item[\answer] \pause \mintinline{haskell}|const :: a -> b -> a|
            \item[\answer] \pause \mintinline{haskell}|const id :: b -> a -> a|
            \item[\answer] \pause \mintinline{haskell}|flip const :: b -> a -> a|
            \item[\answer] \pause \mintinline{haskell}|uncurry (flip const) :: (b, a) -> a|
        \end{itemize}
    \end{frame}
    
    \begin{frame}[fragile]{Конструирование реализаций}
        \begin{minted}{haskell}
            ($) :: (a -> b) -> a -> b
            (&) :: a -> (a -> b) -> b
            (.) :: (b -> c) -> (a -> b) -> a -> c
            f :: Int -> a -> a
            g :: Int -> Double
            h :: ((), Double) -> Int
        \end{minted}
        \begin{itemize}
            \item[\todo] Запишите через доллар \mintinline{haskell}|h x = f 1 (g (h ((), x)))|
            \item[\todo] Запишите через композицию
            \item[\todo] Запишите в бесточечном стиле
            \item[\answer] \pause
            \begin{minted}{haskell}
                example x = f 1 $ g $ curry h () x
                example x = (f 1 . g . curry h ()) x
                example = f 1 . g . curry h ()
            \end{minted}
        \end{itemize}
    \end{frame}

    \begin{frame}[fragile]{Больше полиморфных функций\footnote{\color{blue}\url{http://conal.net/blog/posts/semantic-editor-combinators}}}
        \begin{itemize}
            \item[\todo] Реализуйте функцию, применяющую функцию к первой компоненте пары
            \item[\todo] Прибавьте единицу к первой компоненте \mintinline{haskell}|((Int, a), b)|
            \item[\answer] \pause
            \begin{minted}{haskell}
                first :: (a -> a') -> (a, b) -> (a', b)
                first f (x, y) = (f x, y)
            \end{minted}
            \item[\answer] \pause
            \begin{minted}{haskell}
                first . first :: (a -> a') -> ((a, b), c) -> ((a', b), c)
                (first . first) (+1) value :: ((Int, b), c)
            \end{minted}
        \end{itemize}
    \end{frame}

    \sectionplan{System F \advancedslide}

    \begin{frame}{Полиморфные функции}
        \vspace{-0.5em}
        \begin{itemize}
            \item[\advanced] То, что типизация остаётся валидной после подстановки в тип~--- теорема метатеории
            \vspace{-1.3em}
            \begin{columns}[onlytextwidth]
                \begin{column}[t]{0.05\textwidth}
                \end{column}\hfill%
                \begin{column}[t]{0.95\textwidth}
                    \begin{theorem}[Лемма подстановки типа для $\lambda_{\rightarrow}$]
                        $\Gamma\vdash M : \sigma \Rightarrow [\alpha\mapsto\tau]\Gamma\vdash \hspace{3.25em}M : [\alpha\mapsto\tau]\sigma$ {\color{blue} ($\lambda_{\rightarrow}$ по Карри)}
                        \\
                        $\Gamma\vdash M : \sigma \Rightarrow [\alpha\mapsto\tau]\Gamma\vdash {\color{red} [\alpha\mapsto\tau]}M : [\alpha\mapsto\tau]\sigma$ {\color{red} ($\lambda_{\rightarrow}$ по Чёрчу)}
                    \end{theorem}
                \end{column}
            \end{columns}
            \vspace{0.3em}
            \item Если мы в языке разрешим приписывать типы термам, мы потеряем гибкость:
            \[
                \begin{array}{l}
                    \comb{K} : \alpha \to \beta \to \alpha \\
                    \comb{K} \ap \term{1} : {\color{red}\text{cannot match } \alpha \text{ with } \term{Int}}
                \end{array}
            \]
            \item Введём в теорию абстракцию на типах и аппликацию термов к типам!
            \[
                \begin{array}{l}
                    \comb{K} : \forall \alpha \ap \beta \ldotp \alpha \to \beta \to \alpha
                    \\
                    \comb{K} \ap @\term{Int} : \forall \beta \ldotp \term{Int} \to \beta \to \term{Int}
                    \\
                    \comb{K} \ap @\term{Int} \ap @\term{Float} \ap \term{1} : \term{Float} \to \term{Int}
                \end{array}
            \]
            \item Теперь выбор конкретного типа происходит на стороне пользователя функции
            \item Такие функции называются \textbf{полиморфными} --- умеют работать с различными типами
        \end{itemize}
    \end{frame}

    \begin{frame}[fragile]{Полиморфные функции в Haskell \advancedslide}
        \begin{itemize}
            \item Haskell неявно приписывает \texttt{forall} по всем типовым переменным:
            \begin{minted}{haskell}
                ghci> :set -fprint-explicit-foralls
                ghci> :t const
                const :: forall {a} {b}. a -> b -> a
            \end{minted}
            \item Типовые переменные всегда начинаются с маленькой буквы (\mintinline{haskell}|a|), в отличие от типовых констант (\mintinline{haskell}|Int| --- конкретный тип)
            \item Типовые применения компилятор выставляет автоматически неявно
            \item Иногда бывает полезно типовые применения указывать вручную (\texttt{TypeApplications})
            \begin{itemize}
                \item
                \begin{minted}{haskell}
                    ghci> :set -XTypeApplications
                    ghci> :t const @Int
                    const @Int :: forall {b}. Int -> b -> Int
                \end{minted}
                \item В промышленных языках синтаксис примерно такой:
                \begin{minted}{kotlin}
                    fun <A, B> const(x: A, y: B): A
                    const<Int, Float>(1, 2)
                \end{minted}
            \end{itemize}
        \end{itemize}
    \end{frame}

    \begin{frame}[fragile]{Higher ranked types \advancedslide}
        \begin{itemize}
            \item Бывает полезно в функции высшего порядка уметь задавать типы для функции-параметра:
            \begin{minted}{haskell}
                fst :: forall a b. (#\framebox{forall c}#. (a -> b -> c) -> c) -> a
                fst lambdaPair = lambdaPair #\framebox{@a}# const
            \end{minted}
            \item Требуется расширение компилятора \texttt{RankNTypes}
            \item Вывод типов становится неразрешим, компилятору требуются явные подсказки в виде типовых аннотаций и аппликаций
        \end{itemize}
    \end{frame}

    \sectionplan{Инфраструктура Haskell \practicalslide}

    \begin{frame}{Компилятор GHC}
        \begin{itemize}
            \item Единственный компилятор Haskell
            \item Работает не быстро (зато умный)
            \item Версия языка зафиксирована в стандарте Haskell Report 2010
            \item Язык развивается расширениями компилятора
            \begin{itemize}
                \item Подключение прагмой в начале файла \texttt{\{-\# LANGUAGE ... \#-\}}
                \item Или ключами компиляции (в конфигах проекта)
                \item Добавляют синтаксический сахар (\texttt{LambdaCase}, \texttt{ImplicitParams})
                \item Обогащают систему типов (\texttt{RankNTypes})
                \item Добавляют фичи (\texttt{MultiParamTypeClasses}, \texttt{GADTs})
                \item Добавляют экспериментальные фичи (\texttt{LinearTypes})
            \end{itemize}
            \item \href{https://ghc.gitlab.haskell.org/ghc/doc/users_guide/index.html}{\color{blue} https://ghc.gitlab.haskell.org/ghc/doc/users\_guide/index.html}
        \end{itemize}
    \end{frame}

    \begin{frame}{Системы сборки}
        \begin{itemize}
            \item По конфигурации проекта правильно вызывают компилятор
        \end{itemize}
        \begin{description}
            \item[Cabal] Common Architecture for Building Applications and Libraries
            \begin{itemize}
                \item Система сборки + библиотека
            \end{itemize}
            \item[Stack] Более новая система, надстройка над cabal
            \begin{itemize}
                \item Сама скачает компилятор нужной версии
                \item \texttt{stack new my-project; cd my-project}
                \item \texttt{stack run} --- запустить \texttt{main}
                \item \texttt{stack build} --- собрать проект
                \item \texttt{stack test} --- запустить тесты
                \item \texttt{stack ghci} --- REPL с загруженным проектом
            \end{itemize}
        \end{description}
    \end{frame}

    \begin{frame}[fragile]{REPL, GHCi}
        \begin{block}{REPL (read-eval-print-loop)}
            Интерактивное окружение, исполняющее вводимые в него высказывания и тут же демонстрирующее результат.
        \end{block}
        \begin{itemize}
            \item Компилятор работает долго, а GHCi --- быстро
            \item Пишем код, загружаем в интерпретатор: \texttt{stack ghci}
            \item Основные возможности
            \begin{description}
                \item Исполнить выражение
                \item[:r (reload)] Перезагрузить интерпретатор
                \item[:t (type)] Посмотреть тип выражения
                \item[:i (info)] Посмотреть документацию функции
                \item[:set] Установить расширения языка и флаги компиляции
            \end{description}
        \end{itemize}
    \end{frame}

    \begin{frame}[fragile]{Типовые дыры в Haskell}
        В определении функции можно ставить подчеркивания:
        \begin{minted}{haskell}
            plus' :: Int -> Int -> Int
            plus' x y = x + _
        \end{minted}
        \vspace{1em}
        \begin{minted}{haskell}
            Found hole: _ :: Int
            In the second argument of (+), namely _
            In the expression: x + _
            In an equation for plus': plus' x y = x + _
            Relevant bindings include
            y :: Int
              (bound at /home/user/fp/examples/src/Lib.hs:9:9)
            x :: Int
              (bound at /home/user/fp/examples/src/Lib.hs:9:7)
            plus' :: Int -> Int -> Int
              (bound at /home/user/fp/examples/src/Lib.hs:9:1)
        \end{minted}
    \end{frame}

    \begin{frame}[fragile]{Другие тулы}
        \begin{itemize}
            \item Линтер (\texttt{hlint}) помогает находить ошибки в программах
            \item Очень помогает в обучении
            \item[\NB] На этом курсе обязателен к использованию!
            \item Форматтер автоматически форматирует код: \texttt{brittany}, \texttt{ormolu}\ldots
            \item Обязателен к использованию во многих языках: экономит время и повышает консистентность кода
            \item В Haskell не всегда удовлетворительно работает
        \end{itemize}
    \end{frame}

    \begin{frame}{LSP}
        \begin{itemize}
            \item LSP ~--- Language Server Protocol ~--- это протокол взаимодействия между движками IDE и текстовыми редакторами
            \item Есть language server для большинства языков
            \item \href{https://github.com/haskell/haskell-language-server/releases}{\color{blue}Есть для Haskell}, работает \emph{только} для той же версии ghc
            \item Большинство редакторов кода поддерживает LSP и с его помощью почти превращается в IDE
        \end{itemize}
        \begin{block}{IDE для Haskell на примере vscode}
            \begin{itemize}
                \item Устанавливаете в систему \texttt{haskell-language-server}
                \item Устанавливаете в систему \texttt{hlint}
                \item Ставите плагины: \href{https://marketplace.visualstudio.com/items?itemName=haskell.haskell}{\color{blue} Haskell (может поставить и остальные)}, \href{https://marketplace.visualstudio.com/items?itemName=justusadam.language-haskell}{\color{blue} Haskell Syntax},
                \href{https://marketplace.visualstudio.com/items?itemName=hoovercj.haskell-linter}{\color{blue} haskell-linter}
            \end{itemize}
        \end{block}
    \end{frame}

    \sectionplan{Материалы}

    \begin{frame}{Что посмотреть в транспорте}
        \begin{itemize}
            \item \href{https://youtu.be/n3H_YipBDrY?si=s1_yS-4r8JcZiuUi}{\color{blue} Vitaly Bragilevsky - The clear path to Haskell complexities}
            \item \href{https://youtu.be/sGFdbpGamWk}{\color{blue} Podlodka \#211 – Haskell | Александр Вершилов}
            \item \href{https://youtu.be/ndfLLz-wehE?si=A53s_VYg-UWJVWkW}{\color{blue}  Property-based testing в Haskell} --- так тестируются наши домашки
        \end{itemize}
    \end{frame}

    \begin{frame}{Полезные ссылки}
        \begin{itemize}
            \item \href{https://hoogle.haskell.org/}{\color{blue} Hoogle} --- типизированный поиск по документации библиотек
            \item \href{https://hackage.haskell.org/package/base-4.18.1.0/docs/Debug-Trace.html}{\color{blue} Отладка принтами: Debug.Trace}
            \item \href{https://ghc.gitlab.haskell.org/ghc/doc/users_guide/index.html}{\color{blue} GHC docs} --- про компилятор и расширения языка
            \item \href{https://wiki.haskell.org/Typeclassopedia}{\color{blue} Typeclassopedia}
            \item \href{https://docs.haskellstack.org/en/stable/}{\color{blue} Stack build system}
            \item \href{https://gitlab.haskell.org/ghc/ghc/-/wikis/commentary/rts/haskell-execution}{\color{blue} The Haskell Execution Model}
            \item \href{https://gitlab.haskell.org/ghc/ghc/-/wikis/commentary/compiler/generated-code}{\color{blue} I know kung fu: learning STG by example}
        \end{itemize}
    \end{frame}

    \begin{frame}{Серьёзные материалы}
        \begin{itemize}
            \item \href{https://www.amazon.com/Get-Programming-Haskell-Will-Kurt/dp/1617293768}{\color{blue} Will Kurt ''Get Programming with Haskell``}
            \item Две части курса Дениса Николаевича на степике: \href{https://stepik.org/course/75/syllabus}{\color{blue} раз} и \href{https://stepik.org/course/693/syllabus}{\color{blue} два}
            \item \href{https://www.youtube.com/playlist?list=PL-_cKNuVAYAVX_q9XOKoFm95234G6YfOj}{\color{blue} Лекции Дениса Николаевича на МКН}
            \item \href{https://www.amazon.com/Haskell-Depth-Vitaly-Bragilevsky-ebook/dp/B098C6ZFZR}{\color{blue} Vitaly Bragilevsky ''Haskell in depth``}
            \item \href{https://www.youtube.com/watch?v=DS0dgYVnHy4&list=PLvPsfYrGz3wtQeNnms4dKnqqguoEb5TPx}{\color{blue} Компилятор GHC языка Haskell: теория языков программирования в работе}
            \item \href{https://www.amazon.com/Parallel-Concurrent-Programming-Haskell-Multithreaded/dp/1449335942}{\color{blue} Simon Marlow ''Parallel and Concurrent Programming in Haskell: Techniques for Multicore and Multithreaded Programming``}
        \end{itemize}
    \end{frame}

\end{document}
