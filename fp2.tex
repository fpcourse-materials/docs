%! suppress = Makeatletter
%! suppress = TooLargeSection
%! suppress = MissingLabel
\documentclass[12pt]{article}

% Fields
\usepackage{geometry}
\geometry{top=25mm}
\geometry{bottom=35mm}
\geometry{left=20mm}
\geometry{right=20mm}
% ------------------------------------------------

% Graphics
\usepackage{color}
\usepackage{tabularx}
\usepackage{tikz}
% https://tikz.dev/tikz-graphs
\usetikzlibrary{positioning, shapes.geometric, arrows, automata, graphs}
\tikzset{
    expr/.style={ellipse, draw=gray!60, fill=gray!5, very thick, minimum size=7mm, yshift=0.7cm},
    hexpr/.style={ellipse, draw=gray!60, fill=blue!15, very thick, minimum size=7mm, yshift=0.7cm},
    stmt/.style={rectangle, draw=gray!60, fill=gray!5, very thick, minimum size=5mm, yshift=0.7cm},
    decl/.style={rectangle, draw=blue!60, fill=gray!5, very thick, minimum size=5mm, yshift=0.7cm},
    hdecl/.style={rectangle, draw=blue!60, fill=blue!15, very thick, minimum size=5mm, yshift=0.7cm},
    subtree/.style={shape border rotate=90, isosceles triangle, draw=gray!60, fill=gray!5, very thick, minimum size=5mm, yshift=0.0cm},
}
\usepackage{blkarray}
\usepackage{graphicx}
\usepackage{forest} % https://tex.stackexchange.com/questions/198405/how-to-change-the-color-of-subtrees-in-tikz-qtree
% ------------------------------------------------

% Math
\usepackage{amsmath, amsfonts}
\usepackage{amssymb}
\usepackage{proof}
\usepackage{mathrsfs}
% Crossed-out symbols
% https://tex.stackexchange.com/questions/75525/how-to-write-crossed-out-math-in-latex
\usepackage[makeroom]{cancel}
\usepackage{mathtools}
% ------------------------------------------------

% Additional font sizes
% https://www.overleaf.com/learn/latex/Questions/How_do_I_adjust_the_font_size%3F
\usepackage{moresize}
% Additional colors
% https://www.overleaf.com/learn/latex/Using_colours_in_LaTeX
\usepackage{xcolor}
% \texttimes
\usepackage{textcomp}
% ------------------------------------------------

% Language
\usepackage[utf8] {inputenc}
\usepackage[T2A] {fontenc}
\usepackage[english, russian] {babel}
\usepackage{indentfirst, verbatim}
\usetikzlibrary{cd, babel}
% ------------------------------------------------

% Fonts
\usepackage{stmaryrd}
\usepackage{cmbright}
\usepackage{wasysym}
% ------------------------------------------------

% Code
% https://tex.stackexchange.com/questions/99475/how-to-invoke-latex-with-the-shell-escape-flag-in-texstudio-former-texmakerx
% Colored, requires --shell-escape compiling option
\usepackage{minted}
\setminted{xleftmargin=\parindent, autogobble, escapeinside=\#\#}
\usepackage{listings}
% ------------------------------------------------

% Custom commands
\newcommand{\vocab}[1]{\textbf{#1}} % new word for vocabulary
\newcommand{\point}[1]{{\color{blue}\textit{#1}}} % point out something in text
\newcommand{\sembr}[1]{\llbracket{#1}\rrbracket} % semantic brackets
\newcommand{\comb}[1]{\mathbf{#1}}
\newcommand{\step}{\rightsquigarrow}
\newcommand{\term}[1]{\mathbf{#1}}
\newcommand{\ap}{~}
\newcommand{\termdef}{\coloneqq}
\newcommand{\subst}[3]{\left[#2 \mapsto #3 \right] #1}
\newcommand{\eqbeta}{=_\beta}
\newcommand{\eqeta}{=_\eta}
% ------------------------------------------------

% Head
%\usepackage{fancybox,fancyhdr} not compatible with minted
\usepackage{hyperref}
%\pagestyle{fancy}
% ------------------------------------------------

% Bibliography
\usepackage{csquotes}
% https://tex.stackexchange.com/questions/3802/how-to-get-doi-links-in-bibliography
\usepackage{natbib}
\bibliographystyle{unsrtnat}
%\bibliographystyle{apalike}
%\bibliography{bib}
%\usepackage[backend=bibtex]{biblatex}
%\addbibresource{bib.bib}
%\addbibresource{bib}
% ------------------------------------------------

% Line numbers
% https://tex.stackexchange.com/questions/16010/number-every-line-of-pages
\usepackage{lineno}
\linenumbers
% ------------------------------------------------

\begin{document}

    \begin{center}
    {\LARGE ФП 2.0}
        \\
        осень 2024
    \end{center}

    \tableofcontents

    \newpage

    \section*{Введение}

    В функциональном программировании и соответствующем языковом дизайне существует некоторый набор знаний, считающихся общеизвестным фольклором.
    Сложность в том, что эти знания рассеяны по книгам, статьям и ``культовым'' блог-постам, и требуется довольно много времени и сил для восстановления целостной картины.

    Цель этого курса --- собрать в одном месте такие фольклорные знания и организовать их в некоторую систему.
    Курс будет явным образом опираться на классические работы и помогать в их изучении.
    Просмотр упоминаемых статей является важной частью самостоятельной работы в рамках курсе.

    В связи с широтой контекста, этот курс является обзорным и не всегда глубоким.
    Так, детали реализации в GHC или теор-категорные основания вещей могут даваться в общем виде и без конкретики.
    В то же время в плоскости языкового дизайна через оптику функционального программирования и пользовательского опыта на Haskell курс пытается быть максимально подробным.

    Выбор тем и структура курса во многом обусловлены организацией языка Haskell, который этот курс использует как референсный.
    Не в последнюю очередь выбор тем для курса обусловлен интересами его автора.


    \section{Воспоминания о ФП}

    В этом разделе мы вспомним основные концепции функционального программирования и языка Haskell.

    \subsection{Термы и редукция}

    В ФП программы представляют собой выражения.
    Выполнение программ --- редукция таких выражений до более простых.
    Выражения можно представлять как в виде линейной записи символов, так и в виде дерева, для понимания которого не требуется знания вспомогательных правил ассоциативности и проч.

    Основная идея функционального программирования --- $\lambda$-абстракция.
    Она позволяет взять произвольное выражение и заменить его фрагмент на формальный параметр.
    Формальный параметр должен быть задекларирован выше по дереву с помощью специальной $\lambda$-вершины.
    Вместо формального параметра можно в дальнейшем подставлять различные конкретные параметры, то есть использовать одно выражение для различных целей.

    \begin{tikzpicture}
        \node [expr] (div) {$\div$};
        \node [expr] (x) [below left=of div] {$x$};
        \node [expr] (plus) [below right = of div] {$+$};
        \node [expr] (two) [below right = of plus] {$2$};
        \node [hexpr] (mult) [below left = of plus] {$\times$};
        \node [hexpr] (ten) [below left = of mult] {$10$};
        \node [hexpr] (four) [below right = of mult] {$4$};
        \draw[->] (plus) -- (div);
        \draw[->] (x) -- (div);
        \draw[->] (mult) -- (plus);
        \draw[->] (ten) -- (mult);
        \draw[->] (four) -- (mult);
        \draw[->] (two) -- (plus);
    \end{tikzpicture}
    \begin{tikzpicture}
        \node [hdecl] (f) {$f$};
        \node [hexpr] (lam) [below=of f] {$\lambda$};
        \node [hdecl] (y) [below right =of lam] {$y$};
        \node [expr] (div) [below=of lam] {$\div$};
        \node [expr] (x) [below left=of div] {$x$};
        \node [expr] (plus) [below right = of div] {$+$};
        \node [hexpr] (param) [below left=of plus] {$y$};
        \node [expr] (two) [below right = of plus] {$2$};
        \draw[->] (lam) -- (f);
        \draw[->] (div) -- (lam);
        \draw[->] (plus) -- (div);
        \draw[->] (x) -- (div);
        \draw[->] (two) -- (plus);
        \draw[->] (param) -- (plus);
        \draw[->] (y) -- (lam);
    \end{tikzpicture}
    \begin{tikzpicture}
        \node [hexpr] (app) {$@$};
        \node [hexpr] (f) [below left =of app] {$f$};
        \node [expr] (mult) [below right= of app] {$\times$};
        \node [expr] (ten) [below left= of mult] {$10$};
        \node [expr] (four) [below right= of mult] {$4$};
        \draw[->] (mult) -- (app);
        \draw[->] (f) -- (app);
        \draw[->] (ten) -- (mult);
        \draw[->] (four) -- (mult);
    \end{tikzpicture}

    Простейший функциональный язык --- $\lambda$-исчисление.
    Выражения в нём называются $\lambda$-термами.
    Термы можно сконструировать тремя способами:
    \begin{description}
        \item[Переменные] $x \in \Lambda$, если $x \in V$ \hspace{2em}
        \begin{tikzpicture}
            \node [expr] (x) {x};
        \end{tikzpicture}
        \item[Абстракция] $(\lambda x\ldotp M) \in \Lambda$, если $x \in V, M \in \Lambda$ \hspace{2em}
        \begin{tikzpicture}
            \node [expr] (lam) {$\lambda$};
            \node [decl] (x) [below right = of lam] {$x$};
            \node [subtree] (m) [below = of lam] {$M$};
            \draw[->] (m.north) -- (lam);
            \draw[->] (x) -- (lam);
        \end{tikzpicture}
        \item[Аппликация] $(M~N) \in \Lambda$, если $M \in \Lambda, N \in \Lambda$ \hspace{2em}
        \begin{tikzpicture}
            \node [expr] (dog) {$@$};
            \node [subtree] (m) [below left= of lam] {$M$};
            \node [subtree] (n) [below right= of lam] {$N$};
            \draw[->] (m.north) -- (lam);
            \draw[->] (n.north) -- (lam);
        \end{tikzpicture}
    \end{description}

    Редукция определяется следующим правилом переписывания: ищется применение $\lambda$-функции к аргументу и в её тело осуществляется подстановка аргумента во все свободные вхождения переменной, связанной лямбдой.

    \begin{tikzpicture}
        \node [expr] (top) {$\cdots$};
        \node [hexpr] (ap) [below = of top] {$@$};
        \node [hexpr] (lam) [below left = of ap] {$\lambda$};
        \node [decl] (x) [below right = of lam] {$x$};
        \node [subtree] (l) [below = of lam] {$M$};
        \node [subtree] (r) [below right = of ap] {$N$};
        \draw[->] (ap) -- (top);
        \draw[->] (lam) -- (ap);
        \draw[->] (l) -- (lam);
        \draw[->] (r.north) -- (ap);
        \draw[->] (x) -- (lam);
    \end{tikzpicture}
    \begin{tikzpicture}
        \node [expr] (top) {$\cdots$};
        \node [subtree] (bot) [below = of top] {$\subst{M}{x}{N}$};
        \draw[->] (bot) -- (top);
    \end{tikzpicture}

    \subsection{Типы}

    Программное обеспечение --- это сложно.
    Поэтому постоянно и неизбежно в программах возникают ошибки.
    Их можно искать, в том числе, статически, то есть без запуска программы.
    Одним из видов статического анализа является анализ типов.

    Как известно, выражение можно представить в виде дерева.
    Для примера рассмотрим $(\lambda y\ldotp x \div (y + 2)) \ap 3$:
    \\
    \begin{tikzpicture}
        \node [expr] (ap) {$@$};
        \node [expr] (three) [below right = of ap]{$3$};
        \node [hexpr] (lam) [below left =of ap] {$\lambda$};
        \node [hdecl] (y) [below right =of lam] {$y$};
        \node [expr] (div) [below=of lam] {$\div$};
        \node [expr] (x) [below left=of div] {$x$};
        \node [expr] (plus) [below right = of div] {$+$};
        \node [hexpr] (param) [below left=of plus] {$y$};
        \node [expr] (two) [below right = of plus] {$2$};
        \draw[->] (three) -- (ap);
        \draw[->] (lam) -- (ap);
        \draw[->] (div) -- (lam);
        \draw[->] (plus) -- (div);
        \draw[->] (x) -- (div);
        \draw[->] (two) -- (plus);
        \draw[->] (param) -- (plus);
        \draw[->] (y) -- (lam);
    \end{tikzpicture}

    Идея анализа типов состоит в том, что мы каждой вершине дерева программы пытаемся присвоить некоторую синтаксическую метку по определённым правилам.
    Если каждой вершине метку присвоить можно, то мы считаем, что программа проходит проверку типов, и она ``хорошая''.

    Пример выше проходит проверку типов в некоторой разумной системе типов:

    \begin{tikzpicture}
        \node [expr] (ap) {$@ : int$};
        \node [expr] (three) [below right = of ap]{$3 : int$};
        \node [hexpr] (lam) [below left=of ap] {$\lambda : int \to int$};
        \node [decl] (y) [below right = of lam] {$y {\color{red}~: int}$};
        \node [expr] (div) [below=of lam] {$\div : int$};
        \node [expr] (x) [below left=of div] {$x : int$};
        \node [expr] (plus) [below right = of div] {$+ : int$};
        \node [hexpr] (param) [below left=of plus] {$y : int$};
        \node [expr] (two) [below right = of plus] {$2 : int$};
        \draw[->] (lam) -- (ap);
        \draw[->] (three) -- (ap);
        \draw[->] (div) -- (lam);
        \draw[->] (plus) -- (div);
        \draw[->] (x) -- (div);
        \draw[->] (two) -- (plus);
        \draw[->] (param) -- (plus);
        \draw[->] (y) -- (lam);
    \end{tikzpicture}

    Система типов определяет синтаксис типовых меток и правила, по которым их можно приписывать.
    Синтаксис обычно описывается в классических нотациях а ля BNF, а правила в виде типовых дробей.
    Например, так выглядят дроби для просто-типизированного $\lambda$-исчисления:
    \[
        \begin{array}{l c r}
            \infer[ctx]{\Gamma \vdash x: \sigma}{(x: \sigma) \in \Gamma}
            &
            \infer[elim\to]{\Gamma \vdash M\;N : \tau}{\Gamma \vdash M : \sigma \to \tau & \Gamma \vdash N : \sigma}
            &
            \infer[intro\to]{\Gamma \vdash \lambda x^{\color{red} \sigma}\ldotp M : \sigma \to \tau}{\{x : \sigma\} \cup \Gamma \vdash M : \tau}
        \end{array}
    \]

    Типовые метки имеют чисто-синтаксическую природу, однако их можно проинтерпретировать.
    Самая популярная интерпретация~--- воспринимать типовую метку как множество.
    Так, метке $int \to int$ можно поставить в соответствие множество функций между множествами ограниченных целых чисел.

    \subsection{Функции в Haskell}

    В своей основе Haskell представляет собой расширенное типизированное $\lambda$-исчисление, дополненное примитивными типами, возможностью декларировать новые имена, структурами данных и классами типов.

    Примеры $\lambda$-абстракций в REPL окружении GHCi:

    \begin{minted}{haskell}
        ghci> (\x -> x + 1) 4
        5
    \end{minted}

    Можно узнать тип функции в интерпретаторе (в реальности числа полиморфные, но об этом далее):
    \begin{minted}{haskell}
        ghci> :t \x -> x + 1
        \x -> x + 1 :: Int -> Int
    \end{minted}

    Функциям можно давать имена.
    Именам можно приписывать типы, это рекомендуется делать явно для деклараций на верхнем уровне файлов исходного кода.
    \begin{minted}{haskell}
        f :: Int -> Int
        f x = x + 1
    \end{minted}

    Если имя типа начинается с маленькой буквы, то это не конкретный заранее заданный тип, а типовая переменная, способная принимать различные значения в зависимости от контекста.
    Такая возможность называется \point{параметрическим полиморфизмом}.
    Так, функция, которая просто возвращает свой аргумент, никак не ограничивает тип аргумента.
    Но в то же время тип результата должен совпадать с типом аргумента.
    \begin{minted}{haskell}
        id :: a -> a
        id x = x

        ghci> :t id 5
        id 5 :: Int
    \end{minted}

    Функции могут принимать другие функции в качестве аргументов.
    Имя функции может состоять из специальных символов, тогда она считается оператором и может применяться к своим операндам в инфиксном стиле:
    \begin{minted}{haskell}
        ($) :: (a -> b) -> a -> b
        f $ x = f x
    \end{minted}

    Пример рекурсивной функции, использующей охранные выражения для отличения базового случая рекурсии:
    \begin{minted}{haskell}
        factorial :: Int -> Int
        factorial n
          | n < 1 = 1
          | otherwise = n * factorial (n - 1)
    \end{minted}

    \subsection{Данные в Haskell}

    В Haskell есть встроенная возможность объявлять свои типы данных, а так же создавать их экземпляры.

    Зададим тип данных, описывающий животных:
    \begin{minted}{haskell}
        data Animal
          = Cat String Int
          | Dog String
    \end{minted}

    Мы задали тип данных \texttt{Animal} и два способа создать значения этого типа: для кошек и собак.
    \texttt{Cat} и \texttt{Dog}~--- это \point{конструкторы данных}.
    Они представляют собой функции, реализация которых находится на стороне языка.
    Они выделяют память под экземпляры данного типа и возвращают их.
    Кошек мы описываем именем и оставшимся количеством жизней, а собак~--- только именем.
    \begin{minted}{haskell}
        Cat :: String -> Int -> Animal
        Dog :: String -> Animal
    \end{minted}

    Чтобы воспользоваться информацией, сохранённой в структуре данных, требуется деконструировать её с помощью паттерн-матчинга.
    Мы сопоставляем значение типа с образцом.
    Если образец похож на то, как было сконструировано значение, то он выбирается среди других образцов и переменные, задекларированные в нём, начинают ссылаться на соответствующее содержимое структуры данных:
    \begin{minted}{haskell}
        show :: Animal -> String
        show animal = case animal of
          Cat name nLifes -> "This is cat " ++ name ++ show nLifes
          Dog name -> "This is dog " ++ name
    \end{minted}

    В Haskell если специальный синтаксис для объявления полей с именованными метками.
    \begin{minted}{haskell}
        data Penguin = Penguin { getName :: String, getAge :: Int }
        penguin = Penguin { getName = "Andrey", getAge = 500 }
    \end{minted}
    Haskell генерирует функции-аксессоры для доступа к полям объекта:
    \begin{minted}{haskell}
        ghci> :t getName :: Penguin -> String
    \end{minted}

    Часто функции в программировании частичные --- при некоторых значениях аргументов они могут вернуть результат, а при некоторых~--- нет.
    Давайте моделировать это с помощью специального типа данных.
    Если есть целочисленный результат, будем возвращать его.
    Если нет, будем возвращать специальное значение этого типа~--- \texttt{Nothing}.
    \begin{minted}{haskell}
        data MaybeD = NothingD | JustD Double
        sqrt :: Double -> MaybeD
        sqrt x = if x < 0 then NothingD else JustD (calcSqrt x)
    \end{minted}

    Можно заметить, что так нам придётся объявлять по типу \texttt{MaybeT} для каждого типа \texttt{T}.
    Поэтому Haskell позволяет абстрагироваться в типе, аналогично тому как можно абстрагироваться по значениям в терме.
    \begin{minted}{haskell}
        data Maybe a = Nothing | Just a
        sqrt :: Double -> Maybe Double
        sqrt x = if x < 0 then Nothing else Just (calcSqrt x)
    \end{minted}

    Заметьте, что сейчас \texttt{Maybe} --- это не совсем тип, так как теперь нужно передать типовой параметр, чтобы получить конкретный тип.
    \texttt{Maybe} называют \point{типовым конструктором}.
    Однако нам не всегда будет принципиально и мы для краткости будем типовые конструкторы называть типами.

    Вместе с абстракцией на уровне типов появилась и аппликация типа к типу.
    А что если дать меньше параметров типовому конструктору, чем ожидается?
    А что если больше?
    Контроль за корректностью типовых аппликаций обеспечивает система кайндов.
    Это простейшие ``типы для типов'', то есть синтаксические метки, контролирующие корректность построенных типов.
    Так, обычные типы имеют метку (кайнд) \texttt{*}.
    Типовые конструкторы имеют стрелочные кайнды.
    Например, \texttt{Maybe :: * -> *}.
    Аппликация типового конструктора к типу подходящего кайнда убирает одну стрелку:
    \begin{minted}{haskell}
        ghci> :k Int
        Int :: *
        ghci> :k Maybe
        Maybe :: * -> *
        ghci> :k Maybe Int
        Maybe Int :: *
    \end{minted}

    Кроме совершенно новых типов данных в Haskell можно объявлять типовые синонимы.
    Это имена, которые можно использовать вместо других типов, если, например, запись оригинального типа слишком длинная для повсеместного написания.
    \begin{minted}{haskell}
        type T a = VeryLongType Int (a -> AnotherLongType a)
    \end{minted}

    Если тип данных содержит только один конструктор и только одно поле, то отсутствует необходимость в аллокации новой памяти, содержащей тег конструктора и набор ссылок на поля.
    В таком случае, в качестве значения такого типа можно всегда просто использовать значение оборачиваемого типа, оставляя новый тип присутствовать исключительно во время компиляции, снижая нагрузку на время исполнения.
    Для объявления таких типов-обёрток нужно воспользоваться ключевым словом \texttt{newtype} вместо \texttt{data}:
    \begin{minted}{haskell}
        newtype CourseId = CourseId Int64
        newtype ModuleId = ModuleId Int64
    \end{minted}

    \subsection{Классы типов в Haskell}

    Рассмотренные ранее полиморфные функции ничего не могли делать со своими аргументами, кроме как возвращать их в качестве результата или передавать в другие полиморфные функции.
    Чтобы уметь делать что-то ещё, нужна какая-то дополнительная информация про тип, потому что иначе нет никакой гарантии, что над объектом данного типа можно делать все необходимые операции.
    Так, функция \texttt{suc n = n + 1} не будет работать для строчек, потому что для них, очевидно, не определена операция сложения.
    Поэтому некорректно будет приписать полиморфный тип \texttt{suc :: a -> a}.

    В Haskell есть механизм ограничения полиморфности функций~--- \texttt{классы типов}.
    Мы можем явно задать, что функция требует не произвольный тип на вход, а произвольный тип, для которого определены обязательно нужные нам операции.
    Так, для типа \texttt{suc} достаточно ограничить тип условием наличия плюса для него: \texttt{suc :: Num a => a -> a}.
    Теперь типовая переменная \texttt{a} может быть специализирована только на типы, на которых определена операция сложения.

    Пример объявления класса типов \texttt{Eq}:
    \begin{minted}{haskell}
        class Eq a where
          (==) :: a -> a -> Bool
    \end{minted}

    Для каждого типа можно объявить свою собственную реализацию \texttt{Eq}.
    Возможность использования такой индивидуальной реализации для каждого типа называется ad-hoc полиморфизмом.
    \begin{minted}{haskell}
        instance Eq CourseId where
          CourseId x == CourseID y = x == y

        instance Eq a => Eq [a] where
          [] == [] = True
          x:xs == y:ys = x == y && xs == ys
    \end{minted}

    \subsection{Монады в Haskell}

    Класс типов \texttt{Functor} объявляется для конструкторов типов и позволяет заменить в некотором контейнере все элементы одного типа на все элементы другого, оставляя структуру контейнера неизменной.

    \begin{minted}{haskell}
        class Functor (f :: * -> *) where
          fmap :: (a -> b) -> f a -> f b

        instance Functor [] where
          fmap :: (a -> b) -> [a] -> [b]
          fmap _ [] = []
          fmap f (x:xs) = f x : fmap f xs
    \end{minted}

    В Haskell любая функция просто вычисляет результат некоторого типа.
    Однако в программирования часто требуются функции, которые не только вычисляют результат, но и делают что-то ещё.
    Например, изменяют какое-то состояние или пишут в консоль.
    Иными словами, производят побочные эффекты.
    В любом случае в Haskell мы можем только вернуть из функции только результат, поэтому такие побочные эффекты мы кодируем в качестве дополнительной структуры, оборачивающей чистый результат.
    Т.е. если функция без побочных эффектов возвращала какой-то тип \texttt{a}, то после добавления побочных эффектов в её реализацию она будет возвращать некоторый тип \point{вычислений} \texttt{f a}.

    \begin{itemize}
        \item Если функция кидает ошибку, то \texttt{f = Maybe}.
        \item Если функция читает глобальное состояние, то \texttt{f = e -> \_}.
        \item Если функция читает глобальное состояние и обновляет его, то \texttt{f = s -> (s, \_)}.
    \end{itemize}

    Стандартная библиотека Haskell предоставляет несколько классов типов для работы со значениями вида \texttt{f a}.
    Они позволяют абстрагироваться от структуры \texttt{f} и работать со значениями \texttt{a} внутри, как будто нет никакой дополнительной структуры.

    Первый такой класс типов позволяет писать выражения над вычислениями \texttt{f a}.
    \begin{minted}{haskell}
        class Functor f => Applicative (f :: * -> *) where
          pure :: a -> f a
          liftA2 :: (a -> b -> c) -> f a -> f b -> f c

        instance Applicative Maybe where
          pure :: a -> Maybe a
          pure = Just

          liftA2 :: (a -> b -> c) -> Maybe a -> Maybe b -> Maybe c
          liftA2 _ Nothing _ = Nothing
          liftA2 _ _ Nothing = Nothing
          liftA2 f (Just x) (Just y) = Just (f x y)
    \end{minted}

    Второй класс типов позволяет делать последовательную композицию вычислений в императивном стиле:
    \begin{minted}{haskell}
        class Applicative m => Monad (m :: * -> *) where
          (>>=) :: m a -> (a -> m b) -> m b

        newtype State s a = State { runState :: s -> (s, a) }

        instance Monad (State s) where
          (>>=) :: State s a -> (a -> State s b) -> State s b
          m >>= k = State \s ->
            let (s', x) = runState m s in
            runState (k x) s'
    \end{minted}

    Теперь если мы определим базовые операции работы с состоянием, мы сможем писать код в императивном стиле.
    \begin{minted}{haskell}
        get :: State s s
        get = State \s -> (s, s)

        put :: s -> State s ()
        put newS = State \oldS -> (newS, ())

        example :: State Int Int
        example =
          get >>= \x ->
          put 42 >>= \() ->
          get >>= \y ->
          pure (x + y)

        ghci> runState example 1
        43
    \end{minted}

    Для таких монадических цепочек существует специальный синтаксический сахар:
    \begin{minted}{haskell}
        example :: State Int Int
        example = do
          x <- get
          put 42
          y <- get
          pure (x + y)
    \end{minted}


    \section{Классические интерпретаторы}

    \subsection{Интерпретаторы как основа основ}

    Подобно тому, как в биологии теория эволюция является некоторым сквозным знанием, основой, скрепляющей разрозненные сведения о живом мире, этот курс будет считать интерпретаторы таким стержнем для теории языков программирования.
    Это может показаться внезапным, так как, вроде бы, на практике люди пишут интерпретаторы довольно редко.
    Опровергнем этот тезис и обоснуем выбор интерпретатора как краеугольного камня повествования.

    \subsubsection{Башня интерпретаторов}

    Самым базовым интерпретатором является процессор, он воплощён физически в железе.
    Ему на вход подаётся программа на некотором языке, например, x86, он зачитывает команды и превращает их в действия над памятью.
    Однако человеку крайне сложно программировать на этом языке, нужен новый язык, инкапсулирующий часть сложности и скрывающий лишние детали.

    Чтобы получить новый язык, мы строим программный интерпретатор.
    \vocab{Программный интерпретатор} $U_M^N$ --- это программа на языке $M$\footnote{Под языком мы тут понимаем множество программ на этом языке, иначе говоря, множество деревьев определённого вида.}, получающая на вход программу на языке $N$ и вход для неё --- данные из $D$, и возвращающая результат выполнения этой программы на этих данных: \[U_M^N : N\times D\to D\]

    Например, у нас есть программа $p_N$ и данные для неё $d_{in}$, результат исполнения этой программы $d_{out}$ можно получить как \[d_{out} = U_M^N\left( \underbrace{\langle p_N, d_{in} \rangle}_{\in N\times D} \right)\]

    Но интерпретатор это тоже программа.
    Как её запустить?
    Возьмём наш базовый интерпретатор $U^{x86}$, у него нет языка реализации, так как он реализован в железе, а не программно.
    Возьмём интерпретатор языка ассемблера, реализованный в кодах x86, $U_{x86}^{Asm}$, программу на ассемблере $p_{Asm}$ и вход для неё $d_{in}$.
    Вспомним, что программа --- это тоже данные, просто в некотором специальном формате.
    Тогда результат применения $p_{Asm}$ на данных мы получим следующим образом:
    \[
        d_{out} = U^{x86}\left(\left<\underbrace{U_{x86}^{Asm}}_{\in Asm}, \underbrace{\overbrace{\langle p_{Asm}}^{\in Asm}, \overbrace{d_{in} \rangle}^{\in D}}_{\in D} \right>\right)
    \]

    Но язык ассемблера, тоже не очень приятен для программирования.
    Однако, на нем можно уже написать интерпретатор языка посложнее.
    И так далее.
    Получаем \point{башню интерпретаторов}, на вершине которой находится язык, на котором мы хотим уже решать непосредственно нашу задачу:
    \[
        d_{out} =
        U^{x86}\left(\left<
        U_{x86}^{Asm}, \left<
        U^C_{Asm}, \left<
        U^{Has}_C, \left< p_{Has}, d_{in}
        \right>\right>\right>\right>\right)
    \]

    Иногда язык задают через трансляцию (компиляцию) в другой, но компилятор можно построить автоматически по интерпретатору\footnote{\href{https://habr.com/ru/articles/47418/}{Проекции Футамуры позволяют автоматически строить компиляторы по интерпретаторам.}}.

    \subsubsection{Интерпретаторы повсюду}

    Хорошо, мы пришли к языку нашего сердца (Хаскеллу), почему же мы продолжаем говорить об интерпретаторах?
    Потому что для решения конкретных бизнес-задач прикладные языки всё ещё слишком церемониальны сами по себе~--- программисту приходится думать о большом количестве вещей, нерелевантных его предметной области и решаемой задаче.
    Сложность~--- главный враг программиста, потому что ресурсы человеческого мозга несопоставимы со сложностью реальности, которую приходится описываться в программах.
    Таким образом, в работе постоянно приходится описывать новые языки, наиболее подходящие для решения конкретных прикладных задач.
    А новые языки мы задаём с помощью интерпретаторов.

    Как выглядит классический рекурсивный интерпретатор?
    Он получает программу в виде некоторого дерева и рекурсивно обходит его, считая результаты поддеревьев.
    Когда он посещает вершину дерева, он определяет её тип и понимает, какие действия нужно исполнить.
    Так, простой интерпретатор некоторого языка Лалаланг мог бы выглядеть примерно следующим образом:
    \begin{minted}{haskell}
        eval :: LaProgram -> Int
        eval prog = case prog of
          LaPlus l r -> eval l + eval r
          ...
    \end{minted}

    Видно, что это похоже, например, на обработку запроса пользователя утилиты командной строки --- определяем, какую команду нужно выполнить, и выполняем.
    Или на обработку запроса REST-сервера --- определяем ручку на которую пришел запрос, выполняем соответствующее действие.
    То есть не так редко мы в реальной жизни пишем интерпретаторы.
    Мы просто не видим, что то, что мы пишем --- это на самом деле интерпретатор некоторого языка.
    В общем случае, свёртку некоторой структуры данных уже можно рассматривать как интерпретацию.

    Более того, как мы убедимся в разделе~\ref{sec:wonder-interpreters}, написание любой функции --- это уже задание нового языка.
    Вот был язык, в котором нельзя было пользователя в приложение добавить.
    Написали функцию \texttt{registerUser}~--- появилась новая команда в языке~--- добавить пользователя.
    Далее мы формально покажем, то такой способ эквивалентен добавлению новой ноды в синтаксическое дерево языка.
    Использование функций является примером \vocab{встраивания языка (embedded language)}, когда мы вместо того, чтобы делать новый отдельный язык, мы его реализуем как библиотеку для уже существующего языка\footnote{\url{https://en.wikipedia.org/wiki/Domain-specific_language}}.

    Как мы будем более и более убеждаться по ходу этого курса, почти любую задачу можно свести к придумыванию языка и написанию интерпретатора (например, язык описания реактивного интерфейса пользователя, или язык с любой другой магией).
    Значит, если мы научимся писать интерпретаторы, мы научимся писать любые программы и решать любые задачи!
    И основные наши усилия будут направлены на изучение средств построения интерпретаторов встроенных языков.

    \subsubsection{Интерпретаторы и семантика языков программирования}

    Семантика языков программирования\footnote{\url{https://en.wikipedia.org/wiki/Semantics_(computer_science)}\label{note:sema-wiki}}\footnote{\url{https://www.cs.nmsu.edu/~jcook/posts/pl-semantics-of-pl/}\label{note:sema-cook}} --- это наука, формально изучающая смысл программ --- поведение программ при исполнении\footnote{Различают статическую и динамическую семантики программы. Мы говорим про вторую.}, а так же способы его описания.

    Существует множество стилей описания семантики языковых конструкций и программ\footref{note:sema-wiki}.
    Так, \vocab{операционная семантика} задаёт смысл конструкций языка, в терминах некоторого более простого языка.
    Обычно таким языком выступает некоторый математический формализм.
    Звучит как интерпретация!
    Только язык реализации интерпретатора~--- математика.

    Мы также можем реализовывать интерпретатор на каком-нибудь реальном языке и он тоже будет задавать семантику определяемого языка.
    Однако формальность такого определения будет зависеть от формальности описания семантики самого языка реализации интерпретатора~--- \vocab{мета-языка}.
    Такие интерпретаторы называют ``определяющими'', они задают семантику языка, жертвуя эффективностью ради наглядности.
    Взаимоотношения определяемого языка и мета-языка изучаются в классических статьях~\cite{reynolds1972definitional,reynolds1998definitional}\footnote{Активно используемое автором понятие продолжения будет рассмотрено далее в этом курсе (раздел \ref{sec:continuations}).}.
    Мы будем использовать этот подход для задания семантики новых языков и в качестве мета-языка будем использовать Haskell.

    Нас интересует ещё один стиль описания семантики.
    \vocab{Денотационная семантика}\footnote{\url{https://en.wikipedia.org/wiki/Denotational_semantics}} описывает значение программ путём сопоставления им объектов некоторого множества, \vocab{семантического домена}, которое мы понимаем лучше.
    Обычно домен --- это некоторый математический объект: множество функций из входа в выход, игры между термом и контекстом исполнения\ldots % todo link to game semantics
    Иначе говоря, денотационная семантика языка $L$ --- это функция из программы на этом языке в элемент домена $D$:
    \[
        \sembr{\bullet} : L \to D
    \]

    Приведём примеры конкретных доменов $D$.
    Доменом может выступать множество функций между натуральными числами $\mathbb{N}\to\mathbb{N}$, если программа принимает число на вход и выдаёт число на выход.
    Если результат программы недетерминирован, в качестве домена можно взять булеан $2^\mathbb{N}$.
    И т.д.
    Но мы можем также в качестве домена взять множество значений некоторого языка программирования!
    И интерпретировать программу не в множество функций между натуральными числами, а, скажем, в множество значений функционального типа \mintinline{haskell}|Nat -> Nat| в языке Haskell\footnote{\url{https://okmij.org/ftp/Denotational.html}}\footnote{Любому интересующемуся языками программирования предлагается провести на сайте Олега Киселёва не один месяц жизни: \url{https://okmij.org/ftp/README.html}.}.

    Для задания самой функции $\sembr{\bullet}$ тоже можно использовать Haskell\footref{note:sema-cook}, это будет определяющий интерпретатор этого языка, отправляющий синтаксическое дерево программы в семантический домен, заданный типом языка Haskell.
    Например: \mintinline{haskell}|eval :: Prog -> Nat -> Nat|.

    \subsection{Инициальные интерпретаторы}

    % todo нормальное объяснение, откуда тут слово инициальный

    Слово ``инициальный'' тут относится к теор-категорной концепции инициальных алгебр, которая используется для кодирования рекурсивный структур данных в теории категорий.
    Всё что нам нужно понимать, --- что инициальные интерпретаторы работают с деревом программы, заданным классически с помощью \mintinline{haskell}|data| (да, деревья можно задавать и по-другому).

    % todo mb meta-circular move somewhere
    % todo link to meta-circular origins
    % todo синхронизировать с объяснением eDSL'ей

    Введём также ещё одно важное понятие.
    \vocab{Meta-circular} интерпретатор~--- это интерпретатор, определяющий конструкции определяемого языка через конструкции мета-языка.
    Например:
    \begin{minted}{haskell}
        interpret term = case term of
          App f t -> (interpret f) (interpret t)
          If c t e -> if interpret c then interpret t else interpret e
          ...
    \end{minted}

    Свойства мета-языка в таком случае во многом определяют свойства объектного\cite{reynolds1972definitional,reynolds1998definitional}.
    Мы будем в этом курсе стремиться как можно более переиспользовать возможности мета-языка, при этом добавляя новые конструкции.

    \subsubsection{Untyped tagless interpreters}

    Рассмотрим некоторый нетипизированный язык.
    Его синтаксис зададим следующим образом:
    \begin{minted}{haskell}
        data Term = Const Int | IsZero Term | If Term Term Term
    \end{minted}








    % todo meta-circular interpreter

    % todo crafting interpreters site

    % todo

    \subsection{Реализация функций высших порядков}

    % todo https://craftinginterpreters.com/

    % todo mention peter landin

    % todo дефункционализация

    % todo симуляция наличия higher-kinded типов

    % todo higher-order abstract syntax

    % todo closure conversion closure conversion
    \cite{reynolds1998definitional}

    % todo


    \section{Параметрический полиморфизм}

    \subsection{Напоминание про типы}

    \subsection{Параметрический полиморфизм первого ранга}




    Подобно тому как $\lambda$-выражения позволяют обобщать выражения по значениям на уровне термов, так $\forall$ (\texttt{forall}) в типах позволяет обобщать типы

    % todo

    \subsection{Полиморфизм высшего ранга}

    % TODO

    \subsection{Реализация параметрического полиморфизма}

    % todo

    \subsection{Полиморфизм по runtime-представлению}

    %todo


    \section{Специальный полиморфизм}

    % todo existential types, different vtable implementations
    % todo scala type classes
    % todo type inhabitation
    % todo доклад SPJ
    % todo open unions, data a la carte
    % todo roles and coertions

    % todo


    \section{Дерайвинги и специализация}

    % todo streams and fusion
    % todo CPS

    % todo staged computations and modal types
    % todo Haskell papers about fusion

%    \subsection{Template Haskell}
%
%    % todo порядок на коде для поддержания реификации
%
%    % todo
%
%    \subsection{Рефлексия и реификация}
%
%    % TODO
%
%    \subsection{Zig-generics}
%
%    % todo
%
%    \subsection{GHC.Generics}
%
%    % todo
%
%    \subsection{Uniplate}
%
%    % todo
%
%
%    \section{Оптика}
%
%    % todo


    \section{Интерпретаторы зазеркалья} \label{sec:wonder-interpreters}

    % todo isomorphism proof

    % todo теоретические основы

    % todo church encoding

    % todo Tagless-final интерпретаторы

    % todo codata

    % todo fusion и дефорестация

    % todo наверное после континуэйшенов

    % todo


    \section{Дополнительные главы монад}

    % todo оригинальная статья

    % todo monadic reflection
    % todo monadic reflection & direct style (lib from scala)

    % todo free monads, freer monads

    % todo монады не композируются, но фримонады - композируются

    % todo semantic domains have monad structure

    % todo https://github.com/lampepfl/monadic-reflection/blob/main/TUTORIAL.md

    % todo


    \section{Продолжения (continuations)} \label{sec:continuations}

    % todo

    % todo связь с монадами

    % todo монада Cont

    % todo delimited continuations

    % todo sicp

    % todo raynolds The Discoveries of Continuations

    % todo correspondence to negation in intuitionistic logic?

    % todo CPS correspond to double negation in logic reynolds1998definitional

    % todo defuctionalized continuations, pepers from gibbons talk

    % todo difference lists

    \cite{reynolds1972definitional, reynolds1998definitional}


    \section{Хендлеры эффектов}

    % todo алгебраические эффекты
    % todo связь с delimited continuations
    % todo стратегии компиляции, связь с codata
    % todo эффекты высших порядков
    % todo full vs shallow embeddings
    % todo abstracting definitional interpreters & github semantics
    % todo fused effects and CPS

    % todo


    \section{Системы эффектов}

    % todo


    \section{Корутины}

    % todo


    \section{Компилятор GHC и runtime}

    % todo


    \section{Оптика}

    % todo http://www.timphilipwilliams.com/posts/2019-07-25-minecraft.html

    % todo


    % todo куда-нибудь запихнуть algebra driven design, quick check и доклад про проперти-тесты
    % todo expression problem somewhere
    % todo где-то должно быть содержание книжки про типы Sandy Maguire


    \newpage
    \bibliography{bib}

\end{document}
