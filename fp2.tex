%! suppress = Ellipsis
%! suppress = Quote
%! suppress = LineBreak
%! suppress = Makeatletter
%! suppress = TooLargeSection
%! suppress = MissingLabel
\documentclass[12pt]{article}

% Fields
\usepackage{geometry}
\geometry{top=25mm}
\geometry{bottom=35mm}
\geometry{left=20mm}
\geometry{right=20mm}
% ------------------------------------------------

% Graphics
\usepackage{color}
\usepackage{tabularx}
\usepackage{tikz}
% https://tikz.dev/tikz-graphs
\usetikzlibrary{positioning, shapes.geometric, arrows, automata, graphs}
\tikzset{
    expr/.style={ellipse, draw=gray!60, fill=gray!5, very thick, minimum size=7mm, yshift=0.7cm},
    hexpr/.style={ellipse, draw=gray!60, fill=blue!15, very thick, minimum size=7mm, yshift=0.7cm},
    stmt/.style={rectangle, draw=gray!60, fill=gray!5, very thick, minimum size=5mm, yshift=0.7cm},
    decl/.style={rectangle, draw=blue!60, fill=gray!5, very thick, minimum size=5mm, yshift=0.7cm},
    hdecl/.style={rectangle, draw=blue!60, fill=blue!15, very thick, minimum size=5mm, yshift=0.7cm},
    subtree/.style={shape border rotate=90, isosceles triangle, draw=gray!60, fill=gray!5, very thick, minimum size=5mm, yshift=0.0cm},
}
\usepackage{blkarray}
\usepackage{graphicx}
\usepackage{forest} % https://tex.stackexchange.com/questions/198405/how-to-change-the-color-of-subtrees-in-tikz-qtree
% ------------------------------------------------

% Math
\usepackage{amsmath, amsfonts}
\usepackage{amssymb}
\usepackage{proof}
\usepackage{mathrsfs}
% Crossed-out symbols
% https://tex.stackexchange.com/questions/75525/how-to-write-crossed-out-math-in-latex
\usepackage[makeroom]{cancel}
\usepackage{mathtools}
% ------------------------------------------------

% Additional font sizes
% https://www.overleaf.com/learn/latex/Questions/How_do_I_adjust_the_font_size%3F
\usepackage{moresize}
% Additional colors
% https://www.overleaf.com/learn/latex/Using_colours_in_LaTeX
\usepackage{xcolor}
% \texttimes
\usepackage{textcomp}
% ------------------------------------------------

% Language
\usepackage[utf8] {inputenc}
\usepackage[T2A] {fontenc}
\usepackage[english, russian] {babel}
\usepackage{indentfirst, verbatim}
\usetikzlibrary{cd, babel}
% ------------------------------------------------

% Fonts
\usepackage{stmaryrd}
\usepackage{cmbright}
\usepackage{wasysym}
% ------------------------------------------------

% Code
% https://tex.stackexchange.com/questions/99475/how-to-invoke-latex-with-the-shell-escape-flag-in-texstudio-former-texmakerx
% Colored, requires --shell-escape compiling option
\usepackage{minted}
\setminted{xleftmargin=\parindent, autogobble, escapeinside=??, linenos, numberblanklines=false}
\usepackage{listings}
% ------------------------------------------------

% Custom commands
\newcommand{\vocab}[1]{\textbf{#1}} % new word for vocabulary
\newcommand{\point}[1]{{\color{blue}\textit{#1}}} % point out something in text
\newcommand{\sembr}[1]{\llbracket{#1}\rrbracket} % semantic brackets
\newcommand{\positive}{$+$} % item: pros
\newcommand{\negative}{{\color{red} $-$}} % item: cons
\newcommand{\comb}[1]{\mathbf{#1}}
\newcommand{\step}{\rightsquigarrow}
\newcommand{\term}[1]{\mathbf{#1}}
\newcommand{\ap}{~}
\newcommand{\termdef}{\coloneqq}
\newcommand{\subst}[3]{\left[#2 \mapsto #3 \right] #1}
\newcommand{\eqbeta}{=_\beta}
\newcommand{\eqeta}{=_\eta}
\newcommand{\iso}{\cong}
\newtheorem{task}{Упражнение}
% ------------------------------------------------

% Head
\usepackage{hyperref}
\usepackage{fancyhdr}
% ------------------------------------------------

% Bibliography
\usepackage{csquotes}
% https://tex.stackexchange.com/questions/3802/how-to-get-doi-links-in-bibliography
\usepackage{natbib}
\bibliographystyle{unsrtnat}
%\bibliographystyle{apalike}
%\bibliography{bib}
%\usepackage[backend=bibtex]{biblatex}
%\addbibresource{bib.bib}
%\addbibresource{bib}
% ------------------------------------------------

% Line numbers
% https://tex.stackexchange.com/questions/16010/number-every-line-of-pages
\usepackage{lineno}
\linenumbers
% ------------------------------------------------

% Enumerations
% https://tex.stackexchange.com/questions/10684/vertical-space-in-lists
\usepackage{enumitem}
\setlist{noitemsep} % nosep
%\renewcommand{\labelitemi}{$\triangleright$}
%\renewcommand{\labelitemii}{$\triangleright$}
%\renewcommand{\labelitemiii}{$\triangleright$}
%\renewcommand{\labelitemiv}{$\triangleright$}
% ------------------------------------------------

% Formatting
\usepackage{multicol}
% https://tex.stackexchange.com/questions/282869/putting-two-figures-side-by-side
\usepackage{subcaption}
% https://tex.stackexchange.com/questions/109467/footnote-in-tabular-environment
\usepackage{footnote}
\makesavenoteenv{tabular}
% ------------------------------------------------



\begin{document}

    \begin{center}
    {\LARGE ФП 2.0}\footnote{Автор Андрей Стоян (\texttt{andrey.stoyan.csam@gmail.com}).}
        \\
        осень 2025
    \end{center}

    \tableofcontents

    \newpage

    \section*{Введение}

    Данный курс посвящен основам теории языков программирования.
    Многие сложные концепции могут быть поняты как частные случаи некоторых простых фундаментальных принципов, которые, как правило, считаются общеизвестным фольклором, не требующим дополнительных пояснений.
    Однако, сложность в том, что эти знания рассеяны по книгам, статьям и ``культовым'' блог-постам, и требуется довольно много времени и сил для восстановления целостной картины.

    Цель данного курса --- собрать в одном месте такие фольклорные знания и организовать их в некоторую систему.
    Курс будет явным образом опираться на классические работы, исследующие принципы построения языков, и помогать в их изучении.
    Просмотр упоминаемых статей является важной частью самостоятельной работы в рамках курса.

    Под функциональным программированием, вынесенным даже в заголовок курса, понимается трепетное отношение к понятию эффекта, которое в ФП, в отличие от других школ мысли, не считается аксиоматической данностью, но предметом для изучения, сознательного конструирования и аккуратного обращения.
    Этот подход оказывается очень продуктивным как для изучения языков, так и для построения могущественных языковых конструкций. % todo стиль программирования с эффектами
    Кроме того, функциональные языки сравнительно просты, в результате чего новые идеи и подходы, как правило, зарождаются в них и распространяются далее.

    В качестве основного языка курса выбран Haskell, так как он, с одной стороны, воплощает в себе многие концепции, часто доведенные до некоторого логического завершения, и достаточно могуществен для кодирования других.
    С другой стороны, всё ещё является прикладным промышленным языком программирования.

    В связи с широтой контекста, данный курс не всегда является глубоким.
    Так, детали реализации в GHC или теор-категорные основания вещей могут даваться в общем виде и без конкретики.
    В то же время, в плоскости языкового дизайна через оптику функционального программирования курс пытается быть максимально подробным.

    Таким образом, данный курс может быть полезен тем, кто интересуется дизайном языков и красивыми обобщениями программистских концепций, хочет улучшить свои навыки проектирования API, или планирует вести практическую деятельность на функциональных языках.

    Пререквизитом к прохождению курса является знание основ функционального программирования: алгебраических типов данных, паттерн-матчинга, свёрток, параметрического полиморфизма, классов типов, базовых монад.
    Дополнительно будет полезным умение читать типовые дроби, знакомство с полиморфным $\lambda$-исчислением и кодированием Чёрча.


    \clearpage


    \section{Воспоминания о ФП}

    В этом разделе мы вспомним основные концепции функционального программирования и языка Haskell.

\subsection{Термы и редукция} \label{subsec:terms-reduction}

В ФП программы представляют собой выражения.
Выполнение программ --- редукция таких выражений до более простых.
Выражения можно представлять как в виде линейной записи символов, так и в виде дерева, для понимания которого не требуется знания вспомогательных правил ассоциативности и проч.

Основная идея функционального программирования --- $\lambda$-абстракция.
Она позволяет взять произвольное выражение и заменить его фрагмент на формальный параметр.
Формальный параметр должен быть задекларирован выше по дереву с помощью специальной $\lambda$-вершины.
Вместо формального параметра можно в дальнейшем подставлять различные конкретные параметры, то есть использовать одно выражение для различных целей.

\begin{figure}[h]
    \centering
    \begin{tikzpicture}
        \node [expr] (div) {$\div$};
        \node [expr] (x) [below left=of div] {$x$};
        \node [expr] (plus) [below right = of div] {$+$};
        \node [expr] (two) [below right = of plus] {$2$};
        \node [hexpr] (mult) [below left = of plus] {$\times$};
        \node [hexpr] (ten) [below left = of mult] {$10$};
        \node [hexpr] (four) [below right = of mult] {$4$};
        \draw[->] (plus) -- (div);
        \draw[->] (x) -- (div);
        \draw[->] (mult) -- (plus);
        \draw[->] (ten) -- (mult);
        \draw[->] (four) -- (mult);
        \draw[->] (two) -- (plus);
    \end{tikzpicture}
    \begin{tikzpicture}
        \node [hdecl] (f) {$f$};
        \node [hexpr] (lam) [below=of f] {$\lambda$};
        \node [hdecl] (y) [below right =of lam] {$y$};
        \node [expr] (div) [below=of lam] {$\div$};
        \node [expr] (x) [below left=of div] {$x$};
        \node [expr] (plus) [below right = of div] {$+$};
        \node [hexpr] (param) [below left=of plus] {$y$};
        \node [expr] (two) [below right = of plus] {$2$};
        \draw[->] (lam) -- (f);
        \draw[->] (div) -- (lam);
        \draw[->] (plus) -- (div);
        \draw[->] (x) -- (div);
        \draw[->] (two) -- (plus);
        \draw[->] (param) -- (plus);
        \draw[->] (y) -- (lam);
    \end{tikzpicture}
    \begin{tikzpicture}
        \node [hexpr] (app) {$@$};
        \node [hexpr] (f) [below left =of app] {$f$};
        \node [expr] (mult) [below right= of app] {$\times$};
        \node [expr] (ten) [below left= of mult] {$10$};
        \node [expr] (four) [below right= of mult] {$4$};
        \draw[->] (mult) -- (app);
        \draw[->] (f) -- (app);
        \draw[->] (ten) -- (mult);
        \draw[->] (four) -- (mult);
    \end{tikzpicture}
\end{figure}

Простейший функциональный язык --- $\lambda$-исчисление.
Выражения в нём называются $\lambda$-термами.
Термы можно сконструировать тремя способами:
\begin{description}
    \item[Переменные] $x \in \Lambda$, если $x \in V$ \hspace{2em}
    \begin{tikzpicture}
        \node [expr] (x) {x};
    \end{tikzpicture}
    \item[Абстракция] $(\lambda x\ldotp M) \in \Lambda$, если $x \in V, M \in \Lambda$ \hspace{2em}
    \begin{tikzpicture}
        \node [expr] (lam) {$\lambda$};
        \node [decl] (x) [below right = of lam] {$x$};
        \node [subtree] (m) [below = of lam] {$M$};
        \draw[->] (m.north) -- (lam);
        \draw[->] (x) -- (lam);
    \end{tikzpicture}
    \item[Аппликация] $(M~N) \in \Lambda$, если $M \in \Lambda, N \in \Lambda$ \hspace{2em}
    \begin{tikzpicture}
        \node [expr] (dog) {$@$};
        \node [subtree] (m) [below left= of lam] {$M$};
        \node [subtree] (n) [below right= of lam] {$N$};
        \draw[->] (m.north) -- (lam);
        \draw[->] (n.north) -- (lam);
    \end{tikzpicture}
\end{description}

Редукция определяется следующим правилом переписывания: ищется применение $\lambda$-функции к аргументу и в её тело осуществляется подстановка аргумента во все свободные вхождения переменной, связанной лямбдой.

\begin{figure}[h]
    \centering
    \begin{tikzpicture}
        \node [expr] (top) {$\cdots$};
        \node [hexpr] (ap) [below = of top] {$@$};
        \node [hexpr] (lam) [below left = of ap] {$\lambda$};
        \node [decl] (x) [below right = of lam] {$x$};
        \node [subtree] (l) [below = of lam] {$M$};
        \node [subtree] (r) [below right = of ap] {$N$};
        \draw[->] (ap) -- (top);
        \draw[->] (lam) -- (ap);
        \draw[->] (l) -- (lam);
        \draw[->] (r.north) -- (ap);
        \draw[->] (x) -- (lam);
    \end{tikzpicture}
    \begin{tikzpicture}
        \node [expr] (top) {$\cdots$};
        \node [subtree] (bot) [below = of top] {$\subst{M}{x}{N}$};
        \draw[->] (bot) -- (top);
    \end{tikzpicture}
\end{figure}

\subsection{Типы}

Программное обеспечение --- это сложно.
Поэтому постоянно и неизбежно в программах возникают ошибки.
Их можно искать, в том числе, статически, то есть без запуска программы.
Одним из видов статического анализа является анализ типов.

Как известно, выражение можно представить в виде дерева.
Для примера рассмотрим $(\lambda y\ldotp x \div (y + 2)) \ap 3$:
\\
\begin{figure}[h]
    \centering
    \begin{tikzpicture}
        \node [expr] (ap) {$@$};
        \node [expr] (three) [below right = of ap]{$3$};
        \node [hexpr] (lam) [below left =of ap] {$\lambda$};
        \node [hdecl] (y) [below right =of lam] {$y$};
        \node [expr] (div) [below=of lam] {$\div$};
        \node [expr] (x) [below left=of div] {$x$};
        \node [expr] (plus) [below right = of div] {$+$};
        \node [hexpr] (param) [below left=of plus] {$y$};
        \node [expr] (two) [below right = of plus] {$2$};
        \draw[->] (three) -- (ap);
        \draw[->] (lam) -- (ap);
        \draw[->] (div) -- (lam);
        \draw[->] (plus) -- (div);
        \draw[->] (x) -- (div);
        \draw[->] (two) -- (plus);
        \draw[->] (param) -- (plus);
        \draw[->] (y) -- (lam);
    \end{tikzpicture}
\end{figure}

Идея анализа типов состоит в том, что мы каждой вершине дерева программы пытаемся присвоить некоторую синтаксическую метку по определённым правилам.
Если каждой вершине метку присвоить можно, то мы считаем, что программа проходит проверку типов, и она ``хорошая''.

Пример выше проходит проверку типов в некоторой разумной системе типов:
\begin{figure}[h]
    \centering
    \begin{tikzpicture}
        \node [expr] (ap) {$@ : int$};
        \node [expr] (three) [below right = of ap]{$3 : int$};
        \node [hexpr] (lam) [below left=of ap] {$\lambda : int \to int$};
        \node [decl] (y) [below right = of lam] {$y {\color{red}~: int}$};
        \node [expr] (div) [below=of lam] {$\div : int$};
        \node [expr] (x) [below left=of div] {$x : int$};
        \node [expr] (plus) [below right = of div] {$+ : int$};
        \node [hexpr] (param) [below left=of plus] {$y : int$};
        \node [expr] (two) [below right = of plus] {$2 : int$};
        \draw[->] (lam) -- (ap);
        \draw[->] (three) -- (ap);
        \draw[->] (div) -- (lam);
        \draw[->] (plus) -- (div);
        \draw[->] (x) -- (div);
        \draw[->] (two) -- (plus);
        \draw[->] (param) -- (plus);
        \draw[->] (y) -- (lam);
    \end{tikzpicture}
\end{figure}

Система типов определяет синтаксис типовых меток и правила, по которым их можно приписывать.
Синтаксис обычно описывается в классических нотациях а ля BNF, а правила в виде типовых дробей.
Например, так выглядят дроби для просто-типизированного $\lambda$-исчисления:
\[
    \begin{array}{l c r}
        \infer[ctx]{\Gamma \vdash x: \sigma}{(x: \sigma) \in \Gamma}
        &
        \infer[elim\to]{\Gamma \vdash M\;N : \tau}{\Gamma \vdash M : \sigma \to \tau & \Gamma \vdash N : \sigma}
        &
        \infer[intro\to]{\Gamma \vdash \lambda x^{\color{red} \sigma}\ldotp M : \sigma \to \tau}{\{x : \sigma\} \cup \Gamma \vdash M : \tau}
    \end{array}
\]

Типовые метки имеют чисто-синтаксическую природу, однако их можно проинтерпретировать.
Самая популярная интерпретация~--- воспринимать типовую метку как множество.
Так, метке $int \to int$ можно поставить в соответствие множество функций между множествами ограниченных целых чисел.

\subsection{Функции в Haskell}

В своей основе Haskell представляет собой расширенное типизированное $\lambda$-исчисление, дополненное примитивными типами, возможностью декларировать новые имена, структурами данных и классами типов.

Примеры $\lambda$-абстракций в REPL окружении GHCi:

\begin{minted}{haskell}
    ghci> (\x -> x + 1) 4
    5
\end{minted}

Можно узнать тип функции в интерпретаторе (в реальности числа полиморфные, но об этом далее):
\begin{minted}{haskell}
    ghci> :t \x -> x + 1
    \x -> x + 1 :: Int -> Int
\end{minted}

Функциям можно давать имена.
Именам можно приписывать типы, это рекомендуется делать явно для деклараций на верхнем уровне файлов исходного кода.
\begin{minted}{haskell}
    f :: Int -> Int
    f x = x + 1
\end{minted}

Если имя типа начинается с маленькой буквы, то это не конкретный заранее заданный тип, а типовая переменная, способная принимать различные значения в зависимости от контекста.
Такая возможность называется \point{параметрическим полиморфизмом}.
Так, функция, которая просто возвращает свой аргумент, никак не ограничивает тип аргумента.
Но в то же время тип результата должен совпадать с типом аргумента.
\begin{minted}{haskell}
    id :: a -> a
    id x = x

    ghci> :t id 5
    id 5 :: Int
\end{minted}

Функции могут принимать другие функции в качестве аргументов.
Имя функции может состоять из специальных символов, тогда она считается оператором и может применяться к своим операндам в инфиксном стиле:
\begin{minted}{haskell}
    ($) :: (a -> b) -> a -> b
    f $ x = f x
\end{minted}

Пример рекурсивной функции, использующей охранные выражения для отличения базового случая рекурсии:
\begin{minted}{haskell}
    factorial :: Int -> Int
    factorial n
      | n < 1 = 1
      | otherwise = n * factorial (n - 1)
\end{minted}

\begin{task}
    Что выведет запрос \mintinline{haskell}|ghci> :t uncurry (flip const)|?
\end{task}

\begin{task}
    Что выведет запрос \mintinline{haskell}|ghci> :t first . first| при
    \begin{minted}{haskell}
        first :: (a -> a') -> (a, b) -> (a', b)
    \end{minted}
\end{task}

\begin{task}
    Реализуйте факториал с помощью техники аккумулирующего параметра.
\end{task}

\subsection{Данные в Haskell}

В Haskell есть встроенная возможность объявлять свои типы данных, а так же создавать их экземпляры.

Зададим тип данных, описывающий животных:
\begin{minted}{haskell}
    data Animal
      = Cat String Int
      | Dog String
\end{minted}

Мы задали тип данных \texttt{Animal} и два способа создать значения этого типа: для кошек и собак.
\texttt{Cat} и \texttt{Dog}~--- это \point{конструкторы данных}.
Они представляют собой функции, реализация которых находится на стороне языка.
Они выделяют память под экземпляры данного типа и возвращают их.
Кошек мы описываем именем и оставшимся количеством жизней, а собак~--- только именем.
\begin{minted}{haskell}
    Cat :: String -> Int -> Animal
    Dog :: String -> Animal
\end{minted}

Чтобы воспользоваться информацией, сохранённой в структуре данных, требуется деконструировать её с помощью паттерн-матчинга.
Мы сопоставляем значение типа с образцом.
Если образец похож на то, как было сконструировано значение, то он выбирается среди других образцов и переменные, задекларированные в нём, начинают ссылаться на соответствующее содержимое структуры данных:
\begin{minted}{haskell}
    show :: Animal -> String
    show animal = case animal of
      Cat name nLifes -> "This is cat " ++ name ++ show nLifes
      Dog name -> "This is dog " ++ name
\end{minted}

В Haskell есть специальный синтаксис для объявления полей с именованными метками.
\begin{minted}{haskell}
    data Penguin = Penguin { getName :: String, getAge :: Int }
    penguin = Penguin { getName = "Andrey", getAge = 500 }
\end{minted}
Haskell генерирует функции-аксессоры для доступа к полям объекта:
\begin{minted}{haskell}
    ghci> :t getName :: Penguin -> String
\end{minted}

Часто функции в программировании частичные --- при некоторых значениях аргументов они могут вернуть результат, а при некоторых~--- нет.
Давайте моделировать это с помощью специального типа данных.
Если есть целочисленный результат, будем возвращать его.
Если нет, будем возвращать специальное значение этого типа~--- \texttt{Nothing}.
\begin{minted}{haskell}
    data MaybeD = NothingD | JustD Double
    sqrt :: Double -> MaybeD
    sqrt x = if x < 0 then NothingD else JustD (calcSqrt x)
\end{minted}

Можно заметить, что так нам придётся объявлять по типу \texttt{MaybeT} для каждого типа \texttt{T}.
Поэтому Haskell позволяет абстрагироваться в типе, аналогично тому как можно абстрагироваться по значениям в терме.
\begin{minted}{haskell}
    data Maybe a = Nothing | Just a
    sqrt :: Double -> Maybe Double
    sqrt x = if x < 0 then Nothing else Just (calcSqrt x)
\end{minted}

Заметьте, что сейчас \texttt{Maybe} --- это не совсем тип, так как теперь нужно передать типовой параметр, чтобы получить конкретный тип.
\texttt{Maybe} называют \point{типовым конструктором}.
Однако нам не всегда будет принципиально и мы для краткости будем типовые конструкторы называть типами.

Вместе с абстракцией на уровне типов появилась и аппликация типа к типу.
А что если дать меньше параметров типовому конструктору, чем ожидается?
А что если больше?
Контроль за корректностью типовых аппликаций обеспечивает система кайндов.
Это простейшие ``типы для типов'', то есть синтаксические метки, контролирующие корректность построенных типов.
Так, обычные типы имеют метку (кайнд) \texttt{*}.
Типовые конструкторы имеют стрелочные кайнды.
Например, \texttt{Maybe :: * -> *}.
Аппликация типового конструктора к типу подходящего кайнда убирает одну стрелку:
\begin{minted}{haskell}
    ghci> :k Int
    Int :: *
    ghci> :k Maybe
    Maybe :: * -> *
    ghci> :k Maybe Int
    Maybe Int :: *
\end{minted}

Кроме совершенно новых типов данных в Haskell можно объявлять типовые синонимы.
Это имена, которые можно использовать вместо других типов, если, например, запись оригинального типа слишком длинная для повсеместного написания.
\begin{minted}{haskell}
    type T a = VeryLongType Int (a -> AnotherLongType a)
\end{minted}

Если тип данных содержит только один конструктор и только одно поле, то отсутствует необходимость в аллокации новой памяти, содержащей тег конструктора и набор ссылок на поля.
В таком случае, в качестве значения такого типа можно всегда просто использовать значение оборачиваемого типа, оставляя новый тип присутствовать исключительно во время компиляции, снижая нагрузку на время исполнения.
Для объявления таких типов-обёрток нужно воспользоваться ключевым словом \texttt{newtype} вместо \texttt{data}:
\begin{minted}{haskell}
    newtype CourseId = CourseId Int64
    newtype ModuleId = ModuleId Int64
\end{minted}

\begin{task}
    Определите кайнд конструктора типа
    \begin{minted}{haskell}
        data Free f a = Pure a | Free (f (Free f a))
    \end{minted}
\end{task}

\subsection{Классы типов в Haskell}

Рассмотренные ранее полиморфные функции ничего не могли делать со своими аргументами, кроме как возвращать их в качестве результата или передавать в другие полиморфные функции.
Чтобы уметь делать что-то ещё, нужна какая-то дополнительная информация про тип, потому что иначе нет никакой гарантии, что над объектом данного типа можно делать все необходимые операции.
Так, функция \texttt{suc n = n + 1} не будет работать для строчек, потому что для них, очевидно, не определена операция сложения.
Поэтому некорректно будет приписать полиморфный тип \texttt{suc :: a -> a}.

В Haskell есть механизм ограничения полиморфности функций~--- \texttt{классы типов}.
Мы можем явно задать, что функция требует не произвольный тип на вход, а произвольный тип, для которого определены обязательно нужные нам операции.
Так, для типа \texttt{suc} достаточно ограничить тип условием наличия плюса для него: \texttt{suc :: Num a => a -> a}.
Теперь типовая переменная \texttt{a} может быть специализирована только на типы, на которых определена операция сложения.

Пример объявления класса типов \texttt{Eq}:
\begin{minted}{haskell}
    class Eq a where
      (==) :: a -> a -> Bool
\end{minted}

Для каждого типа можно объявить свою собственную реализацию \texttt{Eq}.
Возможность использования такой индивидуальной реализации для каждого типа называется ad-hoc полиморфизмом.
\begin{minted}{haskell}
    instance Eq CourseId where
      CourseId x == CourseID y = x == y

    instance Eq a => Eq [a] where
      [] == [] = True
      x:xs == y:ys = x == y && xs == ys
\end{minted}

\begin{task}
    Реализуйте инстанс полугруппы для функций.
\end{task}

\begin{task}
    Реализуйте проверку равенства функций.
\end{task}

\subsection{Монады в Haskell}

Класс типов \texttt{Functor} объявляется для конструкторов типов и позволяет заменить в некотором контейнере все элементы одного типа на все элементы другого, оставляя структуру контейнера неизменной.

\begin{minted}{haskell}
    class Functor (f :: * -> *) where
      fmap :: (a -> b) -> f a -> f b

    instance Functor [] where
      fmap :: (a -> b) -> [a] -> [b]
      fmap _ [] = []
      fmap f (x:xs) = f x : fmap f xs
\end{minted}

В Haskell любая функция просто вычисляет результат некоторого типа.
Однако в программирования часто требуются функции, которые не только вычисляют результат, но и делают что-то ещё.
Например, изменяют какое-то состояние или пишут в консоль.
Иными словами, производят побочные эффекты.
В любом случае в Haskell мы можем только вернуть из функции только результат, поэтому такие побочные эффекты мы кодируем в качестве дополнительной структуры, оборачивающей чистый результат.
Т.е. если функция без побочных эффектов возвращала какой-то тип \texttt{a}, то после добавления побочных эффектов в её реализацию она будет возвращать некоторый тип \point{вычислений} \texttt{f a}.

\begin{itemize}
    \item Если функция кидает ошибку, то \texttt{f = Maybe}.
    \item Если функция читает глобальное состояние, то \texttt{f = e -> \_}.
    \item Если функция читает глобальное состояние и обновляет его, то \texttt{f = s -> (s, \_)}.
\end{itemize}

Стандартная библиотека Haskell предоставляет несколько классов типов для работы со значениями вида \texttt{f a}.
Они позволяют абстрагироваться от структуры \texttt{f} и работать со значениями \texttt{a} внутри, как будто нет никакой дополнительной структуры.

Первый такой класс типов позволяет писать выражения над вычислениями \texttt{f a}.
\begin{minted}{haskell}
    class Functor f => Applicative (f :: * -> *) where
      pure :: a -> f a
      liftA2 :: (a -> b -> c) -> f a -> f b -> f c

    instance Applicative Maybe where
      pure :: a -> Maybe a
      pure = Just

      liftA2 :: (a -> b -> c) -> Maybe a -> Maybe b -> Maybe c
      liftA2 _ Nothing _ = Nothing
      liftA2 _ _ Nothing = Nothing
      liftA2 f (Just x) (Just y) = Just (f x y)
\end{minted}

Второй класс типов позволяет делать последовательную композицию вычислений в императивном стиле:
\begin{minted}{haskell}
    class Applicative m => Monad (m :: * -> *) where
      (>>=) :: m a -> (a -> m b) -> m b

    newtype State s a = State { runState :: s -> (s, a) }

    instance Monad (State s) where
      (>>=) :: State s a -> (a -> State s b) -> State s b
      m >>= k = State \s ->
        let (s', x) = runState m s in
        runState (k x) s'
\end{minted}

Теперь если мы определим базовые операции работы с состоянием, мы сможем писать код в императивном стиле.
\begin{minted}{haskell}
    get :: State s s
    get = State \s -> (s, s)

    put :: s -> State s ()
    put newS = State \oldS -> (newS, ())

    example :: State Int Int
    example =
      get >>= \x ->
      put 42 >>= \() ->
      get >>= \y ->
      pure (x + y)

    ghci> runState example 1
    43
\end{minted}

Для таких монадических цепочек существует специальный синтаксический сахар:
\begin{minted}{haskell}
    example :: State Int Int
    example = do
      x <- get
      put 42
      y <- get
      pure (x + y)
\end{minted}

\begin{task}
    Реализуйте \mintinline{haskell}|liftA3| через \mintinline{haskell}|liftA2|.
\end{task}

\begin{task}
    Реализуйте \mintinline{haskell}|>>=| через \mintinline{haskell}|join| и наоборот.
\end{task}

\begin{task}
    Два числа с консоли, поделите одно на другое нацело и распечатайте результат, если остаток не нулевой, распечатайте его тоже.
\end{task}

% todo algebraic data types are relations between data https://stanford-cs242.github.io/f18/lectures/02-2-algebraic-data-types.html
% todo https://codewords.recurse.com/issues/three/algebra-and-calculus-of-algebraic-data-types


    \clearpage


    \section{Общие знания о типах}

    %! suppress = MissingLabel

В этой главе собраны некоторые общие знания о типах.

Разделы~\ref{subsec:variance}, ~\ref{subsec:isomorphism} в основном следуют~\cite[глава 1]{maguire-types}.

\subsection{Вариантность} \label{subsec:variance} % todo

В этом параграфе мы будем рассматривать тему с точки зрения программирования~\cite[глава 3]{maguire-types}, не отдавая должного теории категорий.
Восполнить пробел можно с помощью замечательной статьи, написанной в жанре пьесы~\cite{hinze2012functional}.

\vocab{Ковариантный функтор} --- пара из некоторого типового конструктора \texttt{F} и операции на функциях \texttt{fmap :: (a -> b) -> (F a -> F b)}.
Плюс законы о том, что \texttt{fmap} уважает \texttt{id} и композицию.

\begin{minted}{haskell}
    class Functor f where
      fmap :: (a -> b) -> (f a -> f b)
\end{minted}

\begin{figure}[H]
    \centering
    \includegraphics[width=0.3\textwidth]{figs/functor}
\end{figure}

\vocab{Контравариантный функтор} --- пара из типового конструктора и операции на функциях, разворачивающей стрелку.
Плюс соответствующие законы.

\begin{minted}{haskell}
    class Contravariant f where
      contramap :: (a -> b) -> (f b -> f a)
\end{minted}

\begin{figure}[H]
    \centering
    \includegraphics[width=0.3\textwidth]{figs/contra-functor}
\end{figure}

Типовой конструктор можно объявить ковариантным или контравариантным функтором (или никаким из них) относительно некоторого типового параметра в зависимости от вида декларации соответствующих конструкторов данных.
А именно, от знака позиций, в которых входит этот типовой параметр в тип.

Разовьём интуитивное понимание знаков позиций.
Тип \mintinline{haskell}{A} входит в положительной позиции в \mintinline{haskell}{B} если его значение можно извлечь из \mintinline{haskell}{B}.
И наоборот, тип \mintinline{haskell}{A} входит в отрицательной позиции, если его значение нужно, наоборот, предоставить.
Рассмотрим знаки позиций типов в базовых типовых конструкторах:
\begin{center}
    \begin{tabular}[h]{|c|c|c|}
        \hline
        Тип                              & знак позиции \mintinline{haskell}{A} & знак позиции \mintinline{haskell}{B} \\
        \hline
        \mintinline{haskell}{Either A B} & $+$                                  & $+$                                  \\
        \mintinline{haskell}{(A, B)}     & $+$                                  & $+$                                  \\
        \mintinline{haskell}{A -> B}     & $-$                                  & $+$                                  \\
        \hline
    \end{tabular}
\end{center}

Действительно, из суммы и произведения можно извлечь компоненты с помощью паттерн-матчинга, а из стрелки можно получить правый тип апплицируя её к аргументу.
В то же время значение типа слева от стрелки нужно предоставить.

На плюс и минус действуют интуитивные алгебраические законы при рассмотрении более сложных типов.
Рассмотрим на примере \mintinline{haskell}{f :: ((A, B) -> C) -> (D, E)}.
\begin{itemize}
    \item Плюс на плюс даёт плюс.
    Действительно, нужно лишь применить две элиминации вместо одной, чтобы получить заветный тип.
    В нашем примере, чтобы получить \mintinline{haskell}{D}, нужно сначала апплицировать функцию, а потом разобрать пару.
    \item Плюс на минус (и наоборот) даёт минус.
    Действительно, \mintinline{haskell}{C} нам нужно предоставить: \mintinline{haskell}{f (\ab -> provideC)}.
    \item Минус на минус даёт плюс.
    Пару \mintinline{haskell}{(A, B)} нам предоставляют: \mintinline{haskell}{f (\ab -> ...)}.
\end{itemize}

\begin{task}
    Убедитесь что плюс на минус даёт минус.
\end{task}

Возвращаясь к функторам, если типовой параметр входит в декларацию только в положительных позициях, типовой конструктор можно объявить ковариантным функтором относительно этого параметра.
Если в только в отрицательных~--- контравариантным функтором.
Если в обоих, то никаким функтором объявить нельзя.
Соответственно, будем называть типовые параметры ковариантными, контравариантными и инвариантными.

\begin{task}
    Объявите \mintinline{haskell}{instance Contravariant F} для \mintinline{haskell}{data F a = L (a -> ()) | R Int}.
\end{task}

Таким образом, можно понимать ковариантный функтор как вычисление, результат которого можно пост-обработать, а контравариантный функтор --- как вычисление, аргументы которого можно пред-обработать.

Тип от двух положительных параметров можно объявить \vocab{бифунктором}:
\begin{minted}{haskell}
    class Bifunctor f where
      bimap :: (a -> c) -> (b -> d) -> f a b -> f c d
\end{minted}

Тип от двух параметров, положительного и отрицательного, --- \vocab{профунктором}:
\begin{minted}{haskell}
    class Profunctor p where
      dimap :: (c -> a) -> (b -> d) -> p a b -> p c d
\end{minted}

Профункторы являются некоторыми обобщениями функциональной стрелки.
Например, если у нас есть SQL запрос, который по данным возвращает результат, его можно объявить профунктором с семантикой --- добавить пред-обработку входных данных и пост-обработку выходных:
\begin{minted}{haskell}
    dimap serialize deserialize (query :: Sql Text Text) :: Sql Age [User]
\end{minted}

Также понятие вариантности часто встречается в объектно ориентированных языках для обозначения возможности дополнить отношение подтипизации на полиморфные типы (да и вообще в теории подтипизации).

Действительно, \vocab{отношение подтипизации} \texttt{B <: A} говорит о том, что значение типа \texttt{B} безопасно использовать в позиции, где ожидается значение типа \texttt{A}.
Иначе говоря, существует функция \texttt{upcast :: B -> A}.
Если типовой конструктор \texttt{F a} ковариантен относительно параметра \texttt{a}, то по \texttt{upcast} найдётся \texttt{upcast' :: F B -> F A}.
То есть отношение подтипизации также автоматически включает \texttt{F B <: F A}.
Контравариантный случай аналогично.

\begin{task}
    Убедитесь в вашем любимом языке с поддержкой вариантности, что минус на минус даёт плюс.
\end{task}

\subsection{Изоморфизм} \label{subsec:isomorphism}

Пусть нам нужно спроектировать функцию или модель данных.
Мы начинаем с декларации типа, как её выбрать и из каких вариантов?
Для начала поймём, когда два типа взаимозаменимы, для этого рассмотрим понятия изоморфизма.

Два типа \texttt{A} и \texttt{B} называются \vocab{изоморфными} (обозначают \texttt{A $\iso$ B}) тогда и только тогда, когда существует такая пара функций \texttt{to :: A -> B} и \texttt{from :: B -> A}, что\footnote{Под равенством термов можно понимать разное, например, $\alpha\beta\gamma$-эквивалентность. Мы будем пользоваться \vocab{экстенсиональным равенством} для функций~--- две функции равны, когда равны их результаты на всех входах. \url{https://ncatlab.org/nlab/show/function+extensionality}}
\begin{minted}{c}
    to . from = id
    from . to = id
\end{minted}

Иначе говоря, между обитателями таких типов можно установить взаимно-однозначное соответствие.
Легко понять, что со смысловой точки зрения не принципиально, какой из изоморфных типов использовать --- их можно заменять друг на друга, добавляя вызовы функций перехода.
Такие два типа заключают в себе одинаковое ``количество информации''.
Например, типы \mintinline{haskell}|Bool| и \mintinline{haskell}|Maybe ()| в этом смысле совершенно взаимозаменимы.
Покажем это, предъявив пару взаимообратных функций\footnote{Нужно не забыть показать взаимообратность функций, но это делается тривиально перебором входов (может быть с помощью индукции) и редукцией.}:
\begin{minted}{haskell}
    to :: Bool -> Maybe ()
    to b = if b then Just () else Nothing

    from :: Maybe () -> Bool
    from m = case m of Nothing -> False; Just () -> True
\end{minted}

Несмотря на смысловую взаимозаменимость, для кодирования информации о том, передал ли пользователь программе определённый флаг, мы, скорее всего, воспользуемся типом \mintinline{haskell}|Bool| ввиду нефункциональных соображений о читабельности кода.
Аналогично можно рассматривать соображения эффективности.

С категорным взглядом на происходящее можно ознакомиться в~\cite{hinze2010reason}.
Мы же придерживаемся теоретико-множественной интерпретации типов.

\subsubsection{Кардинальность: суммы, произведения, экспоненты} \label{subsubsec:cardinality}

Типы можно воспринимать как синтаксис для записи множеств, а населяющие их термы~--- как синтаксические записи элементов этих множеств.
Так терм \mintinline{haskell}|(True, False)|~--- запись элемента множества пар, записываемого в синтаксисе типов как \mintinline{haskell}|(Bool, Bool)| (вместо математического $\mathbb{B}\times\mathbb{B}$).
Или же терм \mintinline{haskell}|\x -> x + 1| является записью функции прибавляющей единицу из множества функций над целыми числами, записываемого как \mintinline{haskell}|Integer -> Integer| (вместо математического $\mathbb{Z}\to\mathbb{Z}$).

Заметим, что два типа изоморфны, если соответствующие им множества имеют одинаковое количество элементов.
Более того, таких изоморфизмов $n!$ в случае конечности множеств.
Научимся определять количество таких элементов.
С помощью $|\cdot|$ будем записывать \vocab{кардинальность} типа~--- количество элементов в соответствующем множестве.

\begin{center}
    \begin{tabular}{|l|c|}
        \hline
        Тип и его декларация                                                                                                                                                                            & кардинальность \\
        \hline
        \mintinline{haskell}{data Void}                                                                                                                                                                 & $0$            \\
        \mintinline{haskell}{data Unit = Unit}\footnote{\mintinline{haskell}|Unit| записывается в Haskell с помощью специального синтаксиса \mintinline{haskell}|()|, означающем как бы пустой кортеж.} & $1$ \\
        \mintinline{haskell}{data Bool = False | True}                                                                                                                                                  & $2$            \\
        \hline
    \end{tabular}
\end{center}

Идея алгебраических типов данных в том, что сложные типы можно строить из простых с помощью операции $+$ (``или'') и операции $\times$ (``и'')\footnote{\url{https://stanford-cs242.github.io/f18/lectures/02-2-algebraic-data-types.html}}:
\begin{center}
    \begin{tabular}{|l|c|}
        \hline
        Тип                                                      & кардинальность   \\
        \hline
        \mintinline{haskell}{data Either a b = Left a | Right b} & $|a| + |b|$      \\
        \mintinline{haskell}{data Pair a b = Pair a b}           & $|a| \times |b|$ \\
        \hline
    \end{tabular}
\end{center}

Посчитаем количество обитателей различных типов (вы можете убедиться в справедливости заключения перебрав все термы вручную):
\begin{itemize}
    \item \mintinline{haskell}{|Either Unit (Eigher Bool Bool)| = |Unit| + (|Bool| + |Bool|) = 5}.
    \item \mintinline{haskell}{Pair (Either Bool Unit) (Pair Unit Void)| = 0} ~--- тип \mintinline{haskell}|Void| не населён, как и кортеж, его включающий.
    \item Если \mintinline{haskell}{data Example = FirstAlternative Bool | AnotherOne Unit Bool Bool}, то \\\mintinline{haskell}{|Example| = |Bool| + |Unit| * |Bool| * |Bool| = 2 + 1 * 2 * 2 = 6}.
\end{itemize}

Функциональную стрелку называют экспоненциальным типом.
Действительно, комбинаторно количество обитателей \mintinline{haskell}|A -> B| вычисляется как \[|A \to B| = |B|^{|A|}\]

Так как проектировать типы?
Тому есть несколько соображений:
\begin{itemize}
    \item В типе должно быть не меньше элементов, чем в предметной области, все необходимые объекты были представимы.
    \item В типе должно быть как можно меньше элементов, которых нет в предметной области, чтобы пространство ошибок было минимальным.
    \item Далее среди изоморфных типов выбирается оптимальный исходя из функциональных и нефункциональных требований.
\end{itemize}

Прежде чем работать с некоторым объектом предметной области, информацию о нём, в соответствии со вторым правилом, следует привести в максимально структурное представление, дающее наибольшее количество гарантий\footnote{\url{https://lexi-lambda.github.io/blog/2019/11/05/parse-don-t-validate/}}.

\subsubsection{Алгебраическое представление типа} \label{subsubsec:type-algebra}

Как мы увидели выше, чтобы показать наличие изоморфизма между двумя типами можно либо предъявить пару взаимообратных функций, либо показать, что кардинальности этих двух типов совпадают.
В этом разделе мы научимся сопоставлять типу некоторую алгебраическую запись, отражающую его структуру и кардинальность.
Так, мы сможем синтаксическими преобразованиями формул получать эквивалентные записи, из которых будем восстанавливать типы, заведомо изоморфные данному\footnote{\url{https://codewords.recurse.com/issues/three/algebra-and-calculus-of-algebraic-data-types}}.

В основу алгебраического представления положим вычисление кардинальности типов:
\begin{center}
    \begin{tabular}{|p{0.5\textwidth}|c|}
        \hline
        Тип                                                      & алгебраическая формула      \\
        \hline
        \mintinline{haskell}{data Void}                          & $0$                         \\
        \mintinline{haskell}{data Unit = Unit}                   & $1$                         \\
        \mintinline{haskell}{data Bool = False | True}           & $1 + 1$ (обозначим как $2$) \\
        \mintinline{haskell}{data Maybe a = Nothing | Just a}    & $1 + a$                     \\
        \mintinline{haskell}{data Either a b = Left a | Right b} & $a + b$                     \\
        \mintinline{haskell}{data Pair a b = Pair a b}           & $a \times b$                \\
        \mintinline{haskell}{a -> b}                             & $b^a$                       \\
        \hline
    \end{tabular}
\end{center}

\begin{task}
    Запишите в алгебраическом виде следующий тип:
    \begin{minted}{haskell}
        data T a b = Undefined | Defined a (a -> b)
    \end{minted}
\end{task}

В качестве отношения эквивалентности, будем использовать изоморфизм соответствующих типов.
В такой интерпретации, классические свойства операций сохраняются (рис.~\ref{fig:school-alg}).
Действительно, например:
\begin{minted}{haskell}
    -- ?$(c^b)^a \iso c^{a\times b}$?
    to :: (a -> b -> c) -> (a, b) -> c
    to = uncurry
    from :: ((a, b) -> c) -> a -> b -> c
    from = curry
\end{minted}

\begin{figure}
    \centering
    \includegraphics[width=0.4\textwidth]{figs/school-alg}
    \caption{Законы школьной алгебры ностальгии ради~\cite{hinze2010reason}.}
    \label{fig:school-alg}
\end{figure}

\begin{task}
    Покажите, что $(a + b) + c \iso a + (b + c)$.
\end{task}

\begin{task}
    Покажите, что $c^{a + b} \iso c^a\times c^b$.
\end{task}

Интересным наблюдением может быть то, что функции можно использовать как структуры данных, в соответствие с изоморфизмом $c^{a + b} \iso c^a\times c^b$.
Действительно, в таком случае аргумент функции выступает индексом (его кардинальность должна совпадать с размером коллекции).
Это очень важный изоморфизм, мы к нему вернёмся, чтобы проложить путь к tagless-final интерпретаторам и коданным в главе~\ref{sec:wonder-interpreters}.
\begin{minted}{haskell}
    -- ?$a \times a \iso a^2$?
    get :: (a, a) -> (Bool -> a)
    get (x, y) idx = if idx then x else y
    tabulate :: (Bool -> a) -> (a, a)
    tabulate f = (f True, f False)
\end{minted}

\vocab{Каноническим предствлением типа (canonical representaion)} называют сумму произведений типов:
\[
    \sum_{i}\prod_{j} t_{ij}
\]
Каноническое представление является своего рода нормальной формой, в которой можно записывать алгебраические типы (любой алгебраический тип можно по правилам привести к ней).
Легко узнать в нём вид \mintinline{haskell}{data} деклараций в Haskell.
В дальнейшем мы воспользуемся каноническим представлением для обобщённой работы с типами (глава.~\ref{sec:datatype-generic}).

% todo производная

\subsection{Рекурсивные типы и схемы} \label{subsec:recursive-types}

В этом разделе мы научимся рассуждать о рекурсивных типах, получать их различные формы.
А так же построим универсальную свёртку.

Рассмотрим классический функциональный список.
Список это либо коллекция из нуля элементов, либо одного, либо двух\ldots
Алгебраически это запишется следующим образом:
\[
    L = 1 + a + a^2 + a^3 + \ldots
\]
Фактически получили тип с бесконечной записью.
Поработаем с ним как с формальным рядом.
Вынесем $a$ за скобки:
\[
    L = 1 + a \times (1 + a + a^2 + \ldots)
\]
Заметим, что выражение в скобках представляет собой список, получим такое рекурсивное уравнение\footnote{Либо можно получить то же самое, заметив, что мы имеем дело с рядом Тейлора \url{https://codewords.recurse.com/issues/three/algebra-and-calculus-of-algebraic-data-types}.}:
\[
    L = 1 + a \times L
\]
Легко видеть, что это на самом деле знакомое нам определение списка из Haskell:
\begin{minted}{haskell}
    data List a = Nil | Cons a (List a)
\end{minted}

% todo Смысл рекурсии в том, чтобы в виде конечной записи записать нечто, что в результате некоторого процесса ``разворачивания'' может приобретать сколь угодно

\subsubsection{Неподвижная точка функтора}

Решим полученное рекурсивное уравнение в стиле $\lambda$-исчисления, с помощью некоторого комбинатора неподвижной точки:
\[
    L = FIX\ap\lambda r\ldotp 1 + a \times r
\]
Перепишем на Haskell:
\begin{minted}{haskell}
    data Fix :: (* -> *) -> *
    data Fix f = In { out :: f (Fix f) }

    data ListShape a r = Nil' | Cons' a r
    data List' a = Fix (L a)
\end{minted}

\begin{task}
    Какие типы будут у \mintinline{haskell}|In| и \mintinline{haskell}|out|?
\end{task}

Фактически мы разбили рекурсивную структуру данных на две, одна отвечает за рекурсию, вторая --- за форму дерева.
Иначе говоря, вместо того, чтобы сослаться на себя, тип абстрагируется по рекурсивной ссылке, которую ему предоставят снаружи.
Эта техника называется \vocab{открытой рекурсией}.
Так, пользователь может контролировать, рекурсию, чем мы в дальнейшем будем пользоваться.

Можно показать, что \mintinline{haskell}{List a ?$\iso$? List' a}:
\begin{minted}{haskell}
    to :: List a -> List' a
    to = \case
      Nil -> In Nil'
      Cons x xs -> In $ Cons' x (to xs)

    from :: List' a -> List a
    from (In shape) = case shape of
      Nil' -> Nil
      Cons' x xs -> Cons x (from xs)
\end{minted}

Тип формы можно сделать функтором по последнему параметру.
Это позволит нам в дальнейшем заменять вхождения поддеревьев на что-то полезное.
\begin{minted}{haskell}
    instance Functor (ListShape a) where
      fmap :: (rec -> other) -> ListShape a rec -> ListShape a other
      fmap f = \case
        Nil' -> Nil'
        Cons' x xs -> Cons' x (f xs)
\end{minted}

Таким образом, мы научились кодировать рекурсивный тип в стиле теории категорий, как неподвижную точку функтора.

\begin{task}
    Выразите следующее дерево как неподвижную точку функтора.
    Объявите инстанс функтора для типа-формы.
    \begin{minted}{haskell}
        data Tree a = Leaf a | Node a (Tree a) (Tree a)
    \end{minted}
\end{task}

% todo населить всё это ссылками

\subsubsection{Рекурсивные схемы}

% todo coinduction

TODO % todo

\subsection{Немного категорий} \label{subsec:cats}

Это факультативный раздел, не являющийся необходимым для понимания курса в дальнейшем.

\subsubsection{Категории и алгебры}

Посмотрим совсем немного теории категорий.

\vocab{Категория} --- это коллекция объектов и коллекция стрелок.
Для каждого объекта $X$ существует тождественная стрелка, а для каждой пары стрелок существует способ получить их композицию: $f : Y \to Z, g : X \to Y \Rightarrow f \circ g : X \to Z$.

Определяют категорию, соответствующую Haskell --- $Hask$.
На самом деле это плохая категория с точки зрения теории, но для наших нестрогих рассуждений подойдёт\footnote{\url{https://math.andrej.com/2016/08/06/hask-is-not-a-category/}}.
Объектами в Hask являются типы языка Haskell, а морфизмами --- термы, задающие функции между соответствующими типами.
Тождественный морфизм --- \mintinline{haskell}|id|, композиция задаётся как \mintinline{haskell}|f . g = \x -> f (g x)|.

\begin{task}
    Как в такой категории представить константы?
\end{task}

% todo Category type class

\vocab{Функтором} называется отображение между категориями, которое объектам одной категории сопоставляет объекты другой, а стрелкам одной --- стрелки другой.
В Haskell типовые конструкторы задают отображение между объектами, а \mintinline{haskell}|fmap| --- между стрелками.
Функтор должен сохранять тождественный морфизм и композицию.
\begin{figure}[h!]
    \centering
    \includegraphics[width=0.25\textwidth]{figs/functor}
\end{figure}

\vocab{Алгеброй} в категории $C$ называется пара из объекта категории $X \in Obj(C)$ и морфизма $\phi : F\ap X \to X$, где $F$ --- функтор.
Сам морфизм $F\ap X \to X$ называют \vocab{f-алгеброй}.
Алгебрами в смысле категорий можно описывать алгебры.
Так, в качестве объекта $X$ берём носитель алгебры.
В качестве функтора $F$ --- сигнатуру алгебры в виде типа-формы.
Тогда морфизмом будет интерпретация сигнатуры.
\begin{minted}{haskell}
    data MonoidSig carrier = Mempty | Mappend carrier carrier

    interpretSig :: MonoidSig Int -> Int
    interpretSig = \case Mempty -> 0; Mappend l r -> l + r
\end{minted}

\begin{task}
    Реализуйте алгебру печати, какой тип должен выступить носителем?
\end{task}

\vocab{Морфизмом алгебр} называется такой морфизм между носителями $h : X \to Y$, что следующая диаграмма коммутирует.
Говорят, что \vocab{диаграмма коммутирует}, если все возможные пути по стрелкам в ней равны.
\begin{figure}[h!]
    \centering
    \includegraphics[width=0.3\textwidth]{figs/alg-homomorphism}
\end{figure}

В морфизме алгебр можно обнаружить знакомые черты гомоморфизмов, то есть операций между носителями, которые ``уважают'' операции сигнатуры алгебраической теории.

Алгебры над категорией $C$ образуют \vocab{категорию алгебр}, в которой объектами являются алгебры, а морфизмами --- морфизмы алгебр.

\subsubsection{Рекурсивные типы в категориях}

\vocab{Начальным (инициальным) объектом} категории называется объект, из которого в каждый другой объект существует уникальная стрелка.
\vocab{Терминальным (финальным) объектом} категории называется объект, в который из каждого другого объекта категории существует уникальная стрелка.

Инициальный и терминальный объекты категории не обязательно присутствуют в единственном экземпляре.
Но все инициальные объекты изоморфны друг другу, как и все терминальные.

\begin{task}
    Приведите начальный и терминальный объекты категории $Hask$.
\end{task}

Рекурсивный тип --- это тип, значит ему соответствует объект в категории Hask.
$X$ является рекурсивным типом с формой $F$, если имеет место следующий изоморфизм:
\[X \simeq F\ap X\]

Можно заметить, что свидетель изоморфизма справа налево напоминает f-алгебру, а слева направо --- f-коалгебру (всё то же самое, только все стрелки в обратную сторону).
И действительно, подходящий объект $X$ должен быть либо начальным объектом категории алгебр, либо терминальным объектом категории коалгебр (с соответствующими морфизмами).
Первый вариант соответствует конечным структурам данных, второй --- потенциально бесконечным.

Начальным объектом категории алгебр над $Hask$ для функтора $f$ является следующая алгебра: \mintinline{haskell}|(Fix f, In)| (покажем это далее).
Более того, терминальным объектом категории коалгебр будет \mintinline{haskell}|(Fix f, out)|, благодаря ленивости Haskell.

% todo

\subsubsection{Катаморфизмы: туда и обратно}

Покажем, что \mintinline{haskell}|(Fix f, In)| является инициальным\footnote{\url{https://bartoszmilewski.com/2017/02/28/f-algebras/}}\footnote{\url{https://ncatlab.org/nlab/show/catamorphism}}.
Действительно, для каждого типа \texttt{a} и для каждой f-алгебры \texttt{phi} мы можем построить такой морфизм \mintinline{haskell}|cata phi :: Fix f -> a|, что следующая диаграмма будет коммутировать.
\begin{figure}[h!]
    \centering
    \includegraphics[width=0.4\textwidth]{figs/cata}
\end{figure}

Из диаграммы сразу видно, как такой морфизм построить.
Его называют \vocab{катаморфизмом} f-алгебры \texttt{phi}.
Он является универсальной свёрткой рекурсивных структур данных\footnote{\url{https://reasonablypolymorphic.com/blog/recursion-schemes/index.html}}~\cite{meijer1991functional, meijer1995bananas}, с которой начинается наука рекурсивных схем, ``структурного'' функционального программирования без неструктурной рекурсии.
\begin{minted}{haskell}
    cata :: Functor f => (f a -> a) -> Fix f -> a
    cata phi = phi . fmap (cata phi) . out
\end{minted}

Катаморфизм сначала обрабатывает рекурсивные ссылки, добираясь к ним с помощью \mintinline{haskell}|fmap|, и сворачивает поддеревья, оставляя результаты вместо бывших вхождений поддеревьев.
Таким образом, получается тип формы, у которого вместо рекурсивных ссылок уже значения нужного типа \texttt{a}, а его уже можно непосредственно свернуть с помощью f-алгебры.

\begin{figure}[h]
    \centering
    \includegraphics[width=0.8\textwidth]{figs/cataStep.excalidraw}
\end{figure}

\begin{task}
    Убедитесь, что печатающая алгебра действительно сворачивает терм в строчку.
\end{task}

Покажем, что \mintinline{haskell}|(Fix f, In)| является ещё и терминальным объектом.
Аналогично, для каждого объекта \texttt{a} и f-коалгебры \texttt{psi} найдётся морфизм \mintinline{haskell}|ana psi :: a -> Fix f|.
\begin{figure}[h!]
    \centering
    \includegraphics[width=0.35\textwidth]{figs/ana}
\end{figure}

\begin{minted}{haskell}
    ana :: Functor f => (a -> f a) -> a -> Fix f
    ana psi = In . fmap (ana psi) . psi
\end{minted}

Анаморфизм является универсальной развёрткой.
f-коалгебра показывает, как из некоторого значения получить один слой структуры данных (тип-формы).
Он вместо рекурсивных ссылок хранит зёрнышки, из которых потом прорастут поддеревья.
Анаморфизм как раз сначала разворачивает один слой, а потом рекурсивно разворачивает все поддеревья.

Анаморфизм проясняет интуицию, почему терминальными коалгебрами кодируют потенциально бесконечные структуры данных.
f-коалгебра по некоторому зерну вычисляет следующий слой структуры.
Так, можно слой за слоем лениво вычислять поддеревья, потенциально сколь угодно долго.

TODO % todo


    \clearpage


    \section{Параметрический полиморфизм} \label{sec:parametric-polymorphism}

    %! suppress = MissingLabel

Никакое нетривиальное свойство программ не может быть алгоритмически проверено\footnote{\url{https://en.wikipedia.org/wiki/Rice\%27s_theorem}}.
Чтобы оставаться разрешимыми (в смысле проверки типов и/или вывода), многие системы типов жертвуют полнотой и, помимо некорректных программ, отвергают много корректных.
В то же время системы типов также стараются предоставлять различные возможности, позволяющие протипизировать как можно больше корректных программ.
Одна из них --- параметрический полиморфизм.

Под \vocab{параметрическим полиморфизмом} мы будем подразумевать возможность кода единообразно работать с произвольными типами данных~\cite{strachey2000fundamental, cardelli1985understanding}, что позволяет во многих случаях избегать дублирования кода.

В этой главе мы рассмотрим, как описывают полиморфизм в самом простом виде --- в типизированном $\lambda$-исчислении.
Изучим различные формы параметрического полиморфизма и сопутствующие техники безопасного программирования.
Проанализируем возможные способы эффективной реализации параметрического полиморфизма.
И в завершение рассмотрим полиморфизм по рантайм-представлению, ``полиморфизм по полиморфизму''.

\subsection{Параметрический полиморфизм в языке} \label{subsec:lang-parametric-polumorphism}

$\lambda$-абстракция позволяет обобщать выражения по значениям, каждая абстракция добавляет стрелку в тип выражения.
Аппликация же снимает стрелку.
\[
    \begin{array}{cc}
        \infer[Lam]{\Gamma\vdash \lambda x : \tau\ldotp M : \tau\to\sigma}{x : \tau, \Gamma\vdash M : \sigma}
        &
        \infer[App]{\Gamma\vdash M\ap N : \sigma}{\Gamma\vdash M : \tau\to\sigma & \Gamma\vdash N : \tau}
    \end{array}
\]
В то же время $\Lambda$-абстракция позволяет обобщать выражения по типам, добавляя квантор в тип ($\Pi$-абстракцию)~\cite[глава 23]{pierce2002types}:
\[
    \begin{array}{cc}
        \infer[TLam]{\Gamma\vdash\Lambda\alpha\ldotp M : \forall\alpha\ldotp\tau}{\Gamma\vdash M : \tau}
        &
        \infer[TApp]{\Gamma\vdash M\ap\sigma : [\alpha\to\sigma]\ap\tau}{\Gamma\vdash M : \forall\alpha\ldotp\tau}
    \end{array}
\]

Теперь, например, мы можем дать возможность пользователю выбрать, с каким типом он хочет использовать нашу функцию (применение к типу называют \vocab{универсальной аппликацией (universal application)}):
\[
    \begin{array}{ll}
        id : \forall\alpha\ldotp\alpha\to\alpha            \\
        id = \Lambda\alpha\ldotp\lambda x : \alpha\ldotp x \\
        id \ap nat : nat \to nat                           \\
        id\ap nat \ap 42 : nat
    \end{array}
\]
Функция $id$ фактически принимает два аргумента: тип и значение.

В Haskell типовые абстракции и аппликации приписываются неявно механизмом вывода типов.
Однако, есть расширения языка, которые позволяют их написать явно: \href{https://downloads.haskell.org/ghc/latest/docs/users_guide/exts/type_abstractions.html}{TypeAbstractions}, \href{https://downloads.haskell.org/ghc/latest/docs/users_guide/exts/type_applications.html}{TypeApplications}.
Это может помочь, например, когда информации из терма не достаточно, чтобы вывести тип.
Так, можно явно специализировать \texttt{id} на нужный тип:
\begin{minted}{haskell}
    id :: forall a . a -> a
    ghci> :t id @Int
    id @Int :: Int -> Int
\end{minted}

Кванторы также приписываются неявно в начале типа, следуя конвенции именования: конкретные типы начинаются с большой буквы, а полиморфные --- с маленькой.
Аналогично, у пользователя есть возможность явно приписывать \mintinline{haskell}|forall|'ы с помощью расширения \href{https://downloads.haskell.org/ghc/latest/docs/users_guide/exts/explicit_forall.html\#extension-ExplicitForAll}{ExplicitForAll}.
Это может понадобиться либо за тем, чтобы задать вручную порядок типовых абстракций, либо, чтобы иметь возможность сослаться на абстрагированный тип в теле функции (расширение \href{https://downloads.haskell.org/ghc/latest/docs/users_guide/exts/scoped_type_variables.html#extension-ScopedTypeVariables}{ScopedTypeVariables}).

Полиморфные типы данных задаются с помощью другой конструкции.
Если ранее мы управляли типом с уровня термов универсальной аппликацией, то теперь мы хотим управлять типом на уровне типов.
Для этого мы вводим $\lambda$ абстракцию в типах, аппликацию в типах и, соответственно, $\beta$-редукцию.
Система кайндов (пока) представляет собой простейшую ``систему типов для типов'' и обеспечивает well-formedness типов и строгую нормализуемость\footnote{\vocab{Строгая нормализуемость} --- любой порядок редукций приводит к нормальной форме.}.
Например, мы можем написать тип пары, абстрагированный от конкретных типов компонент, чтобы пользователь мог выбрать нужные ему.
\begin{align*}
    &Pair : * \rightarrow * \rightarrow * \\
    &Pair = \lambda \tau^*~\sigma^*\ldotp\forall \gamma\ldotp(\tau\rightarrow\sigma\rightarrow\gamma)\to\gamma \\
    &pair : \forall \alpha~\beta\ldotp\alpha \rightarrow \beta \rightarrow Pair~\alpha~\beta \\
    &pair = \Lambda \alpha^*~\beta^*\ldotp\lambda x^\alpha~y^\beta\ldotp(\Lambda \gamma^*\ldotp\lambda f^{\alpha\rightarrow\beta\rightarrow\gamma}\ldotp f~x~y) \\
    &fst : \forall \alpha~\beta\ldotp Pair~\alpha~\beta\rightarrow \alpha \\
    &fst = \Lambda \alpha^*~\beta^*\ldotp\lambda p^{Pair~\alpha~\beta}\ldotp p~\alpha~(\mathbf{K}\ap\alpha\ap\beta)
\end{align*}

В Haskell вычислительную семантику полиморфных типов можно проследить в синонимах типов:
\begin{minted}{haskell}
    type Pair a b = forall c . (a -> b -> c) -> c
    intPair :: Pair Int Int -- forall c . (Int -> Int -> c) -> c
\end{minted}
Обычные конструкторы типов номинативны.
Например, \mintinline{haskell}{(Int, Int)} или \mintinline{haskell}{Maybe Int} никуда далее не вычисляются.

Haskell не позволяет создавать функции на типах по месту с помощью явной типовой лямбды\footnote{\url{https://stackoverflow.com/questions/4069840/lambda-for-type-expressions-in-haskell}} ввиду проблематичности этой конструкции для вывода типов.
Однако полноценные функции на типах есть, и мы рассмотрим их далее~\ref{subsec:families}.
В Scala существует нетривиальный трюк\footnote{\href{https://stackoverflow.com/questions/8736164/what-are-type-lambdas-in-scala-and-what-are-their-benefits}{(stackoverflow) Scala type lambdas.}}\footnote{\url{https://stackoverflow.com/questions/9443004/what-does-the-operator-mean-in-scala}}, который позволяет этого добиться.
Scala3, однако, включила эту возможность непосредственно в язык\footnote{\url{https://docs.scala-lang.org/scala3/reference/new-types/type-lambdas.html}}.

\subsubsection{Эмуляция типовых абстракций и аппликаций (\mintinline{haskell}{Proxy})} \label{subsubsec:proxy}

В Haskell расширения, позволяющие вручную задавать типовые аппликации и абстракции появились сравнительно недавно\footnote{\href{https://downloads.haskell.org/ghc/latest/docs/users_guide/exts/type_applications.html\#extension-TypeApplications}{TypeApplications}, \href{https://downloads.haskell.org/ghc/latest/docs/users_guide/exts/type_abstractions.html\#extension-TypeAbstractions}{TypeAbstractions}.}.
До этого пользовались следующей техникой.

В стандартной библиотеке определён тип \mintinline{haskell}|Proxy| с одним параметром.
Это \vocab{фантомный типовой параметр}~--- значения соответствующего типа не хранятся в структуре данных, он только позволяет размещать дополнительную информацию на уровне типов\footnote{\url{https://wiki.haskell.org/Phantom_type}}.
Соответственно, неинформативную константу \mintinline{haskell}|Proxy| можно проаннотировать нужным типом и передать в функцию, чтобы специализировать типовой параметр на нужный тип.
Или можно принять \mintinline{haskell}|Proxy| и воспользоваться \href{https://downloads.haskell.org/ghc/latest/docs/users_guide/exts/scoped_type_variables.html\#pattern-type-signatures}{ScopedTypeVariables} для типовых сигнатур в паттернах\footnote{Типовый параметр на самом деле имеет полиморфные кайнд \mintinline{haskell}{data Proxy (a :: k) = Proxy}, чтобы эта техника работала с типами произвольных кайндов (см. далее\ \ref{subsubsec:promotion}.}.
\begin{minted}{haskell}
    data Proxy a = Proxy

    id :: Proxy a -> a -> a
    ghci> :t id (Proxy :: Proxy Int)
    id (Proxy :: Proxy Int) :: Int -> Int

    id (Proxy :: Proxy ?\framebox{a}?) x = (x :: ?\framebox{a}?)
\end{minted}

Иногда прокси-тип оставляют полиморфным, чтобы пользователь сам мог его задать.
Вместо конкретного значения иногда передают специализированное значение $\bot$, а получатель, не зная тип, не сможет его форсировать (однако, любые вхождения $\bot$ в терм слишком настораживают, поэтому это скорее не очень хорошая практика).
\begin{minted}{haskell}
    id :: proxy a -> a -> a
    id (_ :: proxy a) x = (x :: a)

    ghci> :t id (undefined :: Proxy Int)
    id (undefined :: Proxy Int) :: Int -> Int
\end{minted}

\subsubsection{First-class polymorphism} \label{subsubsec:first-class-polymorphism}

Существует возможность писать функции, которые принимают другие полиморфные функции в качестве аргументов.
Типы таких функций называются \vocab{типами высшего ранга (higher-rank types)}, их можно использовать с расширением \href{https://downloads.haskell.org/ghc/latest/docs/users_guide/exts/rank_polymorphism.html}{RankNTypes}.
Так, типовой параметр функции \texttt{g} определяет функция \texttt{f}, а не вызывающий функцию \texttt{f}:
\begin{minted}{haskell}
    f :: (forall a . a -> a) -> (Int, Char)
    f g = (g @Int 42, g @Char 'a') -- универсальная аппликация для наглядности
    ghci> f (\x -> x)
\end{minted}

Проблема типов высшего ранга в том, что их вывод неразрешим, то есть глобальный вывод типов Haskell в этом случае перестаёт работать.
Но если типы высшего ранга приписать вручную, остальной вывод будет работать как раньше.
Например, числа Чёрча имеют высший ранг\footnote{\url{https://okmij.org/ftp/tagless-final/course/Boehm-Berarducci.html}}\footnote{Любому интересующемуся языками программирования предлагается провести на сайте Олега Киселёва не один месяц жизни: \url{https://okmij.org/ftp/README.html}.}:
\begin{minted}{haskell}
    suc :: (forall a . (a -> a) -> a -> a) -> (a -> a) -> a -> a
    suc n s z = s (n s z)
\end{minted}

%! suppress = LineBreak
\begin{task}
    Какой ранг имеет тип \mintinline{haskell}|Int -> (forall a . a -> a)|?
\end{task}

От многих проблем сопутствующих типам высших рангов можно избавиться, если создавать для них обёртки.
Например, для чисел Чёрча можно создать обёртку \mintinline{haskell}{newtype Church}.
Теперь код, работающий с обёрткой, может быть протипизирован типами первого ранга, только конструктор имеет тип высшего ранга.
\begin{minted}{haskell}
    newtype Church = Church (forall a . (a -> a) -> a -> a)
    (+) :: Church -> Church -> Church -- rank 1
\end{minted}
Аналогичный код можно написать и в Java (Kotlin):
\begin{minted}{kotlin}
    interface Church { fun <a> fold(s: (a) -> a, z: a): a }
    fun plus(n: Church, m: Church): Church = object : Church {
        override fun <a> fold(s: (a) -> a, z: a): a = n.fold(s, m.fold(s, z))
    }
\end{minted}

По умолчанию типовые параметры можно специализировать только на конкретные типы.
Расширение \href{https://downloads.haskell.org/ghc/latest/docs/users_guide/exts/impredicative_types.html}{ImpredicativeTypes} позволяет специализировать типовые параметры на полиморфные типы (включающие \mintinline{haskell}|forall|'ы внутри себя) --- \vocab{импредикативное применение}.
\begin{minted}{haskell}
    runST :: (forall s. ST s a) -> a
    ($) :: forall a b . (a -> b) -> a -> b
    foo = runST $ ... -- типизируется только с ImpredicativeTypes
\end{minted}

Higher-rank типы можно использовать как type-based escape analysis, иначе говоря, не позволять пользователю передавать некоторое значение вовне определённого скоупа.
Так, например, Haskell предоставляет эффективную монаду \mintinline{haskell}{ST}, позволяющую в рамках ограниченного скоупа работать с мутабельными ячейками памяти~\cite{launchbury1995state}\cite[7.2, ST trick]{maguire-types}:
\begin{minted}{haskell}
    newtype ST s a = ST (IO a)
    runST :: (forall s. ST s a) -> a

    sumTo :: Int -> Int
    sumTo n = runST do
      ref <- newSTRef 0
      forM [0..n] \i -> modifySTRef ref (+ i)
      readSTRef ref
\end{minted}
Заметим, что если попытаться вернуть из \texttt{runST} ссылку на мутабельную ячейку, то результирующий тип не пройдёт well-formedness проверку, так как будет содержать фантомный параметр \texttt{s}, который не будет нигде связан:
\begin{minted}{haskell}
    newSTRef :: a -> ST s (Ref s a)
    ghci> runST (newSTRef 0) :: Ref s Int -- ошибка
\end{minted}
На практике, чтобы отличать такие локально связанные типовые переменные, используют концепцию уровней\footnote{\url{https://okmij.org/ftp/ML/generalization.html}}~\cite{peytonjones2019typeinference}.

Типы высших рангов вместе с импредикативным применением образуют \vocab{полиморфизм первого класса (first-class polymorphism)}, когда полиморфные типы могут использоваться почти так же свободно, как и любые другие.
Классический алгоритм глобального вывода Хиндли-Милнера не справляется (и в общем случае задача неразрешима), так что существует большое количество решений, делающих различные компромиссы.
Можно сделать вывод типов локальным, опирающемся только на соседние ноды AST и вспомогательные типовые аннотации~\cite{pierce2000local, christiansen2013bidirectional, dunfield2019sound}.
Либо же можно попытаться помочь глобальному выводу дополнительной предобработкой (Quick Look\footnote{\href{https://youtu.be/ZuNMo136QqI?si=qp8PAEeeF-bioCB_}{(youtube) A Quick Look at Impredicativity (Simon Peyton Jones)}}~\cite{serrano2020quick}, реализованный в Haskell с недавнего времени) или дополнительными регулирующими конструкциями (FreezeML~\cite{emrich2020freezeml}).

\subsubsection{Higher-order/kinded polymorphism}

Haskell позволяет также абстрагироваться по типам произвольных кайндов, а не только \mintinline{haskell}|Type|, как в \mintinline{haskell}{data} декларациях (\vocab{higher-order/kinded types (HKT)}\footnote{\url{https://serokell.io/blog/kinds-and-hkts-in-haskell}}), так и в полиморфных функциях.
Далее мы встретим немало примеров.
Так, \mintinline{haskell}{Fix} имеет кайнд \mintinline{haskell}{(Type -> Type) -> Type}, а катаморфизм абстрагирован по типу стрелочного кайнда:
\begin{minted}{haskell}
    newtype Fix f = Fix (f (Fix f))
    cata :: forall (f :: Type -> Type) a . Functor f => (f a -> a) -> Fix f -> a
\end{minted}

Далее мы рассмотрим технику, позволяющую типы высших порядков закодировать в языке, их не поддерживающем (см. далее~\ref{subsubsec:simulating-hkt}).

% todo kan extensions
%Функции, возвращающие значение какого-то такого вида \texttt{m a} --- слишком полиморфные.
%Код, использующий такие функции может сталкиваться с серьёзными проблемами в производительности, так как компилятор не знает конкретного типа и не может заинлайнить соответствующие вызовы \texttt{fmap}, bind, и т.д.
%Чтобы обойти эту проблему, рекомендуется пользоваться либо конкретными типами, либо воспользоваться библиотекой kan-extensions\footnote{\url{https://hackage.haskell.org/package/kan-extensions}}\footnote{\url{https://bartoszmilewski.com/2017/04/17/kan-extensions/}}~\cite[13.5]{maguire-types}.

\subsubsection{Обобщённые алгебраические типы данных (GADTs)} \label{subsubsec:gadts}

Обобщённые алгебраические типы данных (generalized algebraic data types, GADTs) позволяют приписывать данным на уровне типов больше информации.
В качестве модельного примера возьмём синтаксис крошечного языка программирования.
Зададимся целью не допустить возможности конструирования в Haskell некорректных с точки зрения типов синтаксических деревьев.
\begin{minted}{haskell}
    data Expr = Const Int | IsZero Expr | If Expr Expr Expr
\end{minted}

Как мы знаем, конструкторы данных в Haskell --- это обычные функции с той лишь разницей, что их реализация генерируется компилятором (аллокация памяти, размещение полей\ldots).
У функций есть тип.
Например, \mintinline{haskell}|IsZero :: Expr -> Expr|.

В Haskell есть синтаксис определения \mintinline{haskell}|data| через задание типов конструкторов\footnote{\url{https://downloads.haskell.org/ghc/latest/docs/users_guide/exts/gadt_syntax.html\#gadt-style}}.
Он совершенно аналогичен рассмотренному ранее, только гораздо более удобен для сложно организованных структур данных.
Рассмотренный ранее тип термов \mintinline{haskell}|Expr| будет выглядеть следующим образом:
\begin{minted}{haskell}
    data Expr where
      Const :: Int -> Expr
      IsZero :: Expr -> Expr
      If :: Expr -> Expr -> Expr -> Expr
\end{minted}

Для полиморфных структур данных, на примере списка, используется следующий синтаксис.
Имя \texttt{elem} нужно исключительно для документации и больше никак его использовать нельзя, оно только маркирует наличие типового параметра и позволяет ему вручную задать кайнд\footnote{Кайд можно не писать. Либо можно не писать имена и просто приписать кайнд типовому конструктору: \mintinline{haskell}{data List :: Type -> Type where ...}.}.
\begin{minted}{haskell}
    data List (elem :: Type) where
      Nil :: List a
      Cons :: a -> List a -> List a
\end{minted}

Добавим к \mintinline{haskell}{Expr} фантомный типовой параметр \texttt{ty}, обозначающий тип Haskell, в который должно быть проинтерпретировано данное выражение, и с помощью GADT зададим конкретные значения \texttt{ty} результирующим типам конструкторов.
Так, мы говорим, что программа сконструированная с помощью \mintinline{haskell}{Const} вычисляется в число, \mintinline{haskell}{IsZero} вычисляется в булево значение, а условное выражение --- в тип веток:
\begin{minted}{haskell}
    data Expr ty where
      Const :: Int -> Expr Int
      IsZero :: Expr Int -> Expr Bool
      If :: forall ty . Expr Bool -> Expr ty -> Expr ty -> Expr ty

    eval :: Expr ty -> ty
\end{minted}

Теперь мы можем написать безопасный типизированный интерпретатор.
Обратите внимание, что при сопоставлении с образцами конструкторов, у нас уточняется информация о типовом параметре:\footnote{Тут используется удобное расширение \href{https://downloads.haskell.org/~ghc/9.0.1/docs/html/users_guide/exts/lambda_case.html}{LambdaCase}, позволяющее не вводить лишние имена.}
\begin{minted}{haskell}
    eval :: Expr ty -> ty
    eval = \case
      Const x  -> x         -- ty ?$\sim$? Int
      IsZero t -> eval t == 0 -- ty ?$\sim$? Bool
      If c t e -> if eval c then eval t else eval e
\end{minted}

Далее мы рассмотрим как GADT в Haskell выражаются через более базовые механизмы языка~\ref{subsubsec:system-fc}.

\subsubsection{Структуры на уровне типов, data promotion} \label{subsubsec:promotion}

Чтобы обрести больший контроль корректности программ, научимся кодировать произвольные структуры данных на уровне типов.
В качестве модельной задачи зададим структуру данных, моделирующую вектор, но с контролем длины.

Для начала определим натуральные числа на уровне типов в стиле Пеано:
\begin{minted}{haskell}
    data Zero
    data Suc n
\end{minted}

\begin{task}
    Сколько обитателей типа \mintinline{haskell}|Suc (Suc Zero)|?
\end{task}

Теперь мы можем задать тип вектора, содержащий информацию о длине:
\begin{minted}{haskell}
    data Vec (size :: Type) (elem :: Type) where
      VNil :: Vec Zero a
      VCons :: a -> Vec n a -> Vec (Suc n) a

    example :: Vec (Suc (Suc Zero)) Int
    example = VCons 1 (VCons 2 VNil)
\end{minted}

Для такого типа, например, можно написать безопасную функцию \mintinline{haskell}|zip|, работающую только на векторах одинаковой длины:
\begin{minted}{haskell}
    vzip :: Vec n a -> Vec n b -> Vec n (a, b)
    vzip VNil VNil = VNil                                      -- n ?$\sim$? Zero
    vzip (VCons x xs) (VCons y ys) = VCons (x, y) (vzip xs ys) -- n ?$\sim$? Suc n'
\end{minted}

Заметьте, что в остальных ветках \mintinline{haskell}|vzip| должны возникнуть эквивалентности, начинающиеся с различных конструкторов, например, \mintinline{haskell}|Zero ?$\sim$? Suc n|.
Поскольку невозможно построить такие аргументы функции, Haskell позволяет соответствующие ветки не рассматривать.

\begin{task}
    Напишите функцию добавления в конец элемента вектора.
    Двигайтесь последовательно, заполняя типовые дыры и отслеживая возникающие эквивалентности.
\end{task}

Удивительно, но сейчас наш язык типов не типизирован.
Действительно, кайнд \mintinline{haskell}|Suc| --- \mintinline{haskell}|Suc :: Type -> Type|, соответственно ничто не мешает написать \mintinline{haskell}|Suc (Maybe Int)|.
То есть язык кайндов, который должен контролировать типы, слишком беден.
В то же время он слишком ограничивающий, поскольку не поддерживает полиморфизм, что дало начало большому количеству дублирований а ля \mintinline{haskell}|Typeable (ty :: Type)|, \mintinline{haskell}|Typeable1 (ty :: Type -> Type)|\ldots

Современный Haskell имеет расширение \href{https://downloads.haskell.org/ghc/latest/docs/users_guide/exts/type_data.html#extension-TypeData}{TypeData}, позволяющее объявлять новые типы и кайнды подобно тому, как \mintinline{haskell}{data} позволяет объявлять новые типы.
\begin{minted}{haskell}
    type data Nat = Zero | Suc Nat
\end{minted}

Теперь вектору можно приписать более точный кайнд:
\begin{minted}{haskell}
    data Vec (size :: Nat) (elem :: Type) where
      VNil :: Vec Zero a
      VCons :: a -> Vec n a -> Vec (Suc n) a
\end{minted}

\begin{task}
    Что выведет \mintinline{haskell}|ghci> :k Vec|?
\end{task}

Другим вариантом добиться того же самого является использование \href{https://downloads.haskell.org/ghc/latest/docs/users_guide/exts/data_kinds.html#extension-DataKinds}{DataKinds}~\cite{yorgey2012giving}.
Это расширение автоматически продвигает (promotion) все \mintinline{haskell}|data| декларации на уровень выше.
А именно: любой конструктор типа также становится кайндом, а конструктор данных --- конструктором типа.
Так, в примере с числами, мы можем задекларировать натуральные числа как обычно и использовать на уровне типов:
\begin{minted}{haskell}
    data Nat = Zero | Suc Nat
    ghci> :k Suc :: Nat -> Nat -- тут понятно что Suc используется как тип
\end{minted}

Поскольку типы и термы в Haskell живут в разных пространствах имён, можно называть конструкторы типов и данных одинаково.
Однако если продвинуть такой тип данных, возникнет неоднозначность: мы имеем в виду тип или продвинутый конструктор.
Haskell позволяет указать явно, что речь идёт о продвинутом конструкторе с помощью одинарной кавычки.
\begin{minted}{haskell}
    data T = T Nat
    ghci> :k T
    T :: Type      -- про конструктор типа
    ghci> :k 'T
    'T :: Nat -> T -- про продвинутый конструктор данных
\end{minted}

Не любые \mintinline{haskell}|data| декларации подходят для продвижения, в то же время \mintinline{haskell}|type data| декларации позволяют явно запросить структуру уровня типов и получить внятные ошибки, если декларация написана неправильно.

В случае продвижения полиморфного типа, мы получаем полиморфные кайнды (\href{https://downloads.haskell.org/ghc/latest/docs/users_guide/exts/poly_kinds.html}{PolyKinds}):
\begin{minted}{haskell}
    data [a] = [] | (:) a [a]
    ghci> :k '(:)
    '(:) :: forall k . k -> [k] -> [k]
\end{minted}

\begin{figure}[h]
    \centering
    %! suppress = Quote
    \begin{tabular}{|c|c|c|}
        \hline
        Term                                   & Type                                            & Kind                                            \\
        \hline
        \mintinline{haskell}|Zero|             & \mintinline{haskell}|Nat|                       & \mintinline{haskell}|Type|                      \\
        \mintinline{haskell}|[Zero, Suc Zero]| & \mintinline{haskell}|[Nat]|                     & \mintinline{haskell}|Type|                      \\
        \mintinline{haskell}|[]|               & \mintinline{haskell}|forall a. [a]|             & \mintinline{haskell}|Type|                      \\
        \mintinline{haskell}|(:)|              & \mintinline{haskell}|forall a. a -> [a] -> [a]| & \mintinline{haskell}|Type|                      \\
        & \mintinline{haskell}|'Suc 'Zero|                & \mintinline{haskell}|Nat|                       \\
        & \mintinline{haskell}|'['Zero, 'Suc 'Zero]|      & \mintinline{haskell}|[Nat]|                     \\
        & \mintinline{haskell}|'[Int, Double]|            & \mintinline{haskell}|[Type]|                    \\
        & \mintinline{haskell}|'[]|                       & \mintinline{haskell}|forall k. [k]|             \\
        & \mintinline{haskell}|'(:)|                      & \mintinline{haskell}|forall k. k -> [k] -> [k]| \\
        \hline
    \end{tabular}
    \caption{Пример продвижений в Haskell.}
    \label{fig:universes}
\end{figure}

Примеры продвижения различных конструкций можно увидеть в таблице~\ref{fig:universes}.

В качестве примера, зададим гетерогенный список, индексированный типами элементов:
\begin{minted}{haskell}
    data HList (tys :: [Type]) where
      HNil :: HList '[]
      HCons :: ty -> HList tys -> HList (ty ': tys)

    example :: HList '[Int, Bool, Double]
    example = HCons 42 $ HCons True $ HCons 12.5 HNil
\end{minted}

Структуры данных тоже могут быть полиморфными по кайндам.
Рассмотрим следующий тип \href{https://hackage.haskell.org/package/tagged-0.8.8/docs/Data-Tagged.html#t:Tagged}{\mintinline{haskell}|Tagged|}, позволяющий дополнить тип значения дополнительным типовым тегом.
Кайнд тега может быть произвольным, поэтому, например, можем использовать встроенные в систему типов константы \href{https://ghc.gitlab.haskell.org/ghc/doc/users_guide/exts/type_literals.html}{TypeLits} (другой пример использования полиморфных кайндов мы видели ранее\ \ref{subsubsec:proxy}):
\begin{minted}{haskell}
    newtype Tagged (tag :: k) (a :: Type) = Tagged a
    ghci> :t Tagged
    Tagged :: forall k (tag :: k) a. a -> Tagged tag a

    example :: Tagged ("dbId" :: Symbol) Int
    example = Tagged 42
\end{minted}

Современный Haskell в итоге пришёл к тому, что система типов не делает различий между типами и кайндами (рис.~\ref{fig:types-eq-kinds}).
В частности, \mintinline{haskell}|Type :: Type|.
Это нужно для расширения возможностей Haskell в сторону программирования с зависимыми типами путём добавления несинтаксических эквивалентностей для кайндов (\href{https://ghc.gitlab.haskell.org/ghc/doc/users_guide/exts/poly_kinds.html#extension-TypeInType}{TypeInType}).
$System~FC$ была представлена в работе~\cite{weirich2013system}\footnote{\href{https://www.youtube.com/watch?v=ISGENChlA4M&list=PLvPsfYrGz3wufQguebnCduYgQQ9UMeJRt}{(youtube) Мини-курс на русском языке про развитие Haskell в сторону зависимой типизации.}}\footnote{\href{https://www.youtube.com/watch?v=_HYI7zjkrEs&list=PLvPsfYrGz3wuVAGhNf6-i7uafXg56oqM5&index=1}{(youtube) Мини-курс на русском языке --- система вывода типов Haskell.}}. % todo при чём тут FC

\begin{figure}[h]
    \centering
    \includegraphics[width=0.99\textwidth]{figs/types-eq-kinds}
    \caption{Типы и канды --- одно~\cite{bragilevsky-haskell}.}
    \label{fig:types-eq-kinds}
\end{figure}

\subsection{Реализация параметрического полиморфизма}

\vocab{Конвенция вызова}\footnote{\url{https://en.wikipedia.org/wiki/Calling_convention}} представляет собой набор соглашений между тем как функция компилируется и как должна вызываться.
Например, функция принимает два аргумента, каждый размером в машинное слово, и возвращает один результат размером в машинное слово.
Тогда сгенерированный низкоуровневый код этой функции может, например, ожидать, что оба аргумента передаются через специальную пару регистров, а складывать результат он будет в третий.
В таком случае вызывающий код обязан предоставить аргументы в правильных регистрах и ожидать результата в некотором третьем, заранее оговоренном регистре.

В общем случае, конвенция вызова функции зависит от типов аргументов и результата.
Нужно знать как минимум их размер, чтобы понять, размещать их в регистрах или на стеке.
Нужно знать, это указатель (\vocab{reference type}) или значение само по себе (\vocab{value type}), чтобы понимать, как с ним работать.
В структурах данных нужно знать смещения полей.

Таким образом, реализация параметрического полиморфизма в языке --- это не тривиальная задача.
Разные языки используют различные подходы, все со своими достоинствами и недостатками.

\subsubsection{Mономорфизация} \label{subsubsec:monomorphization}

\vocab{Mономорфизация} --- самый прямолинейный подход, компилируем полиморфные функции и структуры для каждого набора типовых аргументов.
Так, если различных наборов типовых аргументов, с которыми эта функция вызывается, например, 100 (что запросто может быть), то её код будет компилироваться сто раз и занимать в бинарнике в сто раз больше места.
Так делают, например, C++ и Rust.

На самом деле всё ещё хуже.
Если проект многомодульный и состоит из множества единиц компиляции (кусков, которые компилируются отдельно), то одна и та же специализация функции на типовые аргументы будет компилироваться заново во всех единицах компиляции, где такая специализация нужна.
А затем, линкер будет заниматься удалением дубликатов, что тоже не самый быстрый и эффективный процесс.

\begin{itemize}
    \item[\positive] Порождаемый код максимально эффективен для каждого типа;
    \item[\positive] Легко на этапе компиляции отрабатывают \texttt{is}-проверки значений на принадлежность определённому типу (в остальных подходах с этим всё сложно);
    \item[\negative] Время компиляции крайне велико;
    \item[\negative] Существенно увеличивается размер результирующего бинарного файла, что может быть критично для некоторых приложений;
    \item[\negative] Может неэффективно работать из-за засорения кеша кода в процессоре;
    \item[\negative] В интерфейсах не может быть полиморфных методов, так как мы не знаем в месте вызова, к какому именно наследнику относится вызываемый метод, и какой код нужно специализировать (аналогично, не работает higher-rank полиморфзм);
    \item[\negative] К полиморфным функциям нельзя динамически линковаться (у них нет кода до специализации);
    \item[\negative] В общем случае нельзя поддержать variance, потому что код компилируется для конкретного типа и в общем случае не может работать для произвольного подтипа или супертипа (если reference и value типы могут находиться в одной иерархии подтипизации).
\end{itemize}

Некоторые языки не делают инстанциацию скрытой деталью реализации языка, а предоставляют её как инструмент пользователям.
Так делают, например, C++ и Zig.
А именно, это позволяет добиться следующего:
\begin{itemize}
    \item Если разрешить использовать значения в типах, инстанциация может использоваться как механизм вычислений на этапе компиляции.
    \item Если отложить проверку ошибок на стадию инстанциирования, то мы получим своего рода статическую утиную типизацию.
    Это позволит не описывать сложные сигнатуры полиморфных функций.
    Однако тогда функции для тестирования придётся вручную инстанциировать против всевозможных типов, иначе нельзя понять статически, компилируется она хотя бы против этих типов или нет.
\end{itemize}

% todo The Simple Essence of Monomorphization

\subsubsection{Стирание типа} \label{subsubsec:type-erasure}

Можно всё сделать наоборот, унифицировав значения, которые приходят на вход полиморфным функциям и хранятся в полиморфных структурах данных, вместо того, чтобы компилировать код под каждый тип.

Пусть каждое значение будет аллоцировано в куче и передаваться по указателю.
Тогда мы сможем переиспользовать один и тот же код для разных типовых аргументов --- он просто будет ожидать указатели.

\begin{itemize}
    \item[\positive] Каждая функция компилируется ровно один раз --- быстро;
    \item[\positive] Можно динамически загружать новые полиморфные функции и типы и использовать их друг с другом;
    \item[\positive] Гибкость --- вариантность, полиморфные методы в интерфейсах, higher-rank types и т.д. просто работают;
    \item[\negative] Аллокация в куче и разыменование указателя может очень сильно замедлить код;
    \item[\negative] Поскольку информация о типах стирается, нельзя ничего сделать с типовым аргументом, не имея его обитателей (например, запросить рефлексией информацию или сделать \texttt{is} проверку).
\end{itemize}

Такого подхода придерживаются JVM, Haskell и, как правило, другие функциональные языки ввиду его гибкости и скорости компиляции.

Особую проблему вызывает работа с примитивами и другими value-типами, потому что каждое значение приходится сначала боксить (переносить в кучу), а потом уже использовать в полиморфном контексте.
Поэтому языки борются с этим как могут.
Некоторые языки урезают диапазоны значений примитивов, чтобы зарезервировать бит, определяющий, это указатель или значение.
Код консультируется с этим битом для работы (похоже на~\ref{subsubsec:swift-generics}).
Так делают, например, OCaml и \href{https://koka-lang.github.io/koka/doc/book.html#sec-value-types}{Koka}.
Агрессивный инлайнинг тоже помогает.
Java пытается аккуратно двигаться в сторону возможности мономорфизации\footnote{\href{https://youtu.be/JI09cs2yUgY?si=MLkRs31mN1koXIu1}{Type Specialization of Java Generics - What If Casts Have Teeth ?}}\footnote{\url{https://cr.openjdk.org/~jrose/values/parametric-vm.html}}.

\subsubsection{Гибридный подход} \label{subsubsec:hybrid}

С\# реализует гибридный подход\footnote{\href{https://learn.microsoft.com/en-us/dotnet/csharp/programming-guide/generics/generics-in-the-run-time}{Generics in the runtime (C\# programming guide).}}.
Они различают значения, хранимые в куче --- reference types, и значения, хранимые на стеке --- value types.
Для первых они генерируют одну специализацию, работающую с указателями.
Для каждого набора value-типов они генерируют лениво в рантайме специализации.

То есть следы дженериков в таком подходе есть и промежуточном представлении CIL, и в рантайме.

\begin{itemize}
    \item[\positive] value-типы хранятся и передаются as-is без боксинга;
    \item[\positive] Доступна рефлексия по дженерикам;
    \item[\positive] Небольшое время компиляции;
    \item[\negative] Инстанциация в рантайме замедляет исполнение;
    \item[\negative] Variance работает только для reference types (что странно --- есть ``правильная'' подтипизация, а есть ``неправильная'').
\end{itemize}

\subsubsection{Использование виртуальной таблицы свойств типов} \label{subsubsec:swift-generics}

Swift\footnote{\href{https://youtu.be/ctS8FzqcRug?si=y_ZYnuUOulA33d_X}{(youtube) 2017 LLVM Developers’ Meeting: ``Implementing Swift Generics''}} вместе с каждым типовым параметром передаёт value witness table (рис.~\ref{fig:swift-witness-table}).
Это таблица со всей необходимой информацией, о типе: размер и выравнивание, что нужно сделать при копировании и перемещении объекта (например, инкрементировать счётчик ссылок).
Таким образом, скомпилированный код постоянно обращается к этой таблице и делает виртуальные вызовы функций из неё (рис.~\ref{fig:swift-generated-code}).
\begin{figure}
    \centering
    \includegraphics[width=0.7\textwidth]{figs/swift-witness-table}
    \caption{Swift value witness table.}
    \label{fig:swift-witness-table}
\end{figure}
\begin{figure}
    \centering
    \includegraphics[width=0.5\textwidth]{figs/swift-generated-code}
    \caption{Код полиморфной функции, порождаемый компилятором Swift.}
    \label{fig:swift-generated-code}
\end{figure}

\begin{itemize}
    \item[\positive] Небольшое время компиляции;
    \item[\positive] Предсказуемая эффективность (не приводит к неожиданным паузам в рантайме);
    \item[\positive] Эффективная работа с value-значениями;
    \item[\positive] Высокая гибкость;
    \item[\positive] Информация о типах не стирается;
    \item[\negative] Серьёзный константный оверхед на динамические вызовы через таблицу, эффективность очень сильно зависит от компиляторных оптимизаций.
\end{itemize}

Своего рода реализация параметрического полиморфизма через специальный.

\subsection{Полиморфизм по конвенции вызова} \label{subsec:representation-polymorphism}

Как мы уже обсуждали выше~\ref{subsubsec:type-erasure}, параметрический полиморфизм в Haskell реализуется следующим образом: все значения хранятся в куче и передаются в полиморфные функции по указателю.
Однако, если для вычислительного кода важна производительность, такой подход не годится ввиду большой нагрузки на подсистему управления памятью и множества индирекций.
Поэтому Haskell позволяет также писать код с использованием unboxed значений.
А если конвенция вызова не принципиальна, можно по ней абстрагироваться и писать один код для boxed и unboxed значений~\cite{eisenberg2017levity}.

\subsubsection{Разновидности runtime представлений в Haskell}

\begin{figure}[h]
    \centering
    \includegraphics[width=0.5\textwidth]{figs/haskell-value-kinds}
    \caption{Виды значений в Haskell с примерами~\cite{eisenberg2017levity}.}
    \label{fig:haskell-value-kinds}
\end{figure}

На рисунке~\ref{fig:haskell-value-kinds} можно увидеть классификацию значений в Haskell с примерами типов.
\vocab{Unboxed типы} --- их значения удерживаются и передаются по значению.
\vocab{Boxed}, соответственно, наоборот, передаются по указателю и хранятся в куче.
Обычный \mintinline{haskell}|Int| является просто декларацией следующего вида, где \mintinline{haskell}|I#| --- это обычный конструктор с необычным именем, содержащий unboxed значение.
\begin{minted}{haskell}
    data Int = I# Int#
\end{minted}

\vocab{Lifted типы} --- содержат $\bot$ в качестве значения.
Иначе говоря, могут содержать отложенные вычисления (это для них специальным образом обеспечивают компилятор и рантайм).
\vocab{Unlifted типы} --- наоборот, не могут быть отложенными.
Операции, производящие значения unlifted типов всегда энергичные.
Свойство lifted/unlifted называют \vocab{levity}.
Чтобы распространить дальнейшее изложение на энергичные языки, можно levity заменить на boxity и всё останется справедливым.

\# в именах типов и функций --- это конвенция, показывающая, что где-то рядом происходит работа с unlifted значениями\footnote{Нужно подключить расширение \href{https://ghc.gitlab.haskell.org/ghc/doc/users_guide/exts/magic_hash.html}{MagicHash}, чтобы пользоваться \# в идентификаторах.}.

Также в Haskell есть unboxed кортежи, которых не существует на этапе исполнения.
Например, следующая функция как бы возвращает пару значений, но в действительности компилятор может их разместить, например, в паре регистров.
Соответственно, паттерн-матчинг по таким кортежам, просто позволяет сослаться на каждое из этих значений.
\begin{minted}{haskell}
    divMod# :: Int -> Int -> (# Int, Int #)
    case divMod# n k of (# quot, rem #) -> ...
\end{minted}
Соответственно, нет никакого различия между по-разному вложенными unboxed кортежами:
\begin{minted}{haskell}
    (# A, (# B, C #)) ?$\equiv$? (# #( A, B #), C #) ?$\equiv$? (# A, B, C #)
\end{minted}

\subsubsection{Классификация значений по runtime представлению}

Значения различных типов могут быть на этапе исполнения устроены по-разному.
То есть нам нужна некоторая система классификации типов.
Но такая система в Haskell уж есть --- кайнды.
Опишем в виде структур данных предметную область, а потом продвинем на нужный уровень с помощью DataKinds~\ref{subsubsec:promotion}.

Стандартная библиотека Haskell \href{https://downloads.haskell.org/ghc/latest/docs/users_guide/exts/representation_polymorphism.html}{предоставляет} следующие типы данных:
\begin{minted}{haskell}
    TYPE :: RuntimeRep -> Type

    data Levity = Lifted | Unlifted

    data RuntimeRep = BoxedRep Levity
                    | IntRep | DoubleRep
                    | TupleRep [RuntimeRep]
                    | SumRep [RuntimeRep]
                    | ...

    type LiftedRep = BoxedRep Lifted

    type Type = TYPE LiftedRep
\end{minted}

\mintinline{haskell}|TYPE| --- это магический тип, определённый в компиляторе.
Он параметризован runtime-представлением значений.
Теперь привычный \mintinline{haskell}|Type| --- это частный случай с boxed lifted значениями.

%! suppress = Quote
\begin{itemize}
    \item \mintinline{haskell}|Int :: TYPE (BoxedRep Lifted)| или \mintinline{haskell}|:: Type|
    \item \mintinline{haskell}|IntRep| и \mintinline{haskell}|DoubleRep| соответствуют представлению численных констант (в зависимости от архитектуры процессора, целые числа и числа с плавающей запятой может быть необходимо располагать в различных специальных регистрах)\\ \mintinline{haskell}|Int# :: TYPE IntRep|
    \item \mintinline{haskell}|Maybe Int :: Type|
    \item \mintinline{haskell}|Maybe :: Type -> Type|
    \item \mintinline{haskell}|TupleRep| и \mintinline{haskell}|SumRep| --- unboxed алгебраические типы, представления параметризованы представлениями хранимых значений\\
    \mintinline{haskell}|(# Int, Bool #) :: TYPE (TupleRep '[LiftedRep, LiftedRep])|
    \item Для простоты, типы вложенных кортежей не унифицируются
    \begin{minted}{haskell}
        (# Int#, (# Int, Double# #) #)
          :: TYPE (TupleRep '[IntRep, TupleRep '[LiftedRep, DoubleRep]])
    \end{minted}
\end{itemize}

\subsubsection{Representation polymorphism}

Выставив runtime-представление в структуре кайндов, мы теперь можем параметризоваться по ним.
Например, кайнд функциональной стрелки выглядит следующим образом\footnote{Выключить упрощения: \mintinline{haskell}|ghci> :set -fprint-explicit-foralls -fprint-explicit-runtime-reps|}:
\begin{minted}{haskell}
    ghci> :k (->)
    (->) :: forall {q :: RuntimeRep} {r :: RuntimeRep}. TYPE q -> TYPE r -> Type
\end{minted}

\begin{task}
    Подумайте, почему функция имеет boxed тип.
    Может ли быть иначе?
    Может ли это быть полезным?
\end{task}

К сожалению, Haskell выставляет довольно строгое ограничение: связыватели не могут иметь тип, полиморфный по runtime представлению.
Можно легко предположить, почему,~--- нельзя сгенерировать код функции для работы с параметром произвольного рантайм-представления.
Это можно решить только мономорфизацией~\ref{subsubsec:monomorphization}, но Haskell избегает этого подхода\footnote{\url{https://gitlab.haskell.org/ghc/ghc/-/issues/14917}}.
Сообщество также пытается найти другие решения\footnote{\url{https://mail.haskell.org/pipermail/haskell-cafe/2023-January/135770.html}} (что-то вроде~\ref{subsubsec:swift-generics}).

Например, изначально оператор аппликации был обобщён только по возвращаемому типу.
Это не порождает проблем, так как вызывающий код сможет вывести представление и сгенерировать подходящий код:
\begin{minted}{haskell}
    ($) :: forall r a (b :: TYPE r). (a -> b) -> a -> b
    f $ ?\framebox{x}? = f x
\end{minted}

Однако, было замечено, что для оператора аппликации можно получить другую реализацию, не использующую levity-полиморфное связывание\footnote{\url{https://gitlab.haskell.org/ghc/ghc/-/merge_requests/10131}}:
\begin{minted}{haskell}
    ($) :: forall ra rb (a :: TYPE ra) (b :: TYPE rb). (a -> b) -> a -> b
    ($) f = f
\end{minted}

Таким образом, в Haskell полиморфизм по представлениям несколько вырожден и помогает лишь в небольшом количестве случаев, однако немаловажных.
Если позволить мономорфизацию по \mintinline{haskell}{RuntimeRep} параметрам, получится система аналогичная гибридной реализации параметрического полиморфизма~\ref{subsubsec:hybrid}, только с большим контролем со стороны программиста над мономорфизацией.


    \clearpage


    \section{Специальный (ad-hoc) полиморфизм} \label{sec:ad-hoc}
%
    %! suppress = MissingLabel

Как-то Joe Fasel в разговоре с Philip Wadler высказал идею того, что перегрузка функций (overloading) должна находить своё отражение в типах.
Wadler понял его неправильно~\cite{hudak2007history}.
Но то, что он понял, --- оказалось классами типов~\cite{wadler1989make}.

Christopher Strachey ввёл классификацию полиморфизма на две категории.
Параметрический --- один и тот же код работает с данными различных типов.
\vocab{Специальный (ad-hoc) полиморфизм} --- код выбирается в зависимости от типа.
Например, один и тот же символ умножения по-разному действует на целые числа и на числа с плавающей точкой.

Перегрузка в языках обозначает возможность назвать несколько функций одинаково, у которых должны быть различные наборы входных параметров.
В месте вызова компилятор статически определяет по типам аргументов, какую из них действительно следует вызвать.
\begin{minted}{cpp}
    string toString(x: int) { ... }
    string toString(fmt: String, d: double) { ... }
\end{minted}

Классы типов обязуют сначала задекларировать именованную сущность (собственно, класс типов), которая включает в себя пачку деклараций функций, которые могут быть перегружены для различных типов.
\begin{minted}{haskell}
    class Show a where
      show :: a -> String

    instance Show Int where
      show :: Int -> String
      show = ...
\end{minted}

Необходимо заметить, что декларация класса типов содержит формальный типовой параметр, по вхождениям которого в тип функции, собственно, выбирается перегрузка.
Таких параметров может быть много, они могут иметь стрелочные кайнды.
Например, в случае класса типов \mintinline{haskell}|Applicative|, выбор реализации операции \mintinline{haskell}|pure| будет происходить по типовому конструктору результата, то есть даже не по полноценному типу.
Такая гибкость и близко не достигается при классической перегрузке.
\begin{minted}{haskell}
    class Functor f => Applicative (f :: Type -> Type) where
      pure :: a -> f a
      ...

    instance Applicative Maybe where
      pure :: a -> Maybe a
      ...
\end{minted}

Также, в отличие от перегрузки, классы типов совместимы с параметрическим полиморфизмом.
Так, в типе полиморфной функции нельзя указать, что для типа должна присутствовать определённая перегрузка.
Классы типов же позволяют ограничить набор возможных типовых аргументов теми, для которых реализован инстанс нужного класса типов:
\begin{minted}{haskell}
    showPrefixed :: Show a => a -> String -> String
\end{minted}

Если сравнивать классы типов с переопределением (overriding) в ООП языках, то разрешение вызова виртуальной функции происходит с использованием таблицы, хранящейся объекте первого параметра (получателя вызова, receiver).
Классы типов же опираются исключительно на тип, поэтому, например, возможно определение констант в классах типов:
\begin{minted}{haskell}
    class Enum a => Bounded a where
      minBound :: a
      maxBound :: a
\end{minted}

В то же время, классы типов не являются типами, а, скорее, предикатами на типах.
Тип удовлетворяет такому предикату, или свойству, если для него есть соответствующий инстанс.
Поэтому, в частности, привычный способ в ООП создать гетерогенную коллекцию элементов, имеющих общий интерфейс, напрямую не сработает с классами типов.
Например, такой тип не будет корректным: \mintinline{haskell}|[Show]|.
Мы вернёмся к этой проблеме в~\ref{subsubsec:existentials}.

\subsection{Устройство классов типов}

Несмотря на поразительное могущество, идея реализации классов типов крайне проста.
Она была уже во всей полноте представлена в первой работе~\cite{wadler1989make}.
В дальнейших работах уточнялся механизм вывода типов в виде сведения к классической системе типов в стиле Hindley-Milner~\cite{hall1996type}.
Остальные работы, в основном, предлагают огромное разнообразие различных приложений.

\subsubsection{Словари}

Рассмотрим идею реализации классов типов на примере полиморфной сортировки.
Сортировка для списка элементов конкретного типа пишется тривиально:
\begin{minted}{haskell}
    sort :: [Int] -> [Int]
    sort = \case [] -> []; x:xs -> insert x (sort xs)
      where
        insert x xs = let (l, r) = List.partition (?\framebox{<}? x) xs in l ++ x : r)
\end{minted}

В реализации единственная информация о типе, которой мы пользуемся --- порядок на его обитателях.
Таким образом, при переходе к полиморфной сортировке, нам нужно принять словарь с предикатами, задающими нужный порядок для данного типа\footnote{Приятный синтаксис распаковки рекордов доступен с расширением \href{https://ghc.gitlab.haskell.org/ghc/doc/users_guide/exts/record_wildcards.html}{RecordWildCards}.}.
\begin{minted}{haskell}
    data OrdDict a = OrdDict { less :: a -> a -> Bool }

    sort :: OrdDict a -> [a] -> [a]
    sort d@OrdDict {..} = \case [] -> []; x:xs -> insert x (sort d xs)
      where
        insert x xs = let (l, r) = List.partition (?\framebox{`less`}? x) xs in l ++ x : r)
\end{minted}

Теперь, чтобы воспользоваться сортировкой на списке чисел, нужно сконструировать нужный рекорд и вызвать с ним функцию на списке конкретных типов:
\begin{minted}{haskell}
    intOrd :: OrdDict Int
    intOrd = OrdDict { less = (<) }

    ghci> sort intOrd [3, 2, 1]
\end{minted}

Возможна ситуация, когда инстанс для одного типа зависит от инстанса для другого
Например, сравнение списков можно получить автоматически, зная порядок на элементах.
В случае словарей мы это моделируем функцией между словарями:
\begin{minted}{haskell}
    listDict :: OrdDict a -> OrdDict [a]
    listDict d = OrdDict { less = ... ?\framebox{less d}? ... }
\end{minted}

Теперь мы можем сортировать список списков, конструируя нужный словарь:
\begin{minted}{haskell}
    ghci> sort (listDict intDict) [[3, 2], [2, 1], [0]]
\end{minted}

Сравнение явной передачи словарей и классов типов можно увидеть в следующей таблице:

\begin{tabular}{p{8cm}p{8cm}}
    \begin{enumerate}
        \item Определение словаря функций
        \begin{minted}{haskell}
            data MyOrd a = MyOrd
              { less :: a -> a -> Bool }
        \end{minted}
        \item Экземпляр словаря для конкретного типа
        \begin{itemize}
            \item Именованное значение
        \end{itemize}
        \begin{minted}{haskell}
            intMyOrd :: MyOrd Int
            intMyOrd = MyOrd { less = (<) }
        \end{minted}
        \item Явный параметр функции
        \begin{minted}{haskell}
            sort :: MyOrd a -> [a] -> [a]
        \end{minted}
        \item Передаётся пользователем
        \begin{minted}{haskell}
            test = sort ?\framebox{intMyOrd}? [3, 2, 1]
        \end{minted}
    \end{enumerate}
    &
    \begin{enumerate}
        \item Определение класса типов
        \begin{minted}{haskell}
             class MyOrd a where
               less :: a -> a -> Bool
        \end{minted}
        \item Объявление типа представителем класса типов
        \begin{itemize}
            \item Не имеет имени
        \end{itemize}
        \begin{minted}{haskell}
            instance MyOrd Int where
              less = (<)
        \end{minted}
        \item Неявный параметр функции
        \begin{minted}{haskell}
            sort :: MyOrd a => [a] -> [a]
        \end{minted}
        \item Передаётся компилятором
        \begin{minted}{haskell}
            test = sort [3, 2, 1]
        \end{minted}
    \end{enumerate}
\end{tabular}

Таким образом, словарь --- это свидетель (witness) того, что тип удовлетворяет ограничению.
Подобно тому, как коерции свидетельствуют несинтаксические эквивалентности на типах~\ref{subsubsec:gadts}.

\begin{task}
    Какой словарь будет соответствовать higher-kinded классу типов \mintinline{haskell}|Functor|?
\end{task}

\subsubsection{Неявные аргументы}

Можно думать так, что слева от \mintinline{haskell}|=>| передаются неявные аргументы функций, выводимые компилятором из контекста.
То есть, например, не стоит удивляться вхождениям \mintinline{haskell}|=>| в отрицательной позиции, это просто функция с неявным аргументом.
Так, следующий код не скомпилируется, потому что в месте использования переменной \texttt{y} нет значения типа \mintinline{haskell}|Show b|:
\begin{minted}{haskell}
    f :: (Show b => b) -> b
    f x = ?\framebox{x}?
\end{minted}
Можно это значение принять в функции \texttt{f}, тогда оно автоматически пропагируется в \texttt{y}:
\begin{minted}{haskell}
    f :: Show b => (Show b => b) -> b
    f x = ?\framebox{x}?
\end{minted}

Расширение \href{https://ghc.gitlab.haskell.org/ghc/doc/users_guide/exts/implicit_parameters.html}{ImplicitParams} даёт возможность делать некоторое аргументы функции неявными.
Фактически, это реализация динамического связывания в статическом языке~\cite{lewis2000implicit}.
Неявные аргументы берутся из скоупа по имени и подставляются автоматически:
\begin{minted}[escapeinside=##]{haskell}
    sortBy :: (a -> a -> Bool) -> [a] -> [a]

    sort :: (?cmp :: a -> a -> Bool) => [a] -> [a]
    sort = sortBy ?cmp
\end{minted}

Haskell также предоставляет возможность сохранять словари в структуры данных:
\begin{minted}{haskell}
    data ShowDict a where
      ShowDict :: Show a => ShowDict a

    f :: ShowDict b -> (Show b => b) -> b
    f d x = case d of ShowDict -> ?\framebox{x}? -- в скоупе доступен инстанс Show b
\end{minted}

\subsubsection{Вывод инстансов}

Чтобы вызвать ограниченно-полиморфную функцию, GHC производит вывод инстансов или, иначе говоря, автоматически конструирует свидетелей.
Вывод инстансов тесно интегрирован с общей системой вывода типов Haskell~\cite{spj-type-inference}.

В действительности вывод инстансов это не что иное, как \point{задача населения типа}.
Действительно, после трансляции в Core (промежуточное представление в GHC), классы типов представляют собой словари функций.
У нас в контексте имеются конкретные словари и функции, позволяющие из одних словарей получать другие.
Требуется найти терм, конструирующий словарь нужного типа.

Например, внутри функции \mintinline{haskell}|f :: Show a => ..| происходит вызов ограниченно- полиморфной функции
\mintinline{haskell}|g :: Show [a] -> ..|.
У нас имеется словарь \mintinline{haskell}|ShowDict a|, а так же функция \mintinline{haskell}|ShowDict a -> ShowDict [a]|, пришедшая из импортов\footnote{Инстансы можно импортировать пустым импортом: \mintinline{haskell}|import Module ()|.}.
Необходимо сконструировать терм типа \mintinline{haskell}|ShowDict [a]|.
Очевидно, это будет просто аппликации одного к другому.

Вывод инстансов происходит рекурсивно.
Чтобы вывести \mintinline{haskell}|ShowDict [a]|, выводится сначала посылка \mintinline{haskell}|ShowDict a|.
То есть получается рекурсия по структуре типа.
Иначе говоря, вывод инстансов можно эксплуатировать как вычислительный примитив уровня типов.
Так, например, мы можем опускать информацию из типов в термы:
\begin{minted}{haskell}
    type data Nat = Zero | Suc Nat

    class KnownNat (n :: Nat) where
      natVal :: Int

    instance KnownNat Zero where
      natVal = 0

    instance KnownNat n => KnownNat (Suc n) where
      natVal = 1 + natVal @n

    ghci> natVal @(Suc (Suc Zero))
    -- выведется natVal {knownSuc (knownSuc knownZero)}
\end{minted}

В общем случае процесс населения типа, как можно предположить по вычислительной аналогии, неразрешим.
Поэтому GHC накладывает большое количество ограничений на вид инстансов, которые гарантируют тотальность вывода.
Подробно эти ограничения описаны в~\cite{sulzmann2007understanding}.
Также GHC предоставляет различные расширения, ослабляющие эти ограничения и перекладывающие часть ответственности на плечи программиста\footnote{\url{https://downloads.haskell.org/ghc/latest/docs/users_guide/exts/instances.html}}.
Например, c UndecidableInstances можно легко написать разворот списка типов на этапе компиляции, как и любую другую функцию:
\begin{minted}{haskell}
    class Reverse (acc :: [Type]) (tys :: [Type]) where
      showReverse :: String

    instance ShowT acc => Reverse acc '[] where
      showReverse = showTypes @acc

    instance Reverse (ty : acc) tys => Reverse acc (ty : tys) where
      showReverse = showReverse @(ty : acc) @tys

    ghci> showReverse @'[] @'[Char, Int, Double]
\end{minted}

Вывод инстансов опирается только на вид "головы" декларации --- справа от \mintinline{haskell}|=>|, а ограничения слева применяются постфактум.
Это можно использовать, чтобы писать более общие инстансы.
Так, например, работает \vocab{constraint trick}\footnote{\url{https://chrisdone.com/posts/haskell-constraint-trick/}}, позволяющий резолвить ad-hoc полиморфные функции в параметрически-полиморфном контексте.

Этим же пользуется механизм программируемых ошибок компиляции из \href{https://hackage.haskell.org/package/base-4.20.0.1/docs/GHC-TypeLits.html}{GHC.TypeLits}.
Если инстанс сработал и компилятор начал обрабатывать ограничения слева, значит, что-то пошло не так~\cite[глава 12]{maguire-types}.

\begin{minted}{haskell}
    instance (TypeError
      ( Text "Attempting to show a function of type "
        :<>: Text "'" :<>: ShowType (a -> b) :<>: Text "'"
        :$$: Text "Did you forget to apply an argument?"
      )) => Show (a -> b) where
      show = undefined -- реализация не важна, до исполнения дело не дойдёт
\end{minted}

\subsubsection{Построение типа по значению}

После того как мы научились опускать значения из типов, закономерно научиться обратному --- поднимать значения в типы.

Рассмотрим вспомогательную концепцию проксирования.
В стандартной библиотеке определён тип \mintinline{haskell}|Proxy| с одним фантомным параметром.
Исторически он нужен, чтобы обходить отсутствие расширений \href{https://downloads.haskell.org/ghc/latest/docs/users_guide/exts/type_applications.html#extension-TypeApplications}{TypeApplications} и \href{https://downloads.haskell.org/ghc/latest/docs/users_guide/exts/type_abstractions.html#extension-TypeAbstractions}{TypeAbstractions}.
А именно --- неинформативную константу \mintinline{haskell}|Proxy| можно проаннотировать нужным типом и передать в функцию, чтобы специализировать типовой параметр на нужный тип.
Или принять \mintinline{haskell}|Proxy| и воспользоваться \href{https://downloads.haskell.org/ghc/latest/docs/users_guide/exts/scoped_type_variables.html#pattern-type-signatures}{ScopedTypeVariables} для типовых сигнатур в паттернах.
\begin{minted}{haskell}
    data Proxy (a :: k) = Proxy

    id :: Proxy a -> a -> a
    ghci> :t id (Proxy :: Proxy Int)
    id (Proxy :: Proxy Int) :: Int -> Int

    id (Proxy :: Proxy ?\framebox{a}?) x = (x :: ?\framebox{a}?)
\end{minted}

Иногда прокси-тип оставляют полиморфным, чтобы пользователь сам мог его задать.
Вместо конкретного значения иногда передают специализированное значение $\bot$, а получатель, не зная тип, не сможет его форсировать (однако, любые вхождения $\bot$ в тип слишком настораживают).
\begin{minted}{haskell}
    id :: proxy a -> a -> a
    id (_ :: proxy a) x = (x :: a)

    ghci> :t id (undefined :: Proxy Int)
    id (undefined :: Proxy Int) :: Int -> Int
\end{minted}

Теперь напишем функцию поднятия значения в тип с помощью техники через полиморфную рекурсию, описанную, например, в~\cite{kiselyov2004functional}.

В действительности мы, конечно, не можем честно получить синтаксически тип нужного размера, просто потому, что типы существуют строго до стадии исполнения.
Однако, как мы знаем, словари классов типов имеют воплощение в рантайме (случай полиморфной рекурсии~--- как раз пример, когда этого нельзя полностью избежать).
Поэтому воспользуемся continuation passing style, который будет подробно рассмотрен в главе~\ref{sec:continuations}: вместо того, чтобы вернуть результат, примем продолжение, умеющее работать с любым типом с \mintinline{haskell}|KnownNat|:
\begin{minted}{haskell}
    reify :: Int -> (forall n. KnownNat n => Proxy n -> a) -> a
    reify n k
      | n <= 0 = k (Proxy @Zero)
      | otherwise = reify (n - 1) \(Proxy :: Proxy n) -> k (Proxy @(Suc n))
\end{minted}

Продолжение, передаваемое в рекурсивный вызов, неявно принимает словарь для типа \texttt{n} и конструирует словарь для \mintinline{haskell}|Suc n|.

Наконец, можем написать следующую удивительную тождественную функцию, поднимающую сначала значение в тип, а потом опускающее тип обратно в термы:
\begin{minted}{haskell}
    wonderId :: Int -> Int
    wonderId n = reify n (\(Proxy :: Proxy n) -> natVal @n)
\end{minted}

\subsubsection{Имплиситы и когерентность}

Существуют подходы к реализации классов типов, когда они не являются отдельной возможностью языка, а получаются как следствие других, более общих, механизмов.

Так, в Scala существует механизм имплиситов (implicits)~\cite{kvrikava2019scala}.
Параметры функции могут быть помечены \mintinline{scala}|implicit|.
Теперь, если в контексте есть \mintinline{scala}|implicit| значения и \mintinline{scala}|implicit| функции, аргументы сгенерируются автоматически исходя из типа параметра.
Теперь мы можем смоделировать словарь функций, например, с помощью интерфейсов (которые в Scala называются \mintinline{scala}|trait|) и синглтонов, чтобы получить вполне себе классы типов~\cite{oliveira2010type}.
Функции с имплисит-параметрами не являются в полном смысле функциями первого класса в Scala.
\begin{minted}{scala}
    trait Show[T] {
        def show(x: T): String
    }

    def show[T](x: T)(implicit ev: Show[T]): String = ev.show(x)

    implicit object intShow extends Show[Int] {
        def show(x: Int): String = x.toString
    }

    def showAll[T](xs: List[T])(implicit ev: Show[T]): String =
        xs.map(show(_)).join(", ")
\end{minted}

В языках с зависимыми типами неявные параметры\footnote{\url{https://agda.readthedocs.io/en/v2.7.0.1/language/implicit-arguments.html}} особенно нужны, потому что, например, типы --- это такие же параметры функции, как и все остальные.
Поэтому вывод типов~--- это фактически вывод неявных аргументов функций.
Более того, зависимые функции, вместе с аргументами часто принимают доказательства каких-то свойств этих аргументов, которые тоже хочется по возможности выводить из контекста автоматически.
В свете аналогии между классами типов как предикатами на типах, а инстансами как доказательствами, свидетельствующими о выполнении предиката для типа, можно тот же механизм вывода доказательств переиспользовать для эмулирования классов типов\footnote{\url{https://agda.readthedocs.io/en/v2.7.0.1/language/instance-arguments.html}}~\cite{devriese2011bright}.
В обратную сторону можно механизмы зависимой типизации эмулировать классами типов~\cite{mcbride2002faking}.

Выражение классов типов через другие механизмы может выглядеть как абсолютно выигрышная стратегия на фоне реализации их как самостоятельной возможности языка.
Действительно, можно переиспользовать всё то могущество программирования, которое есть с рекордами, в то время как для классов типов и констреинтов то же самое приходится развивать отдельно\footnote{\url{https://downloads.haskell.org/ghc/latest/docs/users_guide/exts/instances.html}}.
Да и неявные аргументы, наверное, полезный механизм сам по себе.

Однако у подхода Haskell есть важное свойство при соблюдении всех ограничений, т.е. при отсутствии \vocab{orphan instances}\footnote{\url{https://stackoverflow.com/questions/3079537/orphaned-instances-in-haskell}}.
\vocab{Когерентность инстансов (coherence)} --- для одного типа все инстансы данного класса типов, полученные разными способами, неотличимы (рис.~\ref{fig:coherence}).
Соответственно, не имеет значения происхождение того или иного инстанса.
Иначе говоря, об этом можно не думать, это снимает существенное количество когнитивной нагрузки и упрощает рефакторинг\footnote{\href{https://youtu.be/hIZxTQP1ifo?si=aG2Lk2eb-5E5SOLb}{Edward Kmett - Type Classes vs. the World.}}.
В то время как остальные подходы требуют трепетного отношения к контексту вызова, потому что из него может прийти неожиданная реализация.

\begin{figure}
    \centering
    \includegraphics[width=0.8\linewidth]{figs/coherence}
    \caption{Когерентность инстансов --- диаграмма коммутирует.}
    \label{fig:coherence}
\end{figure}

\subsubsection{Правила (rules) и специализация}

GHC позволяет прямо в коде, с помощью специально прагмы, указывать оптимизирующие правила переписывания для компилятора\footnote{\url{https://downloads.haskell.org/ghc/latest/docs/users_guide/exts/rewrite_rules.html}}.
Например:
\begin{minted}{haskell}
    {-# RULES
      "map/map"    forall f g xs.  map f (map g xs) = map (f . g) xs
      "map/append" forall f xs ys. map f (xs ++ ys) = map f xs ++ map f ys
     #-}
\end{minted}

В частности, можно переписать полиморфную версию функции на специализированную, если типы подходят.
Для этого нужно реализовать специализированную версию (совпадение семантики --- полностью ответственность программиста) и задать соответствующее правило переписывания:
\begin{minted}{haskell}
    genericLookup :: Ord a => Table a b   -> a   -> b
    intLookup     ::          Table Int b -> Int -> b

    {-# RULES "genericLookup/Int" genericLookup = intLookup #-}
\end{minted}

Основной эффект такой оптимизации --- гарантированное превращение динамических вызовов функций классов типов в статические (потому что тип известен, следовательно, --- и соответствующий ему словарь).

\subsection{Семейства, ассоциированные семейства}

% todo fist-class type families, efunctionalisation (Eisenberg 2013)

% todo \chapter*{FunWithTypeFuns}

% todo

\subsection{Кайнд Constraint}

% todo boxed constraints

Давно появлялись предложения добавить в GHC поддержку синонимов для констреинтов, семейств констреинтов и т.д~\cite{orchard2010haskell}.
В итоге был предложен\footnote{\url{https://gitlab.haskell.org/ghc/ghc/-/wikis/kind-fact}} и реализован\footnote{\url{http://blog.omega-prime.co.uk/2011/09/10/constraint-kinds-for-ghc/}} некоторый механизм унификации типов и констреинтов.
Таким образом, всё, что работало для типов, стало работать и для констреинтов.

В GHC с \href{https://downloads.haskell.org/ghc/9.2-latest/docs/html/users_guide/exts/constraint_kind.html}{ConstraintKinds} был добавлен специальный кайнд \mintinline{haskell}|Constraint|:
\begin{itemize}
    \item Класс типов конструирует констреинт: \mintinline{haskell}|Monad :: (Type -> Type) -> Constraint|;
    \item Эквивалентность является констреинтом: \mintinline{haskell}|(a ?$\sim$? b) :: Constraint|;
    \item Пустой кортеж констреинтов является констреинтом: \mintinline{haskell}|() :: Constraint|;
    \item Кортеж констреинтов является констреинтом: \mintinline{haskell}|(Eq a, a ?$\sim$? b) :: Constraint|.
\end{itemize}

Теперь, например, мы можем реифицировать словарь как объект языка:
\begin{minted}{haskell}
    data Dict (c :: Constraint) where
      Dict :: c => Dict c
\end{minted}

Вспомним гетерогенный список, рассмотренный ранее~\ref{subsubsec:promotion}:
\begin{minted}{haskell}
    data HList (tys :: [Type]) where
      HNil :: HList '[]
      HCons :: ty -> HList tys -> HList (ty : tys)
\end{minted}

Эта структура данных является first-class аналогом variadic generics в C++ или Swift\footnote{\url{https://github.com/swiftlang/swift-evolution/blob/main/proposals/0398-variadic-types.md}}\footnote{Собственно, смысл вариадиков --- не вводить в язык всё что нужно, чтобы сделать \mintinline{haskell}|HList| возможным.}.
Например, мы написать \texttt{map} для такой структуры, если все типы удовлетворяют определённому ограничению.
Для этого сначала реализуем семейство, генерирующее кортеж констреинтов для каждого типа из списка:
\begin{minted}{haskell}
    type family All (c :: k -> Constraint) (tys :: [k]) :: Constraint where
      All c '[] = ()
      All c (ty : tys) = (c ty, All c tys)

    -- All Show [Int, Double] ?$\sim$? (Show Int, (Show Double, ())
\end{minted}

Теперь можем реализовать \texttt{map}:
\begin{minted}{haskell}
    hmap :: forall c res tys . All c tys
         => (forall ty . c ty => ty -> res) -> HList tys -> [res]
    hmap f = \case
      HNil -> []
      HCons x xs -> f x : hmap @c f xs

    ghci> hmap @Show show (HCons (1 :: Int) $ HCons 'a' HNil)
\end{minted}

Больше такого рода упражнений в гетерогенных конструкциях можно найти в~\cite{de2014true}.

% todo https://downloads.haskell.org/ghc/latest/docs/users_guide/exts/quantified_constraints.html



% todo

\subsection{Использование классов типов}

% todo

\subsubsection{Сериализация}

% todo

\subsubsection{Рефлексия на типах}

\cite{peyton2016reflection}

% todo

\subsubsection{Экзистенциальные типы} \label{subsubsec:existentials}

\cite[глава 7]{maguire-types}

% todo dynamic

% todo

\subsubsection{Получение значений с уровня типов}

% todo

\subsubsection{Коерции}

\cite{breitner2014safe}

\cite[глава 8]{maguire-types}

% todo type roles


% todo what do you want to give name to


% todo constraint is a proposition and dict is a proof

% todo constraints package of Edward Kmett

% todo monomorphism restriction

% todo pattern-matching on types

% todo serializer and serializable

% todo proxy

% todo algebra driven design

% todo rules and coersion

% todo type classes vs interfaces
% todo type classes via interfaces and implicits

% todo incoherent instances

% todo instance arguments

% todo Monoid a => Monad m ??

% todo https://downloads.haskell.org/ghc/latest/docs/users_guide/exts/constraints.html

% todo ConstraintKinds

% todo constraints and decidability

% todo https://chrisdone.com/posts/haskell-constraint-trick/

% todo A Reflection on Types SPJ paper

% todo haskell specialization

% todo HasCall stack https://downloads.haskell.org/ghc/latest/docs/users_guide/exts/callstack.html

% todo https://wiki.haskell.org/Inlining_and_Specialisation

% todo existential types, different vtable implementations, value types support
% todo scala type classes
% todo type inhabitation, system FC, Symon's talk
% todo доклад SPJ
% todo open unions, data a la carte
% todo roles and coertions
% todo typeable & reflection
% todo reflecrtion https://www.tweag.io/blog/2017-12-21-reflection-tutorial/
% todo How to make ad-hoc polymorphism less ad-hoc

% todo C. V. Hall, K. Hammond, S. L. Peyton Jones, and P. L. Wadler. Type classes in haskell

% todo S. Peyton Jones, S. Weirich, R. A. Eisenberg, and D. Vytiniotis. A reflection on types. In A list of successes that can change the world,

% todo \chapter*{"Hackett: a metaprogrammable Haskell" by Alexis King}

% todo


% todo call stack

% todo примеры с HList

% todo orphans



% todo functional dependencies & associated types

% todo https://downloads.haskell.org/~ghc/9.0.1/docs/html/users_guide/exts/constraints.html

% todo Orphan rules

% todo builder code serialization format from babylon

% todo associated type families

% todo reference dependent types

% todo рулесы, законы и оптимизация

% todo asking to fill parameters for you because they are very obvious. Implicit in dependently-typed languages

% todo variadics, open sums and products

% todo


    \clearpage


    \section{Персистентность и оптика}
%
%    % todo there are pierce lenses

% todo http://www.timphilipwilliams.com/posts/2019-07-25-minecraft.html

% todo profunctor optics
% todo van laarhoven CPS optics

% todo internal iteration

% todo

%
%    \clearpage


    \section{Интерпретаторы как основа основ}

    % todo expression problem
    % todo Independently Extensible Solutions to the Expression Problem


    \section{Классические интерпретаторы}

%    %! suppress = MissingLabel

\subsubsection{Встроенные доменно-специфичные языки (eDSL)} \label{subsubsec:edsl}

Обсуждение терминологии и сравнение подходов к построению DSL можно найти в~\cite{gibbons2013functional}.
Краткое описание терминов --- в конспекте курса Language Engineering~\cite{languageEngineering}.

Под \vocab{доменно-специфичными языками (domain specific languages, DSL)}\footnote{\url{https://en.wikipedia.org/wiki/Domain-specific_language}} часто понимают специализированные языки для конкретных предметных областей, например, запросов к БД или форматирования документов.
Как правило, такие языки не являются полными по Тьюрингу.

В этом курсе, однако, мы будем считать доменно-специфичным языком любую доменно-специфичную специализацию языка общего назначения\footnote{\url{https://en.wikipedia.org/wiki/Language-oriented_programming}, на русскоязычную страницу тоже следует заглянуть.}.
Это следует из того соображения, что код должен читаться как грамотным проза с уместным словоупотреблением, предоставляющая читателю только необходимое количество подробностей, скрывая несущественное за умолчаниями и терминологией.
В извечной борьбе со сложностью, мы стремимся к такому коду, строя башню из DSL\@.

\vocab{Самостоятельные доменно-специфичные языки (standalone domain specific languages)} --- языки, имеющие свой собственный конкретный синтаксис, а так же инструменты программирования (IDE, исполняющая среда\ldots).
Примеры: SQL, AWK, Antlr\ldots

\vocab{Встроенные доменно-специфичные языки (embedded domain specific languages, eDSL)} --- языки, пользующиеся поддержкой инфраструктуры других языков.
Обычно реализуются как библиотеки для программ на уже существующем языке общего назначения.
Не имеют полностью собственного синтаксиса.
Примеры: ORM, функции обработки строк, библиотека парсер-комбинаторов\ldots

\vocab{Deep eDSL} --- термы на таком языке строят дерево абстрактного синтаксиса для дальнейшей интерпретации:
\begin{minted}{haskell}
    value :: Env -> Int
    value = eval $ Const 1 `Plus` Var "x"
\end{minted}

Можно заметить, что промежуточное дерево, которое получается, нас, как правило, не интересует.
Нам важно только получить элемент домена, которым мы уже умеем пользоваться непосредственно.
\vocab{Shallow eDSL} минуют стадию построения дерева и сразу строят значение в семантическом домене:
\begin{minted}{haskell}
    cnst :: Int -> (Env -> Int)
    cnst x _ = x

    var :: String -> (Env -> Int)
    var name env = env ! name

    plus :: (Env -> Int) -> (Env -> Int) -> (Env -> Int)
    plus l r env = l env + r env

    value :: Env -> Int
    value = cnst 1 `plus` var "x"
\end{minted}

Интерпретаторы часто называют \vocab{наблюдателями (observers)}, которые анализируют термы и дают им некоторый смысл~\cite{gibbons2013functional}.
Можно заметить, что для deep eDSL можно написать сколь угодно много различных наблюдателей.
Однако в случае shallow embedding наблюдатели всегда \texttt{id}.
Мы будем обсуждать возможные решения этой проблемы в разделе~\ref{sec:wonder-interpreters}.

Введём ещё одно важное понятие.
\vocab{Meta-circular интерпретатор}\footnote{\url{https://en.wikipedia.org/wiki/Meta-circular_evaluator}}~--- это интерпретатор, определяющий конструкции определяемого языка через конструкции мета-языка~\cite{reynolds1972definitional}.
Например:
\begin{minted}{haskell}
    interpret term = case term of
      App f t -> (interpret f) (interpret t)
      If c t e -> if interpret c then interpret t else interpret e
      ...
\end{minted}

Свойства мета-языка в таком случае во многом определяют свойства объектного~\cite{reynolds1972definitional,reynolds1998definitional}.
Мы будем в этом курсе стремиться как можно более переиспользовать возможности мета-языка.

\begin{task}
    Предположите, какие свойства наследует определяемый язык.
\end{task}

\subsubsection{Пример: библиотека Accelerate}

Интересным примером встроенного языка, находящегося где-то между deep и shallow является библиотека Accelerate\footnote{\url{https://hackage.haskell.org/package/accelerate}}~\cite[глава 6]{marlow2011parallel}.
Она позволяет на Haskell описать вычисления, которые будут исполняться на GPU\footnote{Другой подход: \href{https://youtu.be/6c0DB2kwF_Q?si=-nB7AkCsDWB_Q-hy}{Java code reflection}, чтобы в рантайме извлекать модель кода. Однако, такой подход не предоставляет статически гарантий программисту и требует глубогоко внедрения в мета-язык.}.

Чтобы исполнить что-то на GPU нужно породить и скомпилировать код на Cuda.
Таким образом, Accelerate должен быть deep embedding, чтобы иметь дерево вычисления, чтобы его транслировать в Cuda наиболее эффективным образом.

В то же время описывать численные вычисления как дерево крайне неудобно.
Неплохо было бы иметь привычные операторы и функции высших порядков для работы с массивами на GPU\@.
Поэтому Accelerate предоставляет на самом деле shallow интерфейс для построения деревьев.

Так, для деревьев выражений определена реализация численных классов типов, например, \mintinline{haskell}|Num|, где операции просто достраивают дерево.
Функции высших порядков реализованы примерно с помощью техники, которую мы будем обсуждать в~\ref{subsec:first-class-functions}.

% todo investigate actual HOF implementation

\subsection{Реализация интерпретаторов}

Итак, мы определили, что интерпретаторы --- это основа основ, потому что главная задача программиста --- борьба со сложностью, а главный инструмент этой борьбы --- использование подходящих доменно-специфичных языков, позволяющих думать только о важном, абстрагируя несущественные детали.

В течение этого курса мы будем учиться писать интерпретаторы.
В этом разделе --- классические, которые принимают деревья на вход и интерпретируют их в семантический домен.
Такие интерпретаторы задают deep eDSL\@.
Далее мы научимся миновать этап построения дерева определяемого языка, напрямую расширяя мета-язык новыми конструкциями (раздел~\ref{sec:wonder-interpreters}).
И наконец, постараемся приблизиться к священного Граалю этой науки~--- расширяемым интерпретаторам (раздел~\ref{sec:effect-handlers}).

``Классические'' интерпретаторы часто называют \vocab{инициальными} интерпретаторами.
Слово ``инициальный'' тут относится к тому, что мы имеем дело с инициальным объектом категории интерпретаций.
Всё что нам нужно понимать, --- что инициальные интерпретаторы работают с деревом программы, заданным классически с помощью \mintinline{haskell}|data| (да, деревья можно задавать и по-другому).

Есть отличная книга~\cite{nystrom2021crafting}, рассказывающая о построении классических интерпретаторов и простых виртуальных машин.
В то же время она покрывает сознание полноценного языка во всех его аспектах, от синтаксического анализа до управления памятью.

\subsubsection{Untyped tagless interpreters}

Для начала рассмотрим некоторый тривиальный нетипизироватный язык.
Под нетипизированностью понимаем отсутствие проверки типов как до исполнения программы, так и во время.
Абстрактный синтаксис этого языка зададим следующим образом:
\begin{minted}{haskell}
    data Term = Const Int | IsZero Term | If Term Term Term
\end{minted}

Значения, возникающие во время исполнения программ на этом языке будем представлять значениями типов \mintinline{haskell}|Bool| и \mintinline{haskell}|Int| языка Haskell\footnote{На самом деле эти значения сами по себе в Haskell несут типовую информацию в runtime, но пока опустим эту деталь для простоты.}.
Соответственно, семантическим доменом программы на этом языке является либо \mintinline{haskell}|Bool|, либо \mintinline{haskell}|Int|, в зависимости от самой программы.
\begin{minted}{haskell}
    interpretUnsafe :: forall res . Term -> res
    interpretUnsafe term = case term of
      Const val -> unsafeCoerce val
      IsZero cond -> unsafeCoerce $ interpretUnsafe @Int cond == 0
      If c t e -> if interpretUnsafe c then interpretUnsafe t else interpretUnsafe e
\end{minted}

Здесь \texttt{unsafeCoerce} используется, чтобы обмануть статическую систему типов Haskell и просто исполнять программы на нашем нетипизированном языке.
Неверное написание программы на этом языке или выбор неправильного домена интерпретации приводят к падению программы.

\subsubsection{Typed tagged interpreters}

Чтобы добиться некоторой безопасности исполнения, будем приписывать значениям некоторые теги, которые будут доступны во время исполнения.
Заведём следующий алгебраический тип:
\begin{minted}{haskell}
    data RtValue = RtBool Bool | RtInt Int
\end{minted}

Теперь семантическим доменов у нас будет тип \mintinline{haskell}|RtValue|, а интерпретатор сможет проверять типы во время исполнения:
\begin{minted}{haskell}
    interpretRt :: Term -> RtValue
    interpretRt (IsZero term) = case eval term of
      RtBool value -> error "Type error"
      RtInt value -> RtBool (value == 0)
    -- ...
\end{minted}

Ситуация с безопасностью программы определённо стала лучше, однако проверка типов во время исполнения~--- это уже поздно: требует дополнительных расходов производительности и удорожает тестирование.

Этот подход часто называют динамической типизацей, когда мы атрибутируем значения некоторой типовой информацией для использования во время исполнения.

\subsubsection{Equality coercions, GADTs} \label{subsubsec:gadts}

Как мы знаем, конструкторы данных в Haskell --- это обычные функции с той лишь разницей, что их реализация генерируется компилятором (аллокация памяти, размещение полей\ldots).
У функций есть сигнатура.
Например, \mintinline{haskell}|IsZero :: Term -> Term|.

В Haskell есть синтаксис определения \mintinline{haskell}|data| через задание типов конструкторов.
Он совершенно аналогичен рассмотренному ранее, только гораздо более удобен для сложно организованных структур данных.
Рассмотренный ранее тип термов \mintinline{haskell}|Term| будет выглядеть следующим образом:
\begin{minted}{haskell}
    data Term where
      Const :: Int -> Term
      IsZero :: Term -> Term
      If :: Term -> Term -> Term -> Term
\end{minted}

Современный Haskell является синтаксически богатым языком, который, однако, несмотря не многообразие конструкций, транслируется в маленький типизированный внутренний язык.
Это язык $System~F_C$, основанный на $System~F$.
Он описан в работе~\cite{sulzmann2007system}.

$System~F_C$ расширяет $System~F$ наличием сложных несинтаксических эквивалентностей типов.
Оказывается, этого достаточно, чтобы поддержать такие возможности Haskell как обобщённые алгебраические типы, ассоциированные семейства типов, функциональные зависимости и т.д.

Синтаксический сахар для типовых эквивалентностей выглядит следующим образом\footnote{Чтобы воспользоваться эквивалентностями и GADT, нужно подключить расширение \href{https://downloads.haskell.org/~ghc/9.0.1/docs/html/users_guide/exts/gadt.html}{GADTs}.}\footnote{\url{https://ghc.gitlab.haskell.org/ghc/doc/users_guide/exts/equality_constraints.html}}:
\begin{minted}{haskell}
    f :: forall a b . a ?$\sim$? b => a -> b
    f = id
\end{minted}
На самом деле это функция от четырёх параметров: двух типовых параметров, коерции, аргумента.
Коерция --- это тип, автоматически выводимый компилятором, который является свидетельством того, что два типа, записанный в кайнде этого типа, эквивалентны.
\begin{minted}{haskell}
    f :: forall (a :: *) (b :: *) . forall (co :: a ?$\sim$? b) . a -> b
\end{minted}

Воспользуемся типовой эквивалентностью следующим образом.
Добавим типу термов фантомный типовой параметр, который будет маркировать тип результата терма.
\begin{minted}{haskell}
    data Term (ty :: Type) where
      Const :: forall ty . ty ?$\sim$? Int => Int -> Term ty
      IsZero :: forall ty . ty ?$\sim$? Bool => Term Int -> Term ty
      If :: forall ty . Term Bool -> Term ty -> Term ty -> Term ty
\end{minted}
Или можно воспользоваться синтаксическим сахаром, который и называется \vocab{обобщёнными алгебраическими типами данных (generalized algebraic data types)}, чтобы скрыть явные эквивалентности:
\begin{minted}{haskell}
    data Term (ty :: Type) where
      Const :: Int -> Term Int
      IsZero :: Term Int -> Term Bool
      If :: Term Bool -> Term ty -> Term ty -> Term ty
\end{minted}

Теперь мы можем написать безопасный типизированный интерпретатор.
Обратите внимание, что при сопоставлении с образцами конструкторов, в скоуп попадают и коерции, которые использовались при применении соответствующих конструкторов.
\begin{minted}{haskell}
    interpret :: Term ty -> ty
    interpret = \case
      Const x  -> x              -- ty ?$\sim$? Int
      IsZero t -> interpret == 0 -- ty ?$\sim$? Bool
      If c t e -> if interpret c then interpret t else interpret e
\end{minted}

Заметим, что в структуре данных типа \mintinline{haskell}|Term ty| не хранятся значения типа \texttt{ty}, а этот типовой параметр существует только для обеспечения большей типовой безопасности.
Такие типовые параметры называют \vocab{фантомными}\footnote{\url{https://wiki.haskell.org/Phantom_type}}.

\subsubsection{Typed tagless initial interpreters}

Опишем систему типов нашего маленького языка.
%! suppress = EscapeAmpersand
\begin{equation*}{}
    \infer[Const]{Const~n : int}{n : Int}
    \quad
    \infer[IsZero]{IsZero~n : bool}{n : int}
    \quad
    \infer[If]{If~c~t~e : \tau}{c : bool & t : \tau & e : \tau}
\end{equation*}
Можем в качестве типовых тегов переиспользовать типы Haskell: \[int \rightsquigarrow Int, bool \rightsquigarrow Bool\]

Заметим, что с помощью обобщённого алгебраического типа данных \mintinline{haskell}|Term ty| выше, мы как раз закодировали эти правила вывода.
Иначе говоря, мы получили типизированный язык программирования, переиспользовав систему типов Haskell.

Таким образом, мы бесплатно получили статически типизированный язык, не нуждающийся в типовых тегах времени исполнения (отсюда слово tagless).

% todo https://jesper.cx/posts/1001-syntax-representations.html

\subsection{Функции первого класса} \label{subsec:first-class-functions}

В этом параграфе мы рассмотрим техники и понятия, относящиеся к реализации функций первого класса.
Эти понятия оказываются крайне полезны и продуктивны, как мы увидим далее.

Те же рассуждения справедливы и для описания \mintinline{haskell}|let|-связываний, поскольку их можно представлять следующим образом:
\[
    \term{let} \ap x \termdef N \ap \term{in} \ap M \equiv (\lambda x\ldotp M) \ap N
\]

\subsubsection{Связывание имён}

\vocab{Динамическое связывание (dynamic scoping)} --- значение свободных переменных функции зависит от области видимости в месте вызова.
То есть разрешение имени происходит в момент обращения к переменной.
Например, следующий код напечатает \texttt{42}:
\begin{minted}{kotlin}
    val f = {
        val x = 4
        fun inner() = x + 1
        inner
    }
    val x = 41
    println(f())
\end{minted}

Этот подход проще в реализации и использовался в ранних версиях Lisp'ов, например.
Однако в таком случае функции не являются надежным барьером абстракции, по-хорошему все свободные переменные должны являться частью сигнатуры (вернёмся в этому в~\ref{sec:effect-systems}).

\vocab{Лексическое/статическое связывание (lexical/static scoping)} --- переменные связываются со значениями в момент объявления функции, в момент вызова результат зависит только от параметров (по модулю изменяемого состояния\footnote{Например, в Kotlin в лямбды можно захватывать изменяемые переменные. Изменения снаружи наблюдаемы внутри лямбды, и наоборот. Иногда это может быть очень удобно, однако нередко приводит к очень неочевидному поведению.}).
Слово ``лексический'' часто употребляется в языках, когда мы что-то можем понять из исходного кода без запуска программы.
Так, код из примера выше напечатает 5.

Далее в этом разделе мы будем говорить о различных способах реализации функций первого класса со статическим связыванием переменных.

\subsubsection{Подстановки}

Как можно заметить, в классическом лямбда-исчислении подстановки от бета-редукции (вспоминали в разделе~\ref{subsec:terms-reduction}) обеспечивают статическое связывание.
Действительно, аргумент немедленно подставляется во все вхождения переменной, соответственно она не остаётся свободной, а просто исчезает.
\[
    (\lambda x\ldotp (\lambda x\ldotp \lambda y\ldotp x + y) \ap 4) \ap 41 \rightsquigarrow (\lambda x\ldotp (\lambda y\ldotp 4 + y)) \ap 41
\]

Такой подход не является самым эффективным, потому что на каждую аппликацию требуется переписывать код функции (!).
В то же время его довольно просто реализовать для некоторых представлений лямбда-термов.
Рассмотрим пример такого представления --- \vocab{locally nameless}~\cite{chargueraud2012locally}.

\begin{minted}{haskell}
    data Term var
      = Var var
      | App (Term var) (Term var)
      | Lam (Term (Maybe var))
\end{minted}

В этом представлении можно выбирать любой тип для именования свободных переменных:
\begin{minted}{haskell}
    example :: Term String
    example = Var "x" `App` Var "y" -- x y
\end{minted}
Добавление каждой связанной переменной добавляет типу переменных нового обитателя \mintinline{haskell}|Nothing| для обращения к ближайшей связанной переменной:
\begin{minted}{haskell}
    -- ?$\lambda x\ldotp x \ap y$?
    example1 = Lam $ Nothing `App` Just "y"
    -- ?$\lambda x \ap y\ldotp x \ap y \ap z$?
    example2 = Lam $ Lam $ Just Nothing `App` Nothing `App` Just (Just "z")
\end{minted}

Удивительно, но монадический bind является реализацией подстановки для таких термов.

\begin{minted}{haskell}
    instance Monad Term where
      (>>=) :: Term var -> (var -> Term var') -> Term var'
      Var var >>= subst = subst var
      App l r >>= subst = App (l >>= subst) (r >>= subst)
      Lam t >>= subst = Lam $ t >>= \case
        Nothing  -> Var Nothing
        Just var -> Just <$> subst var
\end{minted}

\begin{task}
    Подумайте, зачем нужен \texttt{fmap Just} в последней строчке.
\end{task}

Соответственно, call-by-name интерпретатор такого лямбда-исчисления будет выглядеть следующим образом:

\begin{minted}{haskell}
    eval :: Term var -> Term var
    eval = \case
      Var var -> Var var
      App f arg -> case eval f of
        Lam body -> eval $ body >>= maybe arg Var
        t -> App t (eval arg)
      Lam t -> Lam (eval t)
\end{minted}

Можно заметить, что эта реализация не самая эффективная, потому что мы делаем каждое применение функции лишним проходом по терму.

\subsubsection{Окружение}

Можно делать подстановку значений переменных лениво, распространяя окружение, которое ставит в соответствие свободным переменным термы.

\begin{minted}{haskell}
    data Term1 = Var1 String | App1 Term1 Term1 | Lam1 String Term1
    type Env = Map String Term1

    eval1 :: Term1 -> (Env -> Term1)
    eval1 = \case
      Var1 name -> \env -> Map.findWithDefault (Var1 name) name env
      App1 f arg -> \env -> case eval1 f env of
        Lam1 name body -> eval1 body (Map.insert name arg env)
        t -> App1 t (eval1 arg env)
      Lam1 name body -> \env ->
        let env' = Map.delete name env in
        Lam1 name (eval1 body env')
\end{minted}

%! suppress = UnresolvedReference
\begin{task}
    Объясните, зачем окружение модифицируется на строчке~10?
\end{task}

Если ветка \mintinline{haskell}|Lam1| не будет рекурсивно обходить подтерм и подставлять значения переменных, информация о значениях свободных переменных в нём потеряется и мы получим динамическое связывание вместо статического.
Чтобы восстановить статическое связывание, ветка \mintinline{haskell}|Lam1| интерпретатора должна конструировать замыкание, включающее текущее окружение (см. далее~\ref{subsubsec:closures}).

\subsubsection{Верифицированный контекст}

Рассмотрим кодирование, описанное, например, в~\cite{kiselyov2012typed}.

Для начала научимся с помощью системы типов Haskell проверять валидность обращения к окружению.
Представим окружение как список типов, закодированный с помощью вложенных пар:
\begin{minted}{haskell}
    (4, (4.0, "hello")) :: (Int, (Double, String))
\end{minted}

Обращение к окружению будем кодировать числом в унарной записи.
Тип числа (типизированной ссылки внутрь контекста) пусть задаёт множество окружений, из которых на такой позиции можно извлечь нужный тип.
\begin{minted}{haskell}
    data Ref env ty where
      Here :: Ref (ty, env) ty
      There :: Ref env ty -> Ref (ty', env) ty
\end{minted}
Например, тип числа 1 утверждает, что с его помощью можно извлечь значение типа \texttt{ty} из контекста, в котором значение соответствующего типа находится на первой позиции:
\begin{minted}{haskell}
    There Here :: Ref (ty', (ty, env)) ty
\end{minted}

Теперь мы можем закодировать типизированное безопасное обращение к контексту:
\begin{minted}{haskell}
    envLookup :: env -> Ref env ty -> ty
    envLookup env ref = case (ref, env) of
      (Here, (x, _)) -> x
      (There ref', (_, env')) -> envLookup env' ref'
\end{minted}

\begin{task}
    Можно ли разобрать пару сразу на строчке 2?
    Поясните.
\end{task}

\subsubsection{Meta-circular интерпретация}

Крайне не хотелось бы для eDSL самостоятельно реализовывать связывания и функции первого класса.
Построим meta-circular tagless interpreter, который будет переиспользовать функции первого класса мета-языка для реализации их в определяемом языке.

Термы теперь будут не только аннотированы результирующими типами, но и типами необходимых для интерпретации окружений.
Абстрагированному терму доступно большее окружение.

\begin{minted}{haskell}
    data Term2 env ty where
      Var2 :: Ref env ty -> Term2 env ty
      App2 :: Term2 env (arg -> res) -> Term2 env arg -> Term2 env res
      Lam2 :: Term2 (arg, env) res -> Term2 env (arg -> res)
\end{minted}

Теперь абстракцию можем проинтерпретировать в функцию Haskell, а аппликацию --- в аппликацию:
\begin{minted}{haskell}
    eval2 :: Term2 env ty -> env -> ty
    eval2 term env = case term of
      Var2 ref -> env `envLookup` ref
      App2 f arg -> (eval2 f env) (eval2 arg env)
      Lam2 t -> \arg -> eval2 t (arg, env)
\end{minted}

\begin{task}
    Как так получилось, что в последней строчке нужно принять ещё один аргумент?
\end{task}

\begin{task}
    Это call-by-value интерпретатор или call-by-name?
    От чего это зависит?
\end{task}

\begin{task}
    Подумайте, какое решение должно быть более производительно, это или предыдущее?
\end{task}

\subsubsection{Синтаксис высшего порядка} \label{subsubsec:h-syntax}

Ещё чем мы ещё занимаемся вручную --- определяем связыватели (да ещё и в унарной записи).
Хотим переиспользовать языковые идентификаторы.
Для это мы будем прямо в дереве синтаксиса хранить функции мета-языка --- использовать \vocab{синтаксис высшего порядка (higher order abstract syntax)}\footnote{\url{https://en.wikipedia.org/wiki/Higher-order_abstract_syntax}}\footnote{\href{https://cstheory.stackexchange.com/questions/20071/what-is-higher-order-in-higher-order-abstract-syntax}{What is higher-order in higher-order abstract syntax?}}~\cite{pfenning1988higher}.

Теперь мы ссылаемся не на переменные, а на значения, нода абстракции содержит честную функцию:
\begin{minted}{haskell}
    data Term3 ty where
      Val3 :: ty -> Term3 ty
      Plus :: Term3 Int -> Term3 Int -> Term3 Int
      App3 :: Term3 (arg -> res) -> Term3 arg -> Term3 res
      Lam3 :: (Term3 arg -> Term3 res) -> Term3 (arg -> res)

    example3 :: Term3 Int
    example3 = (Lam3 \x -> x `Plus` Val3 41) `App3` Val3 1
\end{minted}

Интерпретация очень простая и абсолютно meta-circular:
\begin{minted}{haskell}
    eval3 :: Term3 ty -> ty
    eval3 term = case term of
      Val3 x -> x
      Plus l r -> eval3 l + eval3 r
      App3 f arg -> (eval3 f) (eval3 arg)
      Lam3 f -> \arg -> eval3 (f (Val3 arg))
\end{minted}

\begin{task}
    Можно ли было объявить \mintinline{haskell}|Lam3| следующим образом?
    \begin{minted}{haskell}
        Lam3 :: (arg -> Term3 res) -> Term3 (arg -> res)
    \end{minted}
\end{task}

\subsubsection{Замыкания} \label{subsubsec:closures}

% todo https://www.youtube.com/watch?v=TcxT0EcA--4&list=PLxMpIvWUjaJv16ZvPEZrwYjBp-6aUIoDL&index=10

Мы можем свести работу с вложенными функциями к работе с глобальными с помощью \vocab{lambda lifting}: вместо каждой вложенной функции заводим по глобальной.
Если вложенная функция имела свободные переменные, добавляем глобальной функции аргумент контекста содержащего значения для этих переменных.
\begin{minted}{kotlin}
    fun f(x) {
        fun g(y) = x + y
        return g
    }
    // lambda lifting породит декларацию
    fun g_glob(ctx: GCtx, y) = ctx.x + y
\end{minted}

Теперь мы можем функциональные объекты представлять как пару из контекста и указателя на глобальную функцию --- то есть как \vocab{замыкание (closure)}\footnote{\url{https://en.wikipedia.org/wiki/Closure_(computer_programming)}}\footnote{Термин closure был предложен Piter Landin, вместе с кучей других вещей.}.
Плавный переход от интерпретаторов с окружением к замыканиям можно посмотреть в гарвардских слайдах~\cite{closures-slides}.
Простую и понятную реализацию для интерпретаторов --- в книжке~\cite[глава 11]{nystrom2021crafting}.

\subsubsection{Дефункционализация} \label{subsubsec:defunctionalization}

\vocab{Дефункционализация (defunctionalization)} --- техника избавление от функций высших порядков в программе\footnote{\url{https://en.wikipedia.org/wiki/Defunctionalization}}~\cite{defunctionalization-slides}.
Впервые предложена в работе~\cite{reynolds1972definitional, reynolds1998definitional}.

Идея заключается в том, чтобы заменить каждую лямбда-функцию вызовом конструктора некоторого алгебраического типа данных.
А каждый call-cite функции заменить на вызов специальной first-order функции \texttt{apply}, интерпретирующей данный алгебраический тип.
Рассмотрим пример.

\begin{minted}{haskell}
    example = (\x -> isRed x) $ (\b -> mkColor 10 50 b) 120
    -- перепишется на
    data Fun = IsRed | MkColor r g
    apply IsRed x = isRed x
    apply (MkColor r g) b -> mkColor r g b
    example = apply IsRed $ apply (MkColor 10 50) 120
\end{minted}

\subsubsection{Сериализация}

В этом разделе мы говорили о возможных реализациях функций первого класса, то есть функций, которые можно использовать так же гибко, как и данные.
Возникает закономерный вопрос: можем ли мы сериализовать функцию первого класса и послать исполняться на другую машину?

Функция состоит из кода и захваченных свободных переменных в случае статического связывания.
Соответственно, если код представлен в сериализуемом виде (например, позиционно-независимый байт-код), то его в принципе можно переслать по сети и исполнить на другом инстансе виртуальной машины.
Так, например, делает Erlang.
Однако, такой подход невероятно неэффективный, так как байт-код нужно интерпретировать.
Таким образом, Erlang жертвует скоростью исполнения ради горизонтальной масштабируемости.

Существуют забавные работы, которые вместо сериализации функции и отправки её на сервер, отправляют некоторый хендл, по которому сервер может вызвать эту функцию на клиенте и получить результат.
Такая разновидность RPC с поддержкой функций высших порядков\footnote{\url{https://github.com/winter-yuki/LambdaRPC.kt}}.

Если мы гарантируем, что на различных узлах кластера исполняется один и тот же кодЮ, как обычно и бывает на практике, можно добиться более эффективной реализации.
Например, используя дефункционализацию~(см.~\ref{subsubsec:defunctionalization}), мы можем сериализовать только объекты алгебраического типа, кодирующие функции.
Поскольку на другом узле кластера исполняется такой же код, мы там можем десериализовать объект и исполнить его с помощью \texttt{apply}.
Однако этот подход не очень поддерживает модульность (сложно один алгебраический тип разбить на много, вернёмся к этой задаче в разделе~\ref{sec:effect-handlers}), а так же \texttt{apply} каждый раз производит декодирование перед исполнением кода (чем, в прочем, можно пренебречь, учитывая работу с сетью).

Подход, реализованный в Haskell\footnote{\url{https://blog.ocharles.org.uk/blog/guest-posts/2014-12-23-static-pointers.html}}\footnote{\url{https://hackage.haskell.org/package/distributed-closure}} позволяет наделить каждую функцию без свободных переменных некоторым статически известным адресом, одинаковым для всех инстансов приложения.
Далее можно сконструировать сериализуемое замыкание путём последовательности частичных применений:
\begin{minted}{haskell}
    data Closure a where
      StaticPtr :: StaticPtr b -> Closure b
      Encoded :: ByteString -> Closure ByteString
      Ap :: Closure (b -> c) -> Closure b -> Closure c

    main = send "some-node" $
      closure (static factorial) `closureAp` closurePure 10
\end{minted}

Подробнее можно прочитать в основополагающей статье про облачный Haskell~\cite{epstein2011towards}.
С практической точки зрения --- в книжке~\cite[глава 16]{marlow2011parallel}.

\begin{task}
    Нужно ли явно добавлять в замыкание свободную переменную \texttt{(*)} (оператор умножения) в реализации факториала?
\end{task}

%
%    \clearpage


    \section{Продолжения (continuations)} \label{sec:continuations}
%
%    %! suppress = MissingLabel

Продолжения с начала 60х не один раз возникали в литературе в различных формах и разнообразных приложениях~\cite{reynolds1993discoveries, landin1997histories}, пока в 70х Wadsworth не придумал общий термин и единую концепцию --- \vocab{continuation}\footnote{\url{https://en.wikipedia.org/wiki/Continuation}} --- ``the meaning of the rest of the program''.

Начальным толчком к размышлениям стал язык Algol 60, имевший нетривиальный механизм меток и прыжков.
Проблемой была как имплементация семантики, так и её денотационное описание вместе с трансляцией в лямбда-исчисление.
Действительно, как математически описать \texttt{goto}?
В каком домене искать семантику таких программ?
Как написать определяющий интерпретатор, отправляющий программу в этот домен?
Ответом стала возможность сослаться на семантику остатка программы в определённой точке (например, на метке).

Итак, продолжение --- это абстрактная концепция, обозначающая остаток вычисления.
Например, когда мы редуцируем простое арифметическое выражение, вычислитель фокусируется внутрь его, чтобы найти редекс.
Оставшееся выражение с дыркой, в которую нужно будет подставить результат вычисления редекса, является продолжением.

\begin{figure}[h]
    \centering
    \begin{tabular}{|c|c|}
        \hline
        Фокус               & Продолжение           \\
        \hline
        $((2 + 3) - 4) * 5$ & $\boxempty$           \\
        $(2 + 3) - 4$       & $\boxempty * 5$       \\
        $2 + 3$             & $(\boxempty - 4) * 5$ \\
        $5$                 & $(\boxempty - 4) * 5$ \\
        $5 - 4$             & $\boxempty * 5$       \\
        $1$                 & $\boxempty * 5$       \\
        $1 * 5$             & $\boxempty$           \\
        $5$                 & $\boxempty$           \\
        \hline
    \end{tabular}
\end{figure}

Практически говоря, продолжение представляет собой некоторую структуру данных, сохраняющее всё что нужно, чтобы продолжить исполнение программы с определённого места.
Иначе говоря, это некоторый снапшот состояния вычислителя, интерпретатора.

Как правило, продолжения существуют вне пользовательского кода.
Это разработчику системы исполнения языка нужно думать, хранение чего нужно поддержать, чтобы продолжать исполнять программу в каждый момент.
Однако, современные языки предоставляют пользователям множество конструкций, позволяющих управлять продолжениями:

\begin{itemize}
    \item Функция \mintinline{haskell}|exit| выбрасывает продолжение программы целиком;
    \item Конструкция \mintinline{kotlin}|try-catch| позволяет выбросить часть продолжения до места поимки исключения и восстановить доставшееся;
    \item Конструкция \mintinline{kotlin}|return| позволяет восстановить исполнение в месте, где функция была вызвана;
    \item Конструкции \mintinline{kotlin}|break| и \mintinline{kotlin}|continue| восстанавливают продолжение после цикла и до\ldots
\end{itemize}

В примерах выше конструкции языка управляют продолжениями неявно.
Однако, иногда вводят операторы, позволяющие явно оперировать продолжениями.

\vocab{Продолжения первого класса (first-class continuations)} --- продолжения, которые представимы в программе в виде значений.
Учитывая, что продолжение имеет вакантное место ещё не вычисленного подвыражения, продолжения первого класса представляют функциями первого класса.

Чтобы получить в коде продолжение первого класса, нужно либо написать код в специальном виде, либо воспользоваться встроенным в язык оператором~\cite[приложение A]{hillerstrom2022foundations}.
Например, $J$, \texttt{escape}~\cite{reynolds1972definitional}, \texttt{call/cc}\ldots

\subsection{Continuation-passing style (CPS)}

Базовым способом получить в программе продолжение первого класса --- написать программу в стиле CPS.
CPS эксплуатирует следующий изоморфизм:
\begin{minted}{haskell}
    to :: a -> (forall r . (a -> r) -> r)
    to x k = k x

    from :: (forall r . (a -> r) -> r) -> a
    from comp = comp id
\end{minted}

Иначе говоря, вместо того, чтобы предоставить значение, можно запросить у вызывающей стороны, как она собирается с этим значением работать, сделать это самостоятельно и вернуть вызывающей стороне\footnote{\url{https://wiki.haskell.org/Cont_computations_as_question-answering_boxes}}.

Корни этого изоморфизма в лемме Йонеды из теории категорий~\cite{hinze2010reason}.
Прикладным же программистам он знаком по технике использования callback'ов.

Например, мы можем переписать факториал в стиле CPS.
Заметьте, что код имеет доступ к продолжению первого класса (однако, никак нетривиально не использует его).
\begin{minted}{haskell}
    facCps :: Int -> (forall r . (Int -> r) -> r)
    facCps n k
      | n <= 1 = k 1
      | otherwise = facCps (n - 1) \res -> k (n * res)
\end{minted}

\begin{task}
    Вручную поредуцируйте определение \mintinline{haskell}|facCps| на простом примере.
\end{task}

\begin{task}
    Сколько функция \mintinline{haskell}|facCps| потребляет стековой памяти?
\end{task}

\subsubsection{Монада \texttt{Cont}}

Из-за CPS код потерял привычную структуру, при которой функции напрямую возвращают свои результаты (i.e. \vocab{direct style}).
При наличии большого количества вызовов трансформированных функций, код становится плохо читаемым.

\begin{minted}{haskell}
    fibCps :: Int -> (forall r . (Int -> r) -> r)
    fibCps n k = if n <= 2 then k 1 else
      fibCps (n - 1) \res1 ->
      fibCps (n - 2) \res2 ->
      k (res1 + res2)
\end{minted}

Однако, можно заметить, что монадическое связывание вторым аргументом тоже принимает продолжение, но ``маленькое'', до конца \mintinline{haskell}|do|-блока.
Таким образом, можно попробовать линеаризовать CPS код с помощью монад.

Заведём \mintinline{haskell}|newtype| обёртку для объявления инстансов:
\begin{minted}{haskell}
    newtype Cont r a = Cont { runCont :: (a -> r) -> r }
\end{minted}

Функтор добавляет пост-процессинг результату перед передачей в продолжение:
\begin{minted}{haskell}
    instance Functor (Cont r) where
      -- fmap :: (a -> b) -> ((a -> r) -> r) -> ((b -> r) -> r)
      fmap f (Cont comp) = Cont \k -> comp (k . f)
\end{minted}

Аппликатив просто передаёт значение продолжению:
\begin{minted}{haskell}
    instance Applicative (Cont r) where
      pure x = Cont \k -> k x
      (<*>) = ap
\end{minted}

Наконец, монада принимает в связывании ``маленькое'' продолжение, композирует его с ``большим'' продолжением, пришедшим снаружи, и передаёт в данное вычисление:
\begin{minted}{haskell}
    instance Monad (Cont r) where
      Cont comp >>= k = Cont \k' -> comp \x -> runCont (k x) k'
\end{minted}

Теперь мы можем писать линейный код, а монадическая машинерия сама конструирует продолжения и подкладывает в предыдущие вычисления:
\begin{minted}{haskell}
    fibCont :: Int -> Cont r Int
    fibCont n = if n <= 2 then pure 1 else do
      res1 <- fibCont (n - 1)
      res2 <- fibCont (n - 2)
      pure (res1 + res2)
\end{minted}

\begin{task}
    Оборвите вычисление, если \texttt{res1} больше \texttt{50}.
\end{task}

\begin{task}
    Оборвите вычисление как только общий результат стал больше 50.
\end{task}

\subsubsection{\texttt{call/cc}}

Можно получить доступ к продолжению просто написав где-то в \mintinline{haskell}|do|-нотации конструктор:
\begin{minted}{haskell}
    do ...; Cont \?\framebox{k}? -> ...; ...
\end{minted}
Однако, для получения продолжений, как правило, пользуются специальными операторами.
Классическим примером является \texttt{call/cc} (call with current continuation):
\begin{minted}{haskell}
    callCC :: ((a -> Cont r b) -> Cont r a) -> Cont r a
    callCC f = Cont \k -> runCont (f \x -> Cont \_ -> k x) k
\end{minted}

\texttt{call/cc} принимает функцию \texttt{f}, в которую передаёт текущее продолжение (рис.~\ref{fig:call-cc}).
При этом вызов продолжения работает как \mintinline{kotlin}|return| для \texttt{f} (продолжение не содержит кода \texttt{f}):
\begin{minted}{haskell}
    foo :: Int -> Cont r String
    foo x = callCC $ \k -> do
      let y = x ^ 2 + 3
      when (y > 20) $ k "over twenty"
      return (show $ y - 4)
\end{minted}

\begin{figure}
    \centering
    \includegraphics[width=0.6\linewidth]{/home/yukio/Diary/projs/fpcourse/fp-2024/docs/figs/call-cc}
    \caption{\texttt{call/cc} вызывает функцию \texttt{f} с продолжением \texttt{k}.
    Аргумент \texttt{k} (или результат \texttt{f}) подставляется вместо вхождения \texttt{call/cc}.}
    \label{fig:call-cc}
\end{figure}

Например, с помощью \texttt{call/cc} можно реализовать кооперативную многозадачность\footnote{\href{https://en.wikibooks.org/wiki/Haskell/Continuation_passing_style\#Example:_coroutines}{(wiki) Continuation-passing style.
Example: coroutines.}}.
Однако, считается, что \texttt{call/cc} --- не самый удачный примитив\footnote{\url{https://okmij.org/ftp/continuations/against-callcc.html}}.

\subsubsection{Continuation semantics}

Существует стиль формальных семантики, называемый \vocab{continuation semantics}.
Фактически он соответствует написанию определяющих интерпретаторов в CPS\footnote{\url{https://ncatlab.org/nlab/show/continuation-passing+style}}.

Заметьте, что то, что мы строили выше, является shallow embedding некоторого языка, заданного в continuation semantics.
Реализация интерпретатора для deep embedding из~\ref{subsubsec:h-syntax}, например, будет выглядеть следующим образом:
\begin{minted}{haskell}
    eval3' :: Term3 ty -> (forall r . (ty -> r) -> r)
    eval3' term k = case term of
      Val3 x -> k x
      Plus l r -> eval3' l \l' -> eval3' r \r' -> k (l' + r')
      App3 f arg -> eval3' f \f' -> eval3' arg \arg' -> k (f' arg')
      Lam3 f -> k \arg -> eval3' (f (Val3 arg)) id
\end{minted}

Язык с CPS интерпретатором расширить управляющими конструкциями вроде \texttt{call/cc}.
Также можно заметить, что реализация интерпретатора стала хвостово-рекурсивной, а стратегии больше не определяется мета-языком~\cite{reynolds1972definitional}.

% todo call/cc

Проследить семантику операций также можно в следующей нотации, эксплицирующей продолжения:

\begin{align*}
    E[abort(e)] &\rightsquigarrow e\\
    E[call/cc(f)] &\rightsquigarrow E[f(\lambda x\ldotp E[x])]
\end{align*}

\subsubsection{Эффективный CPS}

В нашей реализации CPS производится огромное количество аллокаций замыканий, реифицирующих продолжения.
Однако, если CPS получается автоматической трансляцией, можно делать эффективнее.
Изначально продолжения придумывались для описания семантики прыжков, но можно пойти и в обратную сторону, реализовав продолжения эффективно для современных машин через \texttt{goto}.

Так, состояние функции целиком один раз аллоцируется в куче, а перемещения по телу реализованы как машина состояний --- с помощью меток и прыжков.
Таким образом, например, реализованы безстековые корутины в Kotlin\footnote{\url{https://github.com/Kotlin/KEEP/blob/master/proposals/coroutines.md\#state-machines} \label{note:kotlin-state}}, генераторы в C\#\footnote{\url{https://csharpindepth.com/Articles/IteratorBlockImplementation}}\ldots

Даже в таком виде CPS остаётся тяжеловесной трансформацией, способной замедлить исполнение кода на порядки.
Дело, в частности, в том, что переменные в таком подходе сложно размещать в регистрах (у функций много точек выходов и входов\footref{note:kotlin-state}), приходится постоянно записывать их в RAM --- производить spilling\footnote{\url{https://en.wikipedia.org/wiki/Register_allocation}}.

\subsection{Дефункционализация продолжений}





\footnote{\url{https://ncatlab.org/nlab/show/defunctionalization}}

\cite{reynolds1972definitional}

% todo zipper

% todo huet zipper

% todo zipper context and derivations

% todo abstract machines

% todo

\subsection{Delimited continuations}

\footnote{\url{https://www.cl.cam.ac.uk/teaching/2324/R277/handout-delimited-continuations.pdf}}

\cite{dyvbig2007monadic}

% todo https://okmij.org/ftp/continuations/undelimited.html

% todo in haskell

% todo via mutable cell (Representing Monads)

% todo

\subsection{Нативная реализация продолжений}

\subsubsection{State machine}

% todo циклы и спиллинг

% todo

\subsubsection{Continuios stack}

% todo

\subsubsection{Segmented stack}

% todo

\subsection{Использование продолжений}

% todo trampolining

% todo difference lists

% todo lens

% todo CPS интерпретаторы, переходы

% todo корутины

% todo генераторы

% todo эффекты

% todo transducers, pipelines and internal iteration

% todo CEKT

% todo continuation semantics


% todo citations on operators

% todo continuations and operating systems

% todo origins: yoneda (reason isomorphically), logic

% todo continuations and imperative programming

% todo single-shot/multi-shot

% todo codensity

% todo problems with continuations: typing and resources

% todo implementing while & break

% todo

% todo control operator, runtime наизнанку


% todo coroutines example https://en.wikibooks.org/wiki/Haskell/Continuation_passing_style

% todo пример на хаскеле с примитивом


% todo goto не нужен?  


% todo Foundations for Programming and Implementing Effect Handlers

% todo sicp

% todo CPS трансформация интерпретаторов

% todo CPS correspond to double negation in logic reynolds1998definitional

% todo defuctionalized continuations, pepers from gibbons talk

% todo SPJ compiling without continuations

% todo difference lists

% todo The Essence of Cornpiling with Continuations

% todo calculating correct compilers

% todo reference to best refactoring
% todo refrence co calculate correct compilers and stuff

\cite{reynolds1972definitional, reynolds1998definitional, defunctionalization-slides}

% todo Oleg's web cite

% todo tags and prompts

%
%    \clearpage


    \section{Эффекты и хендлеры} \label{sec:effect-handlers}
%
%    % todo obsidian What is a purely functional language?
%
%    %! suppress = MissingLabel

Язык может давать возможность описывать чистые вычисления, предоставляя новый предметно-ориентированный словарный запас.
Однако реальные программы не получится описывать только чистыми вычислениями.
Поэтому нужно иметь возможность создавать языки для описания effectful вычислений.
Об эффектах мы начинаем говорить в этой главе.

\subsection{Понятие эффекта}

Начнём разговор про эффекты апофатично, с того, чем они не являются.
\vocab{Чистое вычисления} --- единственный его наблюдаемый результат --- его значение.
\vocab{Чистая функция} --- результат зависит только от значения аргументов и её аппликация --- чистое вычисление.

Программирование чистыми функциями считается хорошей практикой, так как этот стиль обладает большим количеством хороших свойств.
Так, композиция чистых функций --- чистая функция; всё, что нужно для понимания кода, явно написано в этом коде; системы типов хорошо работают, предоставляя чистоту абстракции, документацию и частичную спецификацию.

Однако, всё чистым кодом записать не получится, несмотря на все эти свойства и то, что даже ввод-вывод так можно описать~\cite{jones2001tackling}.
% todo
\begin{minted}{haskell}
    getList :: Int -> World -> (World, [Int])
    getList n w | n == 0 = (w, [])
                | otherwise =
      let (w', x) = getInt w in
      fmap (x :) (getList (n - 1) w')
\end{minted}

Нужно делегировать весь этот bookkeeping сторонней сущности, чтобы она занималась этим за нас, то есть замести неинтересное под ковёр. % todo

\begin{minted}{haskell}
        getList :: Int -> IO [Int]
        getList n | n == 0 = pure []
                  | otherwise = do
          x <- getInt
          fmap (x :) (getList (n - 1))
\end{minted}

Такой сущностью является интерпретатор языка.
А сам эффект является взаимодействием с контекстом исполнения.
А поскольку мы делаем shallow-embedded встроенные языки для переиспользования максимума мета-языка, нужны высокоуровневые конструкции для конструирования сложных денотаций.

% todo

TODO недетерминизм и matter of perspective\footnote{\url{https://okmij.org/ftp/Computation/having-effect.html}} % todo

% todo semantic domains have monad structure

\subsection{Монады для моделирования эффектов}

\vocab{Монада} --- это тройка $(m, \eta : a \to m a, \mu : m (m a) \to m a)$.
% todo diagrams

Монады для структурирования семантики~\cite{moggi1988computational}, монады как обобщение list comprehension~\cite{wadler1990comprehending}, monads to structure functional programs~\cite{wadler1992essence} % todo

<<Computations of type $\alpha$ correspond to values of type $f\ap\alpha$>>, Moggi's principle. % todo

\subsection{Трансформеры монад и модульные интерпретаторы}

% todo remind of tagless final

% todo lift - monad morphism

% todo ContT, lift = (>>=)

TODO\cite{liang1995monad} % todo

\subsection{Представления монад}

\subsection{Свободные монады}

% todo Stackless Scala with free monads

% todo free monads, freer monads

% todo iteratee O. Kiselyov. Iteratees. In Proc. of the 11th International Symposium on Functional and Logic Programming, pages 166–181, 2012.

% todo Monads and algebras

% todo codensity, reflection without remorse




=% todo obsidian What is a purely functional language?

% todo monad error is a bullshit

% todo iteratee

% todo https://homepages.inf.ed.ac.uk/wadler/papers/expression/expression.txt

% todo https://www.eff-lang.org/handlers-tutorial.pdf

% todo алгебраические эффекты
% todo связь с delimited continuations
% todo стратегии компиляции, связь с codata
% todo эффекты высших порядков
% todo full vs shallow embeddings
% todo abstracting definitional interpreters & github semantics
% todo fused effects and CPS

% todo функция это тоже способ унести код куда-то, обобщённый алгебраический эффект отличается более тонким контролем над континуэйшеном места вызова

% todo expression problem

% todo compare open type families & extensible interpreters

% todo Polymorphic Symmetric Multiple Dispatch with Variance

% todo \textit{multimethods}

% todo  custom schedulers

%    Languages with \textit{multimethods}, like Common Lisp’s CLOS, Dylan, and Julia do support adding both new types and operations easily.
%    What they typically sacrifice is either static type checking, or separate compilation.

% todo ZIO, TS Effect

% todo call-by-push value and how it is related to effects

% todo context polymorphism



    \section{Интерпретаторы зазеркалья} \label{sec:wonder-interpreters}
%
%    



















% todo wadler the expression problem

% todo observers

% todo reactive programming https://youtu.be/sTSQlYX5DU0?si=Ybxux16h5Vt1LnC4

% todo pull and push

% todo expression problem

% todo are custom patterns in haskell connected to this?

% todo вместо того, чтобы доставлять объект к месту деконструирования, доставляем мето деконструирования к месту конструирования

% todo initial and final - isomorphism proof

% todo The Duality of Computation

% todo теоретические основы

% todo church encoding

% todo Tagless-final интерпретаторы

% todo codata, pattern-matching
% todo custom patterns

% todo Extensibility for the Masses: Practical Extensibility with Object Algebras

% todo fusion и дефорестация
% todo haskell inlining
% todo RULES
% todo Oleg about fusion and streams
% todo fused effects

% todo Reading circle about fusion
% todo short cut to deforestation
% todo stream fusion from lists to streams to nothing at all
% todo call-pattern specialization for Haskell programs

% todo композиционность и её восстановление через эксплицирование контекстных зывисимостей.

% todo threaded code

% todo connections to polarity https://ncatlab.org/nlab/show/polarity+in+type+theory

% todo https://okmij.org/ftp/tagless-final/course/Boehm-Berarducci.html

\cite{gibbons2013functional, gibbons2014folding}

% todo
%
%    \clearpage


%    \section{Дополнительные главы монад} \label{sec:monads}
%
%    %! suppress = MissingLabel

Язык может давать возможность описывать чистые вычисления, предоставляя новый предметно-ориентированный словарный запас.
Однако реальные программы не получится описывать только чистыми вычислениями.
Поэтому нужно иметь возможность создавать языки для описания effectful вычислений.
Об эффектах мы начинаем говорить в этой главе.

\subsection{Понятие эффекта}

Начнём разговор про эффекты апофатично, с того, чем они не являются.
\vocab{Чистое вычисления} --- единственный его наблюдаемый результат --- его значение.
\vocab{Чистая функция} --- результат зависит только от значения аргументов и её аппликация --- чистое вычисление.

Программирование чистыми функциями считается хорошей практикой, так как этот стиль обладает большим количеством хороших свойств.
Так, композиция чистых функций --- чистая функция; всё, что нужно для понимания кода, явно написано в этом коде; системы типов хорошо работают, предоставляя чистоту абстракции, документацию и частичную спецификацию.

Однако, всё чистым кодом записать не получится, несмотря на все эти свойства и то, что даже ввод-вывод так можно описать~\cite{jones2001tackling}.
% todo
\begin{minted}{haskell}
    getList :: Int -> World -> (World, [Int])
    getList n w | n == 0 = (w, [])
                | otherwise =
      let (w', x) = getInt w in
      fmap (x :) (getList (n - 1) w')
\end{minted}

Нужно делегировать весь этот bookkeeping сторонней сущности, чтобы она занималась этим за нас, то есть замести неинтересное под ковёр. % todo

\begin{minted}{haskell}
        getList :: Int -> IO [Int]
        getList n | n == 0 = pure []
                  | otherwise = do
          x <- getInt
          fmap (x :) (getList (n - 1))
\end{minted}

Такой сущностью является интерпретатор языка.
А сам эффект является взаимодействием с контекстом исполнения.
А поскольку мы делаем shallow-embedded встроенные языки для переиспользования максимума мета-языка, нужны высокоуровневые конструкции для конструирования сложных денотаций.

% todo

TODO недетерминизм и matter of perspective\footnote{\url{https://okmij.org/ftp/Computation/having-effect.html}} % todo

% todo semantic domains have monad structure

\subsection{Монады для моделирования эффектов}

\vocab{Монада} --- это тройка $(m, \eta : a \to m a, \mu : m (m a) \to m a)$.
% todo diagrams

Монады для структурирования семантики~\cite{moggi1988computational}, монады как обобщение list comprehension~\cite{wadler1990comprehending}, monads to structure functional programs~\cite{wadler1992essence} % todo

<<Computations of type $\alpha$ correspond to values of type $f\ap\alpha$>>, Moggi's principle. % todo

% todo lalalang

% todo оригинальная статья

% todo replicateM

\subsection{Трансформеры монад и модульные интерпретаторы}

% todo remind of tagless final

% todo lift - monad morphism

% todo ContT, lift = (>>=)

TODO\cite{liang1995monad} % todo

\subsection{Представления монад}

% todo Monads and composable continuations by Phill Wadler

% todo Representing Monads Filinsky
% todo paper delimited continuations in WebAssembly

% todo связь с континуэйшенами

% todo mother of all monads

% todo monads in python with nice syntax

% todo monadic reflection & direct style (lib from scala)
% todo https://github.com/lampepfl/monadic-reflection/blob/main/TUTORIAL.md

% todo reflection without remorse

% todo C# expression trees
% todo F# что-то про монады и forM https://learn.microsoft.com/en-us/dotnet/fsharp/language-reference/computation-expressions

% todo https://blog.poisson.chat/posts/2019-10-27-continuation-submonads.htm

% todo https://www.unison-lang.org/docs/fundamentals/abilities/for-monadically-inclined/

% todo https://www.schoolofhaskell.com/school/to-infinity-and-beyond/pick-of-the-week/the-mother-of-all-monads

% todo http://www.valuedlessons.com/2008/01/monads-in-python-with-nice-syntax.html

\subsection{Свободные монады}

% todo Stackless Scala with free monads

% todo free monads, freer monads

% todo iteratee O. Kiselyov. Iteratees. In Proc. of the 11th International Symposium on Functional and Logic Programming, pages 166–181, 2012.

% todo Monads and algebras

% todo codensity, reflection without remorse



    \section{Системы эффектов} \label{sec:effect-systems}
%
%    
% todo merriage of effects and monads

% todo https://github.com/yallop/effects-bibliography

TODO % todo


%Чёткого определения контекста исполнения не существует, однако можно выделить свойства, по которым такие контексты можно распознавать в коде.
%Так, контекст исполнения должен обладать хотя бы одним из следующих двух свойств:
%\begin{enumerate} % todo
%    \item Взаимодействие с контекстом наблюдаемо.
%    Например, в качестве контекста мы можем рассматривать подсистему управления памятью, а в качестве effectful операции --- операцию изменения значения ячейки, так как изменение можно пронаблюдать последующей операцией чтения.
%    \item Действие контекста ограничено определённым скоупом.
%    Например, обработчик исключения является контекстом исполнения для кода внутри \mintinline{kotlin}{try-catch} блока: \mintinline{kotlin}{throw} передаст управление соответствующему обработчику.
%    Также, за пределами такого скоупа операции могут иметь другой смысл.
%    Например, система распространения зависимостей (dependency injection) в различных скоупах может выдавать различную функциональность.
%\end{enumerate}


%
%    \clearpage


    \section{Datatype-generic programming} \label{sec:datatype-generic}
%
%    %! suppress = MissingLabel

Ранее мы говорили о том, что всё, что пишут программисты, --- интерпретаторы.
Оказывается, многие из них можно получать автоматически, выйдя на новый уровень обобщения.
А код, который может не быть написан, --- не должен быть написан.

Как правило, такая ситуация возникает, когда интерпретация полностью определяется описанием (формой) типа.
То есть тем, какие конструкторы данных для него определены, с какими полями, как они называются\ldots
Такими интерпретациями являются, например, сериализация в конеретный формат (e.g. json), структурное равенство и т.д.

Решением является поддержка в языке \vocab{структурного полиморфизма} \vocab{(structural polymorphism)}~\cite[глава 13]{maguire-types} или \vocab{datatype-generic программирование}, как это иногда называют.

% todo не хотим писать код, который можно не писать

Можно вообразить язык, в котором data-generic программирование является обычным программированием~\cite{chapman2010gentle}.
Для этого берётся язык с зависимыми типами, с выделенной вселенной описаний \mintinline{haskell}|Desc|.
Конкретные типы же получаются применением функции интерпретации \mintinline{haskell}|Desc -> Type|.
Таким образом, параметрический полиморфизм воплощается обобщением по типу, а структурный полиморфизм --- обобщением по его описанию.

% todo generics and Gibbons paper about abstractions

% todo Derivable type classes paper

TODO % todo

\subsection{Специализация}

\[
    \sembr{\sembr{spec}(p, x)}(y) \equiv \sembr{p}(x, y)
\]

% todo специализация в Haskell

% todo staged programming

% todo incremental computations

% todo Zig

% todo Oleg's finally tagless partially evaluated paper

% todo modal types for staged programming

% todo haskell SPECIALIZE

TODO\cite{liu2023incremental} % todo

\subsection{Ad-hoc решения}

% todo Kotlinx serialization

TODO % todo

\subsection{Подходы к datatype-generic в Haskell}

TODO % todo https://ora.ox.ac.uk/objects/uuid:cd041aa0-8d69-4f18-bce2-6d75244b69b9/files/m7fd9779b27fcb70ae734e66bcd3ee79f

\subsubsection{Template Haskel}

% todo Рефлексия и реификация

% todo порядок на коде для поддержания реификации

% todo hygene macro & dynamic scoping

% todo Staging with Class

TODO\footnote{\url{https://markkarpov.com/tutorial/th.html}} % todo

\subsubsection{Deriving strategies}

% todo deriving via & Kotlin delegates
% todo newtype derivings & coersions & kotlin delegates

% todo https://ryanglscott.github.io/2018/06/22/quantifiedconstraints-and-the-trouble-with-traversable/

% todo typed template haskell

TODO\footnote{\url{https://ghc.gitlab.haskell.org/ghc/doc/users_guide/exts/deriving_strategies.html}} % todo

\subsubsection{GHC.Generics}

TODO\footnote{\url{https://ghc.gitlab.haskell.org/ghc/doc/users_guide/exts/generics.html}} % todo

\subsubsection{Sum Of Products}

TODO % todo SOP & variadics

\subsubsection{Uniplate}

% todo SYB

TODO % todo

% todo datatype-generic для интерпретаторов зазеркалья

% todo специализация и tagless final

%
%    \clearpage


    % todo субструктурность, ресурсы....


%%    \section{Компилятор GHC и runtime}
%
%    % todo Working with Source Plugins paper

% todo \href{https://gitlab.haskell.org/ghc/ghc/-/wikis/commentary/compiler/generated-code}{\color{blue} I know kung fu: learning STG by example}

% todo


    % todo куда-нибудь запихнуть algebra driven design, quick check и доклад про проперти-тесты
    % todo Perhaps Not The Answer You Were Expecting But You Asked For It
    % todo link to SICP

    % todo зависимые типы в Haskell

    % todo рассовать упражнений по конспекту

    % todo Wadler's "Introduction to functional programming"

%    \begin{center}
%        \vspace{2em}
%        \textit{\LARGE To be continued...}
%        \vspace{2em}
%    \end{center}


%    \newpage
    \bibliography{bib}

\end{document}
