%! suppress = Makeatletter
%! suppress = TooLargeSection
%! suppress = MissingLabel
\documentclass{article}

% Fields
\usepackage{geometry}
\geometry{top=25mm}
\geometry{bottom=35mm}
\geometry{left=20mm}
\geometry{right=20mm}
% ------------------------------------------------

% Graphics
\usepackage{color}
\usepackage{tabularx}
\usepackage{tikz}
% https://tikz.dev/tikz-graphs
\usetikzlibrary{positioning, shapes.geometric, arrows, automata, graphs}
\tikzset{
    expr/.style={ellipse, draw=gray!60, fill=gray!5, very thick, minimum size=7mm, yshift=0.7cm},
    hexpr/.style={ellipse, draw=gray!60, fill=blue!15, very thick, minimum size=7mm, yshift=0.7cm},
    stmt/.style={rectangle, draw=gray!60, fill=gray!5, very thick, minimum size=5mm, yshift=0.7cm},
    decl/.style={rectangle, draw=blue!60, fill=gray!5, very thick, minimum size=5mm, yshift=0.7cm},
    hdecl/.style={rectangle, draw=blue!60, fill=blue!15, very thick, minimum size=5mm, yshift=0.7cm},
    subtree/.style={shape border rotate=90, isosceles triangle, draw=gray!60, fill=gray!5, very thick, minimum size=5mm, yshift=0.0cm},
}
\usepackage{blkarray}
\usepackage{graphicx}
\usepackage{forest} % https://tex.stackexchange.com/questions/198405/how-to-change-the-color-of-subtrees-in-tikz-qtree
% ------------------------------------------------

% Math
\usepackage{amsmath, amsfonts}
\usepackage{amssymb}
\usepackage{proof}
\usepackage{mathrsfs}
% Crossed-out symbols
% https://tex.stackexchange.com/questions/75525/how-to-write-crossed-out-math-in-latex
\usepackage[makeroom]{cancel}
\usepackage{mathtools}
% ------------------------------------------------

% Additional font sizes
% https://www.overleaf.com/learn/latex/Questions/How_do_I_adjust_the_font_size%3F
\usepackage{moresize}
% Additional colors
% https://www.overleaf.com/learn/latex/Using_colours_in_LaTeX
\usepackage{xcolor}
% \texttimes
\usepackage{textcomp}
% ------------------------------------------------

% Language
\usepackage[utf8] {inputenc}
\usepackage[T2A] {fontenc}
\usepackage[english, russian] {babel}
\usepackage{indentfirst, verbatim}
\usetikzlibrary{cd, babel}
% ------------------------------------------------

% Fonts
\usepackage{stmaryrd}
\usepackage{cmbright}
\usepackage{wasysym}
% ------------------------------------------------

% Code
% https://tex.stackexchange.com/questions/99475/how-to-invoke-latex-with-the-shell-escape-flag-in-texstudio-former-texmakerx
% Colored, requires --shell-escape compiling option
% \usepackage{minted}
% \setminted{xleftmargin=\parindent, autogobble, escapeinside=\#\#}
\usepackage{listings}
% ------------------------------------------------

% Custom commands
\newcommand{\comb}[1]{\mathbf{#1}}
\newcommand{\step}{\rightsquigarrow}
\newcommand{\term}[1]{\mathbf{#1}}
\newcommand{\ap}{~}
\newcommand{\termdef}{\coloneqq}
\newcommand{\subst}[3]{\left[#2 \mapsto #3 \right] #1}
\newcommand{\eqbeta}{=_\beta}
\newcommand{\eqeta}{=_\eta}
% ------------------------------------------------

% Head
\usepackage{fancybox,fancyhdr}
\usepackage{hyperref}
%\pagestyle{fancy}
% ------------------------------------------------

\begin{document}
    \begin{center}
    {\LARGE ФП 2.0}
        \\
        осень 2024
    \end{center}

    \tableofcontents

    \section*{Введение}

    В функциональном программировании и языковом дизайне в частности существует некоторый набор знаний, считающихся общеизвестным фольклором.
    Сложность в том, что эти знания рассеяны по книгам, статьям и ``культовым'' блог-постам, и требуется довольно много времени и сил для восстановления целостной картины.

    Цель этого курса --- собрать в одном месте такие фольклорные знания и организовать их в некоторую систему.
    Курс будет явным образом опираться на классические работы и помогать в их изучении.

    В связи с широтой контекста, этот курс является обзорным и не всегда глубоким.
    Так, детали реализации в GHC или теор-категорные основания вещей могут даваться в общем виде и без конкретики.
    В то же время в плоскости языкового дизайна и пользовательского опыта на Haskell курс пытается быть максимально подробным.

    Выбор тем и структура курса во многом обусловлены организацией языка Haskell, который этот курс использует как референсный.
    Не в последнюю очередь выбор тем для курса обусловлен интересами его автора.


    \section{Параметрический полиморфизм}

    \subsection{Напоминание про типы}

    Программное обеспечение --- это сложно.
    Поэтому постоянно и неизбежно в программах возникают ошибки.
    Их можно искать, в том числе, статически, то есть без запуска программы.
    Одним из видов статического анализа является анализ типов.

    Как известно, выражение можно представить в виде дерева.
    Для примера рассмотрим $(\lambda y\ldotp x \div (y + 2)) \ap 3$:
    \\
    \begin{tikzpicture}
        \node [expr] (ap) {$@$};
        \node [expr] (three) [below right = of ap]{$3$};
        \node [hexpr] (lam) [below left =of ap] {$\lambda$};
        \node [hdecl] (y) [below right =of lam] {$y$};
        \node [expr] (div) [below=of lam] {$\div$};
        \node [expr] (x) [below left=of div] {$x$};
        \node [expr] (plus) [below right = of div] {$+$};
        \node [hexpr] (param) [below left=of plus] {$y$};
        \node [expr] (two) [below right = of plus] {$2$};
        \draw[->] (three) -- (ap);
        \draw[->] (lam) -- (ap);
        \draw[->] (div) -- (lam);
        \draw[->] (plus) -- (div);
        \draw[->] (x) -- (div);
        \draw[->] (two) -- (plus);
        \draw[->] (param) -- (plus);
        \draw[->] (y) -- (lam);
    \end{tikzpicture}

    Идея анализа типов состоит в том, что мы каждой вершине дерева программы пытаемся присвоить некоторую синтаксическую метку по определённым правилам.
    Если каждой вершине метку присвоить можно, то мы считаем, что программа проходит проверку типов, и она ``хорошая''.

    Пример выше проходит проверку типов в некоторой разумной системе типов:

    \begin{tikzpicture}
        \node [expr] (ap) {$@ : int$};
        \node [expr] (three) [below right = of ap]{$3 : int$};
        \node [hexpr] (lam) [below left=of ap] {$\lambda : int \to int$};
        \node [decl] (y) [below right = of lam] {$y {\color{red}~: int}$};
        \node [expr] (div) [below=of lam] {$\div : int$};
        \node [expr] (x) [below left=of div] {$x : int$};
        \node [expr] (plus) [below right = of div] {$+ : int$};
        \node [hexpr] (param) [below left=of plus] {$y : int$};
        \node [expr] (two) [below right = of plus] {$2 : int$};
        \draw[->] (lam) -- (ap);
        \draw[->] (three) -- (ap);
        \draw[->] (div) -- (lam);
        \draw[->] (plus) -- (div);
        \draw[->] (x) -- (div);
        \draw[->] (two) -- (plus);
        \draw[->] (param) -- (plus);
        \draw[->] (y) -- (lam);
    \end{tikzpicture}

    Система типов определяет синтаксис типовых меток и правила, по которым их можно приписывать.
    Синтаксис обычно описывается в классических нотациях а ля BNF, а правила в виде типовых дробей.
    Например, так выглядят дроби для просто-типизированного $\lambda$-исчисления:
    \[
        \begin{array}{l c r}
            \infer[ctx]{\Gamma \vdash x: \sigma}{(x: \sigma) \in \Gamma}
            &
            \infer[elim\to]{\Gamma \vdash M\;N : \tau}{\Gamma \vdash M : \sigma \to \tau & \Gamma \vdash N : \sigma}
            &
            \infer[intro\to]{\Gamma \vdash \lambda x^{\color{red} \sigma}\ldotp M : \sigma \to \tau}{\{x : \sigma\} \cup \Gamma \vdash M : \tau}
        \end{array}
    \]

    Типовые метки имеют чисто-синтаксическую природу, однако их можно проинтерпретировать.
    Самая популярная интерпретация~--- воспринимать типовую метку как множество.
    Так, метке $int \to int$ можно поставить в соответствие множество функций между множествами ограниченных целых чисел.

    \subsection{Параметрический полиморфизм первого ранга}




    Подобно тому как $\lambda$-выражения позволяют обобщать выражения по значениям на уровне термов, так $\forall$ (\texttt{forall}) в типах позволяет обобщать типы

    % todo

    \subsection{Полиморфизм высшего ранга}

    % TODO

    \subsection{Реализация параметрического полиморфизма}

    % todo

    \subsection{Полиморфизм по runtime-представлению}

    %todo


    \section{Специальный полиморфизм}

    % todo


    \section{Дерайвинги и специализация}

    \subsection{Template Haskell}

    % todo

    \subsection{Рефлексия и реификация}

    % TODO

    \subsection{Zig-generics}

    % todo

    \subsection{GHC.Generics}

    % todo

    \subsection{Uniplate}

    % todo


    \section{Оптика}

    % todo


    \section{Инициальные интерпретаторы}

    % todo


    \section{Реализация функций высших порядков}

    % todo


    \section{Tagless-final интерпретаторы}

    % todo


    \section{Монады}

    % todo


    \section{Продолжения (continuations)}

    % todo

    % todo связь с монадами


    \section{Алгебраические эффекты}

    % todo


    \section{Коэффекты}

    % todo


    \section{Корутины}

    % todo


    \section{Компилятор GHC}

    % todo


    \section{Персистентные структуры данных}

    % todo

\end{document}
