%! suppress = MissingLabel

Как мы убедились ранее (см.~\ref{subsec:all-interpreters}), программирование состоит из написания новых и новых интерпретаторов поверх друг друга.
Интерпретаторы задают семантику новых языков (см~\ref{subsubsec:semantics}).
В классическом виде язык задаётся как множество деревьев, а интерпретатор отправляет деревья в объект мета-языка.
Если язык встроенный, то такой подход называют deep embedding (см.~\ref{subsubsec:edsl}).
\[
    \sembr{\bullet} : L \to D
\]

Можно заметить, что в конечном итоге мы используем только элемент домена, в который интерпретатор отображает программу.
Сама программа же представляет собой лишь удобную синтаксическую запись элемента домена и является промежуточным шагом, а не самоцелью.
В то же время доменом в случае встроенных языков, заданных интерпретаторами, являются объекты мета-языка.
Можем ли мы миновать стадию интерпретации собственного синтаксиса и сразу строить объект домена в синтаксисе мета-языка?
Да, такое встраивание называется shallow embedding, о нём эта глава.

\subsection{Формальный путь в зазеркалье}

\subsubsection{Алгебраическое представление типа}

% todo reason isomorphically
% todo connection with cardinalities

% todo isomorphism

\cite[глава 1]{maguire-types}

% todo

\subsection{Tagless final интерпретаторы}

% todo

\subsection{Использование зазеркалья в реальности}

\subsubsection{Expression problem}

% todo







% todo presentation by valera


% todo canonnical sums and products

% todo deforestation of a syntax

% todo visitor pattern














% todo wadler the expression problem

% todo observers

% todo reactive programming https://youtu.be/sTSQlYX5DU0?si=Ybxux16h5Vt1LnC4

% todo pull and push

% todo are custom patterns in haskell connected to this?

% todo вместо того, чтобы доставлять объект к месту деконструирования, доставляем мето деконструирования к месту конструирования

% todo initial and final - isomorphism proof

% todo The Duality of Computation

% todo теоретические основы

% todo church encoding

% todo Tagless-final интерпретаторы

% todo codata, pattern-matching

% todo Extensibility for the Masses: Practical Extensibility with Object Algebras

% todo fusion и дефорестация
% todo haskell inlining
% todo Oleg about fusion and streams
% todo fused effects

% todo codensity

% todo Reading circle about fusion
% todo short cut to deforestation
% todo stream fusion from lists to streams to nothing at all
% todo call-pattern specialization for Haskell programs

% todo композиционность и её восстановление через эксплицирование контекстных зывисимостей.

% todo threaded code

% todo connections to polarity https://ncatlab.org/nlab/show/polarity+in+type+theory

% todo https://okmij.org/ftp/tagless-final/course/Boehm-Berarducci.html

\cite{gibbons2013functional, gibbons2014folding}

% todo
