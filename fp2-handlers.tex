%! suppress = MissingLabel

По-другому посмотрим на интерпретаторы встроенных языков, доведём их до абсолюта.
Чтобы менеджить сложность.

% todo

\subsection{Понятие эффекта}

Начнём разговор про эффекты апофатично, с того, чем они не являются.
\vocab{Чистое вычисления} --- единственный его наблюдаемый результат --- его значение.
\vocab{Чистая функция} --- результат зависит только от значения аргументов и её аппликация --- чистое вычисление.

% todo obsidian What is a purely functional language?

Программирование чистыми функциями считается хорошей практикой, так как этот стиль обладает большим количеством хороших свойств.
Так, композиция чистых функций --- чистая функция; всё, что нужно для понимания кода, явно написано в этом коде; системы типов хорошо работают, предоставляя чистоту абстракции, документацию и частичную спецификацию\ldots

Всё можно записать в виде чистых вычислений, даже IO~\cite{jones2001tackling}.
Но результат нам не понравится~--- всё приходится делать вручную:
\begin{minted}{haskell}
    getList :: Int -> World -> (World, [Int])
    getList n w | n == 0 = (w, [])
                | otherwise =
      let (w', x) = getInt w in
      let (w'', xs) = getList (n - 1) w' in
      (w'', x : xs)
\end{minted}

Нужно делегировать весь этот bookkeeping сторонней сущности, чтобы она занималась этим за нас, то есть замести неинтересное под ковёр:
\begin{minted}{haskell}
    getList :: Int -> IO [Int]
    getList n | n == 0 = pure []
              | otherwise = do
      x <- getInt
      xs <- getList (n - 1)
      return (x : xs)
\end{minted}

Эту абстрактную сущность мы будем называть \vocab{контекстом исполнения}.
А \vocab{эффектом}~--- взаимодействие с контекстом исполнения. % todo Получаем красивую картинку
Контекст исполнения характеризуется двумя свойствами (противополагающими его чистым вычислениям):
\begin{enumerate}
    \item Взаимодействие с контекстом наблюдаемо.
    Контекст~--- memory manager, мутации наблюдаемы~--- влияют на чтение.
    \item Действие контекста ограничено определённой областью.
    Эксепшены.
    Инверсия зависимостей.
\end{enumerate}

На практике этим контекстом является интерпретатор встроенного языке, а операции с эффектом~--- его конструкции.
Поэтому мы снова возвращаемся к задаче построения модульных интерпретаторов.

\subsection{Хендлеры эффектов}



% todo

\subsection{Трансформеры монад}

% todo










%TODO недетерминизм и matter of perspective\footnote{\url{https://okmij.org/ftp/Computation/having-effect.html}} % todo
%
% todo diagrams

% todo remind of tagless final

% todo lift - monad morphism

% todo ContT, lift = (>>=)

%TODO\cite{liang1995monad} % todo

%\subsection{Свободные монады}

% todo Stackless Scala with free monads

% todo free monads, freer monads

% todo iteratee O. Kiselyov. Iteratees. In Proc. of the 11th International Symposium on Functional and Logic Programming, pages 166–181, 2012.

% todo Monads and algebras

% todo monad error is a bullshit

% todo iteratee

% todo https://homepages.inf.ed.ac.uk/wadler/papers/expression/expression.txt

% todo https://www.eff-lang.org/handlers-tutorial.pdf

% todo алгебраические эффекты
% todo связь с delimited continuations
% todo стратегии компиляции, связь с codata
% todo эффекты высших порядков
% todo full vs shallow embeddings
% todo abstracting definitional interpreters & github semantics
% todo fused effects and CPS

% todo функция это тоже способ унести код куда-то, обобщённый алгебраический эффект отличается более тонким контролем над континуэйшеном места вызова

% todo compare open type families & extensible interpreters

% todo Polymorphic Symmetric Multiple Dispatch with Variance

% todo \textit{multimethods}

% todo  custom schedulers

%    Languages with \textit{multimethods}, like Common Lisp’s CLOS, Dylan, and Julia do support adding both new types and operations easily.
%    What they typically sacrifice is either static type checking, or separate compilation.

% todo ZIO, TS Effect

% todo call-by-push value and how it is related to effects

% todo context polymorphism
