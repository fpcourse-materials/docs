%! suppress = MissingImport
%! suppress = MissingLabel
%! suppress = LineBreak

% CLI args https://tex.stackexchange.com/a/1501
\newif\ifhandout
\input{flags}

%! suppress = MissingLabel
%! suppress = DocumentclassNotInRoot
%! suppress = DiscouragedUseOfDef

% * Make friends tikz & colors
%   https://en.wikibooks.org/wiki/LaTeX/Colors
% * To enable vertical top alignment globally
%   https://tex.stackexchange.com/questions/9889/positioning-content-at-the-top-of-a-beamer-slide-by-default
% * Set handout from CLI
%   https://tex.stackexchange.com/a/1501
\ifhandout
\documentclass[usenames, dvipsnames, handout]{beamer} % https://tex.stackexchange.com/questions/224091/beamer-how-to-disable-pause-temporarily
\else
\documentclass[usenames, dvipsnames]{beamer}
\fi
% ------------------------------------------------

% Graphics
\usepackage{color}
\usepackage{tabularx}
\usepackage{tikz}
% https://tikz.dev/tikz-graphs
\usetikzlibrary{positioning, shapes.geometric, arrows, automata, graphs}
\tikzset{
    expr/.style={ellipse, draw=gray!60, fill=gray!5, very thick, minimum size=7mm, yshift=0.7cm},
    hexpr/.style={ellipse, draw=gray!60, fill=blue!15, very thick, minimum size=7mm, yshift=0.7cm},
    stmt/.style={rectangle, draw=gray!60, fill=gray!5, very thick, minimum size=5mm, yshift=0.7cm},
    decl/.style={rectangle, draw=blue!60, fill=gray!5, very thick, minimum size=5mm, yshift=0.7cm},
    hdecl/.style={rectangle, draw=blue!60, fill=blue!15, very thick, minimum size=5mm, yshift=0.7cm},
    subtree/.style={shape border rotate=90, isosceles triangle, draw=gray!60, fill=gray!5, very thick, minimum size=5mm, yshift=0.0cm},
}
\usepackage{blkarray}
\usepackage{graphicx}
\usepackage{forest} % https://tex.stackexchange.com/questions/198405/how-to-change-the-color-of-subtrees-in-tikz-qtree
% ------------------------------------------------

% Math
\usepackage{amsmath, amsfonts}
\usepackage{amssymb}
\usepackage{proof}
\usepackage{mathrsfs}
% Crossed-out symbols
% https://tex.stackexchange.com/questions/75525/how-to-write-crossed-out-math-in-latex
\usepackage[makeroom]{cancel}
\usepackage{mathtools}
% ------------------------------------------------

% Additional font sizes
% https://www.overleaf.com/learn/latex/Questions/How_do_I_adjust_the_font_size%3F
\usepackage{moresize}
% Additional colors
% https://www.overleaf.com/learn/latex/Using_colours_in_LaTeX
\usepackage{xcolor}
% Textual math symbols
\usepackage{textcomp}
% ------------------------------------------------

% Language
\usepackage[utf8] {inputenc}
\usepackage[T2A] {fontenc}
\usepackage[english, russian] {babel}
\usepackage{indentfirst, verbatim}
\usetikzlibrary{cd, babel}
% ------------------------------------------------

% Fonts: https://sites.math.washington.edu/~reu/docs/latex_symbols.pdf
\usepackage{stmaryrd}
\usepackage{cmbright}
\usepackage{wasysym}
\usepackage[weather]{ifsym} % https://tex.stackexchange.com/questions/100424/how-to-use-the-ifsym-package
% https://tex.stackexchange.com/questions/615300/pdflatex-builtin-glyph-names-is-empty
\pdfmapline{=dictsym DictSym <dictsym.pfb}
\pdfmapline{=pigpen <pigpen.pfa}
\usepackage{dictsym}
% ------------------------------------------------

% Code
% * Needs -shell-escape build flag
%   https://tex.stackexchange.com/questions/99475/how-to-invoke-latex-with-the-shell-escape-flag-in-texstudio-former-texmakerx
% * Set build directory
%   https://tex.stackexchange.com/questions/339931/latex-minted-package-using-custom-output-directory-build
\usepackage{minted}
\setminted{xleftmargin=\parindent, autogobble, escapeinside=\#\#}
% ------------------------------------------------

% Template
\usetheme{CambridgeUS}
\usecolortheme{dolphin}
% https://tex.stackexchange.com/questions/231439/beamer-how-to-make-font-larger-for-page-numbers
\setbeamerfont{headline}{size=\scriptsize}
\setbeamerfont{footline}{size=\scriptsize}
% Remove heddline
% https://tex.stackexchange.com/questions/33146/how-could-i-remove-a-header-in-a-beamer-presentation
%\setbeamertemplate{headline}{}
% Slide sizes
% https://tex.stackexchange.com/questions/56768/how-to-set-a-small-default-font-size-with-beamer
%\geometry{paperwidth=140mm,paperheight=105mm} % 4:3
\geometry{paperwidth=168mm,paperheight=105mm} % 16:10
% Remove navigation bar
% https://stackoverflow.com/questions/3210205/how-to-get-rid-of-navigation-bars-in-beamer
\beamertemplatenavigationsymbolsempty
% ------------------------------------------------

% Bullets
% https://9to5science.com/change-bullet-style-formatting-in-beamer
% https://tex.stackexchange.com/questions/185742/i-need-to-change-color-of-beamer-itemize-and-subitem-separately
\setbeamertemplate{itemize item}{\scriptsize\raise1.25pt\hbox{\donotcoloroutermaths$\blacktriangleright$}}
\setbeamertemplate{itemize subitem}{\scriptsize\raise1.5pt\hbox{\donotcoloroutermaths$\blacktriangleright$}}
\setbeamertemplate{itemize subsubitem}{\tiny\raise1.5pt\hbox{\donotcoloroutermaths$\blacktriangleright$}}
\setbeamertemplate{enumerate item}{\insertenumlabel.}
\setbeamertemplate{enumerate subitem}{\insertenumlabel.\insertsubenumlabel}
\setbeamertemplate{enumerate subsubitem}{\insertenumlabel.\insertsubenumlabel.\insertsubsubenumlabel}
% ------------------------------------------------

% Table of contents format
% https://tex.stackexchange.com/questions/642927/format-table-of-contents-in-beamer
\setbeamertemplate{section in toc}{%
        {\color{blue}\inserttocsectionnumber.}
    \inserttocsection\par%
}
\setbeamertemplate{subsection in toc}{%
        {\color{blue}\hspace{1em}\scriptsize\raise1.25pt\hbox{\donotcoloroutermaths$\blacktriangleright$}}
    \inserttocsubsection\par%
}
\setbeamertemplate{subsubsection in toc}{%
        {\color{blue}\hspace{2em}\tiny\raise1.25pt\hbox{\donotcoloroutermaths$\blacktriangleright$}}
    \inserttocsubsubsection\par%
}
% ------------------------------------------------

% Misc
\usepackage{multicol}
\usepackage{hyperref}
\usepackage{soul} % https://tex.stackexchange.com/questions/23711/strikethrough-text
% ------------------------------------------------

% Fix \pause for amsmath package envs (black black magic)
% https://tex.stackexchange.com/questions/16186/equation-numbering-problems-in-amsmath-environments-with-pause/75550#75550
% https://tex.stackexchange.com/questions/6348/problem-with-beamers-pause-in-alignments
%! suppress = Makeatletter
\makeatletter
\let\save@measuring@true\measuring@true
\def\measuring@true{%
    \save@measuring@true
    \def\beamer@sortzero##1{\beamer@ifnextcharospec{\beamer@sortzeroread{##1}}{}}%
    \def\beamer@sortzeroread##1<##2>{}%
    \def\beamer@finalnospec{}%
}
%! suppress = Makeatletter
\makeatother
% ------------------------------------------------

% Sections
\newcommand{\sectionplan}[1]{\section{#1}%
    \begin{frame}[noframenumbering]{Содержание}
        \tableofcontents[currentsection]
    \end{frame}
}
\newcommand{\subsectionplan}[1]{\subsection{#1}%
    \begin{frame}[noframenumbering]{Содержание}
        \tableofcontents[currentsubsection]
    \end{frame}
}
% ------------------------------------------------

% Footnotes
\renewcommand{\thefootnote}{\arabic{footnote}}
\renewcommand{\thempfootnote}{\arabic{mpfootnote}}
% https://tex.stackexchange.com/questions/28465/multiple-footnotes-at-one-point
\usepackage{fnpct}
% ------------------------------------------------

% Links
% Colors also links on slide foot.
%\hypersetup{
%    colorlinks=true,
%    citecolor=blue,
%    linkcolor=blue,
%    urlcolor=blue
%}
% ------------------------------------------------

% Appendix
% Slide numbers
% https://tex.stackexchange.com/questions/70448/dont-count-backup-slides
\usepackage{appendixnumberbeamer}
\newcommand{\backupbegin}{
    \newcounter{framenumbervorappendix}
    \setcounter{framenumbervorappendix}{\value{framenumber}}
}
\newcommand{\backupend}{
    \addtocounter{framenumbervorappendix}{-\value{framenumber}}
    \addtocounter{framenumber}{\value{framenumbervorappendix}}
}
% ------------------------------------------------

% Custom commands
% New topic on "in previous series"
\newcommand{\newtopic}[0]{$+$}
\newcommand{\then}{$\Rightarrow$} % item: consequences


\newcommand{\err}[0]{\textcolor{red}{ошибка}} % compilation error
\newcommand{\comb}[1]{\mathbf{#1}} % combinator
\newcommand{\step}{\rightsquigarrow} % reduction step
\newcommand{\sstep}{\twoheadrightarrow} % multiple steps reduction
\newcommand{\term}[1]{\mathbf{#1}} % predefined lambda-term reference
\newcommand{\ap}{~} % lambda-term application
\newcommand{\termdef}{\coloneqq} % lamda term binding
\newcommand{\subst}[3]{\left[#2 \mapsto #3 \right] #1} % substitution
\newcommand{\eqbeta}{=_\beta} % beta equality
\newcommand{\eqeta}{=_\eta} % eta-equality
\newcommand{\eqt}{=} % tree-equality of terms
\newcommand{\tlist}[1]{\term{[}#1\term{]}} % list-term
\newcommand{\pop}[0]{\SunCloud} %item:  general eduation
\newcommand{\popslide}[0]{(\pop)}
\newcommand{\advanced}[0]{$\varhexstar$} % item: advanced science
\newcommand{\advancedslide}[0]{(\advanced)}
\newcommand{\practical}[0]{\dstechnical} % item: practical programming notions
\newcommand{\practicalslide}[0]{(\practical)}
\newcommand{\todo}[0]{todo} % item: question
\newcommand{\answer}[0]{\Lightning} % item: answer to the previous question
\newcommand{\eg}[0]{e.g.} % item: example

\newcommand{\defi}[0]{$\Delta$} % item: definition on smth
\newcommand{\textdefi}[1]{\textbf{#1}}
\newcommand{\positive}{$+$} % item: pros
\newcommand{\negative}{{\color{red} $-$}} % item: cons
\newcommand{\adding}{$+$} % item: something new
\renewcommand{\emph}[1]{{\color{blue} \textit{#1}}}
\newcommand{\vocab}[1]{\textbf{#1}} % item: important new word
\newcommand%! suppress = EscapeHashOutsideCommand
\NB[1][0.3]{N\kern-#1em{B}} % default kern amount: -0.3em
% ------------------------------------------------

% Speaker notes
% https://tex.stackexchange.com/questions/114219/add-notes-to-latex-beamer
% https://tex.stackexchange.com/questions/35444/split-beamer-notes-across-multiple-notes-pages/35496#35496
%\setbeameroption{show notes on second screen=right} % enable speaker notes
%--------------------------------------

\author[]{Андрей Стоян, Илья Колегов, Дмитрий Халанский}
\institute[]{}

\setminted{xleftmargin=\parindent, autogobble, escapeinside=??}
\newcommand{\iso}{\sim}

\title{4. Типы данных}
\author{Андрей Стоян}
\institute[ИПКН ИТМО]{ИПКН ИТМО}

\date{осень 2025}

\begin{document}

    \mymaketitle

    \begin{frame}[noframenumbering]{Содержание}
        \tableofcontents
    \end{frame}

    \sectionplan{Вариантность}

    \begin{frame}[fragile]{(Ковариантный) функтор}
        \pause
        \begin{minted}{haskell}
            class Functor f where
              fmap :: (a -> b) -> (f a -> f b)
        \end{minted}
        \vspace{1em}
        \begin{figure}[H]
            \centering
            \includegraphics[width=0.3\textwidth]{figs/functor}
        \end{figure}
    \end{frame}

    \begin{frame}[fragile]{Контравариантный функтор}
        \pause
        \begin{minted}{haskell}
            class Contravariant f where
              contramap :: (a -> b) -> (f b -> f a)
        \end{minted}
        \vspace{1em}
        \begin{figure}[H]
            \centering
            \includegraphics[width=0.3\textwidth]{figs/contra-functor}
        \end{figure}
    \end{frame}

    \begin{frame}[fragile]{Знак позиции}
        \pause
        \begin{center}
            \begin{tabular}[h]{|c|c|c|}
                \hline
                Тип                              & знак позиции \mintinline{haskell}{A} & знак позиции \mintinline{haskell}{B} \\
                \hline
                \mintinline{haskell}{Either A B} & \pause $+$                                  & $+$                                  \\
                \mintinline{haskell}{(A, B)}     & \pause $+$                                  & $+$                                  \\
                \mintinline{haskell}{A -> B}     & \pause $-$                                  & $+$                                  \\
                \hline
            \end{tabular}
        \end{center}
        \vspace{1em}
        \begin{itemize}
            \item[\todo] \pause \mintinline{haskell}{f :: ((A, B) -> C) -> (D, E)}
            \item[\todo] \pause Объявите \mintinline{haskell}{instance Contravariant F} для \mintinline{haskell}{data F a = L (a -> ()) | R Int}.
        \end{itemize}
    \end{frame}

    \begin{frame}[fragile]{Всякие другие функторы}
        \pause
        \begin{minted}{haskell}
            class Bifunctor f where
              bimap :: (a -> c) -> (b -> d) -> f a b -> f c d
        \end{minted}

        \pause\vspace{1em}
        \begin{minted}{haskell}
            class Profunctor p where
              dimap :: (c -> a) -> (b -> d) -> p a b -> p c d

            dimap serialize deserialize (query :: Sql Text Text) :: Sql Age [User]
        \end{minted}
    \end{frame}

    \sectionplan{Изоморфизм}

    \begin{frame}[fragile]{Определение}
        \pause
        \begin{minted}{c}
    to . from = id
    from . to = id
        \end{minted}
        \vspace{1em}
        \pause
        \begin{minted}{haskell}
    to :: Bool -> Maybe ()
    to b = if b then Just () else Nothing

    from :: Maybe () -> Bool
    from m = case m of Nothing -> False; Just () -> True
        \end{minted}
    \end{frame}

    \begin{frame}[fragile]{Кардинальность типа}
        \pause
        \begin{center}
            \begin{tabular}{|l|c|}
                \hline
                Тип и его декларация                                                                                                                                                                            & кардинальность \\
                \hline
                \mintinline{haskell}{data Void}                                                                                                                                                                 & $0$            \\
                \mintinline{haskell}{data Unit = Unit} & $1$ \\
                \mintinline{haskell}{data Bool = False | True}                                                                                                                                                  & $2$            \\
                \hline
            \end{tabular}
        \end{center}
        \pause
        \begin{center}
            \begin{tabular}{|l|c|}
                \hline
                Тип                                                      & кардинальность   \\
                \hline
                \mintinline{haskell}{data Either a b = Left a | Right b} & $|a| + |b|$      \\
                \mintinline{haskell}{data Pair a b = Pair a b}           & $|a| \times |b|$ \\
                \hline
            \end{tabular}
        \end{center}

        \vspace{1em}
        \pause
        \mintinline{haskell}|A -> B| изоморфно \[|A \to B| = |B|^{|A|}\]
    \end{frame}

    \begin{frame}[fragile]{Примеры}
        \pause
        \begin{itemize}
            \item \mintinline{haskell}{|Either Unit (Eigher Bool Bool)| ?\pause?= |Unit| + (|Bool| + |Bool|) = 5}.
            \item \mintinline{haskell}{Pair (Either Bool Unit) (Pair Unit Void)| ?\pause?= 0} ~--- тип \mintinline{haskell}|Void| не населён, как и кортеж, его включающий.
            \item Если \mintinline{haskell}{data Example = FirstAlternative Bool | AnotherOne Unit Bool Bool}, то \\\mintinline{haskell}{|Example| ?\pause?= |Bool| + |Unit| * |Bool| * |Bool| = 2 + 1 * 2 * 2 = 6}.
        \end{itemize}
    \end{frame}

    \begin{frame}[fragile]{Алгебраическое представление типа}
        \begin{center}
            \begin{tabular}{|p{0.5\textwidth}|c|}
                \hline
                Тип                                                      & алгебраическая формула      \\
                \hline
                \mintinline{haskell}{data Void}                          & $0$                         \\
                \mintinline{haskell}{data Unit = Unit}                   & $1$                         \\
                \mintinline{haskell}{data Bool = False | True}           & $1 + 1$ (обозначим как $2$) \\
                \mintinline{haskell}{data Maybe a = Nothing | Just a}    & $1 + a$                     \\
                \mintinline{haskell}{data Either a b = Left a | Right b} & $a + b$                     \\
                \mintinline{haskell}{data Pair a b = Pair a b}           & $a \times b$                \\
                \mintinline{haskell}{a -> b}                             & $b^a$                       \\
                \hline
            \end{tabular}
        \end{center}

        \pause\vspace{1em}
        Запишите в алгебраическом виде следующий тип:
        \begin{minted}{haskell}
        data T a b = Undefined | Defined a (a -> b)
        \end{minted}
    \end{frame}

    \begin{frame}[fragile]{Школьная алгебра}
        \vspace{-1em}
        \begin{columns}[onlytextwidth]
            \begin{column}[t]{0.485\textwidth}
                \begin{minted}{haskell}
            -- ?$(c^b)^a \iso c^{a\times b}$?
            to :: (a -> b -> c) -> (a, b) -> c
            to = uncurry
            from :: ((a, b) -> c) -> a -> b -> c
            from = curry
                \end{minted}
            \end{column}\hfill%
            \begin{column}[t]{0.485\textwidth}
                \pause
                \begin{figure}
                    \centering
                    \includegraphics[width=1\textwidth]{figs/school-alg}
                \end{figure}
            \end{column}
        \end{columns}
    \end{frame}

    \begin{frame}[fragile]{Больше алгебры}
        \pause
        \begin{itemize}
            \item[\todo] Покажите, что $(a + b) + c \iso a + (b + c)$.
            \item[\todo] Покажите, что $c^{a + b} \iso c^a\times c^b$.
        \end{itemize}
        \pause\vspace{1em}
        \begin{minted}{haskell}
            -- ?$a \times a \iso a^2$?
            get :: (a, a) -> (Bool -> a)
            get (x, y) idx = if idx then x else y
            tabulate :: (Bool -> a) -> (a, a)
            tabulate f = (f True, f False)
        \end{minted}
    \end{frame}

    \begin{frame}[fragile]{Каноническое представление типа}
        \pause
        \[
            \sum_{i}\prod_{j} t_{ij}
        \]
    \end{frame}

    \sectionplan{Рекурсивные типы}

    \begin{frame}[fragile]{Рекурсивные типы}
        \begin{minted}{haskell}
            fac n = if n <= 1 then 1 else n * ?\framebox{fac}? (n - 1)
            data Nat = Zero | Suc ?\framebox{Nat}?
        \end{minted}
    \end{frame}

    \begin{frame}[fragile]{Список через неподвижную точку}
        \pause
        \begin{minted}{haskell}
            data ListShape a r = Either () (a, r) -- ?$\lambda a~r\ldotp 1 + a\times r$?
            data FixList a = FixList (ListShape a (FixList a))
            -- FixList a ?$\iso$? ListShape a (FixList a)
        \end{minted}
        \vspace{1em}
        \pause
        \begin{minted}{haskell}
            foldr :: (Either () (a, r) -> r) -> FixList a -> r
            foldr phi (FixList shape) = case shape of
              Left () -> phi (Left ())
              Right (x, xs) -> phi (Right (x, foldr phi xs))
            -- сравните с классическим определением
            foldr :: r -> (a -> r -> r) -> [a] -> r
        \end{minted}
    \end{frame}

    \begin{frame}[fragile]{Неподвижная точка функтора}
        \pause
        \begin{minted}{haskell}
            newtype Fix :: (Type -> Type) -> Type
            newtype Fix f = In { out :: f (Fix f) }

            data ListF a r = FNil | FCons a r
            type List a = Fix (ListF a)
        \end{minted}
    \end{frame}

\end{document}
